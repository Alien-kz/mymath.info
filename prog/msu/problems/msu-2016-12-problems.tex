\documentclass[10pt, a4paper]{article}

\usepackage[T2A]{fontenc}		%cyrillic output
\usepackage[utf8]{inputenc}		%cyrillic output
\usepackage[english, russian]{babel}	%word wrap
\usepackage{amssymb, amsfonts, amsmath}	%math symbols
\usepackage{mathtext}			%text in formulas
\usepackage{geometry}			%paper format attributes
\usepackage{fancyhdr}			%header
\usepackage{graphicx}			%input pictures
\usepackage{tabularx}			%smart table
\usepackage{longtable}			%long table
\usepackage{tikz, pgfplots}		%draw pictures and graphics
\usetikzlibrary{patterns}		%draw pictures: fill
\usetikzlibrary{positioning}	%draw pictures: below of
\usetikzlibrary{calc}			%draw pictures: $\i$
\usepackage{listofitems}		%draw from tex-list
\usepackage{enumitem}			%enumarate parameters

\geometry{left=2cm, right=2cm, top=2cm, bottom=2cm, headheight=15pt}
\setlist[enumerate]{leftmargin=*}	%remove enumarate indenttion
\sloppy							%correct overfull
\pagestyle{empty}				%no page numbers

\newcommand{\accept}[2]{
	\centerline{\boxed{#1}}
	\newline
	\centerline{\scriptsize{#2}}
}
\newcommand{\reject}[1]{
	\centerline{#1}
}


\newcommand{\head}[4]
{
	\thispagestyle{fancy}
	\fancyhf{}
	\chead{#3, #4}

	\begin{center}
	\begin{large}
	#1 \\
	\textit{#2}\\
	\end{large}
	\end{center}

}


% format

\newcommand{\informat}[1]
{
	\subsubsection*{Ввод} #1
}

\newcommand{\outformat}[1]
{
	\subsubsection*{Вывод} #1
}

\newcommand{\example}[2]
{
	\subsubsection*{Пример}
	\noindent
	\begin{center}
	\begin{tabularx}{\linewidth}{|X|X|}
	\hline
	Ввод & Вывод \\
	\hline
	{\tt #1} & {\tt #2}		\\
	\hline
	\end{tabularx}
	\end{center}
}

\newcommand{\examplelong}[2]
{
	\subsubsection*{Пример}
	\noindent
	\begin{center}
	\begin{tabularx}{\textwidth}{|l|X|}
	\hline
	Ввод & Вывод \\
	\hline
	#1 & #2		\\
	\hline
	\end{tabularx}
	\end{center}
}

\newcommand{\examplee}[4]
{
	\subsubsection*{Пример}
	\noindent
	\begin{center}
	\begin{tabularx}{\linewidth}{|X|X|}
	\hline
	Ввод 	& Вывод  	\\
	\hline
	{\tt #1} & {\tt #2}	\\
	\hline
	{\tt #3} & {\tt #4}	\\
	\hline
	\end{tabularx}
	\end{center}
}

\newcommand{\exampleee}[6]
{
	\subsubsection*{Пример}
	\noindent
	\begin{center}
	\begin{tabularx}{\linewidth}{|X|X|}
	\hline
	Ввод 	& Вывод  	\\
	\hline
	{\tt #1} & {\tt #2}	\\
	\hline
	{\tt #3} & {\tt #4}	\\
	\hline
	{\tt #5} & {\tt #6}	\\
	\hline
	\end{tabularx}
	\end{center}
}

\newcommand{\exampleeee}[8]
{
	\subsubsection*{Пример}
	\noindent
	\begin{center}
	\begin{tabularx}{\linewidth}{|X|X|}
	\hline
	Ввод 	& Вывод  	\\
	\hline
	{\tt #1} & {\tt #2}	\\
	\hline
	{\tt #3} & {\tt #4}	\\
	\hline
	{\tt #5} & {\tt #6}	\\
	\hline
	{\tt #7} & {\tt #8}	\\
	\hline
	\end{tabularx}
	\end{center}
}

\newcommand{\exampleeeee}[5]
{
	\subsubsection*{Пример}
	\begin{center}
	\begin{tabularx}{\linewidth}{|X|X|}
	\hline
	Ввод 	& Вывод  	\\
	\hline
	#1		\\
	\hline
	#2		\\
	\hline
	#3		\\
	\hline
	#4		\\
	\hline
	#5		\\
	\hline
	\end{tabularx}
	\end{center}
}

\newcommand{\examplepic}[3]
{
	\subsubsection*{Пример}
	\noindent
	\begin{center}
	\begin{tabularx}{\linewidth}{|l|l|X|}
	\hline
	Ввод 	& Вывод  	& Пояснение\\
	\hline
	{\tt #1} 		& {\tt #2} 		& #3\\
	\hline
	\end{tabularx}
	\end{center}
}


\newcommand{\excomm}[1]
{
	\subsubsection*{Комментарий}
	\textit{#1}
}

\newcommand{\problemauthor}[1]{
\begin{flushright}
\textit{Автор: #1}
\end{flushright}
}

\newcommand{\problemofferer}[1]{
\begin{flushright}
\textit{Предложил: #1}
\end{flushright}
}

\usepackage{listings}
\lstset{language=C,
        basicstyle=\ttfamily,
        keywordstyle=\color{blue},
        frame=single,
        numbers=left,
        tabsize=4}

\begin{document}

\head{Открытая личная олимпиада по программированию \\ Зимний тур 2016}{13 декабря 2016}{Казахстанский филиал МГУ имени М.В.Ломоносова}{г.~Астана}

\subsection*{A. A train problem}

Айтмухамед во время своего последнего путешествия в поезде заказал обед из вагона ресторана. Официант постоянно проходил мимо, открывая и закрывая двери между вагонами, но заказа всё еще не было. В ожидании трапезы Айтмухамед решил подсчитать сколько же щелчков от закрытия дверей услышал сам официант. При этом он, как истинный математик, четко формализовал эту задачу:

Поезд состоит из $n$ вагонов. Известно, что $k$-й по счету вагон --- это вагон ресторан. В каждом вагоне, кроме вагона ресторана, едет ровно один пассажир, который заказал одну порцию. Официант должен разнести всем заказы, при этом одновременно он может нести только одну порцию. Когда официант проходит через переход между вагонами он открывает и закрывает две двери. Сколько всего дверей он откроет, чтобы разнести все заказы, если блюда разносить он начал из вагона ресторана и закончить свой поход намерен там же?

\informat{Два целых числа $n$ и $k$, где $1 \le n \le 10^{9}$ и $1 \le k \le n$.}

\outformat{Одно целое число --- ответ на задачу.}

\exampleee{3 1}{12}{3 2}{8}{5 2}{28}

\excomm{В первом примере официант сходит во второй вагон (4 двери) и в третий вагон (8 дверей).}



\subsection*{B. Bead garland}

К Новому году Вова уже приготовил $n$ коробок елочных шариков в количестве $a_1$, $a_2$, $\dots$, $a_n$, соответственно (абсолютно все шарики попарно различны). Он хотел собрать гирлянду, взяв из каждой коробки ровно по одному шарику, но получилось слишком много различных вариантов гирлянды (порядок шариков в гирлянде не важен). Чтобы уменьшить количество различных способов собрать гирлянду, Вова решил объединить некоторые коробки (после объединения коробок он опять будет выбирать ровно один шарик из каждой полученной коробки). Так чему же равно минимальное количество различных гирлянд при всех возможных объединениях коробок?

\informat{На первой строке одно целое число $n$, где $1 \le n \le 10^{5}$. \newline
На второй строке $n$ целых чисел $a_1$, $a_2$, $\dots$, $a_n$, где $1 \le a_i \le 10^{9}$.}

\outformat{Одно целое число --- ответ на задачу.}

\examplee{2 \newline 1 2}{2}{3 \newline 3 3 3}{9}

\excomm{В первом примере: если он объединит две коробки, то получит 3 варианта гирлянды из одного шарика; если не будет объединять коробки, то получит 1 * 2 = 2 варианта гирлянды из двух шариков. \newline
Во втором примере: если он объединит 3 коробки, то получит 9 вариантов гирлянды из одного шарика; если он объединит любые 2 коробки, то получит 6 * 3 = 18 вариантов гирлянды из двух шариков; если не будет объединять коробки, то получит 3 * 3 * 3 = 27 вариантов гирлянды из трех шариков.}



\subsection*{C. Champion}

Темирхан, студент 3 курса ВМК, является действующим победителем данной олимпиады прошлого года. Это просто информация.

\informat{Три целых числа.}

\examplee{89 101 115}{Yes}{78 111 32}{No }



\subsection*{D. Digits}

Команда Снежный Куб как опытная команда уже давно привыкла к задачам, в которых условие выглядит значительно проще решения. Вот пример такой простой задачи: найти количество $n$-значных чисел, у которых сумма цифр равна $k$. А сможете ли Вы решить эту задачу?

\informat{Два целых числа $n$ и $k$, где $1 \le n \le 500$ и $1 \le k \le 10^6$.}

\outformat{Одно целое число --- ответ на задачу по модулю $10^9 + 7$.}

\exampleee{2 5}{5}{3 12}{66}{10 50}{349279750}

\excomm{В первом примере подходят числа 14, 23, 32, 41 и 50.}



\subsection*{E. Elimination}

Однажды Павел решил подключить 2 компьютера с помощью сетевого $k$-жильного шнура. Длина шнура равна $n$ метров, а расстояние между компьютерами $m$ метров ($m \leqslant n$). При этом Павел знает, что на некоторых участках у отдельных жил шнура есть разрывы, поэтому он хочет обрезать шнур (возможно, с двух концов) до длины $m$ метров так, чтобы осталось максимальное количество полностью работающих жил. Какое максимальное количество жил будет без разрывов на отрезке из $m$ метров?

\informat{Три целых числа $n$, $k$ и $m$, где $1 \leqslant n \leqslant 10^5$,  $1 \leqslant k \leqslant 10$ и $1 \leqslant m \leqslant n$. \newline
Далее $k$ строк длины $n$ из нулей и единиц, где $i$-я строка описывает состояние $i$-й жилы: 1 --- участок без разрывов, 0 --- участок с разрывом).
}

\outformat{Одно целое число --- ответу на задачу.}

\example{
10 4 3 \newline
1110111111 \newline
1011111111 \newline
0011111100 \newline
1111110111}
{3}



\subsection*{F. Forts}

Куат и Димитрий недавно решили написать двумерную компьютерную игру <<Forts>>. Цель игры --- разрушить базу соперника, которая имеет прямоугольную форму. Как только бомба взрывается в некоторой точке с координатами ($a$, $b$), то уничтожается всё, что попадает в радиус $R$. Поскольку игру разрабатывает Куат, то он сделал себе бесконечную базу, которая занимает весь 3 квадрант ($x < 0$ и $y < 0$). Димитрий уже и не надеется разрушить всю базу, поэтому ему достаточно узнать, какая площадь базы Куата будет уничтожена за данный ход.

\informat{Три целых числа $a$, $b$, $R$, где $0 \le a \le 1000$, $0 \le b \le 1000$ и $1 \le R \le 1000$.}

\outformat{Одно вещественное число --- ответ на задачу с точностью не менее 2 знаков после запятой.}

\examplee{0 0 1}{0.78540}{3 4 10}{22.08995}

\end{document} 
