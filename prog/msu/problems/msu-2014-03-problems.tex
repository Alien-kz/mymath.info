\documentclass[12pt, a4paper]{article}

\usepackage[T2A]{fontenc}		%cyrillic output
\usepackage[utf8]{inputenc}		%cyrillic input
\usepackage[english, russian]{babel}	%word wrap
\usepackage{amssymb, amsfonts, amsmath}	%math symbols
\usepackage{mathtext}			%text in formulas
\usepackage{geometry}			%paper format attributes
\usepackage{fancyhdr}			%header
\usepackage{graphicx}			%input pictures
\usepackage{tikz}				%draw pictures
\usepackage{pgfplots}			%draw graphics
\usetikzlibrary{patterns}		%draw pictures: fill
\usetikzlibrary{calc}			%draw pictures: coordinate calc
% \usetikzlibrary{external}		%draw pictures: cache pictures
% \tikzexternalize				%cache pictures
\usepackage{listofitems}		%list of arguments (pictures)
\usepackage{enumitem}			%enumarate parameters
\usepackage{titlesec}			%new subsection option

\newcommand{\accept}[2]{
	\centerline{\boxed{#1}}
	\newline
	\centerline{\scriptsize{#2}}
}
\newcommand{\reject}[1]{
	\centerline{#1}
}

\geometry{left=1cm, right=1cm, top=2cm, bottom=1cm, headheight=28pt}
\setlist[enumerate]{leftmargin=*}	%remove enumarate indenttion
\sloppy							%correct overfull
\newcommand{\subsectionbreak}{\clearpage} %new section

\newcommand{\head}[4]
{
	\pagestyle{fancy}
	\fancyhf{}
	\lhead{#3 \\ #2}
	\rhead{#1}
}


% format

\newcommand{\informat}[1]
{
	\paragraph{Ввод.\\} #1
}

\newcommand{\outformat}[1]
{
	\paragraph{Вывод.\\} #1
}

\newcommand{\example}[2]
{
	\paragraph{Пример.\\}
	{\tt
	\begin{tabular}{|p{0.4\linewidth}|p{0.4\linewidth}|}
	\hline
	Ввод & Вывод \\
	\hline
	#1 & #2		\\
	\hline
	\end{tabular}
	}
}

\newcommand{\examplee}[4]
{
	\paragraph{Пример.\\}
	{\tt
	\begin{tabular}{|p{0.4\linewidth}|p{0.4\linewidth}|}
	\hline
	Ввод 	& Вывод  	\\
	\hline
	#1 		& #2 		\\
	\hline
	#3		& #4		\\
	\hline
	\end{tabular}
	}
}

\newcommand{\examplEEE}[6]
{
	\paragraph{Пример.\\}
	{\tt
	\begin{tabular}{|p{0.5\linewidth}|p{0.3\linewidth}|}
	\hline
	Ввод 	& Вывод  	\\
	\hline
	#1 		& #2 		\\
	\hline
	#3		& #4		\\
	\hline
	#5		& #6		\\
	\hline
	\end{tabular}
	}
}

\newcommand{\exampleee}[6]
{
	\paragraph{Пример.\\}
	{\tt
	\begin{tabular}{|p{0.4\linewidth}|p{0.4\linewidth}|}
	\hline
	Ввод 	& Вывод  	\\
	\hline
	#1 		& #2 		\\
	\hline
	#3		& #4		\\
	\hline
	#5		& #6		\\
	\hline
	\end{tabular}
	}
}

\newcommand{\exampleeee}[8]
{
	\paragraph{Пример.\\}
	{\tt
	\begin{tabular}{|p{0.4\linewidth}|p{0.4\linewidth}|}
	\hline
	Ввод 	& Вывод  	\\
	\hline
	#1 		& #2 		\\
	\hline
	#3		& #4		\\
	\hline
	#5		& #6		\\
	\hline
	#7		& #8		\\
	\hline
	\end{tabular}
	}
}

\newcommand{\exampleeeee}[5]
{
	\paragraph{Пример.\\}
	{\tt
	\begin{tabular}{|p{0.4\linewidth}|p{0.4\linewidth}|}
	\hline
	Ввод 	& Вывод  	\\
	\hline
	#1		\\
	\hline
	#2		\\
	\hline
	#3		\\
	\hline
	#4		\\
	\hline
	#5		\\
	\hline
	\end{tabular}
	}
}

\newcommand{\examplepic}[3]
{
	\subsection*{Пример.}
	{\tt
	\noindent
	\begin{tabular}{|p{0.1\linewidth}|p{0.1\linewidth}|p{0.5\linewidth}|}
	\hline
	Ввод 	& Вывод  	& Пояснение\\
	\hline
	#1 		& #2 		& #3\\
	\hline
	\end{tabular}
	}
}


\newcommand{\excomm}[1]
{
	\paragraph{Комментарий. \\}
	\textit{#1}
}

\begin{document}

\head{Открытая олимпиада по программированию \\ Весенний тур 2014}{17 марта 2014}{Казахстанский филиал МГУ имени М.В.Ломоносова}{г.~Астана}

\subsection*{A. Ayat and the film}

Аят решил посмотреть какой-нибудь хороший фильм. Он взял пустую флешку, чтобы скопировать на нее фильм с лучшим рейтингом по версии AYAT FILM RATING. Выяснилось, что флешка не резиновая, и далеко не каждый фильм помещается на нее. Поэтому Аят решил выбрать фильм с лучшим рейтингом, который поместится на флешку. Какой фильм он из этого списка выбрал? Или Аят, взгрустнув по поводу несостоявшегося киносеанса, саботированного маленькой флешкой, пошел на пару физической культуры?

\informat{Два целых числа: $S$ --- размер флешки (от 1 до 30000) и $N$ --- количество фильмов в списке (от 1 до 100). Далее $N$ пар целых чисел: $R_i$ --- рейтинг $i$-го фильма (от 1 до 100) и $V_i$ --- размер $i$-го фильма (от 1 до 30000). Рейтинги у всех фильмов различны.}

\outformat{Целое число $K$ --- номер фильма с максимальным рейтингом, который помещается на флешке. Если Аят не смог выбрать фильм, выведите -1.}

\examplee{1000 2 \newline 100 2000 \newline 50 4700}{-1}{4000 5 \newline 100 4700 \newline 20 6200 \newline 50 1400 \newline 40 700 \newline 55 4200}{3}



\subsection*{B. Big dipper}

Команда Big-dipper --- это Денис, Адиль и Надира. Но это никак не помогает решить задачу.

\informat{Два целых числа $A$ и $B$ (оба от 1 до 1000).}

\outformat{Одно целое число.}

\exampleeee{7 2}{95}{2 2}{40}{235 152}{38783}{15 25}{0}



\subsection*{C. Comparing}

Маша и Вадим написали по строке одинаковой длины $N$ из букв латинского алфавита: $a$ и $b$. Когда они сравнили строки, выяснилось, что строки отличаются. <<Так не пойдет! Сейчас мы сделаем из них одинаковые строки!>> --- сказал тот, кто повыше, пошире и носит очки. Они решили привести обе строки к общему виду. Воодушевлённый воспоминаниями годичной давности о методах сортировки, Вадим придумал следующие правила <<приведения>>: за один ход можно переставить две соседних буквы в одной из строк, если эти буквы различны (то есть $ab \rightarrow ba$ или $ba \rightarrow ab$). <<С такими правилами ты точно не приведешь строки $aa$ и $bb$ к одинаковой!>> --- ответила та, кто пониже, стройней и с хорошим зрением. Проверьте, смогут ли ребята привести данные строки к общему виду, и если смогут, то какое минимальное количество ходов понадобится?

\informat{Целое число $N$ --- длина строк (от 1 до 100). Две строки из $N$ латинских символов $a$ и $b$.}

\outformat{Целое число $K$ --- минимальное количество ходов, необходмое для приведения к общему виду. Если строки привести нельзя, выведите -1.}

\exampleee{2 \newline aa \newline bb}{-1}{10 \newline aaaaaaaaab \newline baaaaaaaaa}{9}{6 \newline baaabb \newline abbaab}{3}



\subsection*{D. Dima's divided numbers}

Диму попросили написать программу, которая перебирает все неотрицательные числа, состоящие из не более, чем $M$ цифр. Когда ему давали задание, то ни слова не сказали про систему счисления, в которой должны быть записаны числа. Поэтому хитрый Дима выбрал двоичную систему счисления, чтобы программа работала как можно быстрее (подходящих чисел в ней всего лишь $2^M$). Как только довольный Дима доложил о выполнении задания, ему дали следующее: написать такую же программу, но чтобы она работала параллельно на кластере из $D$ компьютеров, причем каждое число должно быть получено ровно одним компьютером ровно один раз. Так как Дима в глубине души за равенство всех, всего и вся, то он решил разделить числа между компьютерами так, чтобы все компьютеры перебрали одинаковое количество чисел. Выяснилось, что далеко не для любой системы счисления можно распределить все нужные ему числа поровну между $D$ компьютерами. Тогда он решил найти минимальное основание системы счисления, для которой это можно сделать. Помогите Диме! 

\informat{Два целых числа: $M$ --- максимальное количество цифр в числе (от 1 до $10^9$) и $D$ --- количество компьютеров (от 2 до $10^9$).}

\outformat{Целое число $K$ --- минимальное основание системы счисления ($K > 1$), в которой все числа из не более, чем $M$ цифр, можно разделить поровну между $D$ компьютерами.}

\exampleee{3 1000}{10}{2 12}{6}{4 48}{6}



\subsection*{E. Elegant system}
В отличии от Димы у Вани другая позиция по выбору основания системы счисления. Он считает, что двоичная система счисления --- лучшая система счисления в мире. После курса дискретной математики это мнение настолько укрепилось, что он решил в десятичной системе счисления ввести <<двоичное округление>> для чисел из устаревшей десятичной системы счисления в передовую двоичную. Суть округления довольна проста: любое натуральное число заменяется на ближайшее, в записи которого присутствуют только цифры 0 и 1.  Напишите программу, которая <<округляет>> числа. 

\informat{Целое число $N$ (от 1 до $10^{100}$). Ввод заканчивается точкой.}

\outformat{Целое число $K$ --- число, полученное после <<двоичного округления>> без ведущих нулей. Если ближайших числа два, то округлять можно в любую сторону.}

\exampleee{5556.}{10000}{1011556.}{1011111}{101101234567890.}{101101111111111}



\subsection*{F. Fantastic chess}

Андрей и Ануар играют в игру с неадекватным ферзем на прямоугольной шахматной доске. Неадекватный ферзь может ходить вправо, вниз или вправо-вниз по диагонали на любое количество клеток (только в 3 направлениях, а не в 8, как в нормальных шахматных правилах). Хоть этот ферзь и неадекватен, но с правилами этикета знаком: он не может бить другие фигуры и ходить сквозь них. В начале игры ферзь стоит в левом верхнем углу доски. Ходить начинает Андрей. Проигрывает тот, кто не может сделать ход. Кто выиграет при оптимальной игре обоих игроков?

\informat{Целые числа $N$, $M$ --- количество строк и столбцов доски соответственно (оба числа от 1 до 100). Далее матрица $N \times M$, состоящая из символов '0' (ноль --- свободные клетки) и 'x' (икс --- клетки, занятые другими фигурами). Гарантируется, что левый верхний угол помечен свободным.}

\outformat{Строка 'Andrew' (без кавычек), если выиграет Андрей. Строка 'Anuar'  (без кавычек), если выиграет Ануар.}

\exampleee{3 6 \newline
000000 \newline
0xxx00 \newline
000000}
{Andrew}
{1 1 \newline 0}
{Anuar}
{4 4 \newline
00x0 \newline
0x00 \newline
x000 \newline
0000}
{Anuar}



\subsection*{G. Geometry}

Никто уже и не помнит, какой был праздник, но суть была в торте, который Илья принес домой. На празднике было трое друзей, и Илья в магазине выбрал торт в форме прямоугольника (его легко разделить на 4 равных части). Но, транспортируя торт из магазина домой, Илья споткнулся и торт из красивого ровного прямоугольника превратился в непонятный выпуклый четырехугольник. Все 4 вишни, что украшали торт, скатились к вершинам четырехугольника так, что в каждой вершине оказалось по одной вишне. Когда Илья принес торт домой, то перед ним встала непростая задача: как его разделить на 4 равных по площади части (это же не прямоугольник на 4 равных части делить)? Но Илья не растерялся и абстрагировался! Он провел через центр квадратного стола 2 оси параллельно краям стола (хотя бы с делением стола на 4 равных части не возникло проблем) и положил торт так, что все 4 вишни оказались в разных четвертях. Внимательно присмотревшись, Илья понял, что на торте есть такая особенная точка $M$, что если через нее провести две прямые, параллельные осям, то все 4 полученных кусочка будут в форме четырехугольников, равны по площади и на каждом будет ровно по одной вишне. Осталось найти эту точку. Помогите Илье!

\informat{8 целых чисел ($X_1$, $Y_1$, $X_2$, $Y_2$, $X_3$, $Y_3$, $X_4$, $Y_4$), которые задают 4 последовательных вершины четырехугольника. Гарантируется, что N-я вершина лежит в N-й четверти ($N=1, 2, 3, 4$). Модуль каждого числа не менее 1 и не более 100.}

\outformat{2 вещественных числа с точноcтью не менее 2 знаков после запятой --- координаты точки $M$, через которые проведены разрезы. Гарантируется, что решение существует.}

\examplee{2 3 -2 2 -1 -2 3 -1}{0.5 0.5}{1 3 -1 1 -1 -3 1 -5}{0.16 -1.0}



\subsection*{H. Ha-ha-ha}

Двумерная металлическая решетка имеет вид прямоугольника $(N-1) \times (M-1)$. В узлах решетки находятся атомы, которые пронумерованы от $(1, 1)$ --- левый верхний до $(N, M)$ --- правый нижний. У каждого атома есть некоторое число электронов, причем на решетке есть ровно один электрон--непоседа на атоме $(i_1, j_1)$ и один электрон--ускоритель $(i_2, j_2)$. Все электроны, кроме непоседы, всегда остаются на своих атомах. Каждую секунду электрон-непоседа переходит из атома $A$ на соседний по горизонтали или вертикали атом $B$, если число электронов в атоме $B$ на 1 меньше, чем в $A$ (с учетом самого электрона--непоседы). Все атомы, на которых появляется электрон $A$, он отмечает. Если электрон-непоседа добирается до атома, на котором находится электрон-ускоритель, то электрон-непоседа становится сильно--заряженным и теперь может перепрыгивать через один атом. Электрон делает прыжок из атома $A$ в атом $B$ через атом $M$, если:
\begin{enumerate}
\item $A$, $M$, $B$ лежат на прямой параллельной сторонам решетки;
\item $B$ содержит на 1 электрон меньше чем А.
\end{enumerate}
При этом атом M электрон-непоседа не отмечает. Какое наибольшее количество атомов сможет отметить электрон-непоседа?

\informat{Два целых числа $N$, $M$ --- количество строк и столбцов решетки (оба числа от 1 до 100). Матрица $N \times M$, состоящая из целых чисел $A_{ij}$, --- количество электронов на позиции $(i, j)$ (все элементы матрицы от 1 до 100).
Две пары целых чисел $(i_1, j_1)$ и $(i_2, j_2)$ --- координаты электрона--непоседы и электрона--ускорителя (номер строки от 1 до $N$, столбца от 1 до $M$).}

\outformat{Одно целое число --- наибольшее возможное количество отмеченных атомов.}

\exampleee{2 2 \newline
4 3 \newline
4 5 \newline
2 1 \newline
2 2}{1}
{3 4 \newline 
3 3 3 3 \newline
3 1 1 1 \newline
4 1 3 1 \newline        
3 1 \newline
1 4}{7}
{3 3 \newline
5 5 1 \newline
5 1 5 \newline
1 5 6 \newline
3 3 \newline
3 3}{6}

\end{document} 
