\documentclass[12pt, a4paper]{article}

\usepackage[T2A]{fontenc}		%cyrillic output
\usepackage[utf8]{inputenc}		%cyrillic output
\usepackage[english, russian]{babel}	%word wrap
\usepackage{amssymb, amsfonts, amsmath}	%math symbols
\usepackage{mathtext}			%text in formulas
\usepackage{geometry}			%paper format attributes
\usepackage{fancyhdr}			%header
\usepackage{graphicx}			%input pictures
\usepackage{tikz}				%draw pictures
\usetikzlibrary{patterns}		%draw pictures: fill
\usetikzlibrary{calc}			%draw pictures: coordinate calc
% \usetikzlibrary{external}		%draw pictures: cache pitcures
% \tikzexternalize				%cache pictures
\usepackage{listofitems}		%list of arguments (pictures)
\usepackage{enumitem}			%enumarate parameters
\usepackage{titlesec}			%new subsection option


\geometry{left=1cm, right=1cm, top=2cm, bottom=1cm, headheight=15pt}
\setlist[enumerate]{leftmargin=*}	%remove enumarate indenttion
\sloppy							%correct overfull
\newcommand{\subsectionbreak}{\clearpage} %new section

\newcommand{\head}[4]
{
	\pagestyle{fancy}
	\fancyhf{}
	\lhead{#3 \\ #2}
	\rhead{#1}
}


% format

\newcommand{\informat}[1]
{
	\paragraph{Ввод.\\} #1
}

\newcommand{\outformat}[1]
{
	\paragraph{Вывод.\\} #1
}

\newcommand{\example}[2]
{
	\paragraph{Пример.\\}
	{\tt
	\begin{tabular}{|p{0.4\linewidth}|p{0.4\linewidth}|}
	\hline
	Ввод & Вывод \\
	\hline
	#1 & #2		\\
	\hline
	\end{tabular}
	}
}

\newcommand{\examplee}[4]
{
	\paragraph{Пример.\\}
	{\tt
	\begin{tabular}{|p{0.4\linewidth}|p{0.4\linewidth}|}
	\hline
	Ввод 	& Вывод  	\\
	\hline
	#1 		& #2 		\\
	\hline
	#3		& #4		\\
	\hline
	\end{tabular}
	}
}

\newcommand{\examplEEE}[6]
{
	\paragraph{Пример.\\}
	{\tt
	\begin{tabular}{|p{0.5\linewidth}|p{0.3\linewidth}|}
	\hline
	Ввод 	& Вывод  	\\
	\hline
	#1 		& #2 		\\
	\hline
	#3		& #4		\\
	\hline
	#5		& #6		\\
	\hline
	\end{tabular}
	}
}

\newcommand{\exampleee}[6]
{
	\paragraph{Пример.\\}
	{\tt
	\begin{tabular}{|p{0.4\linewidth}|p{0.4\linewidth}|}
	\hline
	Ввод 	& Вывод  	\\
	\hline
	#1 		& #2 		\\
	\hline
	#3		& #4		\\
	\hline
	#5		& #6		\\
	\hline
	\end{tabular}
	}
}

\newcommand{\exampleeee}[8]
{
	\paragraph{Пример.\\}
	{\tt
	\begin{tabular}{|p{0.4\linewidth}|p{0.4\linewidth}|}
	\hline
	Ввод 	& Вывод  	\\
	\hline
	#1 		& #2 		\\
	\hline
	#3		& #4		\\
	\hline
	#5		& #6		\\
	\hline
	#7		& #8		\\
	\hline
	\end{tabular}
	}
}

\newcommand{\exampleeeee}[5]
{
	\paragraph{Пример.\\}
	{\tt
	\begin{tabular}{|p{0.4\linewidth}|p{0.4\linewidth}|}
	\hline
	Ввод 	& Вывод  	\\
	\hline
	#1		\\
	\hline
	#2		\\
	\hline
	#3		\\
	\hline
	#4		\\
	\hline
	#5		\\
	\hline
	\end{tabular}
	}
}

\newcommand{\examplepic}[3]
{
	\subsection*{Пример.}
	{\tt
	\noindent
	\begin{tabular}{|p{0.1\linewidth}|p{0.1\linewidth}|p{0.5\linewidth}|}
	\hline
	Ввод 	& Вывод  	& Пояснение\\
	\hline
	#1 		& #2 		& #3\\
	\hline
	\end{tabular}
	}
}


\newcommand{\excomm}[1]
{
	\paragraph{Комментарий. \\}
	\textit{#1}
}

\begin{document}

\head{Открытая олимпиада по программированию \\ Зимний тур 2013}{11 декабря 2013}{Казахстанский филиал МГУ имени М.В.Ломоносова}{г.~Астана}

\subsection*{A. A}

Во время разговора между Адилем и Денисом иногда звучит шутка <<A ya Denis!>>. Сколько было шуток в разговоре, если других шуток у них нет?

\informat{Строка длиной от 1 до 1000 символов, состоящая из букв английского алфавита \mbox{\tt 'a'..'z'}, \mbox{\tt 'A'..'Z'}, пробелов и 4 видов знаков препинания (,.!?). Ввод оканчивается переносом строки.}

\outformat{Одно неотрицательное число --- количество смешных шуток.}

\example{A ya Denis! A A ya Denis!}{2}

\subsection*{B. Beautiful tree}

Надира любит природу. Особенно красивые деревья. Красота дерева измеряется количеством листьев этого дерева. Посчитайте красоту дерева. 

\informat{Два натуральных числа: $n$ --- количество вершин графа (от 2 до 100), $m$ --- количество ребер графа (от 1 до $\frac{n(n-1)}{2}$). Далее $m$ пар вершин, которые соединены (вершины пронумерованы от 1 до $n$). Гарантируется, что кратных ребер и петель нет.}

\outformat{Одно натуральное число --- красоту дерева, если данный граф является деревом. Иначе выведите~0.}

\examplee
{5 4 \newline
1 3 \newline
2 5 \newline
3 4 \newline
4 2}
{2}
{6 4 \newline
1 6 \newline
2 4 \newline
4 3 \newline
6 2}
{0}

\subsection*{C. Cube}

Адиль предложил Надире и Денису загадать по одному числу от 1 до $n$ независимо друг от друга (причем каждое число с одинаковой вероятностью). Надира никогда не загадывает полные квадраты, а Денис --- полные кубы (по словам Дениса, потому что они --- <<полные>>). Какова вероятность того, что они загадают одно и то же число?

\informat{Одно целое число $n$ (от 2 до 1000).}

\outformat{Вещественное число (с точностью не менее 6 знаков после запятой) --- вероятность того, что они загадали одно и то же число. }

\examplee{3}{0.50000000}{1000}{0.00100280}

\excomm{Вероятностью случайного события $A$ называется отношение числа $n$ равновероятных элементарных событий, составляющих событие $A$, к числу всех возможных элементарных событий $N$: $P(A)= \frac{n}{N}$.}



\subsection*{D. Difficult geometry}

Адиль любит кататься на круглой лодке (радиуса $R$) в треугольном бассейне (со сторонами $a$, $b$ и $c$). Он всегда находится ровно в центре лодки. Хотя Адиль и очень хорошо плавает, он не хочет перевернуться и упасть в воду. Поэтому он плывет так, чтобы лодка всегда касалась хотя бы одной из стенок бассейна. Так Адиль хочет проплыть вдоль всего периметра бассейна. Посчитайте, какой путь проделает Адиль?

\informat{Четыре целых числа: $a$, $b$, $c$  (от 1 до 1000) --- предполагаемые размеры бассейна, $R$ --- радиус лодки (от 1 до 1000).}

\outformat {Вещественное число (с точностью не менее 6 знаков после запятой) --- длина пути, который проделает Адиль (центр лодки). Если такого бассейна нет или лодка не помещается в него, то выведите число -1.}

\exampleee{6 8 10 1}{12.00000000}{2 2 2 2}{-1}{1 2 3 1}{-1}

\excomm{Как это ни удивительно, но размерами Адиля можно пренебречь!}



\subsection*{E. Easy number}

Адиль опять предложил Надире и Денису загадать по одному числу. Но теперь числа могут быть любые целые (даже отрицательные и нули!). После этого он подсчитал <<магическое>> число --- разность квадратов загаданных чисел и сказал его Вам. Ваша задача - подсчитать сколько различных целых пар чисел $(A, B)$ могли загадать Надира и Денис (то есть сколько пар чисел дадут <<магическое>> число, равное N).

\informat{Одно целое число $N$ (от 1 до 5 000 000) --- <<магическое>> число.}

\outformat{Количество различных пар с <<магическим>> числом, равным $N$.}

\examplee{1}{2}{2}{0}



\subsection*{F. Friends}

Денис из поездки решил привезти друзьям магнитики. Он хочет подарить всем друзьям равное число магнитиков. Денис точно не помнит сколько у него друзей, но точно знает, что их не более $n$. Какое наименьшее число магнитиков ему надо купить, чтобы он при любом ненулевом количестве друзей смог раздарить все магнитики?

\informat{Одно целое число $n$ (от 1 до 22) --- максимальное возможное количество друзей.}

\outformat{Одно целое число --- минимальноe возможное количество магнитов.}

\exampleee{1}{1}{2}{2}{3}{6}



\subsection*{G. Game}

Адиль купил игру Дженга (из $N$ деревянных брусков квадратного сечения). Поскольку точных правил игры он не знает, то он придумал свои правила: двое игроков по очереди берут бруски. За ход можно взять 1 брусок, 2 бруска или половину оставшихся брусков (если осталось нечетное количество брусков, то количество округляется в меньшую сторону, например, от 7 брусков можно взять 3). Тот, кто возьмет последний брусок, считается победителем. Адиль будет ходить первый. Может ли он гарантировано обыграть Дениса при правильной игре обоих игроков?

\informat{Одно целое число $N$ (от 1 до 100 000).}

\outformat{Выведите текст 'YES' (без кавычек), если при правильной игре выиграет Адиль и 'NO' (без кавычек) --- иначе.}

\exampleee{1}{YES}{3}{NO}{4}{YES}



\subsection*{H. Hypnoses}

Надира стоит на светофоре и ждет пока загорится зеленый свет (а это случится только через $t$ секунд). Чтобы скоротать время, она следит за машинами, которые проезжают мимо нее по дороге. Некоторые машины едут по прямой слева направо, другие справа налево. Надира в начале смотрит в точку с координатами 0. Как только мимо этой точки проезжает машина, она начинает следить за этой машиной. Если мимо машины, за которой сейчас следит Надира, проезжает другая машина (в любом направлении), она переводит взгляд на нее. Так происходит до тех пор, пока не загорится зеленый свет. Выясните, где остановится взгляд Надиры, когда загорится зеленый цвет?

\informat{Два целых числа: $t$ (от 1 до 1000) --- время, через которое загорится зеленый свет, $n$ (от 1 до 1000) --- количество машин, которые есть на дороге. Далее $2n$ вещественных чисел (от -100 до 100) $x_1$, $v_1$, $x_2$, $v_2$, $\dots$, $x_n$, $v_n$ --- координаты (отличные от нуля) и скорости машин (могут быть нулевыми). Известно, что никакие три машины не бывают в одной точке одномвременно.} 

\outformat{Одно вещественное число с точностью до 2 знаков после запятой --- координата, где остановится взгляд Надиры.}

\exampleee{10 1 \newline 1 1}
{0.000000}
{10 1 \newline 8 -2}
{-12.000000}
{3 4 \newline -2 0.5 \newline -1 1 \newline 1 -0.5 \newline 3 0}
{-0.500000}

\excomm{Размерами машин пренебречь. Никакие 3 машины никогда не оказываются в одной точке (2 машины могут). Машины, которые проезжают через одну и ту же координату, не изменяют скорости друг друга. Машин, которые впервые попадутся Надире на глаза, не более одной.}


\end{document} 
