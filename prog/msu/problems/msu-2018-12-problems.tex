\documentclass[10pt, a4paper]{article}

\usepackage[T2A]{fontenc}		%cyrillic output
\usepackage[utf8]{inputenc}		%cyrillic output
\usepackage[english, russian]{babel}	%word wrap
\usepackage{amssymb, amsfonts, amsmath}	%math symbols
\usepackage{mathtext}			%text in formulas
\usepackage{geometry}			%paper format attributes
\usepackage{fancyhdr}			%header
\usepackage{graphicx}			%input pictures
\usepackage{tabularx}			%smart table
\usepackage{longtable}			%long table
\usepackage{tikz, pgfplots}		%draw pictures and graphics
\usetikzlibrary{patterns}		%draw pictures: fill
\usetikzlibrary{positioning}	%draw pictures: below of
\usetikzlibrary{calc}			%draw pictures: $\i$
\usepackage{listofitems}		%draw from tex-list
\usepackage{enumitem}			%enumarate parameters

\geometry{left=2cm, right=2cm, top=2cm, bottom=2cm, headheight=15pt}
\setlist[enumerate]{leftmargin=*}	%remove enumarate indenttion
\sloppy							%correct overfull
\pagestyle{empty}				%no page numbers

\newcommand{\accept}[2]{
	\centerline{\boxed{#1}}
	\newline
	\centerline{\scriptsize{#2}}
}
\newcommand{\reject}[1]{
	\centerline{#1}
}


\newcommand{\head}[4]
{
	\thispagestyle{fancy}
	\fancyhf{}
	\chead{#3, #4}

	\begin{center}
	\begin{large}
	#1 \\
	\textit{#2}\\
	\end{large}
	\end{center}

}


% format

\newcommand{\informat}[1]
{
	\subsubsection*{Ввод} #1
}

\newcommand{\outformat}[1]
{
	\subsubsection*{Вывод} #1
}

\newcommand{\example}[2]
{
	\subsubsection*{Пример}
	\noindent
	\begin{center}
	\begin{tabularx}{\linewidth}{|X|X|}
	\hline
	Ввод & Вывод \\
	\hline
	{\tt #1} & {\tt #2}		\\
	\hline
	\end{tabularx}
	\end{center}
}

\newcommand{\examplelong}[2]
{
	\subsubsection*{Пример}
	\noindent
	\begin{center}
	\begin{tabularx}{\textwidth}{|l|X|}
	\hline
	Ввод & Вывод \\
	\hline
	#1 & #2		\\
	\hline
	\end{tabularx}
	\end{center}
}

\newcommand{\examplee}[4]
{
	\subsubsection*{Пример}
	\noindent
	\begin{center}
	\begin{tabularx}{\linewidth}{|X|X|}
	\hline
	Ввод 	& Вывод  	\\
	\hline
	{\tt #1} & {\tt #2}	\\
	\hline
	{\tt #3} & {\tt #4}	\\
	\hline
	\end{tabularx}
	\end{center}
}

\newcommand{\exampleee}[6]
{
	\subsubsection*{Пример}
	\noindent
	\begin{center}
	\begin{tabularx}{\linewidth}{|X|X|}
	\hline
	Ввод 	& Вывод  	\\
	\hline
	{\tt #1} & {\tt #2}	\\
	\hline
	{\tt #3} & {\tt #4}	\\
	\hline
	{\tt #5} & {\tt #6}	\\
	\hline
	\end{tabularx}
	\end{center}
}

\newcommand{\exampleeee}[8]
{
	\subsubsection*{Пример}
	\noindent
	\begin{center}
	\begin{tabularx}{\linewidth}{|X|X|}
	\hline
	Ввод 	& Вывод  	\\
	\hline
	{\tt #1} & {\tt #2}	\\
	\hline
	{\tt #3} & {\tt #4}	\\
	\hline
	{\tt #5} & {\tt #6}	\\
	\hline
	{\tt #7} & {\tt #8}	\\
	\hline
	\end{tabularx}
	\end{center}
}

\newcommand{\exampleeeee}[5]
{
	\subsubsection*{Пример}
	\begin{center}
	\begin{tabularx}{\linewidth}{|X|X|}
	\hline
	Ввод 	& Вывод  	\\
	\hline
	#1		\\
	\hline
	#2		\\
	\hline
	#3		\\
	\hline
	#4		\\
	\hline
	#5		\\
	\hline
	\end{tabularx}
	\end{center}
}

\newcommand{\examplepic}[3]
{
	\subsubsection*{Пример}
	\noindent
	\begin{center}
	\begin{tabularx}{\linewidth}{|l|l|X|}
	\hline
	Ввод 	& Вывод  	& Пояснение\\
	\hline
	{\tt #1} 		& {\tt #2} 		& #3\\
	\hline
	\end{tabularx}
	\end{center}
}


\newcommand{\excomm}[1]
{
	\subsubsection*{Комментарий}
	\textit{#1}
}

\newcommand{\problemauthor}[1]{
\begin{flushright}
\textit{Автор: #1}
\end{flushright}
}

\newcommand{\problemofferer}[1]{
\begin{flushright}
\textit{Предложил: #1}
\end{flushright}
}

\usepackage{listings}
\lstset{language=C,
        basicstyle=\ttfamily,
        keywordstyle=\color{blue},
        frame=single,
        numbers=left,
        tabsize=4}

\begin{document}

\head{Открытая личная олимпиада по программированию \\ Зимний тур 2018}{12 декабря 2018}{Казахстанский филиал МГУ имени М.В.Ломоносова}{г.~Астана}

\subsection*{A. Around the problems}

На очередной олимпиаде по программированию Валера обнаружил 7 задач с кодовыми названиями $A$, $B$, $C$, $D$, $E$, $F$, $G$. Валера хочет решить все задачи, поэтому он каждую минуту переходит к следующей задаче. Как только список задач заканчивается, Валера возвращается к задаче $A$, то есть он решает в таком порядке: $A$, $B$, $C$, $D$, $E$, $F$, $G$, $A$, $B$, $C$, $D$, $\ldots$. Чтобы не сбиться со счету, Валера время измеряет периодами, причем длины периодов чередуются: нечетные периоды содержат $n_1$ минут, а четные периоды --- $n_2$ минут. Какую задачу будет решать Валера в $m$-ю минуту $p$-го периода?

\informat{В первой строке 2 целых числа $n_1$ и $n_2$ от 1 до 1000. Во второй строке 2 целых числа $p$ и $m$  от 1 до 1000, где $m \leqslant n_1$ при нечетном $p$ и $m \leqslant n_2$ при четном $p$.}

\outformat{Одна заглавная буква английского алфавита от $A$ до $G$, соответствующая некоторой задаче.}

\exampleee{
10 5 \newline
5 8
}{
C
}{
31 30 \newline
2 1
}{
D
}{
700 1 \newline
11 11
}{
B
}

\excomm{В первом примере периоды идут в порядке 10 минут, 5 минут, 10 минут, 5 минут, 10 минут и т.д. То есть восьмая минута пятого периода --- это 38 минута от начала, что соответствует задаче С.}

\newpage


\subsection*{B. Be lazy}

Чтобы сэкономить сегодня как можно больше энергии для решения задач, Алан решил как можно меньше использовать лестницы. К великому разочарованию Алана, лифт в учебном корпусе останавливается только на некоторых этажах. Какое минимальное число этажей ему надо пройти пешком, чтобы попасть с этажа под номером $P$ на этаж под номером $Q$, если он при необходимости может воспользоваться лифтом?

\informat{В первой строке два целых числа $P$, $Q$ от 1 до $10^9$. Во второй строке одно целое число $n$ от 1 до 100000. На третьей строке даны $n$ различных целых чисел от $1$ до $10^{9}$, отсортированных по возрастанию}

\outformat{Одно целое неотрицательное число --- минимальное количество этажей, которое необходимо пройти пешком.}

\examplee
{2 5
4\newline
1 6 7 8\newline}
{2}
{2 5\newline
2\newline
1 8}
{3}


\excomm{В первом примере Алану надо спуститься со 2-го этажа на 1-й, далее подняться на лифте с 1-го этажа на 6-й и спуститься пешком с 6-го этажа на 5-й.}

\newpage


\subsection*{C. Calculator}

Таир нашел раритетный калькулятор, который отображает числа в двоичной системе счисления. Таир восстановил последнее сохраненное число в калькуляторе, но из-за высокой степени <<раритетности>> экрана не всё на экране отображается хорошо. Таир точно распознал, какие из двоичных цифр числа равны 0, какие равны 1, а какие невозможно определить однозначно. Может ли Вы вывести список всех подходящих кандидатов? Имейте в виду, что ведущие нули допустимы!

\informat{Строка из символов '0', '1' и '?' длиной не более 10 символов (ввод заканчивается переносом строки).}

\outformat{Все подходящие целые числа, отсортированные по возрастанию, с переносом строки после каждого числа.}

\exampleee{1?1?}{10\newline
11\newline
14\newline
15}
{??}
{0\newline
1\newline
2\newline
3}
{10}
{2}

\excomm{В первом примере подходят числа $1010_2 = 10$, $1011_2 = 11$, $1110_2 = 14$, $1111_2 = 15$.}

\newpage


\subsection*{D. Deep rolling}

Али  стал обладателем головоломки <<Тетраэдр Рубика>>, которая представляет собой правильный тетраэдр со стороной 2 и гранями, раскрашенными в белый, синий, красный и зеленый цвета. Собрав головоломку, Али обеспокоился вопросом: что с ней можно сделать. После нескольких перекатываний через ребра по плоскости он заметил удивительный факт: на любую точку плоскости всегда попадает грань одного и того же цвета! А можно ли узнать цвет грани, которая попадет на данную точку плоскости, не производя самих перекатываний?

\begin{center}
\begin{tikzpicture}[yslant=0, xslant=0.5, scale=0.75]
	\draw (0, 0) grid (6, 6);


	\foreach \x in {1, ..., 6}{
		\draw (\x, 0) -- (0, \x);
	}
	\foreach \x in {1, ..., 5}{
		\draw (\x, 6) -- (6, \x);
	}
	\draw[->] (0, 0) -- (7, 0);
	\node at (7.5, 0) (a) {$a$};

	\draw[->] (0, 0) -- (0, 7);
	\node at (0, 7.5) (b) {$b$};

	\foreach \x in {1, 3, 5}{
		\foreach \y in {1, 3, 5}{	
			\fill[red] (\x, \y) -- (\x + 1, \y) -- (\x, \y + 1) -- cycle;
			\node at (\x+0.365, \y+0.3) (R) {R};
		}
	}
	\foreach \x in {0, 2, 4}{
	    \foreach \y in {0, 2, 4}{
			\node at (\x+0.365, \y+0.3) (W) {W};
		}
	}
	\foreach \x in {1, 3, 5}{
	    \foreach \y in {0, 2, 4}{
			\fill[blue] (\x, \y) -- (\x + 1, \y) -- (\x, \y + 1) -- cycle;
			\node[white] at (\x+0.365, \y+0.3) (B) {B};
		}
	}
	\foreach \x in {0, 2, 4}{
		\foreach \y in {1, 3, 5}{	
			\fill[green] (\x, \y) -- (\x + 1, \y) -- (\x, \y + 1) -- cycle;
			\node at (\x+0.365, \y+0.3) (G) {G};
		}
	}
	\foreach \x in {1, 3, 5}{
		\foreach \y in {1, 3, 5}{	
			\fill[red] (\x, \y) -- (\x - 1, \y) -- (\x, \y - 1) -- cycle;
			\node at (\x-0.365, \y-0.3) (R) {R};
		}
	}
	\foreach \x in {2, 4, 6}{
	    \foreach \y in {2, 4, 6}{
			\node at (\x-0.365, \y-0.3) (W) {W};
		}
	}
	\foreach \x in {1, 3, 5}{
	    \foreach \y in {2, 4, 6}{
			\fill[blue] (\x, \y) -- (\x - 1, \y) -- (\x, \y - 1) -- cycle;
			\node[white] at (\x-0.365, \y-0.3) (B) {B};
		}
	}
	\foreach \x in {2, 4, 6}{
		\foreach \y in {1, 3, 5}{	
			\fill[green] (\x, \y) -- (\x - 1, \y) -- (\x, \y - 1) -- cycle;
			\node at (\x-0.365, \y-0.3) (G) {G};
		}
	}


\end{tikzpicture}
\end{center}

Изначально нижняя грань тетраэдра стоит в точках с координатами $(0, 0)$, $(2, 0)$, $(1, \sqrt{3})$. Нижняя грань окрашена в белый цвет, а боковые соответственно в красный, зеленый и синий (смотрите картинку). Координаты точки плоскости $(X, Y)$, для которой необходимо узнать цвет, задаются через два параметра $A$ и $B$:

$$
\begin{pmatrix}
X \\
Y
\end{pmatrix}
= 
A * 
\begin{pmatrix}
2 \\
0
\end{pmatrix}
+ B * 
\begin{pmatrix}
1 \\
\sqrt{3}
\end{pmatrix}
$$

\informat{Два вещественных числа $A$ и $B$ от 0 до 100 с точностью в 1 знак после запятой. Гарантируется, что числа $A$, $B$ и $A + B$ не являются целыми.}

\outformat{Одну из четырех строк: white, red, green или blue.}

\exampleee{11.1 11.1}{red}{12.3 45.6}{green}{65.4 32.1}{blue}

\excomm{$A$ и $B$ являются координатами данной точки в системе координат с базисными векторами $e_1(2, 0)$ и $e_2(1, \sqrt{3})$. В таком базисе начальные координаты вершин нижней грани будут равны $(0, 0)$, $(1, 0)$, $(0, 1)$.}

\newpage


\subsection*{E. Elementary balance}

Ануар очень любит придумывать различные термины. Вот его последнее творение: элементарный балансир массива --- это такой элемент массива, что сумма всех элементов справа и сумма всех чисел слева от этого элемента отличаются на минимальное число. Осталось дело за малым --- найти позиции всех элементарных балансиров данного массива!


\informat{В первой строке одно целое число $n$ от 1 до 100000. Во второй строке $n$ целых чисел, по модулю не превосходящих $10^{9}$.}

\outformat{Одно или несколько целых чисел через пробел --- позиции элементарных балансиров массива.}

\exampleee{7\newline
7 6 5 4 3 2 1}{3}
{2 \newline
3 4}{2}
{3 \newline
1 -1 1}
{1 2 3}

\excomm{В первом примере слева от третьего числа сумма равна 13, справа от него --- 10. Между этими числами разница равна 3 --- минимальная возможная.}

\newpage


\subsection*{F. Full overlapping}

Первое, чему удивился Азат, когда приступил к учебе в Москве, --- это количество одновременно выданных домашних заданий. За семестр ему выдали $n$ домашних заданий. Причем $i$-e домашнее задание Азату выдали в 12:00 $L_i$-го дня, а сдать его нужно было не позднее 12:00 $R_i$-го дня. Азат уверен, что иногда были моменты, когда количество актуальных домашних работ было просто космическим! Правда ли это? Найдите, какое максимальное количество перекрывающихся домашних заданий было у Азата. Два задания являются перекрывающимися, если существует интервал времени ненулевой длины, который принадлежит временным отрезкам обоих заданий.

\informat{В первой строке одно целое число $n$ от 1 до 100000. Во второй строке $n$ пар целых чисел $L_i < R_i$ от 1 до $10^{18}$.}

\outformat{Одно целое неотрицательное число --- максимальное количество перекрывающихся домашних заданий.}

\example{
5\newline
1 5\newline
2 6\newline
3 4\newline
1 4\newline
1 2
}{4}

\excomm{В примере интервал $(3; 4)$ принадлежит временным отрезкам первых четырех заданий.}

\newpage


\subsection*{G. Galaxy number}

Бекарыс знает ответ на <<главный вопрос жизни, вселенной и всего такого>>. Поэтому он может найти минимальное 42-значное число, кратное данному $k$. А вы можете?

\informat{Одно целое число $k$ от 1 до $10^{15}$.}

\outformat{Одно целое 42-значное число.}

\examplelong{2019}{100000000000000000000000000000000000000218}


\end{document} 
