\documentclass[10pt, a4paper]{article}

\usepackage[T2A]{fontenc}		%cyrillic output
\usepackage[utf8]{inputenc}		%cyrillic output
\usepackage[english, russian]{babel}	%word wrap
\usepackage{amssymb, amsfonts, amsmath}	%math symbols
\usepackage{mathtext}			%text in formulas
\usepackage{geometry}			%paper format attributes
\usepackage{fancyhdr}			%header
\usepackage{graphicx}			%input pictures
\usepackage{tabularx}			%smart table
\usepackage{longtable}			%long table
\usepackage{tikz, pgfplots}		%draw pictures and graphics
\usetikzlibrary{patterns}		%draw pictures: fill
\usetikzlibrary{positioning}	%draw pictures: below of
\usetikzlibrary{calc}			%draw pictures: $\i$
\usepackage{listofitems}		%draw from tex-list
\usepackage{enumitem}			%enumarate parameters

\geometry{left=2cm, right=2cm, top=2cm, bottom=2cm, headheight=15pt}
\setlist[enumerate]{leftmargin=*}	%remove enumarate indenttion
\sloppy							%correct overfull
\pagestyle{empty}				%no page numbers

\newcommand{\accept}[2]{
	\centerline{\boxed{#1}}
	\newline
	\centerline{\scriptsize{#2}}
}
\newcommand{\reject}[1]{
	\centerline{#1}
}


\newcommand{\head}[4]
{
	\thispagestyle{fancy}
	\fancyhf{}
	\chead{#3, #4}

	\begin{center}
	\begin{large}
	#1 \\
	\textit{#2}\\
	\end{large}
	\end{center}

}


% format

\newcommand{\informat}[1]
{
	\subsubsection*{Ввод} #1
}

\newcommand{\outformat}[1]
{
	\subsubsection*{Вывод} #1
}

\newcommand{\example}[2]
{
	\subsubsection*{Пример}
	\noindent
	\begin{center}
	\begin{tabularx}{\linewidth}{|X|X|}
	\hline
	Ввод & Вывод \\
	\hline
	{\tt #1} & {\tt #2}		\\
	\hline
	\end{tabularx}
	\end{center}
}

\newcommand{\examplelong}[2]
{
	\subsubsection*{Пример}
	\noindent
	\begin{center}
	\begin{tabularx}{\textwidth}{|l|X|}
	\hline
	Ввод & Вывод \\
	\hline
	#1 & #2		\\
	\hline
	\end{tabularx}
	\end{center}
}

\newcommand{\examplee}[4]
{
	\subsubsection*{Пример}
	\noindent
	\begin{center}
	\begin{tabularx}{\linewidth}{|X|X|}
	\hline
	Ввод 	& Вывод  	\\
	\hline
	{\tt #1} & {\tt #2}	\\
	\hline
	{\tt #3} & {\tt #4}	\\
	\hline
	\end{tabularx}
	\end{center}
}

\newcommand{\exampleee}[6]
{
	\subsubsection*{Пример}
	\noindent
	\begin{center}
	\begin{tabularx}{\linewidth}{|X|X|}
	\hline
	Ввод 	& Вывод  	\\
	\hline
	{\tt #1} & {\tt #2}	\\
	\hline
	{\tt #3} & {\tt #4}	\\
	\hline
	{\tt #5} & {\tt #6}	\\
	\hline
	\end{tabularx}
	\end{center}
}

\newcommand{\exampleeee}[8]
{
	\subsubsection*{Пример}
	\noindent
	\begin{center}
	\begin{tabularx}{\linewidth}{|X|X|}
	\hline
	Ввод 	& Вывод  	\\
	\hline
	{\tt #1} & {\tt #2}	\\
	\hline
	{\tt #3} & {\tt #4}	\\
	\hline
	{\tt #5} & {\tt #6}	\\
	\hline
	{\tt #7} & {\tt #8}	\\
	\hline
	\end{tabularx}
	\end{center}
}

\newcommand{\exampleeeee}[5]
{
	\subsubsection*{Пример}
	\begin{center}
	\begin{tabularx}{\linewidth}{|X|X|}
	\hline
	Ввод 	& Вывод  	\\
	\hline
	#1		\\
	\hline
	#2		\\
	\hline
	#3		\\
	\hline
	#4		\\
	\hline
	#5		\\
	\hline
	\end{tabularx}
	\end{center}
}

\newcommand{\examplepic}[3]
{
	\subsubsection*{Пример}
	\noindent
	\begin{center}
	\begin{tabularx}{\linewidth}{|l|l|X|}
	\hline
	Ввод 	& Вывод  	& Пояснение\\
	\hline
	{\tt #1} 		& {\tt #2} 		& #3\\
	\hline
	\end{tabularx}
	\end{center}
}


\newcommand{\excomm}[1]
{
	\subsubsection*{Комментарий}
	\textit{#1}
}

\newcommand{\problemauthor}[1]{
\begin{flushright}
\textit{Автор: #1}
\end{flushright}
}

\newcommand{\problemofferer}[1]{
\begin{flushright}
\textit{Предложил: #1}
\end{flushright}
}

\usepackage{listings}
\lstset{language=C,
        basicstyle=\ttfamily,
        keywordstyle=\color{blue},
        frame=single,
        numbers=left,
        tabsize=4}

\begin{document}

\head{Открытая личная олимпиада по программированию \\ Зимний тур 2016}{13 декабря 2016}{Казахстанский филиал МГУ имени М.В.Ломоносова}{г.~Астана}

\subsubsection*{A. A train problem}

\problemauthor{ Баев А.Ж.}

Ответ: $2 k (k-1) + 2 (n-k) (n-k+1)$.

Асимптотика: $O(1)$.



\subsubsection*{B. Bead garland}

\problemauthor{ Баев А.Ж.}

Количество различных гирлянд изначально равно $a_1 a_2 ... a_n$. Гирлянду длины один не выгодно объединять ни к какой другой гирлянде (вместо $a \cdot 1 < a + 1$), гирлянды большей длины наоборот выгоднее объединять между собой ($a_1 \cdot a_2 > a_1 + a_2$). Значит, выгоднее всего объединить все гирлянд, с длиной больше 1. При этом стоит обратить внимание на случай, когда имеются гирлянды только единичной длины.

Асимптотика: $O(n)$.



\subsubsection*{C. Champion}

\problemauthor{ Абдикалыков А.К.}

Вывод является буквами, ascii-код которых дан на вводе (32-й символ таблицы является пробел).

Асимптотика: $O(1)$.



\subsubsection*{D. Digits}

\problemauthor{ Абдикалыков А.К.}

Пусть $d[n][k]$ --- количество $n$-значных чисел, у которых сумма цифр равна $k$. Ясно, что если у числа отбросить последнюю цифру $z$, то получим число с суммой цифр $k - z$. Значит,
$$d[i][j] = \sum_{z=0}^{\min(9,j)} d[i-1][j-z].$$
Что легко просчитать от для всех $i$ от 1 до $n$ и $j$ от 1 до $k$. Начальные значения $d[0][0] = 1$ и $d[i][0] = 0$.

Асимптотика: $O(n k)$.



\subsubsection*{E. Elimination}

\problemauthor{ Баев А.Ж.}

Пусть $d[i][j]$ --- количество работающих участков у $i$-й жилы среди участков с $(j-m+1)$-го до $j$-го включительно, которые можно вычислить за $O(1)$ каждый:$$d[i][j] = d[i][j-1] + a[i][j] - a[i][j-m]$$
для всех $i$ от 1 до $k$ и $j$ от $m$ до $n$.
Ответом на задачу будет $\max_{m\leqslant j \leqslant n} c_j$,
где $c_j$ --- количество $d[i][j] = m$, для всех $i$ от 1 до $k$.

Асимптотика: $O(n k)$.



\subsubsection*{F. Forts}

\problemauthor{ Баев А.Ж.}

\begin{center}
\definecolor{light}{rgb}{0.85,0.85,0.85}
\begin{tikzpicture}[x=6, y=6]
\begin{scope}
  \clip (-10,-10) -- (-10,0) -- (0,0) -- (0, -10);
  \fill[fill=light, draw=none] (3, 4) circle (10);
\end{scope}
\draw (-7, 4) arc (180:270:10);
\draw (-10, 0) -- (0, 0);
\draw (0, -10) -- (0, 0);
\draw (3, 4) -- (0, 0);
\draw (3, 4) -- (-6.165, 0);
\draw (3, 4) -- (0, -5.539);
\node at (3, 4.5) {O};
\node at (-7, -1) {A};
\node at (-1, -7) {B};
\node at (-1, -1) {C};
\end{tikzpicture}
\end{center}

Искомая площадь равна нулю, если $a^2 + b^2 > r^2$. Иначе ее можно найти как разность площали кругового сектора $OAB$ и двух треугольников $AOC$ и $BOC$.

Асимптотика: $O(1)$.

\end{document} 
