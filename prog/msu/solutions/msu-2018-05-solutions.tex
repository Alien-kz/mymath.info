\documentclass[10pt, a4paper]{article}

\usepackage[T2A]{fontenc}		%cyrillic output
\usepackage[utf8]{inputenc}		%cyrillic output
\usepackage[english, russian]{babel}	%word wrap
\usepackage{amssymb, amsfonts, amsmath}	%math symbols
\usepackage{mathtext}			%text in formulas
\usepackage{geometry}			%paper format attributes
\usepackage{fancyhdr}			%header
\usepackage{graphicx}			%input pictures
\usepackage{tabularx}			%smart table
\usepackage{longtable}			%long table
\usepackage{tikz, pgfplots}		%draw pictures and graphics
\usetikzlibrary{patterns}		%draw pictures: fill
\usetikzlibrary{positioning}	%draw pictures: below of
\usetikzlibrary{calc}			%draw pictures: $\i$
\usepackage{listofitems}		%draw from tex-list
\usepackage{enumitem}			%enumarate parameters

\geometry{left=2cm, right=2cm, top=2cm, bottom=2cm, headheight=15pt}
\setlist[enumerate]{leftmargin=*}	%remove enumarate indenttion
\sloppy							%correct overfull
\pagestyle{empty}				%no page numbers

\newcommand{\accept}[2]{
	\centerline{\boxed{#1}}
	\newline
	\centerline{\scriptsize{#2}}
}
\newcommand{\reject}[1]{
	\centerline{#1}
}


\newcommand{\head}[4]
{
	\thispagestyle{fancy}
	\fancyhf{}
	\chead{#3, #4}

	\begin{center}
	\begin{large}
	#1 \\
	\textit{#2}\\
	\end{large}
	\end{center}

}


% format

\newcommand{\informat}[1]
{
	\subsubsection*{Ввод} #1
}

\newcommand{\outformat}[1]
{
	\subsubsection*{Вывод} #1
}

\newcommand{\example}[2]
{
	\subsubsection*{Пример}
	\noindent
	\begin{center}
	\begin{tabularx}{\linewidth}{|X|X|}
	\hline
	Ввод & Вывод \\
	\hline
	{\tt #1} & {\tt #2}		\\
	\hline
	\end{tabularx}
	\end{center}
}

\newcommand{\examplelong}[2]
{
	\subsubsection*{Пример}
	\noindent
	\begin{center}
	\begin{tabularx}{\textwidth}{|l|X|}
	\hline
	Ввод & Вывод \\
	\hline
	#1 & #2		\\
	\hline
	\end{tabularx}
	\end{center}
}

\newcommand{\examplee}[4]
{
	\subsubsection*{Пример}
	\noindent
	\begin{center}
	\begin{tabularx}{\linewidth}{|X|X|}
	\hline
	Ввод 	& Вывод  	\\
	\hline
	{\tt #1} & {\tt #2}	\\
	\hline
	{\tt #3} & {\tt #4}	\\
	\hline
	\end{tabularx}
	\end{center}
}

\newcommand{\exampleee}[6]
{
	\subsubsection*{Пример}
	\noindent
	\begin{center}
	\begin{tabularx}{\linewidth}{|X|X|}
	\hline
	Ввод 	& Вывод  	\\
	\hline
	{\tt #1} & {\tt #2}	\\
	\hline
	{\tt #3} & {\tt #4}	\\
	\hline
	{\tt #5} & {\tt #6}	\\
	\hline
	\end{tabularx}
	\end{center}
}

\newcommand{\exampleeee}[8]
{
	\subsubsection*{Пример}
	\noindent
	\begin{center}
	\begin{tabularx}{\linewidth}{|X|X|}
	\hline
	Ввод 	& Вывод  	\\
	\hline
	{\tt #1} & {\tt #2}	\\
	\hline
	{\tt #3} & {\tt #4}	\\
	\hline
	{\tt #5} & {\tt #6}	\\
	\hline
	{\tt #7} & {\tt #8}	\\
	\hline
	\end{tabularx}
	\end{center}
}

\newcommand{\exampleeeee}[5]
{
	\subsubsection*{Пример}
	\begin{center}
	\begin{tabularx}{\linewidth}{|X|X|}
	\hline
	Ввод 	& Вывод  	\\
	\hline
	#1		\\
	\hline
	#2		\\
	\hline
	#3		\\
	\hline
	#4		\\
	\hline
	#5		\\
	\hline
	\end{tabularx}
	\end{center}
}

\newcommand{\examplepic}[3]
{
	\subsubsection*{Пример}
	\noindent
	\begin{center}
	\begin{tabularx}{\linewidth}{|l|l|X|}
	\hline
	Ввод 	& Вывод  	& Пояснение\\
	\hline
	{\tt #1} 		& {\tt #2} 		& #3\\
	\hline
	\end{tabularx}
	\end{center}
}


\newcommand{\excomm}[1]
{
	\subsubsection*{Комментарий}
	\textit{#1}
}

\newcommand{\problemauthor}[1]{
\begin{flushright}
\textit{Автор: #1}
\end{flushright}
}

\newcommand{\problemofferer}[1]{
\begin{flushright}
\textit{Предложил: #1}
\end{flushright}
}

\usepackage{listings}
\lstset{language=C,
        basicstyle=\ttfamily,
        keywordstyle=\color{blue},
        frame=single,
        numbers=left,
        tabsize=4}

\begin{document}

\head{Открытая командная олимпиада по программированию \\ Весенний тур 2018}{19 мая 2018}{Казахстанский филиал МГУ имени М.В.Ломоносова}{г.~Астана}

\subsubsection*{A. Azat and bookshelf}
\problemauthor{ Баев А.Ж.}

Ответ равен разности объемов двух прямоугольных параллелепипедов:
$$W \cdot H \cdot L - (W - 2 D) (H - 2 D) (L - D).$$

Асимптотика: $O(1)$. 

\begin{lstlisting}
#include <iostream>
using namespace std;
int main() {
    long long d, w, h, l, volume_ext, volume_int;
    cin >> d >> w >> h >> l;
    volume_ext = w * h * l;
    volume_int = (w - 2 * d) * (h - 2 * d) * (l - d);
    cout << volume_ext - volume_int << endl;
    return 0;
}
\end{lstlisting}



\subsubsection*{B. Bekarys and khet}
\problemauthor{ Абдикалыков А.К.}

Длина пути луча в каждой клетка равна 1. Значит, ответ --- количество клеток, через которые пройдет луч. Для этого можно построить простой автомат, определяющий $(x_k, y_k)$ --- текущее положение, $v_k$ --- направление на входе в клетку. 

Направление меняется следующим образом, если в клетке $(x_k, y_k)$ находится зеркало вида {\tt '/'} и вида {\tt '\textbackslash'} соответственно:
$$
v_{k+1} =
\begin{cases}
UP, & v_k = RIGHT \\
DOWN, & v_k = LEFT \\
RIGHT, & v_k = UP \\
LEFT, & v_k = DOWN \\
\end{cases}
\hspace{1cm}
v_{k+1} =
\begin{cases}
DOWN, & v_k = RIGHT \\
UP, & v_k = LEFT \\
LEFT, & v_k = UP \\
RIGHT, & v_k = DOWN \\
\end{cases}
$$
Положение меняется в зависимости от направления естественным образом.

Асимптотика: $O(M \cdot N)$ (так как каждую клетку луч перечесет не более двух раз). 

\begin{lstlisting}
#include <iostream>
using namespace std;
int main() {
    long long n, m;
    char field[101][101];
    cin >> n >> m;
    for (int i = 0; i < n; i++)
        cin >> field[i];

    int dx[] = {0, 0, -1, 1};
    int dy[] = {1, -1, 0, 0};
    enum dir{RIGHT, LEFT, UP, DOWN};
    int forward_slash[] = {UP, DOWN, RIGHT, LEFT};
    int backward_slash[] = {DOWN, UP, LEFT, RIGHT};
    int x = 0, y = 0, v = RIGHT;
    int answer = 0;
    while (x >= 0 && x < n && y >= 0 && y < m) {
        if (field[x][y] == '/')
            v = forward_slash[v];
        if (field[x][y] == '\\')
            v = backward_slash[v];
        x += dx[v];
        y += dy[v];
        answer++;
    }

    cout << answer << endl;
    return 0;
}
\end{lstlisting}



\subsubsection*{C. Computer vision}
\problemauthor{ Абдикалыков А.К.}

Обозначим: $ dp[m][r]$ --- количество $m$-значных чисел, которые соответствует первым $m$ символам шаблона и дают остаток $r$ при делении на 11.

В случае, если символ $a[m]$ в шаблоне является цифрой, то количество $dp[m][r]$ определяется как количество соответствующих $(m-1)$-значных чисел:
$$
dp[m][r_k] = dp[m-1][k], 
$$
где $r_k = 10 k + a[m] \pmod {11}$, для всех $k = 0, \dots, 10$.

В случае, если символ $a[m]$ является звездочкой, то вместо неё можно подставить любую цифру. Для фиксированного $k$ от 0 до 9, значение $10 * k + r \pmod {11}$ дает все возможные остатки, кроме $10 - k$ (так как никакая цифра не может давать остаток $r = 10$ при делении на 11): 
$$
dp[m][r] = \sum_{k + r \neq 10, k = 0}^{10} dp[m-1][k] = dp[m][r] = 
$$
$$
=\sum_{k = 0}^{10} dp[m-1][k] - dp[m-1][10 - r],
$$

Асимптотика: $O(N)$. 

\begin{lstlisting}
#include <iostream>
#include <string>
using namespace std;

int dp[100001][11];
int main() {
    string s;
    cin >> s;
    long long mod = 1000000007LL;
    int n = s.size();
    dp[0][0] = 1;
    for (int m = 1; m <= n; m++) {
        if (s[m - 1] != '*') {
            for (int k = 0; k < 11; k++) {
                int r = (10 * k + s[m - 1] - '0') % 11;
                dp[m][r] = dp[m - 1][k];
            }
        } else {
            long long s = 0;
            for (int k = 0; k < 11; k++)
                s = (s + dp[m - 1][k]) % mod;
            for (int r = 0; r < 11; r++) {
                dp[m][r] = (s + mod - dp[m - 1][10 - r]) % mod;
            }
        }
    }
    cout << dp[n][0] << endl;
    return 0;
}
\end{lstlisting}



\subsubsection*{D. DPK rover}
\problemauthor{ Баев А.Ж.}

При $M = 1$, ответ очевидно будет $D / 2$. Пусть $M$ --- простое число. Обозначим $x$ --- расстояние, после которого первые $(M-1)$ марсоходов отдадут топливо $M$-му марсоходу. На обратную дорогу они должны оставить себе столько же топлива, сколько потратили. Значит, каждый из них отдаст $M$-му марсоходу топлива, на котором можно проехать расстояние $(D - 2x)$. Итого: $(M - 1) (D - 2x)$.

У $M$-го марсохода бак свободен ровно настолько, сколько он уже проехал. Чтобы всё полученное топливо поместилось в его в бак необходимо выполенение:
$$
x \geqslant (M - 1) (D - 2x).
$$
Cхема движения обратно соответствует и прямой схеме (если произвести все движения в обратном порядке). То есть полученного топлива должно хватать для движения обратно:
$$
x \leqslant (M - 1) (D - 2x).
$$
Таким образом, до точки разворота группа проедет:
$$
x = \frac{M - 1}{2 M - 1} D.
$$

Покажем, что в случае с составным числом $M = A \cdot B$ всегда выгодно делиться на группы. Два деления (на группы размером $B$ и далее на группы размером $A$) выгоднее, чем одно (на группу размера $AB$). Для этого проверим неравенство:
$$ \frac{A - 1}{2 A - 1} D + \frac{B - 1}{2 B - 1} D\geqslant \frac{AB - 1}{2 AB - 1} D$$
$$ \frac{(4 A B - 1)(B - 1)(A - 1)}{(2 AB - 1)(2 B - 1)(2 A - 1)} \geqslant 0$$
Это верно с учетом $A > 1$ и $B > 1$.

Для решения задачи в общем виде достаточно получить каноническое разложение числа $M = p_1^{\alpha_1} p_2^{\alpha_2} \dots p_k^{\alpha_k}$ на простые множители. Сделать это можно стандартным алгоритмом разложения на простые множители, проверяя делители до $\sqrt{M}$. Ответ равен: 
$$L = D \left( \frac{1}{2} + \sum_{i=1}^{k} \alpha_i \frac{p_i - 1}{2 p_i - 1} \right). $$

Асимптотика: $O(\sqrt{M})$. 

Решение за $O(M)$ не проходило ограничения по времени.



\begin{lstlisting}
#include <iostream>
#include <iomanip>
using namespace std;
int main() {
    long long d, m;
    long double answer = 0.5;
    cin >> d >> m; 
    for (long long p = 2; p * p <= m; p++) {
        while (m % p == 0) {
            answer += (p - 1.0) / (2.0 * p - 1.0);
            m /= p;
        }
    }
    if (m != 1) {
        answer += (m - 1.0) / (2.0 * m - 1.0);
    }
    cout << fixed << setprecision(10) << answer * d << endl;
    return 0;
}

\end{lstlisting}



\subsubsection*{E. Excellent idea}
\problemauthor{ Абдикалыков А.К.}

Необходимо было вывести $k$-й символ текста, где $k$ --- это число в этом же тексте. Число было во всех тестах на последнем месте. В некоторых тестах числа были больше 10. Во всех тестах числа не превышали длину теста.

Асимптотика: $O(N)$, где $N$ --- длина теста.

\begin{lstlisting}
#include <iostream>
#include <string>
using namespace std;
int main() {
    string s;
    getline(cin, s);
    int index = 0, n = s.size();
    for (int i = 0; i < n; i++)
        if (s[i] >= '0' && s[i] <= '9')
                index = 10 * index + s[i] - '0';
    cout << s[index - 1] << endl;
    return 0;
}
\end{lstlisting}


\subsubsection*{F. Forced reduction}
\problemauthor{ Баев А.Ж.}

\textbf{Решение 1.} Дерево представим в виде списка ребер ориентированного графа (от родителя к детям). Для быстрого выделения ребер с одинаковыми буквами у каждой вершины булдем хранить сразу 26 списков (по одному списку для каждой буквы).

Запустим обход в ширину из корня. При этом для каждой вершины, которая изымается из очереди, будет объединять всех её детей с одинаковыми ребрами. Например, у вершины $v$ есть дети $a_1$, $a_2$, \dots, $a_k$, в которые ведут ребра с буквой {\tt a}. В отличии от классического добавления в очередь всех вершин $a_i$, в очередь добавим только $a_1$. При этом в список детей вершины $a_1$ добавим детей из $a_2$, $a_3$, \dots, $a_k$. То же самое исполним для списков $b_i$, $c_i$, \dots, $z_i$.

Каждый ребенок (точнее ребро) будет перенесено не более одного раза. Поэтому асимптотика аккумулировано не превосходит $O(N)$. Поиск ребер с одинаковыми буквами требует $O(1)$ действий, так как у каждой вершины есть отдельный список для каждой буквы. Остается алгоритм обхода в ширину: $O(N)$ действий.  

Асиптотика: $O(N)$.

\textbf{Решение 2 (которое, увы, проходило на олимпиаде).} Достаточно промоделировать требуемый процесс при обходе дерева в глубину или в ширину, начиная с корня. При этом поддерживать ответ в какой-либо структуре с элементами в виде строк (без повторов), например {\tt set <string>}.

Такое решение в худшем будет работать на дереве типа <<бамбук>> (у каждого родителя ровно один ребенок). Длина $k$-й строки будет равна $k$. Поиск будет производится за $O(\log{k})$ в худшем, с добавлением за $O(k)$ в худшем. Общая сложность обработки $\sum_{k=1}^{N} k = O(N^2)$. Но оптимизиции и кэширование значительно ускоряют процесс копирования строк.

Асиптотика: $O(N^2)$.



\begin{lstlisting}
#include <iostream>
#include <vector>
#include <queue>
using namespace std;

vector < vector < vector <int> > > son;
int main() {
    int n;
    cin >> n;
    son.resize(n + 1);
    for (int from = 1; from <= n; from++)
        son[from].resize(26);
    for (int to = 2; to <= n; to++) {
        int from;
        char ch;
        cin >> from >> ch;
        son[from][ch - 'a'].push_back(to);
    }

    queue <int> q;
    q.push(1);
    int answer = 0;
    while (!q.empty()) {
        int current = q.front();
        q.pop();
        for (char ch = 0; ch < 26; ch++) {
            if (son[current][ch].size() > 0) {
                int dst = son[current][ch][0];
                for (size_t i = 1; 
                     i < son[current][ch].size(); i++) {
                    int src = son[current][ch][i];
                    for (char c = 0; c < 26; c++) {
                        for (size_t j = 0; 
                             j < son[src][c].size(); j++) {
                             int sonson = son[src][c][j];
                            son[dst][c].push_back(sonson);
                            }
                    }
                }
                q.push(dst);
                answer++;
            }
        }
    }

    cout << answer << endl;
    return 0;
}
\end{lstlisting}



\subsubsection*{G. Giant chandelier}
\problemauthor{ Баев А.Ж.}

Центр тяжести люстры:
$$\vec{m} = \frac{1}{n} \sum_{i = 1}^{n} \vec{v}_i.$$

Чтоб понять насколько отклоняется люстра вдоль центра тяжести, найдем длины проекций каждого вектора на вектор $\vec{m}$ через скалярное произведение:
$$p_i = (\vec{v}_i, \vec{e}),$$
где $\vec{e} = \frac{\vec{m}}{|\vec{m}|}$.

Ответ на задачу: $\max{p_i} - \min{p_i}$.

Асиптотика: $O(N)$.

\begin{lstlisting}
#include <iostream>
#include <iomanip>
#include <cmath>
using namespace std;

struct Point {
    double x, y, z;
};
double dot(const Point &a, const Point &b) {
    return a.x * b.x + a.y * b.y + a.z * b.z;
}

Point lamp[100000];
int main() {
    int n;
    cin >> n;
    Point m = {0, 0, 0};
    for (int i = 0 ; i < n; i++) {
        cin >> lamp[i].x >> lamp[i].y >> lamp[i].z;
        m.x += lamp[i].x; m.y += lamp[i].y; m.z += lamp[i].z;
    }
    double length = sqrt(dot(m, m));
    m.x /= length; m.y /= length; m.z /= length;
    double p_min = 0.0, p_max = 0.0;
    for (int i = 0; i < n; i++) {
        double d = dot(lamp[i], m);
        if (p_min > d)
            p_min = d;
        if (p_max < d)
            p_max = d;
    }
    cout << fixed << setprecision(10) << p_max - p_min << endl;
    return 0;
}
\end{lstlisting}


\subsubsection*{H. Hyper numbers}
\problemauthor{ Абдикалыков А.К.}

Заметим, что всегд верно $(n, x) = (n, n - x)$. По определению гипер-чисел, не может быть несколько взаимно простых чисел. Значит, $x = n - x$. Тогда $n = 2x$ и $(n, x) = x$. Опять же по опредению эти числа должны быть взаимно просты, то есть $x = 1$, $n = 2$. Таким образом, достаточно проверить, входит ли число 2 в указанный диапазон.

Асиптотика: $O(1)$.

\begin{lstlisting}
#include <iostream>
using namespace std;
int main() {
    long long L, R, answer;
    cin >> L >> R;
    answer = (L == 2);
    cout << answer << endl;
    return 0;
}
\end{lstlisting}

\end{document} 
