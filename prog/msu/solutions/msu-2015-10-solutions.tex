\documentclass[10pt, a4paper]{article}

\usepackage[T2A]{fontenc}		%cyrillic output
\usepackage[utf8]{inputenc}		%cyrillic output
\usepackage[english, russian]{babel}	%word wrap
\usepackage{amssymb, amsfonts, amsmath}	%math symbols
\usepackage{mathtext}			%text in formulas
\usepackage{geometry}			%paper format attributes
\usepackage{fancyhdr}			%header
\usepackage{graphicx}			%input pictures
\usepackage{tabularx}			%smart table
\usepackage{longtable}			%long table
\usepackage{tikz, pgfplots}		%draw pictures and graphics
\usetikzlibrary{patterns}		%draw pictures: fill
\usetikzlibrary{positioning}	%draw pictures: below of
\usetikzlibrary{calc}			%draw pictures: $\i$
\usepackage{listofitems}		%draw from tex-list
\usepackage{enumitem}			%enumarate parameters

\geometry{left=2cm, right=2cm, top=2cm, bottom=2cm, headheight=15pt}
\setlist[enumerate]{leftmargin=*}	%remove enumarate indenttion
\sloppy							%correct overfull
\pagestyle{empty}				%no page numbers

\newcommand{\accept}[2]{
	\centerline{\boxed{#1}}
	\newline
	\centerline{\scriptsize{#2}}
}
\newcommand{\reject}[1]{
	\centerline{#1}
}


\newcommand{\head}[4]
{
	\thispagestyle{fancy}
	\fancyhf{}
	\chead{#3, #4}

	\begin{center}
	\begin{large}
	#1 \\
	\textit{#2}\\
	\end{large}
	\end{center}

}


% format

\newcommand{\informat}[1]
{
	\subsubsection*{Ввод} #1
}

\newcommand{\outformat}[1]
{
	\subsubsection*{Вывод} #1
}

\newcommand{\example}[2]
{
	\subsubsection*{Пример}
	\noindent
	\begin{center}
	\begin{tabularx}{\linewidth}{|X|X|}
	\hline
	Ввод & Вывод \\
	\hline
	{\tt #1} & {\tt #2}		\\
	\hline
	\end{tabularx}
	\end{center}
}

\newcommand{\examplelong}[2]
{
	\subsubsection*{Пример}
	\noindent
	\begin{center}
	\begin{tabularx}{\textwidth}{|l|X|}
	\hline
	Ввод & Вывод \\
	\hline
	#1 & #2		\\
	\hline
	\end{tabularx}
	\end{center}
}

\newcommand{\examplee}[4]
{
	\subsubsection*{Пример}
	\noindent
	\begin{center}
	\begin{tabularx}{\linewidth}{|X|X|}
	\hline
	Ввод 	& Вывод  	\\
	\hline
	{\tt #1} & {\tt #2}	\\
	\hline
	{\tt #3} & {\tt #4}	\\
	\hline
	\end{tabularx}
	\end{center}
}

\newcommand{\exampleee}[6]
{
	\subsubsection*{Пример}
	\noindent
	\begin{center}
	\begin{tabularx}{\linewidth}{|X|X|}
	\hline
	Ввод 	& Вывод  	\\
	\hline
	{\tt #1} & {\tt #2}	\\
	\hline
	{\tt #3} & {\tt #4}	\\
	\hline
	{\tt #5} & {\tt #6}	\\
	\hline
	\end{tabularx}
	\end{center}
}

\newcommand{\exampleeee}[8]
{
	\subsubsection*{Пример}
	\noindent
	\begin{center}
	\begin{tabularx}{\linewidth}{|X|X|}
	\hline
	Ввод 	& Вывод  	\\
	\hline
	{\tt #1} & {\tt #2}	\\
	\hline
	{\tt #3} & {\tt #4}	\\
	\hline
	{\tt #5} & {\tt #6}	\\
	\hline
	{\tt #7} & {\tt #8}	\\
	\hline
	\end{tabularx}
	\end{center}
}

\newcommand{\exampleeeee}[5]
{
	\subsubsection*{Пример}
	\begin{center}
	\begin{tabularx}{\linewidth}{|X|X|}
	\hline
	Ввод 	& Вывод  	\\
	\hline
	#1		\\
	\hline
	#2		\\
	\hline
	#3		\\
	\hline
	#4		\\
	\hline
	#5		\\
	\hline
	\end{tabularx}
	\end{center}
}

\newcommand{\examplepic}[3]
{
	\subsubsection*{Пример}
	\noindent
	\begin{center}
	\begin{tabularx}{\linewidth}{|l|l|X|}
	\hline
	Ввод 	& Вывод  	& Пояснение\\
	\hline
	{\tt #1} 		& {\tt #2} 		& #3\\
	\hline
	\end{tabularx}
	\end{center}
}


\newcommand{\excomm}[1]
{
	\subsubsection*{Комментарий}
	\textit{#1}
}

\newcommand{\problemauthor}[1]{
\begin{flushright}
\textit{Автор: #1}
\end{flushright}
}

\newcommand{\problemofferer}[1]{
\begin{flushright}
\textit{Предложил: #1}
\end{flushright}
}

\usepackage{listings}
\lstset{language=C,
        basicstyle=\ttfamily,
        keywordstyle=\color{blue},
        frame=single,
        numbers=left,
        tabsize=4}

\begin{document}

\head{Открытая командная олимпиада по программированию \\ Осенний тур 2015}{20 октября 2015}{Казахстанский филиал МГУ имени М.В.Ломоносова}{г.~Астана}

\subsubsection*{A. Alternative result} 

\problemauthor{Абдикалыков А.К.}

Несложно убедиться, что можно получить все значения от 0 до $3n$, кроме $3n-1$.

Асимптотика: $O(n)$.



\subsubsection*{B. Boolean} 

\problemauthor{Абдикалыков А.К.}

Необходимо было вывести $N$-е слово из текста.

Асимптотика: $O(1)$.



\subsubsection*{C. Car collection} 

\problemauthor{Баев А.Ж.}

Ответом на задачу является:
$$\sum_{i=1}^{n-1} \sum_{j=i+1}^{n} a_i a_j = \frac{1}{2} \left( (\sum_{i=1}^{n} {a_i})^2 - \sum_{i=1}^{n} a_i^2 \right).$$

Асимптотика: $O(n)$.

Замечание: наивное решение не проходит ограничения по времени.



\subsubsection*{D. Domino} 

\problemauthor{Баев А.Ж.}

Промоделируем падения слева направо и справа налево. Для этого найдем максимальную длину положительной подстроки массива $l_i$, где $l_i = a_i - a_{i-1} - h_{i-1}$, и максимальную длину положительной подстроки массива $r_i$, где $r_i = a_i - a_{i+1} - h_{i+1}$. Ответов будет максимум из первого и второго случая.

Асимптотика: $O(n)$.



\subsubsection*{E. Enlarged triangle} 

\problemauthor{Баев А.Ж.}

Пусть $S(a, b, c)$ --- площадь треугольника со сторонами $a$, $b$, $c$. Несложно проверить, что функция $f(m) = S(a + m, b + m, c + m)$ является монотонно возрастающей (при условии, что $m > 0$ и треугольник с данными сторонами существует). Значит, ответ можно найти бинарным поиском по $m$ на отрезке $[0; \sqrt{2 S}]$.

Асимптотика: $O(\log S)$.



\subsubsection*{F. Footprints} 

\problemauthor{Баев А.Ж.}

Обозначим начальную позицию (0, 0). Далее промоделируем шаги $(x_i, y_i)$. Минимальные размеры лабиринта будут $(\max\limits_{1 \leqslant i \leqslant n} x_i - \min\limits_{1 \leqslant i \leqslant n} x_i)$ и $(\max\limits_{1 \leqslant i \leqslant n} y_i - \min\limits_{1 \leqslant i \leqslant n} y_i)$ соответственно.

Асимптотика: $O(N)$.



\subsubsection*{G. Great divisors} 

\problemauthor{Абдикалыков А.К.}

Максимальный собственный делитель числа $n$ равен $n / p_n$, где $p_n$ --- минимальное простое число, на которое делится $n$. Последовательность $p_n$ легко построить, используя стандартный алгоритм решета Эратосфена (у всех еще не вычеркнутых чисел вида $p^2 + p \cdot j$ минимальным простым делителем будет $p$).

Асимптотика: $O(n \log n)$.



\subsubsection*{H. Honest gifts} 

\problemauthor{Баев А.Ж.}

Ясно, что максимальным количество наборов с общим количество $p$ синих и $q$ красных карандашей будет $(p, q)$ --- наибольший общий делитель $p$ и $q$. Поэтому достаточно перебрать все числа $i$ от 0 до $k$ и выбрать максимум из $gcd(a - i, b - (k - i))$.

Асимптотика: $O(k \log \max(a, b))$.



\subsubsection*{I. Inner subset} 

\problemauthor{Баев А.Ж.}

Необходимо посчитать количество способов выбрать подпоследовательность так, чтобы сумма чисел была кратна $k$. Обозначим $d[i][r]$ --- количество подпоследовательностей из первых $i$ элементов, которые в сумму дают остаток $r$ при делении на $k$. Каждое такое подножество можно получить, либо добавив $a[i]$ элемент к подножествами множества из первых $i-1$ с остатком суммы равным $(r - a[i]) \bmod k$, либо не добавляя $a[i]$ элемент:
$$d[i][r] = d[i-1][r] + d[i-1][(r - a[i]) \bmod k].$$
Инициализировать динамику можно $d[0][0] = 1$ и $d[0][r] = 0$ при $r$ от 1 до $k-1$.

Асимптотика: $O(n k)$.

Замечание: не стоит забывать производить каждое сложение по модулю $10^9 + 7$, иначе произойдет переполнение ответа.

\end{document} 
