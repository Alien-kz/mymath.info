\documentclass[10pt, a4paper]{article}

\usepackage[T2A]{fontenc}		%cyrillic output
\usepackage[utf8]{inputenc}		%cyrillic output
\usepackage[english, russian]{babel}	%word wrap
\usepackage{amssymb, amsfonts, amsmath}	%math symbols
\usepackage{mathtext}			%text in formulas
\usepackage{geometry}			%paper format attributes
\usepackage{fancyhdr}			%header
\usepackage{graphicx}			%input pictures
\usepackage{tabularx}			%smart table
\usepackage{longtable}			%long table
\usepackage{tikz, pgfplots}		%draw pictures and graphics
\usetikzlibrary{patterns}		%draw pictures: fill
\usetikzlibrary{positioning}	%draw pictures: below of
\usetikzlibrary{calc}			%draw pictures: $\i$
\usepackage{listofitems}		%draw from tex-list
\usepackage{enumitem}			%enumarate parameters

\geometry{left=2cm, right=2cm, top=2cm, bottom=2cm, headheight=15pt}
\setlist[enumerate]{leftmargin=*}	%remove enumarate indenttion
\sloppy							%correct overfull
\pagestyle{empty}				%no page numbers

\newcommand{\accept}[2]{
	\centerline{\boxed{#1}}
	\newline
	\centerline{\scriptsize{#2}}
}
\newcommand{\reject}[1]{
	\centerline{#1}
}


\newcommand{\head}[4]
{
	\thispagestyle{fancy}
	\fancyhf{}
	\chead{#3, #4}

	\begin{center}
	\begin{large}
	#1 \\
	\textit{#2}\\
	\end{large}
	\end{center}

}


% format

\newcommand{\informat}[1]
{
	\subsubsection*{Ввод} #1
}

\newcommand{\outformat}[1]
{
	\subsubsection*{Вывод} #1
}

\newcommand{\example}[2]
{
	\subsubsection*{Пример}
	\noindent
	\begin{center}
	\begin{tabularx}{\linewidth}{|X|X|}
	\hline
	Ввод & Вывод \\
	\hline
	{\tt #1} & {\tt #2}		\\
	\hline
	\end{tabularx}
	\end{center}
}

\newcommand{\examplelong}[2]
{
	\subsubsection*{Пример}
	\noindent
	\begin{center}
	\begin{tabularx}{\textwidth}{|l|X|}
	\hline
	Ввод & Вывод \\
	\hline
	#1 & #2		\\
	\hline
	\end{tabularx}
	\end{center}
}

\newcommand{\examplee}[4]
{
	\subsubsection*{Пример}
	\noindent
	\begin{center}
	\begin{tabularx}{\linewidth}{|X|X|}
	\hline
	Ввод 	& Вывод  	\\
	\hline
	{\tt #1} & {\tt #2}	\\
	\hline
	{\tt #3} & {\tt #4}	\\
	\hline
	\end{tabularx}
	\end{center}
}

\newcommand{\exampleee}[6]
{
	\subsubsection*{Пример}
	\noindent
	\begin{center}
	\begin{tabularx}{\linewidth}{|X|X|}
	\hline
	Ввод 	& Вывод  	\\
	\hline
	{\tt #1} & {\tt #2}	\\
	\hline
	{\tt #3} & {\tt #4}	\\
	\hline
	{\tt #5} & {\tt #6}	\\
	\hline
	\end{tabularx}
	\end{center}
}

\newcommand{\exampleeee}[8]
{
	\subsubsection*{Пример}
	\noindent
	\begin{center}
	\begin{tabularx}{\linewidth}{|X|X|}
	\hline
	Ввод 	& Вывод  	\\
	\hline
	{\tt #1} & {\tt #2}	\\
	\hline
	{\tt #3} & {\tt #4}	\\
	\hline
	{\tt #5} & {\tt #6}	\\
	\hline
	{\tt #7} & {\tt #8}	\\
	\hline
	\end{tabularx}
	\end{center}
}

\newcommand{\exampleeeee}[5]
{
	\subsubsection*{Пример}
	\begin{center}
	\begin{tabularx}{\linewidth}{|X|X|}
	\hline
	Ввод 	& Вывод  	\\
	\hline
	#1		\\
	\hline
	#2		\\
	\hline
	#3		\\
	\hline
	#4		\\
	\hline
	#5		\\
	\hline
	\end{tabularx}
	\end{center}
}

\newcommand{\examplepic}[3]
{
	\subsubsection*{Пример}
	\noindent
	\begin{center}
	\begin{tabularx}{\linewidth}{|l|l|X|}
	\hline
	Ввод 	& Вывод  	& Пояснение\\
	\hline
	{\tt #1} 		& {\tt #2} 		& #3\\
	\hline
	\end{tabularx}
	\end{center}
}


\newcommand{\excomm}[1]
{
	\subsubsection*{Комментарий}
	\textit{#1}
}

\newcommand{\problemauthor}[1]{
\begin{flushright}
\textit{Автор: #1}
\end{flushright}
}

\newcommand{\problemofferer}[1]{
\begin{flushright}
\textit{Предложил: #1}
\end{flushright}
}

\usepackage{listings}
\lstset{language=C,
        basicstyle=\ttfamily,
        keywordstyle=\color{blue},
        frame=single,
        numbers=left,
        tabsize=4}

\begin{document}

\head{Открытая личная олимпиада по программированию \\ Зимний тур 2018}{12 декабря 2018}{Казахстанский филиал МГУ имени М.В.Ломоносова}{г.~Астана}

\subsection*{A. Around the problems}

\problemauthor{ Баев А.Ж.}

Сгруппируем периоды по парам, чтобы найти суммарное количество минут, которое прошло от начала олимпиады. Если $p$ нечетное, то количество минут от начала олимпиады равно 
$$t = \left[ \frac{p-1}{2} \right] * (n_1 + n_2) + m.$$
Если $p$ четное, то количество минут от начала олимпиады равно 
$$t = \left[ \frac{p-1}{2} \right] * (n_1 + n_2) + n_1 + m.$$

Ответ зависит от остатка при делении $t$ на 7. Если $t$ делится на 7, то ответ  $G$, иначе это ответ $(t - 1) \bmod 7$-я задача.

Асимптотика по времени $O(1)$.

\lstinputlisting{code/2018-12-A.cpp}




\subsection*{B. Be lazy}

\problemauthor{ Абдикалыков А.К.}

Найдем расстояния до ближайших к числам $p$ и $q$ элементов: $mp = \min_{i} |a_i - p|$ и $mq = \min_{i} |a_i - q|$. Ответом будет $\min( mp + mq, |p - q|)$.

Асимптотика по времени $O(n)$.

\lstinputlisting{code/2018-12-B.cpp}




\subsection*{C. Calculator}

\problemauthor{ Жусупов Али}

Переберём все числа $k$ от 0 до $2^n - 1$, где $n$ --- количество символов данной строки. Число $k$ не подходит, если хотя бы в одной позиции $i$ в строке $s$ будут символ отличный от вопроса (то есть 0 или 1) и он не будет совпадать с $i$-м битом числа $k$. Выбор соответствующего бита удобны сделать через битовый сдвиг числа.

Асимптотика по времени $O(2^n \cdot n)$.

\lstinputlisting{code/2018-12-C.cpp}




\subsection*{D. Deep rolling}

\problemauthor{ Баев А.Ж.}

\begin{center}
\begin{tikzpicture}[yslant=0, xslant=0.5, scale=0.75]
	\draw (0, 0) grid (6, 6);


	\foreach \x in {1, ..., 6}{
		\draw (\x, 0) -- (0, \x);
	}
	\foreach \x in {1, ..., 5}{
		\draw (\x, 6) -- (6, \x);
	}
	\draw[->] (0, 0) -- (7, 0);
	\node at (7.5, 0) (a) {$a$};

	\draw[->] (0, 0) -- (0, 7);
	\node at (0, 7.5) (b) {$b$};

	\foreach \x in {1, 3, 5}{
		\foreach \y in {1, 3, 5}{	
			\fill[red] (\x, \y) -- (\x + 1, \y) -- (\x, \y + 1) -- cycle;
			\node at (\x+0.365, \y+0.3) (R) {R};
		}
	}
	\foreach \x in {0, 2, 4}{
	    \foreach \y in {0, 2, 4}{
			\node at (\x+0.365, \y+0.3) (W) {W};
		}
	}
	\foreach \x in {1, 3, 5}{
	    \foreach \y in {0, 2, 4}{
			\fill[blue] (\x, \y) -- (\x + 1, \y) -- (\x, \y + 1) -- cycle;
			\node[white] at (\x+0.365, \y+0.3) (B) {B};
		}
	}
	\foreach \x in {0, 2, 4}{
		\foreach \y in {1, 3, 5}{	
			\fill[green] (\x, \y) -- (\x + 1, \y) -- (\x, \y + 1) -- cycle;
			\node at (\x+0.365, \y+0.3) (G) {G};
		}
	}
	\foreach \x in {1, 3, 5}{
		\foreach \y in {1, 3, 5}{	
			\fill[red] (\x, \y) -- (\x - 1, \y) -- (\x, \y - 1) -- cycle;
			\node at (\x-0.365, \y-0.3) (R) {R};
		}
	}
	\foreach \x in {2, 4, 6}{
	    \foreach \y in {2, 4, 6}{
			\node at (\x-0.365, \y-0.3) (W) {W};
		}
	}
	\foreach \x in {1, 3, 5}{
	    \foreach \y in {2, 4, 6}{
			\fill[blue] (\x, \y) -- (\x - 1, \y) -- (\x, \y - 1) -- cycle;
			\node[white] at (\x-0.365, \y-0.3) (B) {B};
		}
	}
	\foreach \x in {2, 4, 6}{
		\foreach \y in {1, 3, 5}{	
			\fill[green] (\x, \y) -- (\x - 1, \y) -- (\x, \y - 1) -- cycle;
			\node at (\x-0.365, \y-0.3) (G) {G};
		}
	}


\end{tikzpicture}
\end{center}

Если $[a]$ и $[b]$ одной четности, то это либо белый, либо красный цвет. Белый цвет бывает в случае, если $[a]$ --- четное и $\{a\} + \{b\} < 1$ или если $[a]$ --- нечетное и $\{a\} + \{b\} > 1$. В противном случае, цвет --- красный.

Если $[a]$ и $[b]$ разной четности, то это либо синий, либо зеленый цвет. Зеленый цвет бывает в случае, если $[a]$ --- четное и $\{a\} + \{b\} < 1$ или если $[a]$ --- нечетное и $\{a\} + \{b\} > 1$. В противном случае, цвет --- зеленый.

Асимптотика по времени $O(1)$.

\lstinputlisting{code/2018-12-D.cpp}




\subsection*{E. Elementary balance}

\problemauthor{ Баев А.Ж.}

Суффиксные суммы считаем за один проход слева направо. 
$$l_i = l_{i - 1} + a_i$$

Сумма всего массива $s$ равна сумме префиксной суммы $l_k$, элемента $a_k$ и суффиксной суммы $r_k$.
$$s = \sum_{i=1}^{k - 1} a_i + a_k + \sum_{i=k + 1}^{n} a_i = l_k + a_k + r_k $$
Значит префиксные суммы можно выразить через суффиксные.

Ответ $ans = \min_i |l_i - r_i|$ --- минимум модуля их разности. Несложно заметить, что префиксные и суффиксные суммы можно не хранить в массивах, а вычислять <<на лету>>.

Асимптотика по времени $O(n)$.

\lstinputlisting{code/2018-12-E.cpp}




\subsection*{F. Full overlapping}

\problemofferer{ Жусупов Али}

Сохраним все левые концы в один массив $l$, все правые концы --- в массив $r$. Отсортируем оба массива по возрастанию. Заведем переменную-счетчик $c$ для вычисления количества наложений. Теперь пройдемся сканирующей прямой по объединению значений в обоих массивов слева направо. Для этого модифицируем алгоритм слияния двух отсортированных массивов через два указателя $i$ и $j$. Если текущий правый конец $l_i$ не меньше текущего левого конца $r_j$, то указатель $i$ передвигаем к следующему правому концу и уменьшаем счетчик $c$. В противном случае, указатель $j$ передвигаем к следующему левому концу и увеличиваем счетчик $c$. Максимальное значение $c$ за всё время --- ответ на задачу. Обратите внимание, если $l_i = r_j$, то необходимо сдвигать указатель $j$.

Асимптотика по времени $O(n)$.

\lstinputlisting{code/2018-12-F.cpp}




\subsection*{G. Galaxy number}

\problemauthor{ Абдикалыков А.К.}

Пусть $m$ --- искомое число. Тогда $m - 10^{41} < k$ и $m \bmod k \equiv 0$. То есть $m = 10^{41} + r$, где $r$ остаток, дополняющий остаток при делении $10^{41}$ на $k$ до $0$ или $k$. Таким образом:
$$ m = 10^{41} + (k - r) \bmod k$$
где $r = 10^{41} \bmod k$, который можно вычислить наивным алгоритмом, не забывая вычислить остаток при делении на $k$ после каждого умножения.

Стоит обратить внимание, что наличие встроенной длинной арифметики на python позволяет очень написать ответ прямой формулой.

Асимптотика по времени $O(1)$.

\lstinputlisting{code/2018-12-G.cpp}





\end{document} 
