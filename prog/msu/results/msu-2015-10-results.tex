\documentclass[10pt, a4paper, landscape]{article}

\usepackage[T2A]{fontenc}		%cyrillic output
\usepackage[utf8]{inputenc}		%cyrillic output
\usepackage[english, russian]{babel}	%word wrap
\usepackage{amssymb, amsfonts, amsmath}	%math symbols
\usepackage{mathtext}			%text in formulas
\usepackage{geometry}			%paper format attributes
\usepackage{fancyhdr}			%header
\usepackage{graphicx}			%input pictures
\usepackage{tabularx}			%smart table
\usepackage{longtable}			%long table
\usepackage{tikz, pgfplots}		%draw pictures and graphics
\usetikzlibrary{patterns}		%draw pictures: fill
\usetikzlibrary{positioning}	%draw pictures: below of
\usetikzlibrary{calc}			%draw pictures: $\i$
\usepackage{listofitems}		%draw from tex-list
\usepackage{enumitem}			%enumarate parameters

\geometry{left=2cm, right=2cm, top=2cm, bottom=2cm, headheight=15pt}
\setlist[enumerate]{leftmargin=*}	%remove enumarate indenttion
\sloppy							%correct overfull
\pagestyle{empty}				%no page numbers

\newcommand{\accept}[2]{
	\centerline{\boxed{#1}}
	\newline
	\centerline{\scriptsize{#2}}
}
\newcommand{\reject}[1]{
	\centerline{#1}
}


\newcommand{\head}[4]
{
	\thispagestyle{fancy}
	\fancyhf{}
	\chead{#3, #4}

	\begin{center}
	\begin{large}
	#1 \\
	\textit{#2}\\
	\end{large}
	\end{center}

}


% format

\newcommand{\informat}[1]
{
	\subsubsection*{Ввод} #1
}

\newcommand{\outformat}[1]
{
	\subsubsection*{Вывод} #1
}


\newcommand{\example}[2]
{
	\subsubsection*{Пример}
	{\tt
	\begin{tabularx}{\linewidth}{|X|X|}
	\hline
	Ввод & Вывод \\
	\hline
	#1 & #2		\\
	\hline
	\end{tabularx}
	}
}

\newcommand{\examplee}[4]
{
	\subsubsection*{Пример}
	{\tt
	\begin{tabularx}{\linewidth}{|X|X|}
	\hline
	Ввод 	& Вывод  	\\
	\hline
	#1 		& #2 		\\
	\hline
	#3		& #4		\\
	\hline
	\end{tabularx}
	}
}

\newcommand{\exampleee}[6]
{
	\subsubsection*{Пример}
	{\tt
	\begin{tabularx}{\linewidth}{|X|X|}
	\hline
	Ввод 	& Вывод  	\\
	\hline
	#1 		& #2 		\\
	\hline
	#3		& #4		\\
	\hline
	#5		& #6		\\
	\hline
	\end{tabularx}
	}
}

\newcommand{\exampleeee}[8]
{
	\subsubsection*{Пример}
	{\tt
	\begin{tabularx}{\linewidth}{|X|X|}
	\hline
	Ввод 	& Вывод  	\\
	\hline
	#1 		& #2 		\\
	\hline
	#3		& #4		\\
	\hline
	#5		& #6		\\
	\hline
	#7		& #8		\\
	\hline
	\end{tabularx}
	}
}

\newcommand{\exampleeeee}[5]
{
	\subsubsection*{Пример}
	{\tt
	\begin{tabularx}{\linewidth}{|X|X|}
	\hline
	Ввод 	& Вывод  	\\
	\hline
	#1		\\
	\hline
	#2		\\
	\hline
	#3		\\
	\hline
	#4		\\
	\hline
	#5		\\
	\hline
	\end{tabularx}
	}
}

\newcommand{\examplepic}[3]
{
	\subsubsection*{Пример}
	{\tt
	\noindent
	\begin{tabularx}{\linewidth}{|X|X|X|}
	\hline
	Ввод 	& Вывод  	& Пояснение\\
	\hline
	#1 		& #2 		& #3\\
	\hline
	\end{tabularx}
	}
}


\newcommand{\excomm}[1]
{
	\subsubsection*{Комментарий}
	\textit{#1}
}

\newcommand{\problemauthor}[1]{
\begin{flushright}
\textit{Автор: #1}
\end{flushright}
}

\newcommand{\problemofferer}[1]{
\begin{flushright}
\textit{Предложил: #1}
\end{flushright}
}

\usepackage{listings}
\lstset{language=C,
        basicstyle=\ttfamily,
        keywordstyle=\color{blue},
        frame=single,
        numbers=left,
        tabsize=4}

\begin{document}

\head{Открытая командная олимпиада по программированию \\ Осенний тур 2015}{20 октября 2015}{Казахстанский филиал МГУ имени М.В.Ломоносова}{г.~Астана}

\renewcommand{\arraystretch}{1.5}
\begin{center}
\begin{longtable}{|c|p{0.2\linewidth}|p{0.2\linewidth}|*{9}{p{0.025\linewidth}|}c|c|}
\hline 
№ & Команда & Состав & A & B & C & D & E & F & G & H & I & Итог & Штраф\\
\hline
\endhead
1 & Snowy Cube	 & Камалбеков Тимур (ВМ-2) \newline Абайулы Ерулан (ВМ-2) & 
\accept{+2}{0:27}  &
\accept{+}{0:10}  &
\accept{+}{0:47}  &
\accept{+1}{2:44}  &
\accept{+2}{1:11}  &
\accept{+}{0:44}  &
\accept{+2}{1:23}  &
\accept{+}{1:31}  &
  &
8 &
677\\
\hline
2 & Банка огурцов	 & Турганбаев Сатбек (ВМ-2) \newline Омаров Темирхан (ВМ-2) \newline Жусупов Али (ММ-2) & 
\accept{+2}{2:11}  &
\accept{+6}{1:57}  &
\accept{+1}{1:09}  &
\reject{-3} &
  &
\accept{+3}{2:17}  &
\accept{+1}{2:33}  &
\accept{+1}{2:46}  &
  &
6 &
1053\\
\hline
3 & School Dominators	 & Сёмочкин Константин (школа) \newline Ещанов Данияр (школа) \newline Сапаргалиев Олжас (школа) & 
\accept{+1}{0:32}  &
\reject{-4} &
\reject{-2} &
  &
\accept{+}{1:20}  &
\accept{+}{1:46}  &
  &
\reject{-5} &
  &
3 &
238\\
\hline
4 & Pascal/C++	 & Абдыкалик Чингиз (ВМ-1) \newline Кожемяк Виталий (ВМ-1) \newline Малик Алмат (ММ-2) & 
\accept{+}{0:32}  &
\reject{-7} &
\reject{-2} &
  &
  &
  &
\reject{-5} &
  &
  &
1 &
32\\
\hline
5 & HAWK	 & Керопян Агаси (ВМ-1) \newline Ахметов Дастан (ВМ-1) \newline Мырзабеков Руслан (ВМ-1) & 
\accept{+1}{1:45}  &
\reject{-8} &
  &
  &
  &
  &
  &
  &
  &
1 &
125\\
\hline
6 & Avadakedavra	 & Болотников Дмитрий (ММ-1) \newline Савицкая Нина (ММ-1) \newline Коробов Павел (ММ-1) & 
\accept{+}{2:45}  &
\reject{-5} &
  &
  &
  &
  &
  &
  &
  &
1 &
165\\
\hline
 & & Успешных попыток &
6  &
2  &
2  &
1  &
2  &
3  &
2  &
2  &
0  &
20 & \\
\hline 
 & & Всего попыток &
12  &
34  &
8  &
5  &
4  &
6  &
14  &
8  &
0  &
91 & \\
\hline 
\end{longtable} 
\end{center}
\renewcommand{\arraystretch}{1}

\end{document} 
