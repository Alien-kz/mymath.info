\documentclass[10pt, a4paper, landscape]{article}

\usepackage[T2A]{fontenc}		%cyrillic output
\usepackage[utf8]{inputenc}		%cyrillic output
\usepackage[english, russian]{babel}	%word wrap
\usepackage{amssymb, amsfonts, amsmath}	%math symbols
\usepackage{mathtext}			%text in formulas
\usepackage{geometry}			%paper format attributes
\usepackage{fancyhdr}			%header
\usepackage{graphicx}			%input pictures
\usepackage{tabularx}			%smart table
\usepackage{longtable}			%long table
\usepackage{tikz, pgfplots}		%draw pictures and graphics
\usetikzlibrary{patterns}		%draw pictures: fill
\usetikzlibrary{positioning}	%draw pictures: below of
\usetikzlibrary{calc}			%draw pictures: $\i$
\usepackage{listofitems}		%draw from tex-list
\usepackage{enumitem}			%enumarate parameters

\geometry{left=2cm, right=2cm, top=2cm, bottom=2cm, headheight=15pt}
\setlist[enumerate]{leftmargin=*}	%remove enumarate indenttion
\sloppy							%correct overfull
\pagestyle{empty}				%no page numbers

\newcommand{\accept}[2]{
	\centerline{\boxed{#1}}
	\newline
	\centerline{\scriptsize{#2}}
}
\newcommand{\reject}[1]{
	\centerline{#1}
}


\newcommand{\head}[4]
{
	\thispagestyle{fancy}
	\fancyhf{}
	\chead{#3, #4}

	\begin{center}
	\begin{large}
	#1 \\
	\textit{#2}\\
	\end{large}
	\end{center}

}


% format

\newcommand{\informat}[1]
{
	\subsubsection*{Ввод} #1
}

\newcommand{\outformat}[1]
{
	\subsubsection*{Вывод} #1
}


\newcommand{\example}[2]
{
	\subsubsection*{Пример}
	{\tt
	\begin{tabularx}{\linewidth}{|X|X|}
	\hline
	Ввод & Вывод \\
	\hline
	#1 & #2		\\
	\hline
	\end{tabularx}
	}
}

\newcommand{\examplee}[4]
{
	\subsubsection*{Пример}
	{\tt
	\begin{tabularx}{\linewidth}{|X|X|}
	\hline
	Ввод 	& Вывод  	\\
	\hline
	#1 		& #2 		\\
	\hline
	#3		& #4		\\
	\hline
	\end{tabularx}
	}
}

\newcommand{\exampleee}[6]
{
	\subsubsection*{Пример}
	{\tt
	\begin{tabularx}{\linewidth}{|X|X|}
	\hline
	Ввод 	& Вывод  	\\
	\hline
	#1 		& #2 		\\
	\hline
	#3		& #4		\\
	\hline
	#5		& #6		\\
	\hline
	\end{tabularx}
	}
}

\newcommand{\exampleeee}[8]
{
	\subsubsection*{Пример}
	{\tt
	\begin{tabularx}{\linewidth}{|X|X|}
	\hline
	Ввод 	& Вывод  	\\
	\hline
	#1 		& #2 		\\
	\hline
	#3		& #4		\\
	\hline
	#5		& #6		\\
	\hline
	#7		& #8		\\
	\hline
	\end{tabularx}
	}
}

\newcommand{\exampleeeee}[5]
{
	\subsubsection*{Пример}
	{\tt
	\begin{tabularx}{\linewidth}{|X|X|}
	\hline
	Ввод 	& Вывод  	\\
	\hline
	#1		\\
	\hline
	#2		\\
	\hline
	#3		\\
	\hline
	#4		\\
	\hline
	#5		\\
	\hline
	\end{tabularx}
	}
}

\newcommand{\examplepic}[3]
{
	\subsubsection*{Пример}
	{\tt
	\noindent
	\begin{tabularx}{\linewidth}{|X|X|X|}
	\hline
	Ввод 	& Вывод  	& Пояснение\\
	\hline
	#1 		& #2 		& #3\\
	\hline
	\end{tabularx}
	}
}


\newcommand{\excomm}[1]
{
	\subsubsection*{Комментарий}
	\textit{#1}
}

\newcommand{\problemauthor}[1]{
\begin{flushright}
\textit{Автор: #1}
\end{flushright}
}

\newcommand{\problemofferer}[1]{
\begin{flushright}
\textit{Предложил: #1}
\end{flushright}
}

\usepackage{listings}
\lstset{language=C,
        basicstyle=\ttfamily,
        keywordstyle=\color{blue},
        frame=single,
        numbers=left,
        tabsize=4}

\begin{document}

\head{Открытая командная олимпиада по программированию \\ Весенний тур 2015}{18 марта 2015}{Казахстанский филиал МГУ имени М.В.Ломоносова}{г.~Астана}

\renewcommand{\arraystretch}{1.5}
\begin{center}
\begin{longtable}{|c|p{0.15\linewidth}|p{0.25\linewidth}|*{10}{p{0.025\linewidth}|}c|c|}
\hline 
№ & Команда & Состав & A & B & C & D & E & F & G & H & I & J & Итог & Штраф \\
\hline
\endhead
1 & Lord Bendtner Team	 & Седякин Илья (ВМ-21) \newline Таскынов Ануар (ВМ-21) \newline Вержбицкий Владислав (ВМ-21) & 
\accept{+2}{0:39}&
  &
\accept{+}{3:01}&
\accept{+}{3:08}&
  &
  &
  &
\accept{+2}{1:16}&
  &
\accept{+2}{1:49}&
5 &
713
\\
\hline
2 & prosti Denis	 & Шокетаева Надира (ММ-21) \newline Жусупов Али (ММ-11) & 
\accept{+1}{0:05}&
\reject{-1} &
  &
\accept{+}{1:14}&
  &
  &
  &
\accept{+6}{3:55}&
\accept{+4}{2:03}&
\accept{+2}{0:46}&
5 &
743
\\
\hline
3 & Snow	 & Журавская Александра (ВМ-11) \newline Камалбеков Тимур (ВМ-11) \newline Абайулы Ерулан (ВМ-11) & 
\accept{+}{0:27}&
\accept{+2}{1:38}&
  &
\accept{+4}{1:04}&
\accept{+4}{3:48}&
  &
  &
\accept{+5}{3:06}&
  &
\reject{-5} &
5 &
903
\\
\hline
4 & МАРЖЫ	 & Таранов Денис (ВМ-21) \newline Васильев Андрей (ВМ-21) & 
\accept{+1}{0:07}&
\accept{+}{1:46}&
\reject{-1} &
\accept{+7}{0:58}&
  &
  &
  &
\reject{-47} &
  &
\reject{-1} &
3 &
331
\\
\hline
5 & A contrario	 & Омаров Темирхан (ВМ-11) \newline Иглымов Алишер (ВМ-11) \newline Колесников Владислав (ВМ-11) &
\accept{+1}{0:23}&
\reject{-1} &
  &
\accept{+1}{1:13}&
  &
\reject{-4} &
  &
\accept{+2}{3:33}&
\reject{-2} &
  &
3 &
389
\\ 
\hline
6 & 012	 & Амир Мирас (ВМ-21) \newline Шабхатов Асылжан (ВМ-21) \newline Жусупекова Зинель (ВМ-21) & 
\accept{+}{0:20}&
  &
\reject{-7} &
\accept{+}{3:21}&
  &
  &
  &
\reject{-28} &
\reject{-4} &
\reject{-1} &
2 &
221
\\
\hline
7 & FATAL ERROR! & Ганиев Адилет (ММ-11) \newline Малик Алмат (ММ-11) \newline Абдильманов Ернар (ММ-11) & 
\accept{+}{0:32}&
  &
  &
\reject{-4} &
  &
  &
  &
\reject{-2} &
  &
  &
1 &
32
\\
\hline
8 & The Big Bang	 & Вишневский Виктор (ММ-11) \newline Исакова Жаркын (ММ-11) \newline Ким Виолетта (ММ-11) & 
\accept{+1}{0:37}&
\reject{-1} &
  &
  &
  &
  &
  &
\reject{-2} &
  &
  &
1 &
57
\\
\hline
9 & DKA	& Даку Ансар (ВМ-11) &
\accept{+1}{1:40}&
  &
  &
\reject{-1} &
  &
  &
  &
\reject{-7} &
  &
  &
1 &
120
\\
\hline
 & & Успешных попыток &
9  &
2  &
1  &
6  &
1  &
0  &
0  &
4  &
1  &
2  &
26 & \\
\hline 
 & & Всего попыток &
16  &
7  &
9  &
23  &
5  &
4  &
0  &
105  &
11  &
13  &
193 & \\
\hline 
\end{longtable}
\end{center}
\renewcommand{\arraystretch}{1}

\end{document} 
