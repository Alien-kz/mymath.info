\documentclass[10pt, a4paper, landscape]{article}

\usepackage[T2A]{fontenc}		%cyrillic output
\usepackage[utf8]{inputenc}		%cyrillic output
\usepackage[english, russian]{babel}	%word wrap
\usepackage{amssymb, amsfonts, amsmath}	%math symbols
\usepackage{mathtext}			%text in formulas
\usepackage{geometry}			%paper format attributes
\usepackage{fancyhdr}			%header
\usepackage{graphicx}			%input pictures
\usepackage{tabularx}			%smart table
\usepackage{longtable}			%long table
\usepackage{tikz, pgfplots}		%draw pictures and graphics
\usetikzlibrary{patterns}		%draw pictures: fill
\usetikzlibrary{positioning}	%draw pictures: below of
\usetikzlibrary{calc}			%draw pictures: $\i$
\usepackage{listofitems}		%draw from tex-list
\usepackage{enumitem}			%enumarate parameters

\geometry{left=2cm, right=2cm, top=2cm, bottom=2cm, headheight=15pt}
\setlist[enumerate]{leftmargin=*}	%remove enumarate indenttion
\sloppy							%correct overfull
\pagestyle{empty}				%no page numbers

\newcommand{\accept}[2]{
	\centerline{\boxed{#1}}
	\newline
	\centerline{\scriptsize{#2}}
}
\newcommand{\reject}[1]{
	\centerline{#1}
}


\newcommand{\head}[4]
{
	\thispagestyle{fancy}
	\fancyhf{}
	\chead{#3, #4}

	\begin{center}
	\begin{large}
	#1 \\
	\textit{#2}\\
	\end{large}
	\end{center}

}


% format

\newcommand{\informat}[1]
{
	\subsubsection*{Ввод} #1
}

\newcommand{\outformat}[1]
{
	\subsubsection*{Вывод} #1
}


\newcommand{\example}[2]
{
	\subsubsection*{Пример}
	{\tt
	\begin{tabularx}{\linewidth}{|X|X|}
	\hline
	Ввод & Вывод \\
	\hline
	#1 & #2		\\
	\hline
	\end{tabularx}
	}
}

\newcommand{\examplee}[4]
{
	\subsubsection*{Пример}
	{\tt
	\begin{tabularx}{\linewidth}{|X|X|}
	\hline
	Ввод 	& Вывод  	\\
	\hline
	#1 		& #2 		\\
	\hline
	#3		& #4		\\
	\hline
	\end{tabularx}
	}
}

\newcommand{\exampleee}[6]
{
	\subsubsection*{Пример}
	{\tt
	\begin{tabularx}{\linewidth}{|X|X|}
	\hline
	Ввод 	& Вывод  	\\
	\hline
	#1 		& #2 		\\
	\hline
	#3		& #4		\\
	\hline
	#5		& #6		\\
	\hline
	\end{tabularx}
	}
}

\newcommand{\exampleeee}[8]
{
	\subsubsection*{Пример}
	{\tt
	\begin{tabularx}{\linewidth}{|X|X|}
	\hline
	Ввод 	& Вывод  	\\
	\hline
	#1 		& #2 		\\
	\hline
	#3		& #4		\\
	\hline
	#5		& #6		\\
	\hline
	#7		& #8		\\
	\hline
	\end{tabularx}
	}
}

\newcommand{\exampleeeee}[5]
{
	\subsubsection*{Пример}
	{\tt
	\begin{tabularx}{\linewidth}{|X|X|}
	\hline
	Ввод 	& Вывод  	\\
	\hline
	#1		\\
	\hline
	#2		\\
	\hline
	#3		\\
	\hline
	#4		\\
	\hline
	#5		\\
	\hline
	\end{tabularx}
	}
}

\newcommand{\examplepic}[3]
{
	\subsubsection*{Пример}
	{\tt
	\noindent
	\begin{tabularx}{\linewidth}{|X|X|X|}
	\hline
	Ввод 	& Вывод  	& Пояснение\\
	\hline
	#1 		& #2 		& #3\\
	\hline
	\end{tabularx}
	}
}


\newcommand{\excomm}[1]
{
	\subsubsection*{Комментарий}
	\textit{#1}
}

\newcommand{\problemauthor}[1]{
\begin{flushright}
\textit{Автор: #1}
\end{flushright}
}

\newcommand{\problemofferer}[1]{
\begin{flushright}
\textit{Предложил: #1}
\end{flushright}
}

\usepackage{listings}
\lstset{language=C,
        basicstyle=\ttfamily,
        keywordstyle=\color{blue},
        frame=single,
        numbers=left,
        tabsize=4}

\begin{document}

\head{Открытая командная олимпиада по программированию \\ Весенний тур 2017}{31 мая 2017}{Казахстанский филиал МГУ имени М.В.Ломоносова}{г.~Астана}

\renewcommand{\arraystretch}{1.5}
\begin{center}
\begin{longtable}{|c|c|p{0.15\linewidth}|p{0.28\linewidth}|*{9}{p{0.03\linewidth}|}c|c|}
\hline
№  & КФ  & Команда & Состав & A & B & C & D & E & F & G & H & I & Итог & Штраф 
\\
\hline
\endhead
1 &  & WeeD & Орынкул Батырхан (НУ) \newline  Иманмалик Ержан (НУ)  & \accept{+}{0:07}  & \accept{+1}{2:08}  & \accept{+}{0:18}  & \accept{+}{1:14}  & \accept{+3}{1:21}  & \accept{+1}{3:44}  & \accept{+}{0:34}  & \accept{+43}{2:28}  & \accept{+}{0:51}  & 9 & 1725\\
\hline
2 & 1 & гРиБы & Аскергали Ануар (ВМ-1) \newline  Бекмаганбетов Бекарыс (ММ-1) \newline Шарипов Азат (ВМ-1)  & \accept{+}{0:11}  & \reject{-1} & \accept{+}{1:50}  & \accept{+}{1:28}  & \accept{+}{0:42}  &   & \accept{+}{0:52}  & \reject{-4} & \accept{+1}{1:37}  & 6 & 420\\
\hline
3 &  & Premium  Basher & Сайланбаев Алибек   (НУ) \newline  Жусупов Али (ЕНУ) \newline Дуйсенбаев Азамат (ЕНУ) & \accept{+}{0:11}  & \reject{-5} & \accept{+}{1:01}  & \accept{+}{1:48}  & \accept{+1}{0:44}  &   & \accept{+}{2:39}  & \reject{-29} & \accept{+2}{1:26}  & 6 & 529\\
\hline
4 & 2 & Snowy Cube & Журавская Александра (ВМ-3) \newline  Абайулы Ерулан (ВМ-3)   \newline Камалбеков Тимур (ВМ-3)  & \accept{+}{0:07}  &   & \accept{+2}{1:41}  & \accept{+1}{0:41}  & \accept{+}{0:21}  &   &   &   & \accept{+5}{1:30}  & 5 & 420\\
\hline
5 &  & FM-coders & Айкен (ЕНУ)   \newline  Адилет (ЕНУ) \newline Нурлан (ЕНУ)& \accept{+}{0:28}  &   & \accept{+3}{2:49}  & \accept{+1}{2:17}  & \accept{+6}{3:18}  &   & \accept{+}{1:22}  & \reject{-24} & \reject{-2} & 5 & 814\\
\hline
6 & 3 & Жизнь за Нерзула & Болотников Димитрий   (ММ-2) \newline  Газизов Куат (ММ-2) \newline Кунакбаев Рамазан (ММ-1)  & \accept{+}{2:45}  &   & \reject{-2} & \accept{+2}{3:15}  & \accept{+}{2:00}  &   & \accept{+}{2:51}  &   & \accept{+1}{3:41}  & 5 & 932\\
\hline
7 &  & Тот самый  бэт-спрей  \newline против акул & Погребная Александра  \newline (6 гимназия)   \newline  Сапаргалиев Олжас (60 лицей)    \newline Агаси Керопян (ВМ-2) & \accept{+}{1:29}  &   &   & \accept{+1}{2:20}  & \accept{+5}{1:15}  & \reject{-1} & \reject{-2} & \accept{+43}{3:56}  & \reject{-16} & 4 & 1520\\
\hline
8 & 4 & AAC & Козган Айжас (ВМ-2) \newline  Ким Владимир (ВМ-2) \newline Абдыклик Чингиз (ВМ-2) & \accept{+}{2:10}  & \reject{-1} &   & \accept{+1}{2:58}  & \reject{-6} &   & \accept{+}{2:51}  &   & \reject{-7} & 3 & 499\\
\hline
9 & 5 & VirusVAR & Татин Алмаз  (ВМ-2)  \newline  Кожемяк Виталий (ВМ-2)   \newline Мырзабеков Руслан (ВМ-2)  & \accept{+}{0:37}  &   &   &   & \reject{-1} &   & \accept{+1}{3:03}  &   &   & 2 & 240\\
\hline
10 & 6 & tomorrow exam & Омаров Темирхан (ВМ-3) & \accept{+}{0:41}  &   &   &   & \reject{-3} &   &   &   &   & 1 & 41\\
\hline
11 & 7 & ZhalgasIChiki & Ержанов Жалгас (ВМ-1)  & \accept{+1}{1:37}  &   &   &   & \reject{-5} &   &   &   &   & 1 & 117\\
\hline
  &  & Успешных попыток &   & 11 & 1 & 5 & 8 & 7 & 1 & 7 & 2 & 5 & 47 &  \\
\hline
  &  & Всего попыток &   & 13 & 9 & 12 & 14 & 50 & 4 & 10 & 148 & 46 & 306 &  \\
\hline
  &  & \%: &   & 85\% & 11\% & 42\% & 57\% & 14\% & 25\% & 70\% & 1\% & 11\% & 15\% &  \\
\hline
\end{longtable}
\end{center}
\renewcommand{\arraystretch}{1}

\end{document} 
