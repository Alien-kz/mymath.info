\documentclass[10pt, a4paper, landscape]{article}

\usepackage[T2A]{fontenc}		%cyrillic output
\usepackage[utf8]{inputenc}		%cyrillic output
\usepackage[english, russian]{babel}	%word wrap
\usepackage{amssymb, amsfonts, amsmath}	%math symbols
\usepackage{mathtext}			%text in formulas
\usepackage{geometry}			%paper format attributes
\usepackage{fancyhdr}			%header
\usepackage{graphicx}			%input pictures
\usepackage{tabularx}			%smart table
\usepackage{longtable}			%long table
\usepackage{tikz, pgfplots}		%draw pictures and graphics
\usetikzlibrary{patterns}		%draw pictures: fill
\usetikzlibrary{positioning}	%draw pictures: below of
\usetikzlibrary{calc}			%draw pictures: $\i$
\usepackage{listofitems}		%draw from tex-list
\usepackage{enumitem}			%enumarate parameters

\geometry{left=1cm, right=1cm, top=2cm, bottom=1cm, headheight=15pt}
\setlist[enumerate]{leftmargin=*}	%remove enumarate indenttion
\sloppy							%correct overfull
\pagestyle{empty}				%no page numbers

\newcommand{\accept}[2]{
	\centerline{\boxed{#1}}
	\newline
	\centerline{\scriptsize{#2}}
}
\newcommand{\reject}[1]{
	\centerline{#1}
}


\newcommand{\head}[4]
{
	\thispagestyle{fancy}
	\fancyhf{}
	\chead{#3, #4}

	\begin{center}
	\begin{large}
	#1 \\
	\textit{#2}\\
	\end{large}
	\end{center}

}


% format

\newcommand{\informat}[1]
{
	\subsubsection*{Ввод} #1
}

\newcommand{\outformat}[1]
{
	\subsubsection*{Вывод} #1
}


\newcommand{\example}[2]
{
	\subsubsection*{Пример}
	{\tt
	\begin{tabularx}{\linewidth}{|X|X|}
	\hline
	Ввод & Вывод \\
	\hline
	#1 & #2		\\
	\hline
	\end{tabularx}
	}
}

\newcommand{\examplee}[4]
{
	\subsubsection*{Пример}
	{\tt
	\begin{tabularx}{\linewidth}{|X|X|}
	\hline
	Ввод 	& Вывод  	\\
	\hline
	#1 		& #2 		\\
	\hline
	#3		& #4		\\
	\hline
	\end{tabularx}
	}
}

\newcommand{\exampleee}[6]
{
	\subsubsection*{Пример}
	{\tt
	\begin{tabularx}{\linewidth}{|X|X|}
	\hline
	Ввод 	& Вывод  	\\
	\hline
	#1 		& #2 		\\
	\hline
	#3		& #4		\\
	\hline
	#5		& #6		\\
	\hline
	\end{tabularx}
	}
}

\newcommand{\exampleeee}[8]
{
	\subsubsection*{Пример}
	{\tt
	\begin{tabularx}{\linewidth}{|X|X|}
	\hline
	Ввод 	& Вывод  	\\
	\hline
	#1 		& #2 		\\
	\hline
	#3		& #4		\\
	\hline
	#5		& #6		\\
	\hline
	#7		& #8		\\
	\hline
	\end{tabularx}
	}
}

\newcommand{\exampleeeee}[5]
{
	\subsubsection*{Пример}
	{\tt
	\begin{tabularx}{\linewidth}{|X|X|}
	\hline
	Ввод 	& Вывод  	\\
	\hline
	#1		\\
	\hline
	#2		\\
	\hline
	#3		\\
	\hline
	#4		\\
	\hline
	#5		\\
	\hline
	\end{tabularx}
	}
}

\newcommand{\examplepic}[3]
{
	\subsubsection*{Пример}
	{\tt
	\noindent
	\begin{tabularx}{\linewidth}{|X|X|X|}
	\hline
	Ввод 	& Вывод  	& Пояснение\\
	\hline
	#1 		& #2 		& #3\\
	\hline
	\end{tabularx}
	}
}


\newcommand{\excomm}[1]
{
	\subsubsection*{Комментарий}
	\textit{#1}
}

\newcommand{\problemauthor}[1]{
\begin{flushright}
\textit{Автор: #1}
\end{flushright}
}

\newcommand{\problemofferer}[1]{
\begin{flushright}
\textit{Предложил: #1}
\end{flushright}
}

\usepackage{listings}
\lstset{language=C,
        basicstyle=\ttfamily,
        keywordstyle=\color{blue},
        frame=single,
        numbers=left,
        tabsize=4}

\begin{document}

\head{Открытая командная олимпиада по программированию \\ Весенний тур 2014}{17 марта 2014}{Казахстанский филиал МГУ имени М.В.Ломоносова}{г.~Астана}

\renewcommand{\arraystretch}{1.5}
\begin{center}
\begin{longtable}{|c|p{0.2\linewidth}|p{0.2\linewidth}|*{8}{p{0.025\linewidth}|}c|c|}
\hline 
№ & Команда & Состав & A & B & C & D & E & F & G & H & Итог & Штраф \\
\hline
\endhead
1 & Big Dipper &	Шокетаева Надира (ММ-11) 	\newline Таранов Денис (ВМ-11)		\newline Тлеубаев Адиль (ВМ-21) &
\accept{+}{0:09}  &
\accept{+}{0:12}  &
\accept{+}{2:33}  &
\accept{+2}{0:55}  &
\accept{+10}{2:07}  &
  &
  &
  &
5 &
596
 \\
\hline 
2 & msu\_25		&	Автайкина Мария (ВМ-21)		\newline Журавлев Вадим (ВМ-21)	\newline Овчинников Дмитрий (ВМ-21) &
\accept{+4}{1:11}  &
\accept{+}{0:12}  &
\accept{+1}{2:49}  &
  &
\accept{+}{2:01}  &
\reject{-3} &
  &
  &
4 &
473 \\
\hline 
3 & CMC AID		&	Оспанов Аят (ВМ-21)			\newline Ламонов Иван (ВМ-21)		\newline Солтанова Дана (ВМ-21) &
\accept{+1}{0:36}  &
\accept{+1}{0:09}  &
\accept{+}{1:33}  &
\reject{-1} &
  &
  &
  &
  &
3 &
178 \\
\hline 
4 & msu\_23		&	Седякин Илья (ВМ-11) \newline Васильев Андрей (ВМ-11)		\newline Таскынов Ануар (ВМ-11) &
\accept{+1}{0:08}  &
\accept{+}{0:10}  &
\accept{+1}{2:21}  &
  &
\reject{-2} &
  &
  &
  &
3 &
199 \\
\hline
5 & VM		&	Амир Мирас (ВМ-11)		\newline Матвеева Виктория (ВМ-11) &
\reject{-4} &
\accept{+1}{0:25}  &
\reject{-4} &
  &
  &
  &
  &
  &
1 &
45 \\
\hline
 & & Успешных попыток &
4  &
5  &
4  &
1  &
2  &
0  &
0  &
0  &
16  &
  \\
\hline 
 & & Всего попыток &
14  &
7  &
10  &
4  &
14  &
3  &
0  &
0  &
52  &
  \\
\hline 
\end{longtable} 
\end{center}
\renewcommand{\arraystretch}{1}

\end{document} 
