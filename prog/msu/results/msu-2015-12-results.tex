\documentclass[10pt, a4paper, landscape]{article}

\usepackage[T2A]{fontenc}		%cyrillic output
\usepackage[utf8]{inputenc}		%cyrillic output
\usepackage[english, russian]{babel}	%word wrap
\usepackage{amssymb, amsfonts, amsmath}	%math symbols
\usepackage{mathtext}			%text in formulas
\usepackage{geometry}			%paper format attributes
\usepackage{fancyhdr}			%header
\usepackage{graphicx}			%input pictures
\usepackage{tabularx}			%smart table
\usepackage{longtable}			%long table
\usepackage{tikz, pgfplots}		%draw pictures and graphics
\usetikzlibrary{patterns}		%draw pictures: fill
\usetikzlibrary{positioning}	%draw pictures: below of
\usetikzlibrary{calc}			%draw pictures: $\i$
\usepackage{listofitems}		%draw from tex-list
\usepackage{enumitem}			%enumarate parameters

\geometry{left=1cm, right=1cm, top=2cm, bottom=1cm, headheight=15pt}
\setlist[enumerate]{leftmargin=*}	%remove enumarate indenttion
\sloppy							%correct overfull
\pagestyle{empty}				%no page numbers

\newcommand{\accept}[2]{
	\centerline{\boxed{#1}}
	\newline
	\centerline{\scriptsize{#2}}
}
\newcommand{\reject}[1]{
	\centerline{#1}
}


\newcommand{\head}[4]
{
	\thispagestyle{fancy}
	\fancyhf{}
	\chead{#3, #4}

	\begin{center}
	\begin{large}
	#1 \\
	\textit{#2}\\
	\end{large}
	\end{center}

}


% format

\newcommand{\informat}[1]
{
	\subsubsection*{Ввод} #1
}

\newcommand{\outformat}[1]
{
	\subsubsection*{Вывод} #1
}


\newcommand{\example}[2]
{
	\subsubsection*{Пример}
	{\tt
	\begin{tabularx}{\linewidth}{|X|X|}
	\hline
	Ввод & Вывод \\
	\hline
	#1 & #2		\\
	\hline
	\end{tabularx}
	}
}

\newcommand{\examplee}[4]
{
	\subsubsection*{Пример}
	{\tt
	\begin{tabularx}{\linewidth}{|X|X|}
	\hline
	Ввод 	& Вывод  	\\
	\hline
	#1 		& #2 		\\
	\hline
	#3		& #4		\\
	\hline
	\end{tabularx}
	}
}

\newcommand{\exampleee}[6]
{
	\subsubsection*{Пример}
	{\tt
	\begin{tabularx}{\linewidth}{|X|X|}
	\hline
	Ввод 	& Вывод  	\\
	\hline
	#1 		& #2 		\\
	\hline
	#3		& #4		\\
	\hline
	#5		& #6		\\
	\hline
	\end{tabularx}
	}
}

\newcommand{\exampleeee}[8]
{
	\subsubsection*{Пример}
	{\tt
	\begin{tabularx}{\linewidth}{|X|X|}
	\hline
	Ввод 	& Вывод  	\\
	\hline
	#1 		& #2 		\\
	\hline
	#3		& #4		\\
	\hline
	#5		& #6		\\
	\hline
	#7		& #8		\\
	\hline
	\end{tabularx}
	}
}

\newcommand{\exampleeeee}[5]
{
	\subsubsection*{Пример}
	{\tt
	\begin{tabularx}{\linewidth}{|X|X|}
	\hline
	Ввод 	& Вывод  	\\
	\hline
	#1		\\
	\hline
	#2		\\
	\hline
	#3		\\
	\hline
	#4		\\
	\hline
	#5		\\
	\hline
	\end{tabularx}
	}
}

\newcommand{\examplepic}[3]
{
	\subsubsection*{Пример}
	{\tt
	\noindent
	\begin{tabularx}{\linewidth}{|X|X|X|}
	\hline
	Ввод 	& Вывод  	& Пояснение\\
	\hline
	#1 		& #2 		& #3\\
	\hline
	\end{tabularx}
	}
}


\newcommand{\excomm}[1]
{
	\subsubsection*{Комментарий}
	\textit{#1}
}

\newcommand{\problemauthor}[1]{
\begin{flushright}
\textit{Автор: #1}
\end{flushright}
}

\newcommand{\problemofferer}[1]{
\begin{flushright}
\textit{Предложил: #1}
\end{flushright}
}

\usepackage{listings}
\lstset{language=C,
        basicstyle=\ttfamily,
        keywordstyle=\color{blue},
        frame=single,
        numbers=left,
        tabsize=4}

\begin{document}

\head{Открытая личная олимпиада по программированию \\ Зимний тур 2015}{18 декабря 2015}{Казахстанский филиал МГУ имени М.В.Ломоносова}{г.~Астана}

\renewcommand{\arraystretch}{1.5}
\begin{center}
\begin{longtable}{|c|c|c|c|*{7}{c|}c|c|}
\hline 
Место & Участник & ВУЗ & Курс & A & B & C & D & E & F & G & Задач & Итог\\
\hline
\endhead
1 &
Омаров Темирхан  &
ВМК & 2 &
100 &
68 &
  &
100 &
100 &
100 &
52 &
4 &
520 \\
\hline
2 &
Сапаргалиев Олжас &
60 лицей & 9 &
 100 &
 100 &
  &
100 &
  &
16 &
100 &
4 &
416 \\
\hline
3 &
Колесников Владислав &
ВМК & 2 &
100 &
42 &
38 &
100 &
  &
100 &
  &
3 &
380 \\
\hline
4 &
Турукпаев Исламбек  &
ВМК & 1 &
98 &
100 &
  &
80 &
12 &
0 &
0 &
1 &
290 \\
\hline
5 &
Пикулина Алиса &
ММ & 1 &
100 &
  &
  &
80 &
  &
100 &
  &
2 &
280 \\
\hline
6 &
Ким Владимир &
ВМК & 1 &
98 &
50 &
  &
100 &
  &
  &
22 &
1 &
270 \\
\hline
7-8 &
Алмат Малик &
ММ & 2 &
100 &
76 &
  &
80 &
  &
  &
  &
1 &
256 \\
\hline
7-8 &
Елешов Данияр &
ВМК & 1 &
100 &
28 &
28 &
80 &
  &
20 &
  &
1 &
256 \\
\hline
9 &
Абдыкалик Шынгыс &
ВМК & 1 &
100 &
  &
  &
100 &
  &
26 &
  &
2 &
226 \\
\hline
10 &
Алибеков Ануар &
ВМК & 2 &
100 &
24 &
36 &
24 &
2 &
2 &
22 &
1 &
210 \\
\hline
11-12 &
Керопян Агаси &
ВМК & 1 &
100 &
12 &
6 &
24 &
18 &
26 &
0 &
1 &
186 \\
\hline
11-12 &
Болотников Димитрий &
ММ & 1 &
100 &
6 &
38 &
42 &
  &
0 &
  &
1 &
186 \\
\hline
13 &
Кобекбай Бауыржан &
ВМК & 1 &
100 &
  &
0 &
24 &
  &
26 &
  &
1 &
150 \\
\hline
14 &
Кинжикеева Дина &
ВМК & 2 &
100 &
  &
36 &
  &
  &
  &
  &
1 &
136 \\
\hline
15 &
Михно Ксения &
ММ & 1 &
86 &
32 &
  &
  &
  &
  &
  &
0 &
118 \\
\hline
16-17 &
Мустафин Ануар &
ММ & 2 &
100 &
0 &
  &
  &
  &
16 &
  &
1 &
116 \\
\hline
16-17 &
Бигалиева Айымкоз &
ВМК & 1 &
100 &
  &
  &
16 &
  &
0 &
  &
1 &
116 \\
\hline
18 &
Мырзабеков Руслан &
ВМК & 1 &
100 &
  &
  &
  &
  &
0 &
  &
1 &
100 \\
\hline
19 &
Алимгазиев Сержан &
ВМК & 1 &
  &
  &
30 &
60 &
  &
  &
  &
0 &
90 \\
\hline
20 &
Кожемяк Виталий &
ВМК & 1 &
34 &
  &
  &
16 &
  &
  &
  &
0 &
50 \\
\hline
21 &
Татин Алмаз &
ВМК & 1 &
4 &
  &
  &
  &
  &
  &
  &
0 &
4 \\
\hline
 & Успешных попыток & & &
15 &
2 &
0 &
5 &
1 &
3 &
1 & & \\
\hline 
 & Всего попыток & & &
20 &
11 &
7 &
16 &
4 &
10 &
4 & & \\
\hline 
\end{longtable} 
\end{center}
\renewcommand{\arraystretch}{1}

\end{document} 
