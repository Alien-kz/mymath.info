\documentclass[10pt]{beamer}
\usetheme{Madrid}
\usepackage[T2A]{fontenc}
\usepackage[utf8]{inputenc}
\usepackage[russian]{babel}
\usepackage{amsmath}
\usepackage{amsfonts}
\usepackage{amssymb}
\usepackage{comment}
\usepackage[all]{xy}

%\newtheorem{theorem}{Теорема}

\author{Баев А.Ж.}
\title{Введение в численные методы. \\ Итерационные методы решения СЛАУ}
\institute{Казахстанский филиал МГУ} 
\date{11 февраля 2019}
%\subject{} 

\usepackage{tikz}
\usetikzlibrary{patterns} 
\usetikzlibrary{arrows, matrix, positioning, decorations.pathreplacing}

\usepackage{listings}
\lstset{language=python}
\lstset{basicstyle=\ttfamily}  
\lstset{keywordstyle=\color{blue}}
\lstset{frame=single}
\lstset{numbers=left}
\lstset{tabsize=4}



\usepackage{graphicx}

\begin{document}

\maketitle


\begin{frame}[fragile]
\frametitle{План на семестр}

\begin{enumerate}
\item СЛАУ (точные методы)
\item \textbf{СЛАУ (итерационные методы)}
\item решение нелинейных уравнений
\item интерполяция 
\item аппроксимация
\item интегрирование
\item дифференцирование
\end{enumerate}

\end{frame}


\begin{frame}[fragile]
\frametitle{Линейная алгебра}

Решение системы линейных алгебраических уравнений:

$$A x = f$$

где $A \in \mathbb{R}^{n \times n}$, $x \in \mathbb{R}^{n}$, $f \in \mathbb{R}^{n}$.

Дано $A$, $f$. Найти $x$.

\end{frame}


\begin{frame}[fragile]
\frametitle{Каноническая форма}

\vfill
$x_0$ --- задается произвольным образом.

\vfill
$$ B \frac{x_{k+1} - x_k}{\tau} + A x_k = f$$
где $A$ --- исходная матрица, $B$ --- невырожденная матрица, зависящая от метода, $\tau$ --- итерационный параметр.

\vfill
$x_k$ сходится к решению.

\vfill

\end{frame}

\begin{frame}[fragile]
\frametitle{Каноническая форма}

$$ B x_{k+1} = B x_k - \tau A x_k + f \tau$$
где $A$ --- исходная матрица, $B$ --- невырожденная матрица, зависящая от метода, $\tau$ --- итерационный параметр.

\vfill

Необходимо выбирать $B$ такую, что её можно обратить быстрее, чем матрицу $A$.

\end{frame}


\begin{frame}[fragile]
\frametitle{Определения}

\vfill
$A^*$ --- сопряженный к $A$ оператор, если $(Ax, y) = (x, A^* y)$.

\vfill
$A$ --- самосопряженный оператор, если $A = A^*$.

Свойство: все собственные значения самосопряженного оператора вещественны и существует базис из собственных векторов.

\vfill

$A$ --- положительно определённый оператор, если $(A x, x) > 0$ для всех $x \neq 0$.

Свойство: все собственные значения положительно определённого самосопряжённого оператора вещественны и положительны.

\end{frame}


\begin{frame}[fragile]
\frametitle{Вспомогательные свойства}

\begin{theorem}[1]
Для любого самосопряженного положительно определенного оператора $A$ справедлива оценка
$$ \gamma_1 \cdot ||x||_2^2 \leqslant (Ax, x) \leqslant \gamma_2 \cdot ||x||_2^2 $$
где $\gamma_1$, $\gamma_2$ --- наименьшее и наибольшее собственные значения оператора.

\end{theorem}

\begin{proof}

Разложим произвольный вектор в базис из собственных векторов:
$$x = c_1 \vec{e}_1 + c_2 \vec{e}_2 + \ldots + c_n \vec{e}_n$$

$$||x||_2 = c_1^2 + c_2^2 + \ldots + c_n^2$$

$$(Ax, x) = \lambda c_1^2 + \lambda c_2^2 + \ldots + \lambda c_n^2$$

\end{proof}

\end{frame}


\begin{frame}[fragile]
\frametitle{Вспомогательные свойства}

\begin{theorem}[2]
Для любого положительно определенного оператора $A$ существует $\delta_1 > 0$ такое, что
$$\delta_1 \cdot ||x||_2^2 \leqslant (Ax, x) \leqslant \delta_2 \cdot ||x||_2^2$$

\end{theorem}

\begin{proof}

Если $A = A^*$, то $\delta = \gamma_1$. В противном случае введём самосопряжённый оператор:
$$ 
A_1 = \frac{A + A^*}{2} > 0 
$$

Тогда 
$$(A x, x) = (A^* x, x) = \frac12 ( (A x, x) + (A^* x, x) ) = (A_1 x, x)$$

\end{proof}

\end{frame}

\begin{frame}[fragile]
\frametitle{Энергетическая норма}

\vfill
Можно рассматривать энергетическую норму $||x||_A = (Ax, x)$, индуцированную оператором $A>0$.
\vfill
Получаем, что для последоветельности векторов $z_k$ верно:
$$||z_k||_A \to 0 \Leftrightarrow ||z_k||_2 \to 0$$
\vfill
\end{frame}

\begin{frame}[fragile]
\frametitle{Теорема Самарского}

\begin{theorem}[Теорема Самарского]
Пусть $A$ --- самосопряженная положительно определенная матрица и
$$ B > \frac{\tau}{2} A$$
Тогда при любом выборе начального приближения $x_0$ итерационный процесс, который определяется формулой
$$ B \frac{x_{k+1} - x_k}{\tau} + A x_k = f$$
сходится к решению системы $Ax = f$.

\end{theorem}
\end{frame}


\begin{frame}[fragile]
\frametitle{Теорема Самарского}
\begin{proof}

Пусть $x_k = x + z_k$, где $x$ --- точное решение, $z_k$ --- погрешность.


$$ B \frac{z_{k+1} - z_k}{\tau} + A z_k = 0$$

$$ z_{k+1} = z_k - \tau B^{-1} A z_k = (E - \tau B^{-1} A) z_k$$

Рассмотрим последовательность:
$$ 
c_{k} = (A z_k, z_k)
$$

Покажем, что:
\begin{enumerate}
\item $c_k > 0$;
\item $c_{k+1} < c_k$;
\item $c_k \to 0$.
\end{enumerate}
\end{proof}
\end{frame}


\begin{frame}[fragile]
\frametitle{Теорема Самарского}
\begin{proof}[]
1) По условию $A > 0$, значит $c_{k} = (A z_k, z_k) > 0$.

2) Монотонность:
$$c_{k+1} = (A z_{k + 1}, z_{k+1}) = (A z_k - \tau A B^{-1} A z_k, z_k - \tau B^{-1} A z_k) =  $$
$$ = c_k - 2 \tau ( A z_k, B^{-1} A z_k) + \tau^2 (A B^{-1} A z_k, B^{-1} A z_k)$$
Замена $\omega_k = B^{-1} A z_k$, то есть $B \omega_k = A z_k$

$$ c_{k+1} = c_k - 2 \tau (B \omega_k, \omega_k) + \tau^2 (A \omega_k, \omega_k) = c_k - 2 \tau \left((B - \frac{\tau}{2} A) \omega_k, \omega_k\right)$$
С учётом того, что $B - \frac{\tau}{2} A > 0$, получим, что $c_{k+1} < c_k$.
\end{proof}
\end{frame}


\begin{frame}[fragile]
\frametitle{Теорема Самарского}
\begin{proof}[]
3) Сходимость:

$$\left((B - \frac{\tau}{2} A) \omega_k, \omega_k\right) \to 0$$

Из эквивалентности энергетической нормы, индуцированной положительно определённым оператором $G = B - \frac{\tau}{2} A > 0 $, и нормы $||\cdot||_2$:
$$ (G \omega_k, \omega_k) \to 0 \Leftrightarrow (\omega_k, \omega_k) \to 0$$
получаем $||\omega_k||_2 \to 0$.

Оценим норму оператора
$$||x|| = ||H^{-1} H x|| \leqslant ||H^{-1}|| \cdot ||H x|| \Leftrightarrow ||H x|| \geqslant \frac{||x||}{||H^{-1}||}$$
для $H = B^{-1} A$:
$$||\omega_k||_2 = ||B^{-1} A z_k||_2 \geqslant \frac{||z_k||_2}{||A^{-1} B||_2}$$
Получили $||z_k||_2 \to 0$.
\end{proof}

\end{frame}


\end{document}
