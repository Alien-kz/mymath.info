\documentclass[10pt]{beamer}
\usetheme{Madrid}
\usepackage[T2A]{fontenc}
\usepackage[utf8]{inputenc}
\usepackage[russian]{babel}
\usepackage{amsmath}
\usepackage{amsfonts}
\usepackage{amssymb}
\usepackage{comment}
\usepackage[all]{xy}

%\newtheorem{theorem}{Теорема}

\author{Баев А.Ж.}
\title{Введение в численные методы. \\ Нелиныйные уравнения}
\institute{Казахстанский филиал МГУ} 
\date{16 февраля 2019}
%\subject{} 

\usepackage{tikz, pgfplots}
\usetikzlibrary{patterns} 
\usetikzlibrary{arrows, matrix, positioning, decorations.pathreplacing}

\usepackage{listings}
\lstset{language=python}
\lstset{basicstyle=\ttfamily}  
\lstset{keywordstyle=\color{blue}}
\lstset{frame=single}
\lstset{numbers=left}
\lstset{tabsize=4}



\usepackage{graphicx}

\begin{document}

\maketitle


\begin{frame}[fragile]
\frametitle{План на семестр}

\begin{enumerate}
\item СЛАУ (точные методы)
\item СЛАУ (итерационные методы)
\item \textbf{решение нелинейных уравнений}
\item интерполяция 
\item аппроксимация
\item интегрирование
\item дифференцирование
\end{enumerate}

\end{frame}


\begin{frame}[fragile]
\frametitle{Линейная алгебра}

Дана функция $f \in C[a, b]$. Найти решение:

$$f(x) = 0$$

на отрезке $[a, b]$. Считаем, что корень существует и единственный. Необходимо вычислить корень $x$ с заранее заданной точностью $\varepsilon$:

$$|x^{*} - x| < \varepsilon, f(x) = 0.$$

\end{frame}


\begin{frame}[fragile]
\frametitle{Метод деления отрезка пополам}

Этот метод также называется <<бинарный поиск>> или <<дихотомия>>.

Пусть
$$f(l) f(r) < 0,$$
если $l < x < r$, где $x$ --- корень. 

\vfill
Вычислим значение в середине отрезка $m = \frac{l+r}{2}$. 

\vfill

Если $f(m)$ и $f(r)$ одного знака, то $x^* \in [l; m]$.

\vfill

Если $f(m)$ и $f(r)$ разного знака, то $x^* \in [m; r]$.

\vfill
\end{frame}


\begin{frame}[fragile]
\frametitle{Метод деления отрезка пополам (пример)}

Найдем решение уравнения $x^2 - 2 = 0$ на отрезке $[0; 2]$ с точностью $\varepsilon = 0.2$.

\begin{minipage}{0.4\linewidth}
\begin{tikzpicture}
\begin{axis}
[
unit vector ratio*=1 1 1,
xlabel=x,
ylabel=y,
ytick={-3.0, -2.0, ..., 5.0},
xmin=0, xmax=3,
ymin=-2, ymax=2,
grid=major,
axis lines = middle
]
\addplot[mark=none,samples=100]{x*x - 2};
\draw[fill=black] (axis cs:0.0, 0.0) circle (0.3);
\draw[fill=black] (axis cs:1.0, 0.0) circle (0.3);
\draw[fill=black] (axis cs:1.25, 0.0) circle (0.3);
\draw[fill=white] (axis cs:1.5, 0.0) circle (0.3);
\draw[fill=white] (axis cs:2.0, 0.0) circle (0.3);
\end{axis}
\end{tikzpicture}
\end{minipage}
\begin{minipage}{0.51\linewidth}
Рассмотрим $f(x) = x^2 - 2$. Знаки на границах: $f(0) = -2 < 0$, $f(2) = 2 > 0$.

\begin{enumerate}

\item $l = 0$, $r = 2$. $f(m) = f(1) = -1$. Уменьшаем отрезок до $[1; 2]$.

\item $l = 1$, $r = 2$. $f(m) = f\left(\frac{3}{2}\right) = \frac{1}{4}$. Уменьшаем отрезок до $\left[1; \frac{3}{2} \right]$.

\item $l = 1$, $r = \frac{3}{2}$, $f(m) = f\left(\frac{5}{4}\right) = - \frac{7}{16}$. Уменьшаем отрезок до $\left[\frac{5}{4}; \frac{3}{2} \right]$.

\item $l = \frac{5}{4}$, $r = \frac{3}{2}$, $f(m) = f\left(\frac{11}{8}\right) = - \frac{7}{64}$. Уменьшаем отрезок до $\left[\frac{11}{8}; \frac{3}{2} \right]$.

\item $\left| \frac{7}{5} - \frac{4}{3} \right| = \frac{1}{15} < 0.2$. 

\item Ответ: $\frac{23}{16}$ с точностью до $0.2$.

\end{enumerate}
\end{minipage}

\end{frame}


\begin{frame}[fragile]
\frametitle{Метод деления отрезка пополам (код)}

\begin{lstlisting}
l := a
r := b
s := f(b)
while |r - l| > eps
    m := (r + l) / 2
    if s * f(m) > 0
        l := m
    else
        r := m
return (l + r) / 2
\end{lstlisting}
\end{frame}


\begin{frame}[fragile]
\frametitle{Метод деления отрезка пополам (сходимость)}

Количество итераций можно определить из неравенства:

$$ \frac{b - a}{2^n} \le \varepsilon .$$

Откуда легко найти количество итераций:

$$ n \ge \left[ \log_2 {\frac{b - a}{\varepsilon}} \right].$$

Скорость сходимости <<линейная>> с параметром $\frac{1}{2}$:

$$|x - x_{k+1}| \le \frac{1}{2} |x - x_{k}|.$$
\end{frame}


\begin{frame}[fragile]
\frametitle{Метод хорд}

Дана функцию $f(x) \in C^{2}[a, b]$. Функции $f'(x)$ и $f''(x)$ не изменяет знак на всем отрезке. 

\vfill

Проведем хорду через точки $(a; f(a))$ и $(b; f(b))$:
$$\frac{x-a}{b-a} = \frac{y-f(a)}{f(b)-f(a)}.$$

\vfill

Если $f'(x) f''(x) > 0$: 
$$
\begin{cases}
x_0 &= a,\\
x_{k+1} &= b - f(b) \frac{b-x_k}{f(b)-f(x_k)}
\end{cases}
$$

\vfill

Если $f'(x) f''(x) < 0$: 
$$
\begin{cases}
x_0 &= b,\\
x_{k+1} &= a - f(a) \frac{x_k-a}{f(x_k)-f(a)}
\end{cases}
$$


\end{frame}





\begin{frame}[fragile]
\frametitle{Метод хорд (пример)}
Найдем решение уравнения $x^2 - 2 = 0$ на отрезке $[0; 2]$ с точностью $\varepsilon = 0.2$. Так как $f'(x) f''(x) > 0$,  итерации слева направо ($x_0 = 0$) с фиксированным правым концом хорд. 

\begin{minipage}{0.4\linewidth}
\begin{tikzpicture}
\begin{axis}
[
unit vector ratio*=1 1 1,
xlabel=x,
ylabel=y,
ytick={-3.0, -2.0, ..., 5.0},
xmin=0, xmax=3,
ymin=-2, ymax=2,
grid=major,
axis lines = middle
]
\addplot[mark=none,samples=100]{x*x - 2};
\draw[red] (axis cs:0.0, -2.0) -- (axis cs:2.0, 2.0);
\draw[red] (axis cs:1.0, -1.0) -- (axis cs:2.0, 2.0);
\draw[fill=black] (axis cs:1.0, 0.0) circle (0.3);
\draw[fill=black] (axis cs:1.33, 0.0) circle (0.3);
\draw[fill=white] (axis cs:1.0, -1.0) circle (0.3);
\draw[fill=white] (axis cs:0.0, -2.0) circle (0.3);
\draw[fill=white] (axis cs:2.0, 2.0) circle (0.3);
\end{axis}
\end{tikzpicture}
\end{minipage}
\begin{minipage}{0.51\linewidth}

Вычисления:
$$x_{k+1} = 2 - 2 * \frac{2 - x_k}{2 - f(x_k)}.$$

\begin{enumerate}
\item $x_0 = 0$, $f(x_0) = -2$.

$x_{1} = 2 - 2 * \frac{2-x_0}{2-f(x_0)} = 1$.

\item $x_1 = 1$, $f(x_1) = -1$.

$x_{2} = 2 - 2 * \frac{2-x_1}{2-f(x_1)} = \frac{4}{3}$.

\item $x_1 = \frac{4}{3}$, $f(x_1) = -\frac{2}{9}$.

$x_{3} = 2 - 2 * \frac{2-x_2}{2-f(x_2)} = \frac{7}{5}$.

\item $\left| \frac{7}{5} - \frac{4}{3} \right| = \frac{1}{15} < 0.2$. 

\item Ответ: $\frac{7}{5}$ с точностью до $0.2$.
\end{enumerate}
\end{minipage}
\end{frame}





\begin{frame}[fragile]
\frametitle{Метод хорд (код)}
{\tt f(x)} --- исходную функцию,

{\tt f1(x)} --- производная (вычисленная аналитически),

{\tt f2(x)} --- вторая производная (вычисленная аналитически).

\begin{lstlisting}
m := (r + l) / 2
if f1(m) * f2(m) > 0
    xnew := a
    fb := f(b)
    do
        xold := xnew
        xnew := b - fb * (b - xold) / (fb - f(xold))
    while |xold - xnew| > eps
else
    xnew := b
    fa := f(a)
    do
        xold := xnew
        xnew := a - fa * (xold - a) / (f(xold) - fa)
    while |xold - xnew| > eps
\end{lstlisting}

\end{frame}





\begin{frame}[fragile]
\frametitle{Метод хорд (сходимость)}

Скорость сходимости <<линейная>>, то есть, существует такая $0 < L < 1$, что:

$$|x - x_{k+1}| \le L |x - x_{k}|.$$

\end{frame}





\begin{frame}[fragile]
\frametitle{Метод касательных}

Дана функцию $f(x) \in C^{2}[a, b]$. Функции $f'(x)$ и $f''(x)$ не изменяет знак на всем отрезке. 

\vfill

Проведем касательную к графику через точку $(x_k; f(x_k))$. Уравнение соответствующей прямой:

$$y = f'(x_k) (x - x_k) + f(x_k).$$

\vfill

Найдем точку пересечения хорды и с осью абсцисс ($y = 0$). Откуда легко получить формулу для вычисления корня методом касательных:

$$x_{k+1} = x_k - \frac{f(x_k)}{f'(x_k)}.$$

Если $f'(x) f''(x) > 0$ на всем отрезке $[a; b]$, то $x_0 = b$, а иначе $x_0 = a$. 


\end{frame}





\begin{frame}[fragile]
\frametitle{Метод касательных (пример)}

Найдем решение уравнения $x^2 - 2 = 0$ на отрезке $[0; 2]$ с точностью $\varepsilon = 0.2$. Так как $f'(x) = 2 x > 0$ и $f''(x) = 2 > 0$ итерации справа налево ($x_0 = 2$).

\begin{minipage}{0.4\linewidth}
\begin{tikzpicture}
\begin{axis}
[
unit vector ratio*=1 1 1,
xlabel=x,
ylabel=y,
ytick={-3.0, -2.0, ..., 5.0},
xmin=0, xmax=3,
ymin=-2, ymax=2,
grid=major,
axis lines = middle
]
\addplot[mark=none,samples=100]{x*x - 2};
\draw[red] (axis cs:1.50, 0.0) -- (axis cs:2.0, 2.0);
\draw[red] (axis cs:1.42, 0.0) -- (axis cs:1.5, 0.25);
\draw[fill=black] (axis cs:1.50, 0.0) circle (0.3);
\draw[fill=black] (axis cs:1.42, 0.0) circle (0.3);
\draw[fill=white] (axis cs:2.0, 2.0) circle (0.3);
\draw[fill=white] (axis cs:1.5, 0.25) circle (0.3);
\end{axis}
\end{tikzpicture}
\end{minipage}
\begin{minipage}{0.5\linewidth}
Вычисления:

$$x_{k+1} = x_k - \frac{f(x_k)}{f'(x_k)}.$$

\begin{enumerate}
\item $x_0 = 2$, $f(x_0) = 2$, $f'(x_0) = 4$.

$x_{1} = x_0 - \frac{f(x_0)}{f'(x_0)} = \frac{3}{2}$.

\item $x_1 = \frac{3}{2}$, $f(x_1) = \frac{1}{4}$, $f'(x_1) = 3$.

$x_{2} = x_1 - \frac{f(x_1)}{f'(x_1)} = \frac{17}{12} $.

\item $\left| \frac{17}{12} - \frac{3}{2} \right| = \frac{1}{12} < 0.2$

\item Ответ: $\frac{17}{12}$ с точностью до $0.2$.
\end{enumerate}
\end{minipage}


\end{frame}





\begin{frame}[fragile]
\frametitle{Метод касательных (код)}
\begin{lstlisting}
m := (r + l) / 2
if f1(m) * f2(m) > 0
    xnew := a
else
    xnew := b
do
    xold := xnew
    xnew := b - f(xold) / f1(xold)
while |xold - xnew| > eps
\end{lstlisting}


\end{frame}





\begin{frame}[fragile]
\frametitle{Метод касательных}

Скорость сходимости <<квадратичная>>, то есть, существует такой $0 < L < 1$, что:

$$|x - x_{k+1}| \le L |x - x_{k}|^2.$$

\end{frame}

\begin{frame}[fragile]
\frametitle{Метод касательных}

\begin{theorem}
Пусть в некоторой окрестности корня $x^*$ выполнены следующие условия:
$$ |f'(x)| \geqslant m_1 > 0$$
$$ |f''(x)| \leqslant M_2$$
$$ \frac{M_2}{2 m_1} |x_0 - x^*| \leqslant q < 1$$
где $x_0$ --- начальное приближение. Тогда итерационный метод Ньютона
$$x_{k+1} = x_k - \frac{f(x_k)}{f'(x_k)}$$
сходится и справедлива оценка:

$$|x_n - x^*| \le C q^{2^n}$$
\end{theorem}
\end{frame}


\begin{frame}[fragile]
\frametitle{Метод касательных}

\begin{proof}
Разложим $f(x)$ в ряд Тейлора в окрестности точки $x_k$
$$f(x) = f(x_k) + (x - x_k) f'(x_k) + \frac{(x - x_k)^2}{2} f''(\xi)$$

Подставим вместо $x$ корень уравнения $f(x^*) = 0$.
$$0 = f(x_k) + (x^* - x_k) f'(x_k) + \frac{(x^* - x_k)^2}{2} f''(\xi)$$

Подставим $x_{k+1}$.
$$x_{k+1} - x^{*} = \frac{(x^* - x_k)^2}{2 f'(x_k)} f''(\xi)$$
\end{proof}
\end{frame}


\begin{frame}[fragile]
\frametitle{Метод простой итерации}
Итерационный процесс
$$x_{k+1} = \varphi(x_k) $$
где $ \varphi(x) = x + \rho(x) f(x)$, $\rho(x)$ постоянного знака.

\begin{theorem}
Пусть $x^*$ --- корень уравнения $\varphi(x) = x$ и функция  
$\varphi(x)$ удовлетворяет на отрезке $[a, b]$ условию Липшица 
$$ |\varphi(x_1) - \varphi(x_2)| \leqslant L |x_1 - x_2| $$
с константой $L < 1$. Тогда при любом выборе $x_O$ итерационный процесс сходится к $x^*$.
\end{theorem}

\end{frame}



\begin{frame}[fragile]
\frametitle{Литература}
Подробно с методами можно ознакомить в книге \cite[стр. 138]{Kalitkin} и \cite[стр. 130]{Samarsky}


\begin{thebibliography}{99}

\bibitem{Kalitkin}
Калиткин Н.Н. Численные методы. - Спб.: БХВ-Петербург, 2014.

\bibitem{Samarsky}
Самарский А.А., Гулин А.В. Численные методы. - М.: Наука, 1989.

\end{thebibliography}
\end{frame}

\end{document}
