\documentclass[10pt]{beamer}
\usetheme{Madrid}
\usepackage[T2A]{fontenc}
\usepackage[utf8]{inputenc}
\usepackage[russian]{babel}
\usepackage{amsmath}
\usepackage{amsfonts}
\usepackage{amssymb}
\usepackage{comment}
\usepackage[all]{xy}

%\newtheorem{theorem}{Теорема}

\author{Баев А.Ж.}
\title{Введение в численные методы. \\ Интерполяция и аппроксимация}
\institute{Казахстанский филиал МГУ} 
\date{18 февраля 2019}
%\subject{} 

\usepackage{tikz, pgfplots}
\usetikzlibrary{patterns} 
\usetikzlibrary{arrows, matrix, positioning, decorations.pathreplacing}

\usepackage{listings}
\lstset{language=python}
\lstset{basicstyle=\ttfamily}  
\lstset{keywordstyle=\color{blue}}
\lstset{frame=single}
\lstset{numbers=left}
\lstset{tabsize=4}



\usepackage{graphicx}

\begin{document}

\maketitle


\begin{frame}[fragile]
\frametitle{План на семестр}

\begin{enumerate}
\item СЛАУ (точные методы)
\item СЛАУ (итерационные методы)
\item решение нелинейных уравнений
\item \textbf{интерполяция}
\item аппроксимация
\item интегрирование
\item дифференцирование
\end{enumerate}

\end{frame}


\begin{frame}[fragile]
\frametitle{Интерполяция}

Интерполяция --- способ нахождения промежуточных значений величины по имеющемуся дискретному набору известных значений. 

\vfill
Дана сетка порядка $n$: 
$$x_0 < x_1 < \ldots < x_n$$

\vfill
Дан набор измерений:
$$y_i = f(x_i)$$

\vfill
Необходимо восстановить значение функции $f$ в другом наборе точек $z_0 < z_1 < \ldots < z_m$,
\end{frame}


\begin{frame}[fragile]
\frametitle{Кусочно-линейная интерполяция}

$$
f(x) =
\begin{cases}
\frac{y_{i+1} - y_i}{x_{i+1} - x_i} \left( x - x_i \right) + y_i, x \in [x_i; x_{i+1})
\end{cases}
$$

\vfill

\begin{center}
\begin{tikzpicture}
\begin{axis}
[
unit vector ratio*=1 1 1,
xlabel=x,
ylabel=y,
xmin=0, xmax=6,
ymin=0, ymax=5,
grid=major,
axis lines = middle
]
\addplot coordinates { (2, 1) (3, 4) (4, 2) (5, 5) };

\end{axis}
\end{tikzpicture}
\end{center}
\end{frame}

\begin{frame}[fragile]
\frametitle{Кусочно-линейная интерполяция}

Недостатки: недифференцируема (большинство физический и экономических процессов имеют более гладкий характер).

\end{frame}



\begin{frame}[fragile]
\frametitle{Кусочно-линейная интерполяция}

Для восстановления зависимости в точке $z$ необходимо определить $x_i \leqslant z < x_{i+1}$. 

\vfill

Асимптотика для восстановления зависимости в $m$ точках: $O(m \log{n})$. 


\end{frame}

\begin{frame}[fragile]
\frametitle{Интерполяционный многочлен Лагранжа}

Строим $P(x)$ --- многочлен $n$-й степени, который будет проходить через заданные точки $(x_i, y_i)$.


\begin{center}
\begin{tikzpicture}
\begin{axis}
[
unit vector ratio*=1 1 1,
xlabel=x,
ylabel=y,
xmin=0, xmax=6,
ymin=0, ymax=5,
grid=major,
axis lines = middle
]

\addplot[red, domain=2:5] { 
1 * (x-3) * (x-4) * (x-5) / (2-3) / (2-4) / (2-5)
+
4 * (x-2) * (x-4) * (x-5) / (3-2) / (3-4) / (3-5)
+
2 * (x-2) * (x-3) * (x-5) / (4-2) / (4-3) / (4-5)
+
5 * (x-2) * (x-3) * (x-4) / (5-2) / (5-3) / (5-4)
};

\addplot[only marks]  coordinates { (2, 1) (3, 4) (4, 2) (5, 5) };

\end{axis}
\end{tikzpicture}
\end{center}
\end{frame}

\begin{frame}[fragile]
\frametitle{Интерполяционный многочлен Лагранжа}
Введём базис из многочленов $\varphi_i(x)$ таких, что 

$$
\varphi_i(x) = \frac{\prod_{j \ne i} (x - x_j)}{\prod_{j \ne i} (x_i - x_j)}
$$

Основное свойство: $\varphi_i(x_j) = \delta_{ij}$.

\begin{center}
\begin{tikzpicture}[scale=0.7]
\begin{axis}
[
%unit vector ratio*=1 1 1,
xlabel=x,
ylabel=y,
xmin=0, xmax=6,
ymin=-0.5, ymax=1.5,
grid=major,
axis lines = middle
]
\addplot[blue,domain=2:5] { (x-3) * (x-4) * (x-5) / (2-3) / (2-4) / (2-5)};
\addlegendentry{$\varphi_0$}


\addplot[green, domain=2:5] { (x-2) * (x-4) * (x-5) / (3-2) / (3-4) / (3-5)};
\addlegendentry{$\varphi_1$}


\addplot[only marks] coordinates { (2, 1) (3, 1) };
\addplot[only marks] coordinates { (2, 0) (3, 0) (4, 0) (5, 0) };

\end{axis}
\end{tikzpicture}
\begin{tikzpicture}[scale=0.7]
\begin{axis}
[
%unit vector ratio*=1 1 1,
xlabel=x,
ylabel=y,
xmin=0, xmax=6,
ymin=-0.5, ymax=1.5,
grid=major,
axis lines = middle
]

\addplot[red, domain=2:5] { (x-2) * (x-3) * (x-5) / (4-2) / (4-3) / (4-5)};
\addlegendentry{$\varphi_2$}

\addplot[black, domain=2:5] { (x-2) * (x-3) * (x-4) / (5-2) / (5-3) / (5-4)};
\addlegendentry{$\varphi_3$}

\addplot[only marks] coordinates { (4, 1) (5, 1) };
\addplot[only marks] coordinates { (2, 0) (3, 0) (4, 0) (5, 0) };

\end{axis}
\end{tikzpicture}
\end{center}

$$
P(x) = \sum_{i=0}^{n} y_i \varphi_i(x)
$$

\end{frame}


\begin{frame}[fragile]
\frametitle{Интерполяционный многочлен Лагранжа}

Недостатки: большая вариация для некоторых классов функций $f(x)$.


\end{frame}

\begin{frame}[fragile]
\frametitle{Интерполяционный многочлен Лагранжа}

\begin{lstlisting}
x = np.zeros(n + 1)
y = np.zeros(n + 1)

...

def phi(i, z):
	p := 1
	for j := 0 .. n
		p := p * (z - x[j]) / (x[i] - x[j])
	return p

def P(z):
	s := 0
	for i := 0 .. n
		s := s + y[i] * phi(i, z)
	return s
\end{lstlisting}
\end{frame}

\begin{frame}[fragile]
\frametitle{Интерполяционный многочлен Лагранжа}

Для восстановления зависимости в точке $z$ необходимо вычислить значения для каждого $\varphi_i(z)$. 

\vfill

Асимптотика для восстановления зависимости в $m$ точках: $O(m n^2)$. 


\vfill

Асимптотику можно улучшить. Подумайте как!
\end{frame}

\begin{frame}[fragile]
\frametitle{Сплайновая интерполяция}

$f(x)$ --- сплайновая интерполяция, то есть на $i$-м отрезке $[x_{i}; x_{i+1}]$ равна:

$$ P_i(x) = A_i * (x - x_{i})^3 + B_i * (x - x_{i})^2 + C_i * (x - x_{i}) + D_i, i = \overline{0,n-1} $$

\begin{center}
\begin{tikzpicture}
\begin{axis}
[
unit vector ratio*=1 1 1,
xlabel=x,
ylabel=y,
xmin=0, xmax=6,
ymin=0, ymax=5,
grid=major,
axis lines = middle
]

\addplot[red, domain=2:3] { 
-5/3*(x-2)*(x-2)*(x-2) + 0*(x-2)*(x-2) + 14/3*(x-2) + 1 
};
\addplot[red, domain=3:4] { 
10/3*(x-3)*(x-3)*(x-3) - 5*(x-3)*(x-3) - 1/3*(x-3) + 4
};
\addplot[red, domain=4:5] { 
-5/3*(x-4)*(x-4)*(x-4)+ 5*(x-4)*(x-4) -1/3*(x-4) + 2
};

\addplot[only marks]  coordinates { (2, 1) (3, 4) (4, 2) (5, 5) };

\end{axis}
\end{tikzpicture}
\end{center}

\end{frame}

\begin{frame}[fragile]
\frametitle{Сплайновая интерполяция}

Для более простых соотношений считаем, что сетка $x_i$ --- равномерная. То есть $x_i = a + i * h$, где $h = \frac{x_n - x_0}{n}$. Строим аппроксимацию достаточно гладкую в узлах (дважды непрерывно дифференцируемая).


1. Значение $i$-го cплайна на левом конце равно $y_{i}$:

$$ P_i(x_{i}) = y_{i}, i = \overline{0, n-1} $$

2. Значение $i$-го cплайна на правом конце равно $y_{i+1}$:

$$ P_i(x_{i+1}) = y_{i+1}, i = \overline{1, n} $$

3. Первые производные сплайнов непрерывны в узлах:

$$ P_{i}'(x_{i+1}) = P_{i+1}'(x_{i+1}), i = \overline{0, n-2} $$

4. Вторые производные сплайнов непрерывны в узлах:

$$ P_{i}''(x_{i+1}) = P_{i+1}''(x_{i+1}), i = \overline{0, n-2} $$

5. Края свободны:

$$ P_0''(x_{0}) = 0, P_{n-1}''(x_{n}) = 0$$

\end{frame}

\begin{frame}[fragile]
\frametitle{Сплайновая интерполяция}
$$
\begin{cases}
D_i = y_{i}, 										& i = \overline{0, n-1} \\
A_i h^3 + B_i h^2 + C_i h + y_{i} = y_{i+1},		& i = \overline{0, n-1} \\
3 A_i h^2 + 2 B_i h + C_i = C_{i+1}, 				& i = \overline{0, n-2} \\
6 A_i h + 2 B_i = 2 B_{i+1},						& i = \overline{0, n-2} \\
2 B_0 = 0 											&\\
6 A_{n-1} h + 2 B_{n-1} = 0 						&\\
\end{cases}
$$

Ответ для $D_i$:
$$
	D_i = y_{i}, i = \overline{0, n-1}
$$

Выразим $A_i$ через $B_i$ и введем фиктивный элемент $B_{n} = 0$:
$$
	A_i = \frac{B_{i+1} - B_i}{3h},  i = \overline{0, n-1}
$$
\end{frame}

\begin{frame}[fragile]
\frametitle{Сплайновая интерполяция}
Подставим $A_i$ в оставшиеся неразрешенные соотношения:
$$
\begin{cases}
(B_{i+1} - B_i) \frac{h^2}{3} + B_i h^2 + C_i h + y_{i} = y_{i+1},	& i = \overline{0, n-1} \\
(B_{i+1} - B_i) h + 2 B_i  h + C_i = C_{i+1}, 						& i = \overline{0, n-2} \\
B_0 = 0 																&\\
B_n = 0 																&\\
\end{cases}
$$

Выразим $C_i$ через $B_i$:

$$
C_i = \frac{y_{i+1} - y_{i}}{h} - (B_{i+1} + 2 B_i) \frac{h}{3}, i = \overline{0, n-1}
$$ 

Получим замкнутую систему для $B_i$:
$$B_i +  4 B_{i+1} + B_{i+2} = 3 \frac{y_{i} - 2 y_{i+1} + y_{i+2}}{h^2} , i = \overline{0, n-2}$$
\end{frame}

\begin{frame}[fragile]
\frametitle{Сплайновая интерполяция}
$$
\left\lbrace
\begin{aligned}
&B_0 = B_n = 0\\
&B_{i-1} +  4 B_{i} + B_{i+1} = 3 \frac{y_{i-1} - 2 y_{i} + y_{i+1}}{h^2} , i = \overline{1, n-1}\\
\end{aligned}
\right.
$$

Сводится к СЛАУ $(n+1)$-го порядка с трехдиагональной матрицей:
$$
\begin{pmatrix}
   1 & 0 & 0 & 0 & &\hdots& 0 & 0 & 0 \\
   1 & 4 & 1 & 0 & &\hdots& 0 & 0 & 0 \\
   0 & 1 & 4 & 1 & &\hdots& 0 & 0 & 0 \\
   0 & 0 & 1 & 4 & &\hdots& 0 & 0 & 0 \\
  \vdots& \vdots& \vdots& \vdots& \ddots& \vdots& \vdots& \vdots \\
   0 & 0 & 0 & 0 & &\hdots&  & 4 & 1 & 0 \\
   0 & 0 & 0 & 0 & &\hdots&  & 1 & 4 & 1 \\
   0 & 0 & 0 & 0 & &\hdots&  & 0 & 0 & 1 \\
\end{pmatrix}
\begin{pmatrix}
B_0\\
B_1\\
B_2\\
B_3\\
\vdots\\
B_{n-2}\\
B_{n-1}\\
B_n\\
\end{pmatrix}
=
\begin{pmatrix}
0\\
3 {y}_{xx, 1} \\
3 {y}_{xx, 2} \\
3 {y}_{xx, 3} \\
\vdots\\
3 {y}_{xx, n-2} \\
3 {y}_{xx, n-1} \\
0\\
\end{pmatrix}
$$
где 
${y}_{xx, i} = \frac{y_{i-1} - 2 y_{i} + y_{i+1}}{h^2}$

\end{frame}

\begin{frame}[fragile]
\frametitle{Сплайновая интерполяция}
По диагоналям:
$$ a = (0, 1, 1, 1, ..., 1, 1, 0) $$
$$ b = (1, 4, 4, 4, ..., 4, 4, 1) $$
$$ c = (0, 1, 1, 1, ..., 1, 1, 0) $$
$$ f = 3 \cdot (0, {y}_{xx, 1}, {y}_{xx, 2}, {y}_{xx, 3}, ..., {y}_{xx, n-2}, {y}_{xx, n-1}, 0) $$

Методом прогонки можно решить систему $A s = f$ и получить вектор коэффициентов $B$. Остальные коэффициенты определяются по формулам:
$$
\begin{cases}
B_i &= s_i \\
A_i &= \frac{B_{i+1} - B_i}{3 h}\\
C_i &= \frac{y_{i+1} - y_i}{h} - \left(B_{i+1} + 2 B_{i}\right) \frac{h}{3}\\
D_i &= y_i\\
\end{cases}
$$
\end{frame}

\begin{frame}[fragile]
\frametitle{Сплайновая интерполяция}

Дана сетка на $n=3$ отрезка со значениями в узлах: $(2, 1)$, $(3, 4)$, $(4, 2)$, $(5, 5)$.

Шаг сетки: $h = 1$. Посчитаем правые части для СЛАУ:
$$3 y_{xx, 1} = 3 (y_{0} - 2 y_{1} + y_{2}) /{h^2} = 3 * (1 - 2 * 4 + 2)  = - 15$$
$$3 y_{xx, 2} = 3 (y_{1} - 2 y_{2} + y_{3}) /{h^2} = 3 * (4 - 2 * 2 + 5)  = 15$$

Решаем методом прогонки:
$$
\begin{pmatrix}
  1 & 0 & 0 & 0 \\
  1 & 4 & 1 & 0 \\
  0 & 1 & 4 & 1 \\
  0 & 0 & 0 & 1 \\
\end{pmatrix}
\begin{pmatrix}
B_0\\
B_1\\
B_2\\
B_3\\
\end{pmatrix}
=
\begin{pmatrix}
0\\
-15 \\
15 \\
0 \\
\end{pmatrix}
$$

Решение СЛАУ: $B = (0, -5, 5, 0)$. 

\end{frame}

\begin{frame}[fragile]
\frametitle{Сплайновая интерполяция}

Коэффициенты $B$:
$$ B_0 = 0, B_1 = -5,  B_2 = 5$$

Коэффициенты $A$:
$$ A_0 = \frac{B_{1} - B_{0}}{3 h} = -\frac53,  A_1 = \frac{B_{2} - B_{1}}{3 h} = \frac{10}{3},  A_2 = \frac{B_{3} - B_{2}}{3 h} = -\frac{5}{3}$$

Коэффициенты $C$:
$$ C_0 = \frac{y_1 - y_0}{h} - \frac{(B_1 + 2 B_0) h}{3}= \frac{14}{3} ,  C_1 = \frac{y_2 - y_1}{h} - \frac{(B_2 + 2 B_1) h}{3} = -\frac13$$
$$C_2 = \frac{y_3 - y_2}{h} - \frac{(B_3 + 2 B_2) h}{3} = -\frac13$$

Коэффициенты $D$:
$$ D_0 = y_0 = 1,  D_1 = y_1 = 4,  D_2 = y_2 = 2$$

Сплайновая интерполяция:
$$
P(x) = 
\begin{cases}
-\frac{5}{3}(x-2)^3+0(x-2)^2 + \frac{14}{3}(x-2)+1, & x \in \left[2;3\right) \\
\frac{10}{3}\left(x-3\right)^3 - 5\left(x-3\right)^2 -\frac13\left(x-3\right) + 4,& x \in \left[3;4\right)	\\
-\frac{5}{3}\left(x-4\right)^3 + 5\left(x-4\right)^2 -\frac13\left(x-4\right) + 2,& x \in \left[4;5\right]	\\
\end{cases}
$$
\end{frame}

\begin{frame}[fragile]
\frametitle{Сплайновая интерполяция}

\begin{lstlisting}
def generateSpline(x, y)
	n = x.shape[0] - 1
    h = (x[n] - x[0]) / n

    a = np.array([0] + [1] * (n - 1) + [0])
    b = np.array([1] + [4] * (n - 1) + [1])
    c = np.array([0] + [1] * (n - 1) + [0])
    f = np.zeros(n + 1)
    for i in range(1, n):
        f[i] := 3 * (y[i-1] - 2 * y[i] + y[i+1]) / h**2
    s = sweep(n+1, a, b, c, f)

    for i in range(0, n):
        B[i] := s[i]
        A[i] := (B[i+1] - B[i]) / (3 * h)
        C[i] := (y[i+1] - y[i]) / h - \
                (B[i+1] + 2 * B[i]) * h / 3
        D[i] := y[i]
    return A, B, C, D
\end{lstlisting}


\end{frame}

\begin{frame}[fragile]
\frametitle{Интерполяционный многочлен Лагранжа}

Для определения коэффициентов необходимо решить прогонкой СЛАУ $O(n)$. 

\vfill

Асимптотика для восстановления зависимости в $m$ точках: $O(m n)$. 

\end{frame}

\begin{frame}[fragile]
\frametitle{Литература}
Подробнее в \cite[стр. 44]{Kalitkin} и \cite[стр. 65]{Samarsky}.



\begin{thebibliography}{99}

\bibitem{Kalitkin}
Калиткин Н.Н. Численные методы. - Спб.: БХВ-Петербург, 2014.

\bibitem{Samarsky}
Самарский А.А., Гулин А.В. Численные методы. - М.: Наука, 1989.

\end{thebibliography}
\end{frame}

\end{document}
