\documentclass[10pt]{beamer}
\usetheme{Madrid}
\usepackage[T2A]{fontenc}
\usepackage[utf8]{inputenc}
\usepackage[russian]{babel}
\usepackage{amsmath}
\usepackage{amsfonts}
\usepackage{amssymb}
\usepackage{comment}
\usepackage[all]{xy}

%\newtheorem{theorem}{Теорема}

\author{Баев А.Ж.}
\title{Введение в численные методы. \\ Интерполяция и аппроксимация}
\institute{Казахстанский филиал МГУ} 
\date{18 февраля 2019}
%\subject{} 

\usepackage{tikz, pgfplots}
\usetikzlibrary{patterns} 
\usetikzlibrary{arrows, matrix, positioning, decorations.pathreplacing}

\usepackage{listings}
\lstset{language=python}
\lstset{basicstyle=\ttfamily}  
\lstset{keywordstyle=\color{blue}}
\lstset{frame=single}
\lstset{numbers=left}
\lstset{tabsize=4}



\usepackage{graphicx}

\begin{document}

\maketitle


\begin{frame}[fragile]
\frametitle{План на семестр}

\begin{enumerate}
\item СЛАУ (точные методы)
\item СЛАУ (итерационные методы)
\item решение нелинейных уравнений
\item интерполяция
\item \textbf{аппроксимация}
\item интегрирование
\item дифференцирование
\end{enumerate}

\end{frame}


\begin{frame}[fragile]
\frametitle{Аппроксимация}

Аппроксимация --- математический метод, состоящий в замене одних математических объектов другими, в том или ином смысле близкими к исходным. 

\vfill
Дана сетка порядка $n$: 
$$x_0 < x_1 < \ldots < x_n$$

\vfill
Дан набор измерений:
$$y_i = f(x_i)$$

\vfill
Необходимо восстановить вид функции $f$.
\end{frame}


\begin{frame}[fragile]
\frametitle{Линейное нормирование пространство}

Линейное пространство $L$ нормировано, если каждому элементу $v \in L$ поставлено в соответствие вещественное число $||v||$ такое, что:
\begin{enumerate}
\item $||v|| \geqslant 0$;
\item $||\alpha v|| = |\alpha| \cdot ||v||$;
\item $||v_1 + v_2|| \leqslant ||v_1|| + ||v_2||$.
\end{enumerate}

\vfill
Линейное пространство $L$ строго нормировано, если из равенства $||v_1 + v_2|| = ||v_1|| + ||v_2||$ следует, что $v_1 = \alpha v_2$.
\end{frame}

\begin{frame}[fragile]
\frametitle{Задача наилучшего приближения}

Пусть имеется некоторое линейное нормированное пространство $L$, элемент $f \in L$ и набор линейно--независимых элементов $\varphi_i \in L$, где $i = 1, \dots, n$. Требует найти наилучшее линейное приближение по норме, то есть:
$$g = \underset{c_i}{Argmin} ||f - \sum_{i=1}^{n} c_i \varphi_i|| .$$

\end{frame}

\begin{frame}[fragile]
\frametitle{Задача наилучшего приближения}

\begin{theorem}
В любом нормированном пространстве существует элемент наилучшего приближения.
\end{theorem}
\end{frame}

\begin{frame}[fragile]
\frametitle{Задача наилучшего приближения}

\begin{proof}
1) Функционал
$$G_f(c_1, ..., c_n) = ||f - \sum_{i=1}^{n} c_i \varphi_i||$$
является непрерывной функцией переменных $c_i$ (по неравенству треугольника).

2) Рассмотрим функционал на $f = 0$ на единичном шаре $a_1^2 + \cdot + a_n^2 = 1$. $G_0(a_1, ..., a_n)$ достигает нижней грани $\tilde{G}_0$ в некоторой точке $(\tilde{a}_1, ..., \tilde{a}_n)$.

3) Рассмотрим функционал на произвольной функции $f$ в шаре радиуса $\alpha$. $G_f(c_1, ..., c_n)$ достигает нижней грани $\tilde{G}_f$ в некоторой точке $(\hat{c}_1, ..., \hat{c}_n)$:
$$ G_f(\hat{c}) \leqslant G_f(0) = ||f|| $$

4) Рассмотрим функционал на произвольной функции $f$ вне шара радиуса $\alpha$.
$$ G_f(c) = ||c_1 \varphi_1 + ... + c_n \varphi_n - f || \geqslant ||c_1 \varphi_1 + ... + c_n \varphi_n || - || f || = $$
$$ = |c| \cdot ||a_1 \varphi_1 + ... + a_n \varphi_n || - ||f|| \geqslant \alpha \tilde{G}_0 - ||f|| \geqslant \{ \text{ Определим $\alpha$ } \} \geqslant ||f|| \geqslant G_f(\hat{c})$$
\end{proof}
\end{frame}

\begin{frame}[fragile]
\frametitle{Задача наилучшего приближения}

\begin{theorem}
В любом строго нормированном пространстве элемент наилучшего приближения единственный.
\end{theorem}

\begin{proof}
Пусть $\varphi_1 \neq \varphi_2$ --- различные элементы наилучшего приближения. Рассмотрим элемент $\frac12 (\varphi_1 + \varphi_2)$. Тогда 
$$\Delta \leqslant ||\frac12 (f - \varphi_1)  + \frac12 (f - \varphi_2)|| \leqslant ||\frac12 (f - \varphi_1)|| + ||\frac12 (f - \varphi_2)|| = \Delta$$

Из строгой нормированности
$$\frac12 (f - \varphi_1) = \alpha \frac12 (f - \varphi_2)$$

Получаем, что $\alpha = 1$ и $\varphi_1 = \varphi_2$.
\end{proof}
\end{frame}

\begin{frame}[fragile]
\frametitle{Гильбертово пространство}

Линейной пространство со скалярным произведением.

\vfill

$$||f||^2 = (f, f)$$

\end{frame}

\begin{frame}[fragile]
\frametitle{Гильбертово пространство}

\begin{multline*}
F(c_0, c_1, ..., c_n) = (f - \sum_{i=0}^{n} c_i \varphi_i, f - \sum_{j=0}^{n} c_j \varphi_j) = \\
= (f, f) - 2 \sum_{i=0}^{n} c_i (\varphi_i, f) + \sum_{i=0}^{n} \sum_{j=0}^{n} c_i c_j (\varphi_i, \varphi_j)
\end{multline*}

Производные по $c_k$ должны обнулиться:
$$ - 2 (\varphi_k, f) + 2 \sum_{j=0}^{n} c_j (\varphi_k, \varphi_j) = 0$$.

Получаем систему с матрицей Грамма:
$$
\begin{pmatrix}
(\varphi_0, \varphi_0) & (\varphi_0, \varphi_1) & ... & (\varphi_0, \varphi_n) \\
(\varphi_1, \varphi_0) & (\varphi_1, \varphi_1) & ... & (\varphi_1, \varphi_n) \\
...						& ...					  & ... & ...					  \\
(\varphi_n, \varphi_0) & (\varphi_n, \varphi_1) & ... & (\varphi_n, \varphi_n) \\
\end{pmatrix}
\begin{pmatrix}
c_0 \\
c_1 \\
... \\
c_n \\
\end{pmatrix}
=
\begin{pmatrix}
(\varphi_0, f) \\
(\varphi_1, f) \\
... \\
(\varphi_n, f) \\
\end{pmatrix}
$$


\end{frame}




\begin{frame}[fragile]
\frametitle{Пример 1}

Рассмотрим кусочно--линейную аппроксимацию $f(x) \in C[0, 1]$ по норме, согласованной со скалярным произведением:

$$(f, g) = \int_{0}^{1} f(x) g(x) dx .$$
\end{frame}




\begin{frame}[fragile]
\frametitle{Пример 1}

В качестве функции для ядра выберем:

$$
\varphi(x) = 
\left\{
\begin{aligned}
& 1 - x ,&& x \in [0; 1]\\ 
& 1 + x ,&& x \in [-1; 0)\\
& 0 ,&& x \not\in [-1; 1]\\
\end{aligned}
\right.
$$

\begin{center}
\begin{tikzpicture}
\begin{axis}
[
unit vector ratio*=1 1 1,
xlabel=x,
ylabel=y,
xmin=-3, xmax=3,
ymin=0, ymax=2,
ytick={0.0, 1.0, 2.0},
grid=major,
axis lines = middle
]


\addplot coordinates { (-2, 0) (-1, 0) (0, 1) (1, 0) (2, 0) };

\end{axis}
\end{tikzpicture}
\end{center}
\end{frame}




\begin{frame}[fragile]
\frametitle{Пример 1}
Базисные функции построим на равномерной сетке с шагом $h = \frac{1}{n}$.

\vfill
Данную базисную функцию сожмем до интервала $[-h; h]$ и сдвинем на $x_i$ для всех $i = 0, ..., n$. Получим базис:
\begin{equation*}
\varphi_i(x) = \varphi \left( \frac{x - x_i}{h} \right)
\end{equation*}
 
 \begin{center}
\begin{tikzpicture}[scale=0.7]
\begin{axis}
[
%unit vector ratio*=1 1 1,
xlabel=x,
ylabel=y,
xmin=0.0, xmax=1.0,
ymin=-0.5, ymax=1.5,
grid=major,
axis lines = middle
]
\addplot coordinates { (0, 1) (1/3, 0) (1, 0) };
\addlegendentry{$\varphi_0$}


\addplot coordinates { (0, 0) (1/3, 1) (2/3, 0) (1, 0) };
\addlegendentry{$\varphi_1$}


\end{axis}
\end{tikzpicture}
\begin{tikzpicture}[scale=0.7]
\begin{axis}
[
%unit vector ratio*=1 1 1,
xlabel=x,
ylabel=y,
xmin=0.0, xmax=1.0,
ymin=-0.5, ymax=1.5,
grid=major,
axis lines = middle
]

\addplot coordinates { (0, 0) (1/3, 0) (2/3, 1) (1, 0) };
\addlegendentry{$\varphi_0$}


\addplot coordinates { (0, 0) (2/3, 0) (1, 1) };
\addlegendentry{$\varphi_1$}

\end{axis}
\end{tikzpicture}
\end{center}

\end{frame}




\begin{frame}[fragile]
\frametitle{Пример 1}
Вычислим элементы матрицы Грамма. Легко увидеть, что: 
$$\int_{-1}^{1} \varphi^2(x) dx = 2 \int_{0}^{1} \varphi^2(x) dx = 2 \int_{0}^{1} x^2 dx = \frac{2}{3} .$$

Значит диагональные элементы равны ($i = 1, ..., n-1$):
$$(\varphi_i(x), \varphi_i(x)) = \frac{2h}{3} .$$

Для первого и последнего элемента ответ будет в 2 раза меньше:
$$(\varphi_0(x), \varphi_0(x)) = (\varphi_n(x), \varphi_n(x)) = \frac{h}{3} .$$

Далее посмотрим скалярное произведение 2 подряд идущих базисных функций (над и под главной диагональю):
$$(\varphi_i(x), \varphi_{i+1}(x)) = h \int_{0}^{1} (1 - x) x dx = \frac{h}{6} .$$

Скалярное произведение элементов при $|i-j|>1$ будет нулевое:
$$(\varphi_i(x), \varphi_j(x)) = 0.$$
\end{frame}




\begin{frame}[fragile]
\frametitle{Пример 1}
В итоге, необходимо найти решение СЛАУ (все строки матрицы умножены на $\frac{6}{h}$) с трехдиагональной матрицей:
\begin{equation}
\begin{pmatrix}
2 & 1 & 0 & \cdots & 0 & 0 & 0 \\
1 & 4 & 1 & \cdots & 0 & 0 & 0 \\
0 & 1 & 4 & \cdots & 0 & 0 & 0 \\
\cdots & \cdots & \cdots & \cdots & \cdots & \cdots & \cdots\\
0 & 0 & 0 & \cdots & 4 & 1 & 0\\
0 & 0 & 0 & \cdots & 1 & 4 & 1\\
0 & 0 & 0 & \cdots & 0 & 1 & 2\\
\end{pmatrix}
\begin{pmatrix}
c_0 \\
c_1 \\
c_2 \\
\cdots \\
c_{n-2} \\
c_{n-1} \\
c_{n} \\
\end{pmatrix}
=
\frac{6}{h} 
\begin{pmatrix}
(\varphi_0, f) \\
(\varphi_1, f) \\
(\varphi_2, f) \\
\cdots \\
(\varphi_{n-2}, f) \\
(\varphi_{n-1}, f) \\
(\varphi_n, f) \\
\end{pmatrix}
\end{equation}

Интегралы для правых частей надо вычислять численно. Причем для первой и последней компоненты с особенными границами:
$$(\varphi_i, f) = 
\begin{cases}
\int_{x_{i}}^{x_{i+1}} \varphi_i(x) f(x) dx, i = 0\\
\int_{x_{i-1}}^{x_{i+1}} \varphi_i(x) f(x) dx, i = 1 .. (n-1)\\
\int_{x_{i-1}}^{x_{i}} \varphi_i(x) f(x) dx, i = n
\end{cases}
.$$
\end{frame}

\begin{frame}[fragile]
\frametitle{Пример 1}
Дана сетка размером $n=3$ на отрезке $[0;1]$. Построить кусочно--линейную аппроксимацию функции 
$$f(x) = 54 x^2$$
 методом наименьших квадратов с минимальным интегральным отклонением.

$$
\varphi_0 (x) = \varphi \left(\frac{x - 0}{1/3} \right) =
\begin{cases}
1 - 3x &, x \in \left[0, \frac{1}{3} \right]\\
0      &, x \not \in \left[0, \frac{1}{3} \right]\\
\end{cases}
$$

$$
\varphi_1 (x) = \varphi \left(\frac{x - 1/3}{1/3} \right) =
\begin{cases}
0 + 3x &, x \in \left[0, \frac{1}{3} \right]\\
2 - 3x &, x \in \left[\frac{1}{3}, \frac{2}{3} \right]\\
0      &, x \not \in \left[0, \frac{2}{3} \right]\\
\end{cases}
$$

$$
\varphi_2 (x) = \varphi \left(\frac{x - 2/3}{1/3} \right) =
\begin{cases}
-1 + 3x &, x \in \left[\frac{1}{3}, \frac{2}{3} \right]\\
3 - 3x &, x \in \left[\frac{2}{3}, 1 \right]\\
0      &, x \not \in \left[\frac{1}{3}, 1 \right]\\
\end{cases}
$$

$$
\varphi_3 (x) = \varphi \left(\frac{x - 1}{1/3} \right) =
\begin{cases}
-2 + 3x &, x \in \left[ \frac{2}{3}, 1 \right]\\
0      &, x \not \in \left[\frac{2}{3}, 1 \right]\\
\end{cases}
$$
\end{frame}

\begin{frame}[fragile]
\frametitle{Пример 1}
Вычислим проекции функции $f(x) = 54 x^2$ на базисные функции $\varphi_i(x)$:
\begin{equation*}
\begin{aligned}
&(f, \varphi_0) = \int_{\frac{0}{3}}^{\frac{1}{3}} 54 x^2 (1 - 3x) dx = \frac{1}{6} \approx 0.16667\\
&(f, \varphi_1) = \int_{\frac{0}{3}}^{\frac{1}{3}} 54 x^2 (0 + 3x) dx + \int_{\frac{1}{3}}^{\frac{2}{3}} 54 x^2 (2 - 3x) dx= \frac{7}{3} \approx 2.33333\\
&(f, \varphi_2) = \int_{\frac{1}{3}}^{\frac{2}{3}} 54 x^2 (-1 + 3x) dx + \int_{\frac{2}{3}}^{\frac{3}{3}} 54 x^2 (3 - 3x) dx= \frac{25}{3} \approx 8.33333\\
&(f, \varphi_3) = \int_{\frac{2}{3}}^{\frac{3}{3}} 54 x^2 (-2 + 3x) dx = \frac{43}{6} \approx 7.16667\\
\end{aligned}
\end{equation*}
\end{frame}

\begin{frame}[fragile]
\frametitle{Пример 1}
Умножаем полученные значения на $\frac{6}{h} = 18$ и получаем вектор--столбец $(3, 42, 150, 129)^T$ для СЛАУ, определяющая коэффициенты аппрокосимации:

$$
\begin{pmatrix}
2 & 1 & 0 & 0 \\
1 & 4 & 1 & 0 \\
0 & 1 & 4 & 1 \\
0 & 0 & 1 & 2\\
\end{pmatrix}
\begin{pmatrix}
c_0 \\
c_1 \\
c_2 \\
c_3 \\
\end{pmatrix}
=
\begin{pmatrix}
3\\
42\\
150\\
129\\
\end{pmatrix}
$$
Решая данную систему методом прогонки, находим коэффициенты:
$$
\begin{cases}
c_0 = -1 \\
c_1 = 5 \\
c_2 = 23 \\
c_3 = 53 \\
\end{cases}
$$
\end{frame}

\begin{frame}[fragile]
\frametitle{Пример 1}
Аппроксимирующая функция выглядит так:
$$
p(x) = -\varphi_0(x) + 5 \varphi_1(x) + 23 \varphi_2(x) + 53 \varphi_3(x).
$$

Это кусочно линейная функция, которая проходит через узлы: $(0, -1)$, $\left(\frac{1}{3}, 5 \right)$, $\left( \frac{2}{3}, 23 \right)$, $(1, 53)$.
\begin{center}
\begin{tikzpicture}
\begin{axis}
[
%unit vector ratio*=1 1 1,
xlabel=x,
ylabel=y,
domain=0:1,
%xmin=0.00, xmax=1.00,
ymin=-10, ymax=60,
grid=major,
axis lines = middle
]

\addplot[blue, solid] { 54 * x * x };
\addlegendentry{$f(x)$}

\addplot coordinates { (0, -1) (1/3, 5) (2/3, 23) (1, 53) };
\addlegendentry{$p(x)$}
\end{axis}
\end{tikzpicture}
\end{center}

\end{frame}


\begin{frame}[fragile]
\frametitle{Пример 1}
Аппроксимирующая функция выглядит так:
$$
p(x) = -\varphi_0(x) + 5 \varphi_1(x) + 23 \varphi_2(x) + 53 \varphi_3(x).
$$

\begin{center}
\begin{tikzpicture}
\begin{axis}
[
%unit vector ratio*=1 1 1,
xlabel=x,
ylabel=y,
domain=0.3:0.7,
xmin=0.3, xmax=0.7,
ymin=0, ymax=40,
grid=major,
axis lines = middle
]

\addplot[blue, solid] { 54 * x * x };
\addlegendentry{$f(x)$}

\addplot coordinates { (0, -1) (1/3, 5) (2/3, 23) (1, 53) };
\addlegendentry{$p(x)$}
\end{axis}
\end{tikzpicture}
\end{center}

\end{frame}

\begin{frame}[fragile]
\frametitle{Пример 1}

Образ базиса и базисные функции:
\begin{lstlisting}
n = int()
xi = linspace(0.0, 1.0, n + 1)
h = 1.0 / n

def phi(x):
    if -1 <= x <= 0:
        return 1 + x
    if 0 <= x <= 1:
        return 1 - x
    return 0

def phi(i, x)
    return phi((x - xi[i]) / h)
\end{lstlisting}
\end{frame}

\begin{frame}[fragile]
\frametitle{Пример 1}
Функция, которая вычисляет интеграл
$$\int_{0}^{1} f(t) \varphi_i(t) dt= \int_{x_i-h}^{x_i+h} f(t) \varphi_i(t) dt$$
методом трапеций (1000 трапеций).
\begin{lstlisting}
def integral(f, i):
    m = 1000
    l = 0 if i == 0 else xi[i] - h
    r = 1 if i == n else xi[i] + h

    s = 0.5 * (f(l) * phi(i, l) + f(r) * phi(i, r))
    for t in linspace(l, r, m + 1)[1: -1]:
        s = s + f(t) * phi(i, t)
    return s * (r - l) / m
\end{lstlisting}
\end{frame}

\begin{frame}[fragile]
\frametitle{Пример 1}
Функция вычисления весовых коэффициентов $c_i$ с помощью функций $integral$ --- численное интегрирование и $sweep$ --- метод прогонки.
\begin{lstlisting}
def generateLeastSquares():
    a = np.array([0] + [1] * (n - 1) + [1])
    b = np.array([2] + [4] * (n - 1) + [2])
    c = np.array([1] + [1] * (n - 1) + [0])
    f = np.zeros(n + 1)
    for i in range(n + 1):
        f[i] = 6 / h * integral(f, i)
    
    return sweep(n+1, a, b, c, f)[:-1]
\end{lstlisting}

\end{frame}



\begin{frame}[fragile]
\frametitle{Пример 2}

Рассмотрим аппроксимацию $f \in R^n$ элементами из подпространства ($m < n$):
$$v = c_1 \varphi_1 + ... + c_m \varphi_m$$
по норме, согласованной со скалярным произведением:
$$(f, g) = \sum_i {f_i g_i}$$

Метод наименьших квадратов.
$$
\begin{pmatrix}
(\varphi_1, \varphi_1) & (\varphi_1, \varphi_2) & ... & (\varphi_1, \varphi_m) \\
(\varphi_2, \varphi_1) & (\varphi_2, \varphi_2) & ... & (\varphi_2, \varphi_m) \\
...						& ...					  & ... & ...					  \\
(\varphi_m, \varphi_1) & (\varphi_m, \varphi_2) & ... & (\varphi_m, \varphi_m) \\
\end{pmatrix}
\begin{pmatrix}
c_1 \\
c_2 \\
... \\
c_m \\
\end{pmatrix}
=
\begin{pmatrix}
(\varphi_1, f) \\
(\varphi_2, f) \\
... \\
(\varphi_m, f) \\
\end{pmatrix}
$$
\end{frame}


\begin{frame}[fragile]
\frametitle{Пример 2}
Дан набор $n$ измерений $y_i = f(x_i)$. Необходимо аппроксимировать вектор $y$ вектором $\tilde{y}$, который получен линейной аппроксимацией 
$$\tilde{y}_i = a x_i + b$$
с минимальным отклонением: 
$$\min_{a,b} (y - \tilde{y}, y - \tilde{y}).$$

\end{frame}


\begin{frame}[fragile]
\frametitle{Пример 2}
Пусть $m = 2$.

$$\varphi_1 = (x_1, x_2, ..., x_n)$$ 
$$\varphi_2 = (1, 1, ..., 1)$$ 

Метод наименьших квадратов.
$$(y - a \varphi_1 - b \varphi_2, y - a \varphi_1 - b \varphi_2) \to \min$$ 

Система.
$$
\begin{pmatrix}
x_1^2 + ... + x_n^2 & x_1 + ... + x_n \\
x_1 + ... + x_n & n \\
\end{pmatrix}
\begin{pmatrix}
a \\
b \\
\end{pmatrix}
=
\begin{pmatrix}
x_1 y_1 + ... + x_n y_n \\
y_1 + ... + y_n  \\
\end{pmatrix}
$$ 

Введем обозначение средних величин:
$$\overline{z} = \frac{1}{n} \sum_{i=1}^{n} {z_i} .$$

Система.
$$
\begin{pmatrix}
\overline{x^2} & \overline{x} \\
\overline{x} & 1 \\
\end{pmatrix}
\begin{pmatrix}
a \\
b \\
\end{pmatrix}
=
\begin{pmatrix}
\overline{x y} \\
\overline{y}  \\
\end{pmatrix}
$$ 
\end{frame}

\begin{frame}[fragile]
\frametitle{Пример 2}

Минимизируем по школьному:
$$F(a,b) = \sum_{i=1}^n (a x_i + b - y_i)^2$$

Производные по $a$ и $b$ должны обнулиться:
$$
\begin{cases}
\sum_{i=1}^n x_i (a x_i + b - y_i) &= 0\\
\sum_{i=1}^n (a x_i + b - y_i) &= 0\\
\end{cases}
\Leftrightarrow
\begin{cases}
a \sum_{i=1}^n x_i^2 + b \sum_{i=1}^n x_ i &= \sum_{i=1}^n x_i y_i \\
a \sum_{i=1}^n x_i + b \sum_{i=1}^n 1 &= \sum_{i=1}^n y_i\\
\end{cases}
$$

Поделим каждое уравнение системы на $n$ и получим: 
\begin{equation}
\begin{cases}
a \overline{x^2} + b \overline{x} &= \overline{x y} \\
a \overline{x} + b &= \overline{y}\\
\end{cases}
\Leftrightarrow
\begin{cases}
a &= \frac{\overline{x y} - \overline{x} *  \overline{y}}{\overline{x^2} - \overline{x}^2} \\
b &= \overline{y} - a \overline{x}\\
\end{cases}
\end{equation}
\end{frame}


\begin{frame}[fragile]
\frametitle{Пример 2}

Дана сетка на $n=3$ отрезка со значениями в узлах: $(2, 1)$, $(3, 4)$, $(4, 2)$, $(5, 5)$.

Построить линейную аппроксимацию методом наименьших квадратов. 

Рассмотрим 2 вектора: $x = ( 2, 3, 4, 5 )$ и $y = (1, 4, 2, 5)$.

Вычислим средние величины:
$$
\overline{x} = \frac{1}{4} \left(2 + 3 + 4 + 5 \right) = \frac{7}{2}
$$
$$
\overline{y} = \frac{1}{4} (1 + 4 + 2 + 5) = 3
$$
$$
\overline{x y} = \frac{1}{4} \left( 2 \cdot 1 + 3 \cdot 4 + 4 \cdot 2 + 5 \cdot 5 \right) = \frac{47}{4}
$$
$$
\overline{x^2} = \frac{1}{4} \left(2^2 + 3^2 + 4^2 + 5^2 \right) = \frac{54}{4}
$$

Коэффициенты прямой:
$$
a = \frac{\overline{x y} - \overline{x} \cdot \overline{y}}{\overline{x^2} - \overline{x}^2} = \frac{ \frac{47}{4} - \frac{7}{2}  \cdot 3 }{ \frac{54}{4} - \frac{49}{4}} = 1
$$
$$
b = \overline{y} - a  \cdot \overline{x} = 3 - 1 \cdot \frac{7}{2} = -\frac{1}{2}
$$
\end{frame}


\begin{frame}[fragile]
\frametitle{Пример 2}

\begin{center}
\begin{tikzpicture}
\begin{axis}
[
unit vector ratio*=1 1 1,
xlabel=x,
ylabel=y,
xmin=0, xmax=6,
ymin=0, ymax=5,
grid=major,
axis lines = middle
]
\addplot[blue,only marks] coordinates { (2, 1) (3, 4) (4, 2) (5, 5) };

\addplot[red, domain=2:5] { -1/2 + x };
\end{axis}
\end{tikzpicture}
\end{center}

\end{frame}


\begin{frame}[fragile]
\frametitle{Пример 2}

Код, вычисляющий коэффициенты:
\begin{lstlisting}
def linearRegression(x, y)
    a = (x * y).mean() - x.mean() * y.mean()
    a /= (x * x).mean() - x.mean() ** 2
    b = y.mean() - a * x.mean()
    return a, b
\end{lstlisting}

\end{frame}



\begin{frame}[fragile]
\frametitle{Литература}
Подробнее в \cite[стр. 44]{Kalitkin} и \cite[стр. 65]{Samarsky}.



\begin{thebibliography}{99}

\bibitem{Kalitkin}
Калиткин Н.Н. Численные методы. - Спб.: БХВ-Петербург, 2014.

\bibitem{Samarsky}
Самарский А.А., Гулин А.В. Численные методы. - М.: Наука, 1989.

\end{thebibliography}
\end{frame}

\end{document}
