\documentclass[10pt]{beamer}
\usetheme{Madrid}
\usepackage[T2A]{fontenc}
\usepackage[utf8]{inputenc}
\usepackage[russian]{babel}
\usepackage{amsmath}
\usepackage{amsfonts}
\usepackage{amssymb}
\usepackage{comment}

%\newtheorem{theorem}{Теорема}

\author{Баев А.Ж.}
\title{Введение в численные методы. \\ Точные методы решения СЛАУ}
\institute{Казахстанский филиал МГУ} 
\date{07 февраля 2019}
%\subject{} 

\usepackage{tikz}
\usetikzlibrary{patterns} 
\usetikzlibrary{arrows, matrix, positioning, decorations.pathreplacing}

\usepackage{listings}
\lstset{language=python}
\lstset{basicstyle=\ttfamily}  
\lstset{keywordstyle=\color{blue}}
\lstset{frame=single}
\lstset{numbers=left}
\lstset{tabsize=4}



\usepackage{graphicx}

\begin{document}

\maketitle


\begin{frame}[fragile]
\frametitle{План на семестр}

\begin{enumerate}
\item СЛАУ (точные методы)
\item СЛАУ (итерационные методы)
\item решение нелинейных уравнений
\item интерполяция 
\item аппроксимация
\item интегрирование
\item дифференцирование
\end{enumerate}

\end{frame}


\begin{frame}[fragile]
\frametitle{Линейная алгебра}

Решение системы линейных алгебраических уравнений:

$$A x = f$$

где $A \in \mathbb{R}^{n \times n}$, $x \in \mathbb{R}^{n}$, $f \in \mathbb{R}^{n}$.

Дано $A$, $f$. Найти $x$.

\end{frame}


\begin{frame}[fragile]
\frametitle{Линейная алгебра (точные методы)}

\begin{enumerate}
\item Метод Гаусса.
\item Метод прогонки (если $A$ --- трёхдиагональная).
\item Метод Холецкого (если $A > 0$).
\end{enumerate}

\end{frame}




\begin{frame}[fragile]
\frametitle{Метод Гаусса}

Пусть $A = L U$, где $L$ --- нижнетреугольная (lower), $U$ --- верхнетреугольная (upper) матрица с единицами на диагонали:
$$ L U x = f$$

Прямой ход: привести матрицу к улучшенному верхнетреугольному виду элементарными преобразованиями строк.
$$ U x = L^{-1} f$$

Обратный ход: привести матрицу к диагональному виду элементарными преобразованиями строк.
$$ x = U^{-1} L^{-1} f$$

\end{frame}


\begin{frame}[fragile]
\frametitle{Метод Гаусса}
$$
\begin{pmatrix}
1 & 1 & 1 & 1  \\
1 & 3 & 2 & 2 \\
2 & 3 & 3 & 3 \\
-1 & 0 & 0 & 1 
\end{pmatrix}
\begin{pmatrix}
x_1 \\
x_2 \\
x_3 \\
x_4
\end{pmatrix}
=
\begin{pmatrix}
2 \\
3 \\
4 \\
-3
\end{pmatrix}
$$

\end{frame}


\begin{frame}[fragile]
\frametitle{Метод Гаусса (прямой ход)}

Шаг 1. Ведущий элемент $a_{1,1} = 1$. 
$$
\begin{pmatrix}
1 & 1 & 1 & 1 \\
1 & 3 & 2 & 2 \\
2 & 3 & 3 & 3 \\
-1 & 0 & 0 & 1 
\end{pmatrix}
\begin{pmatrix}
x_1 \\
x_2 \\
x_3 \\
x_4
\end{pmatrix}
=
\begin{pmatrix}
2 \\
3 \\
4 \\
-3
\end{pmatrix}
\Rightarrow
\begin{pmatrix}
1 & 1 & 1 & 1 \\
0 & 2 & 1 & 1 \\
0 & 1 & 1 & 1 \\
0 & 1 & 1 & 2 
\end{pmatrix}
\begin{pmatrix}
x_1 \\
x_2 \\
x_3 \\
x_4
\end{pmatrix}
=
\begin{pmatrix}
2 \\
1 \\
0 \\
-1
\end{pmatrix}
$$

Действие эквивалентно умножению уравнения слева на матрицу:
$$
L_1 = 
\begin{pmatrix}
1 & 0 & 0 & 0 \\
-1 & 1 & 0 & 0 \\
-2 & 0 & 1 & 0 \\
1 & 0 & 0 & 1 
\end{pmatrix}
$$
\end{frame}


\begin{frame}[fragile]
\frametitle{Метод Гаусса (прямой ход)}

Шаг 2. Ведущий элемент $a_{2,2} = 2$. 

$$
\begin{pmatrix}
1 & 1 & 1 & 1 \\
0 & 2 & 1 & 1 \\
0 & 1 & 1 & 1 \\
0 & 1 & 1 & 2 
\end{pmatrix}
\begin{pmatrix}
x_1 \\
x_2 \\
x_3 \\
x_4
\end{pmatrix}
=
\begin{pmatrix}
2 \\
1 \\
0 \\
-1
\end{pmatrix}
\Rightarrow
\begin{pmatrix}
1 & 1 & 1 & 1 \\
0 & 1 & \frac{1}{2} & \frac{1}{2} \\
0 & 0 & \frac{1}{2} & \frac{1}{2} \\
0 & 0 & \frac{1}{2} & \frac{3}{2} 
\end{pmatrix}
\begin{pmatrix}
x_1 \\
x_2 \\
x_3 \\
x_4
\end{pmatrix}
=
\begin{pmatrix}
2 \\
\frac{1}{2} \\
-\frac{1}{2} \\
-\frac{3}{2}
\end{pmatrix}
$$
Действие эквивалентно умножению уравнения слева на матрицу:
$$
L_2 = 
\begin{pmatrix}
1 & 0 & 0 & 0 \\
0 & \frac{1}{2} & 0 & 0 \\
0 & -\frac12 & 1 & 0 \\
0 & -\frac12 & 0 & 1 
\end{pmatrix}
$$

\end{frame}


\begin{frame}[fragile]
\frametitle{Метод Гаусса (прямой ход)}

Шаг 3. Ведущий элемент $a_{3,3} = \frac{1}{2}$.

$$
\begin{pmatrix}
1 & 1 & 1 & 1 \\
0 & 1 & \frac{1}{2} & \frac{1}{2} \\
0 & 0 & \frac{1}{2} & \frac{1}{2} \\
0 & 0 & \frac{1}{2} & \frac{3}{2} 
\end{pmatrix}
\begin{pmatrix}
x_1 \\
x_2 \\
x_3 \\
x_4
\end{pmatrix}
=
\begin{pmatrix}
2 \\
\frac{1}{2} \\
-\frac{1}{2} \\
-\frac{3}{2}
\end{pmatrix}
\Rightarrow
\begin{pmatrix}
1 & 1 & 1 & 1 \\
0 & 1 & \frac{1}{2} & \frac{1}{2} \\
0 & 0 & 1 & 1 \\
0 & 0 & 0 & 1 
\end{pmatrix}
\begin{pmatrix}
x_1 \\
x_2 \\
x_3 \\
x_4
\end{pmatrix}
=
\begin{pmatrix}
2 \\
\frac{1}{2} \\
-1 \\
-1
\end{pmatrix}
$$
Действие эквивалентно умножению уравнения слева на матрицу:
$$
L_3 = 
\begin{pmatrix}
1 & 0 & 0 & 0 \\
0 & 1 & 0 & 0 \\
0 & 0 & 2 & 0 \\
0 & 0 & -1 & 1 
\end{pmatrix}
$$
\end{frame}

\begin{frame}[fragile]
\frametitle{Метод Гаусса (прямой ход)}
Прямой ход сводится к действиям:
$$
L_3 L_2 L_1 A x = L_3 L_2 L_1 f
$$

Обозначим $L_3 L_2 L_1 = L^{-1}$, то есть 
$$ L = L_1^{-1} L_2^{-1} L_3^{-1}$$

После прямого хода остается матричное уравнение

$$U x = \tilde{f}$$
где $U = L^{-1} A$, $\tilde{f} = L^{-1} f$.
\end{frame}

\begin{frame}[fragile]
\frametitle{Метод Гаусса (обратный ход)}
$$x_4 = f_4 = -1$$
$$x_3 = f_3 - u_{3,4} x_4 = 0$$
$$x_2 = f_2 - u_{2,3} x_3 - u_{2,4} x_4 = 1$$
$$x_1 = f_1 - u_{1,2} x_2 - u_{1,3} x_3 - u_{1,4} x_4 = 2$$

\end{frame}


\begin{frame}[fragile]
\frametitle{Метод Гаусса (прямой ход)}

Прямой ход: привести матрицу к треугольному виду элементарными преобразованиями строк.

Для ведущего элемента $a_{k, k}$:
$$
\begin{pmatrix}
a_{1,1} & a_{1,2} &  \dots & a_{1,k-1}   & a_{1,k}     & \dots & a_{1,j} & \dots & a_{1,n}   \\
0       & a_{2,2} &  \dots & a_{2,k-1}   & a_{2,k}     & \dots & a_{2,j}& \dots & a_{2,n}   \\
\dots   & \dots   &  \dots & \dots       & \dots       & \dots & \dots& \dots & \dots     \\
0       & 0       &  \dots & a_{k-1,k-1} & a_{k-1,k}   & \dots & a_{3,j}& \dots & a_{3,n}   \\
0       & 0       &  \dots & 0           & a_{k,k}     & \dots & a_{k,j}& \dots & a_{k,n}   \\
0       & 0       &  \dots & 0           & a_{k+1,k}   & \dots & a_{k+1,j}& \dots & a_{k+1,n} \\
\dots   & \dots   &  \dots & \dots       & \dots       & \dots & \dots& \dots & \dots     \\
0       & 0       &  \dots & 0           & a_{i,k}     & \dots & a_{i,j}& \dots & a_{i,n}   \\
\dots   & \dots   &  \dots & \dots       & \dots       & \dots & \dots& \dots & \dots     \\
0       & 0       &  \dots & 0           & a_{n,k}     & \dots & a_{n,j}& \dots & a_{n,n}
\end{pmatrix}
$$

$$ \tilde{a}_{k,j} := \frac{a_{k,j}}{a_{k,k}} $$
$$ \tilde{a}_{i,j} := a_{i,j} - \tilde{a}_{k,j} a_{i,k}$$

для всех $i$ от $(k+1)$ до $n$ и для всех $j$ от $(k+1)$ до $n$.


\end{frame}


\begin{frame}[fragile]
\frametitle{Метод Гаусса (обратный ход)}

Обратный ход: обращаем улучшенную верхне--треугольную матрицу.

Для искомого $x_{i}$:

$$x_i = \tilde{f_i} - \sum_{j=i+1}^{n} u_{i,j} x_j$$

\end{frame}

\begin{frame}[fragile]
\frametitle{Метод Гаусса (псевдокод и асимптотика)}

1. Прямой ход:

\begin{lstlisting}
for k := 1 .. n
    for j := k+1 .. n
	    A[k][j] := A[k][j] / A[k][k]

    for i := k+1 .. n
        for j := k+1 .. n
            A[i][j] := A[i][j] - A[i][k] * A[k][j]
        f[i] := f[i] - A[i][k] * f[k]
        A[i][k] := 0
\end{lstlisting}

$$f_1(n) = \sum_{k=1}^{n} \sum_{i=k+1}^{n} \left(2 + \sum_{j=k+1}^{n} 1 \right) = 
\sum_{k=1}^{n} \sum_{i=k+1}^{n} (2 + n - k) = 
\sum_{k=1}^{n} (n - k)(n + 2 - k)$$

Сделаем замену $k = n + 1 - s$:
$$f_1(n) = \sum_{s=1}^{n} (s - 1)(s + 1) = \sum_{s=1}^{n} s^2 - \sum_{s=1}^{n} 1 =\frac{n(n+1)(2n+1)}{6} - n = \frac{2 n^3 + 3 n^2 - 5 n}{6}$$

\end{frame}

\begin{frame}[fragile]
\frametitle{Метод Гаусса (псевдокод и асимптотика)}

2. Обратный ход:

\begin{lstlisting}
for i := n .. 1
    x[i] := f[i]
    for j := i+1 .. n
        x[i] := x[i] - A[i][j] * x[j]
\end{lstlisting}


$$f_2(n) = \sum_{i=1}^{n} \left(1 + \sum_{j=i+1}^{n} 1 \right) = 
\sum_{i=1}^{n} (1 + n - i)$$

Сделаем замену $i = n + 1 - s$:
$$f_2(n) = \sum_{s=1}^{n} s = \frac{n^2 + n}{2}$$
\end{frame}

\begin{frame}[fragile]
\frametitle{Метод Гаусса (асимптотика)}

Алгоритмическая сложность метода Гаусса для решения СЛАУ:

$$f(n) = f_1(n) + f_2(n) = \frac{2 n^3 + 3 n^2 - 5 n}{6} + \frac{n^2 + n}{2} = \frac{n^3 + 3 n^2 - n}{3}$$
 
Асимптотика алгоритмической сложности:
\begin{enumerate}
\item прямого хода метода Гаусса $f_1(n) = \Theta(n^3)$;
\item обратного хода метода Гаусса $f_2(n) = \Theta(n^2)$;
\item метода Гаусса для решения СЛАУ $f(n) = \Theta(n^3)$.
\end{enumerate}
\end{frame}


\begin{frame}[fragile]
\frametitle{Метод Гаусса (теорема об $LU$--разложении)}

\begin{theorem}[теорема об $LU$--разложении]
Пусть все угловые миноры матрицы $A$ отличны от нуля. Тогда матрицу $A$ можно представить единственным образом в виде произведения
$$A = L U$$
где $L$ --- нижняя треугольная матрица с ненулевыми диагональными элементами и $U$ --- верхняя треугольная матрица с единичной диагональю.
\end{theorem}
\end{frame}

\begin{frame}[fragile]
\frametitle{Метод Гаусса (теорема об $LU$--разложении)}

\begin{proof}

Существование.

Математическая индукция. База: $n = 2$.

$$
\begin{pmatrix}
a_{1,1} & a_{1,2} \\
a_{2,1} & a_{2,2} \\
\end{pmatrix}
=
\begin{pmatrix}
l_{1,1} & 0 \\
l_{2,1} & l_{2,2} \\
\end{pmatrix}
\begin{pmatrix}
1 & u_{1,2} \\
0 & 1 \\
\end{pmatrix}
$$

Решается в явном виде:
$$
\begin{cases}
l_{1, 1} = a_{1, 1}\\
l_{2, 1} = a_{2, 1}\\
l_{2, 2} = \frac{\Delta}{a_{1, 1}}\\
u_{1, 2} = \frac{a_{1, 2}}{a_{1, 1}}\\
\end{cases}
$$
\end{proof}

\end{frame}



\begin{frame}[fragile]
\frametitle{Метод Гаусса (теорема об $LU$--разложении)}

\begin{proof}[]
Обозначим: 

$A_k$ --- угловой минор матрицы,

$b_{k-1}$ --- часть нижней вектор--строки, 

$a_{k-1}$ --- часть правого вектор--столбца.

Предположение $n = k - 1$: $A_{k-1} = L_{k-1} U_{k-1}$. Переход $n = k$: $A_{k} = L_{k} U_{k}$. 



$$
\begin{pmatrix}
A_{k-1} & a_{k-1} \\
b_{k-1} & a_{k,k} \\
\end{pmatrix}
=
\begin{pmatrix}
L_{k-1} & 0 \\
l_{k-1} & l_{k,k} \\
\end{pmatrix}
\begin{pmatrix}
U_{k-1} & u_{k-1} \\
0 & 1 \\
\end{pmatrix}
$$

Решается в явном виде:
$$
\begin{cases}
L_{k - 1} u_{k-1} = a_{k - 1}\\
l_{k - 1} U_{k-1} = b_{k - 1}\\
l_{k - 1} u_{k-1} + l_{k, k} = a_{k, k}\\
\end{cases}
\Leftrightarrow
\begin{cases}
u_{k-1} = L_{k - 1}^{-1} a_{k - 1}\\
l_{k - 1} = b_{k - 1} U_{k-1}^{-1}\\
l_{k, k} = a_{k, k} - l_{k - 1} u_{k-1}\\
\end{cases}
$$

По условию $det A \neq 0 \Leftrightarrow l_{k, k} \neq 0$ 

$\Rightarrow$ 

Разложение $A_k = L_k U_k$ существует, причем $L_{k}$ и $U_{k}$ имеют ненулевые диагональные элементы.
\end{proof}

\end{frame}


\begin{frame}[fragile]
\frametitle{Метод Гаусса (теорема об $LU$--разложении)}

\begin{proof}[]
Единственность.

$$ L_1 U_1 = L_2 U_2$$

$$ U_1 U_2^{-1} = L_1^{-1} L_2$$

Слева верхнетреугольная, справа нижнетреугольная $\Rightarrow$ 

все матрицы диагональные $\Rightarrow$ 

слева единичная матрица $\Rightarrow$ 

$L_1 = L_2$.

\end{proof}

\end{frame}


\begin{frame}[fragile]
\frametitle{Метод Гаусса (перестановка ведущего элемента)}

Прямой ход метод Гаусса. Усиленно делим на нуль:

$$
\begin{pmatrix}
0 & 3 & 0 & 0 \\
1 & 0 & 0 & 0 \\
0 & 0 & 4 & 0 \\
2 & 0 & 0 & 5 
\end{pmatrix}
\begin{pmatrix}
x_1 \\
x_2 \\
x_3 \\
x_4
\end{pmatrix}
=
\begin{pmatrix}
1 \\
2 \\
3 \\
4
\end{pmatrix}
\Rightarrow
\begin{pmatrix}
2 & 0 & 0 & 5 \\
1 & 0 & 0 & 0 \\
0 & 0 & 4 & 0 \\
0 & 3 & 0 & 0 \\
\end{pmatrix}
\begin{pmatrix}
x_1 \\
x_2 \\
x_3 \\
x_4
\end{pmatrix}
=
\begin{pmatrix}
4 \\
2 \\
3 \\
1
\end{pmatrix}$$
Действие эквивалентно умножению уравнения слева на матрицу:
$$
P = 
\begin{pmatrix}
0 & 0 & 0 & 1 \\
0 & 1 & 0 & 0 \\
0 & 0 & 1 & 0 \\
1 & 0 & 0 & 0 
\end{pmatrix}
$$
\end{frame}

\begin{frame}[fragile]
\frametitle{Метод Гаусса (перестановка ведущего элемента)}
Обобщение $LU$ разложения: 

$$A = PLU$$
где 
$P$ --- перестановочная матрица, 

$L$ --- нижнетреугольная, 

$U$ --- верхнетругольная c единицами.

\end{frame}


\begin{frame}[fragile]
\frametitle{Метод Гаусса (перестановка ведущего элемента)}

$$L_nP_nL_{n-1}P_{n-1}\ldots L_2P_2L_1P_1 A = U$$

Введём новые матрицы
$$
\begin{cases}
L'_n &= L_n\\
L'_{n-1} &= P_nL_nP_{n}^{-1}\\
L'_{n-2} &= P_nP_{n-1}L_{n-1}P_n^{-1}P_{n-1}^{-1}\\
&\ldots\\
L'_1 &= P_nP_{n-1}\ldots P_2L_1P_2^{-1}\ldots P_{n-1}^{-1}P_n^{-1}
\end{cases}
$$

Получаем
$$\underbrace{L'_nL'_{n-1}\ldots L'_1}_{L^{-1}}\underbrace{P_nP_{n-1}\ldots P_1}_{P^{-1}}\cdot A$$
\end{frame}

\begin{frame}[fragile]
\frametitle{python}

Список списков
\begin{lstlisting}
l = [ [1, 1, 1, 1],
      [1, 3, 2, 2],
      [2, 3, 3, 3],
      [-1, 0, 0, 1] ]
\end{lstlisting}
не поддерживает матричных действий.

Массивы numpy
\begin{lstlisting}
import numpy as np
...
a = np.array(l)
\end{lstlisting}
поддерживает матричные действия

\end{frame}


\begin{frame}[fragile]
\frametitle{python}

Список списков
\begin{lstlisting}
l = [ [1, 1, 1, 1],
      [1, 3, 2, 2],
      [2, 3, 3, 3],
      [-1, 0, 0, 1] ]
print(l[3][0])
\end{lstlisting}
не поддерживает матричных действий.

Массивы numpy
\begin{lstlisting}
import numpy as np
...
a = np.array(l)
print(a[3][0])
\end{lstlisting}
поддерживает матричные действия

\end{frame}


\begin{frame}[fragile]
\frametitle{Примитивы numpy}
Вектор из 5 единиц
\begin{lstlisting}
a = np.ones(5)
\end{lstlisting}

Матрица 5 на 6 из чисел 7.0
\begin{lstlisting}
a = np.ones((5, 6)) * 7.0
\end{lstlisting}


Единичная матрица размера 5
\begin{lstlisting}
a = np.eye(5)
\end{lstlisting}

Диагональная матрица 5 на 5 с 2 по диагонали и 1 в остальных позициях:
\begin{lstlisting}
a = np.eye(5) + np.ones((5, 5))
\end{lstlisting}

Найти сумму и максимум всех элементов матрицы
\begin{lstlisting}
v = np.sum(a)
m = np.maximum(a)
\end{lstlisting}
\end{frame}


\begin{frame}[fragile]
\frametitle{Срезы numpy}
Создать матрицу размера $n \times n$ со случайными значениями:
\begin{lstlisting}
n = 5
a = np.random.rand(n,n)
\end{lstlisting}

Применить первую строку ко второй строке
\begin{lstlisting}
a[:,1] += a[:,0]
\end{lstlisting}

Обнулить все элемент первого столбца кроме $a[0][0]$:
\begin{lstlisting}
a[1:,0] = np.zeros(n - 1)
\end{lstlisting}

Найти максимальный по модулю элемент в последнем столбце матрицы
\begin{lstlisting}
m = np.maximum(np.abs(a[-1,:]))
\end{lstlisting}
\end{frame}

\begin{frame}[fragile]
\frametitle{Треугольные матрицы numpy}
Оставляет нижнетреугольную матрицу
\begin{lstlisting}
np.tril([[1, 2, 3],
         [4, 5, 6],
         [7, 8, 9]])
np.tril([[1, 2, 3],
         [4, 5, 6],
         [7, 8, 9]],
         -1)
\end{lstlisting}

Получаем
\begin{lstlisting}
1 0 0
4 5 0
7 8 9
\end{lstlisting}
и
\begin{lstlisting}
0 0 0
4 0 0
7 8 0
\end{lstlisting}

Аналогично {\tt np.triu} --- верхнетреугольная.
\end{frame}

\begin{frame}[fragile]
\frametitle{linalg}
\begin{lstlisting}
import scipy.linalg as sla
\end{lstlisting}

Разложение матрицы
\begin{lstlisting}
P, L, U = sla.lu(A)
\end{lstlisting}

Проверка матриц на равенство: $A = LU$ и $P = E$.
\begin{lstlisting}
np.allclose(A, L.dot(U))
np.allclose(P, np.eye(n))
\end{lstlisting}

Решение уравнения $Ax = b$ методом Гаусса:
\begin{lstlisting}
x = numpy.linalg.solve(A, b)
\end{lstlisting}

Обратная матрица $C = A^{-1}$:
\begin{lstlisting}
C = numpy.linalg.inv(A)
\end{lstlisting}

\end{frame}

\begin{frame}[fragile]
\frametitle{Упражнение}
Упражнение 1. Обратная матрица к нижнетреугольной матрице --- нижнетреугольная матрица.

Упражнение 2. Даны две верхнетреугольные матрицы $n \times n$. Определить количество умножений пар элементов матриц при перемножении этих матриц.

Упражнение 3. Доказать соотношения для $PLU$ разложения.
\end{frame}


\begin{frame}[fragile]
\frametitle{reference}

Допуск к зачету:

\url{https://pythontutor.ru/}

\vfill{2cm}

К первой домашней работе:

\url{https://docs.scipy.org/doc/numpy/reference/}

\end{frame}
\end{document}
