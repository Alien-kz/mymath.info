\documentclass[10pt]{beamer}
\usetheme{Madrid}
\usepackage[T2A]{fontenc}
\usepackage[utf8]{inputenc}
\usepackage[russian]{babel}
\usepackage{amsmath}
\usepackage{amsfonts}
\usepackage{amssymb}
\usepackage{comment}
\usepackage[all]{xy}
\usepackage{gnuplottex}

%\newtheorem{theorem}{Теорема}

\author{Баев А.Ж.}
\title{Введение в численные методы. \\ Дифференцирование и задача Коши}
\institute{Казахстанский филиал МГУ} 
\date{25 февраля 2019}
%\subject{} 

\usepackage{tikz, pgfplots}
\usepgfplotslibrary{fillbetween}
%\pgfplotsset{compat=1.10}


\usetikzlibrary{patterns} 
\usetikzlibrary{arrows, matrix, positioning, decorations.pathreplacing}

\usepackage{listings}
\lstset{language=python}
\lstset{basicstyle=\ttfamily}  
\lstset{keywordstyle=\color{blue}}
\lstset{frame=single}
\lstset{numbers=left}
\lstset{tabsize=4}



\usepackage{graphicx}

\begin{document}

\maketitle


\begin{frame}[fragile]
\frametitle{План на семестр}

\begin{enumerate}
\item СЛАУ (точные методы)
\item СЛАУ (итерационные методы)
\item решение нелинейных уравнений
\item интерполяция
\item аппроксимация
\item интегрирование
\item \textbf{дифференцирование}
\end{enumerate}

\end{frame}


\begin{frame}[fragile]
\frametitle{Дифференцирования}

Дана дифференцируемая на отрезке $(a; b)$ функция $f(x)$. Необходимо вычислить значение производной:
$$f'(x).$$
в точках равномерно сетки $x_i = a + i h$, где $h = \frac{b-a}{n}$.

Основная идея: формула Тейлора.


\end{frame}

\begin{frame}[fragile]
\frametitle{Первая производная}

\begin{center}
\begin{tikzpicture}[scale=0.75]
\begin{axis}
[domain=0:1]
\addplot[blue!30] {27 * x^3 - 45 * x^2 + 18 * x + 2};
\addplot[blue, only marks, samples = 11] {27 * x^3 - 45 * x^2 + 18 * x + 2};
\legend{$f(x)$};
\end{axis}
\end{tikzpicture}
\begin{tikzpicture}[scale=0.75]
\begin{axis}
[domain=0:1]
\addplot[red!30] {81*x^2 - 90 * x + 18 };
\addplot[red, only marks, samples = 11] {81*x^2 - 90 * x + 18};
\legend{$f'(x)$};
\end{axis}
\end{tikzpicture}
\end{center}
\end{frame}


\begin{frame}[fragile]
\frametitle{Первая производная (вперёд)}

Определим значение производную в точке $x_i$ через значения $f_i$ и $f_{i+1}$.
$$f_{x, i} = \frac{f_{i+1} - f_i}{h}$$

Оценим погрешность с помощью формулы Тейлора
$$f(x) = f(x_i) + (x - x_i) f'(x_i) + \frac{(x-x_i)^2}{2} f''(\xi)$$

Подставим вместо $x = x_{i+1}$
$$f_{i+1} = f_i + h f'_i + \frac{h^2}{2} f''(\xi)$$

Погрешность:
$$f_{x, i} = \frac{f_{i+1} - f_i}{h} = 
\frac{f_i + h f'_i + \frac{h^2}{2} f''(\xi) - f_i}{h} =
f'(x_i) + O(h)$$


\end{frame}


\begin{frame}[fragile]
\frametitle{Первая производная (вперёд)}

\begin{center}
\begin{tikzpicture}[scale=0.75]
\begin{axis}
[domain=0:1]
\addplot[blue!30] {27 * x^3 - 45 * x^2 + 18 * x + 2};
\addplot[blue, only marks, samples = 11] {27 * x^3 - 45 * x^2 + 18 * x + 2};
\legend{$f(x)$};
\end{axis}
\end{tikzpicture}
\begin{tikzpicture}[scale=0.75]
\begin{axis}
[domain=0:1]
\addplot[red!30] {81*x^2 - 90 * x + 18 };
\addplot[red, only marks, samples = 11] {81*x^2 - 90 * x + 18};
\legend{$f'(x)$};
\addplot[green, only marks] coordinates
{
( 0.0 , 13.77 )
( 0.1 , 6.39 )
( 0.2 , 0.63 )
( 0.3 , -3.51 )
( 0.4 , -6.03 )
( 0.5 , -6.93 )
( 0.6 , -6.21 )
( 0.7 , -3.87 )
( 0.8 , 0.09 )
( 0.9 , 5.67 )
};
\end{axis}
\end{tikzpicture}
\end{center}
\end{frame}

\begin{frame}[fragile]
\frametitle{Первая производная (назад)}


Определим значение производную в точке $x_i$ через значения $f_{i-1}$ и $f_{i}$.
$$f_{x, i} = \frac{f_{i+1} - f_i}{h}$$

Из формулы Тейлора в точке $x_i$:
$$f_{i-1} = f_i - h f'_i + \frac{h^2}{2} f''(\xi)$$

Погрешность:
$$f_{\overline{x}, i} = \frac{f_{i} - f_{i - 1}}{h}  = \frac{f_{i} - f_i + h f'_i - \frac{h^2}{2} f''(\xi)}{h} = f'(x_i) + O(h)$$

\end{frame}


\begin{frame}[fragile]
\frametitle{Первая производная (назад)}

\begin{center}
\begin{tikzpicture}[scale=0.75]
\begin{axis}
[domain=0:1]
\addplot[blue!30] {27 * x^3 - 45 * x^2 + 18 * x + 2};
\addplot[blue, only marks, samples = 11] {27 * x^3 - 45 * x^2 + 18 * x + 2};
\legend{$f(x)$};
\end{axis}
\end{tikzpicture}
\begin{tikzpicture}[scale=0.75]
\begin{axis}
[domain=0:1]
\addplot[red!30] {81*x^2 - 90 * x + 18 };
\addplot[red, only marks, samples = 11] {81*x^2 - 90 * x + 18};
\legend{$f'(x)$};
\addplot[green, only marks] coordinates
{
( 0.1 , 13.77 )
( 0.2 , 6.39 )
( 0.3 , 0.63 )
( 0.4 , -3.51 )
( 0.5 , -6.03 )
( 0.6 , -6.93 )
( 0.7 , -6.21 )
( 0.8 , -3.87 )
( 0.9 , 0.09 )
( 1.0 , 5.67 )
};
\end{axis}
\end{tikzpicture}
\end{center}
\end{frame}

\begin{frame}[fragile]
\frametitle{Первая производная (центральная)}

Определим значение производную в точке $x_i$ через значения $f_{i-1}$ и $f_{i+1}$.
$$f_{x, i} = \frac{f_{i+1} - f_{i-1}}{2h}$$

Из формулы Тейлора в точке $x_i$:
$$f_{i+1} = f_i + h f'_i + \frac{h^2}{2} f''(\xi_1)$$
$$f_{i-1} = f_i - h f'_i + \frac{h^2}{2} f''(\xi_2)$$


Погрешность:
$$\frac{f_{i + 1} - f_{i - 1}}{2 h} = \frac{f_{i} + h f'_i + \frac12 h^2 f''_i + \frac16 h^3 f''(\xi_1) - f_{i} + h f'_i - \frac12 h^2 f''_i + \frac16 h^3 f''(\xi_2)}{2 h} = $$
$$ = \frac{2 h f'_i + \frac16 h^3 (f''(\xi_1) + f''(\xi_2))}{2 h} = f'(x_i) + \frac{h^2}{6} f''(\xi)$$

$$f_{\hat{x}, i} = \frac{f_{i + 1} - f_{i - 1}}{2 h} = f'(x_i) + O(h^2)$$

\end{frame}

\begin{frame}[fragile]
\frametitle{Первая производная (центральная)}

\begin{center}
\begin{tikzpicture}[scale=0.75]
\begin{axis}
[domain=0:1]
\addplot[blue!30] {27 * x^3 - 45 * x^2 + 18 * x + 2};
\addplot[blue, only marks, samples = 11] {27 * x^3 - 45 * x^2 + 18 * x + 2};
\legend{$f(x)$};
\end{axis}
\end{tikzpicture}
\begin{tikzpicture}[scale=0.75]
\begin{axis}
[domain=0:1]
\addplot[red!30] {81*x^2 - 90 * x + 18 };
\addplot[red, only marks, samples = 11] {81*x^2 - 90 * x + 18};
\legend{$f'(x)$};
\addplot[green, only marks] coordinates
{
( 0.1 , 10.08 )
( 0.2 , 3.51 )
( 0.3 , -1.44 )
( 0.4 , -4.77 )
( 0.5 , -6.48 )
( 0.6 , -6.57 )
( 0.7 , -5.04 )
( 0.8 , -1.89 )
( 0.9 , 2.88 )
};
\end{axis}
\end{tikzpicture}
\end{center}
\end{frame}


\begin{frame}[fragile]
\frametitle{Первая производная (несимметричная второго порядка)}
Определим значение производную в точке $x_i$ через значения $f_{i}$, $f_{i+1}$ и $f_{i+2}$.

Из формулы Тейлора в точке $x_i$ получим
$$f_{i+2} = f_{i} + (2 h) f'_i + \frac12 (2 h)^2 f''_i + \frac16 (2 h)^3 f''(\xi_1)$$ 

Пояснение:
$$\frac{- 3 f_i + 4 f_{i + 1} -f_{i + 2}}{2 h} = $$
$$= \frac{
- 3 f_i
+ 4 (f_{i} + h f'_i + \frac{h^2}{2} f''_i + \frac{h^3}{6} f'''(\xi_2)) 
-(f_{i} + (2 h) f'_i + \frac{(2 h)^2}{2}  f''_i + \frac{(2 h)^3}{6} f''(\xi_1)) 
}{2 h} =$$
$$ = \frac{2 h f'_i + \frac23 h^3 f'''(\xi_2) - \frac43 h^3 f'''(\xi_1) }{2 h} = $$
$$= f'_i + \frac13 h^2 (f'''(\xi_2) - f'''(\xi_1) ) - \frac23 h^2 f'''(\xi_1)$$

$$f_{\tilde{x}, i} = \frac{-3f_{i} + 4 f_{i + 1} - f_{i+2}}{2 h} = f'(x_i) + O(h^2)$$
\end{frame}

\begin{frame}[fragile]
\frametitle{Первая производная (несимметричная второго порядка)}
\begin{center}
\begin{tikzpicture}[scale=0.75]
\begin{axis}
[domain=0:1]
\addplot[blue!30] {27 * x^3 - 45 * x^2 + 18 * x + 2};
\addplot[blue, only marks, samples = 11] {27 * x^3 - 45 * x^2 + 18 * x + 2};
\legend{$f(x)$};
\end{axis}
\end{tikzpicture}
\begin{tikzpicture}[scale=0.75]
\begin{axis}
[domain=0:1]
\addplot[red!30] {81*x^2 - 90 * x + 18 };
\addplot[red, only marks, samples = 11] {81*x^2 - 90 * x + 18};
\legend{$f'(x)$};
\addplot[green, only marks] coordinates
{
( 0.0 , 17.46 )
( 0.1 , 9.27 )
( 0.2 , 2.7 )
( 0.3 , -2.25 )
( 0.4 , -5.58 )
( 0.5 , -7.29 )
( 0.6 , -7.38 )
( 0.7 , -5.85 )
( 0.8 , -2.7 )
};
\end{axis}
\end{tikzpicture}
\end{center}
\end{frame}



\begin{frame}[fragile]
\frametitle{Вторая производная}

\begin{center}
\begin{tikzpicture}[scale=0.75]
\begin{axis}
[domain=0:1]
\addplot[blue!30] {27 * x^4 - 45 * x^3 + 18 * x^2 + 2 * x};
\addplot[blue, only marks, samples = 11] {27 * x^4 - 45 * x^3 + 18 * x^2 + 2*x};
\legend{$f(x)$};
\end{axis}
\end{tikzpicture}
\begin{tikzpicture}[scale=0.75]
\begin{axis}
[domain=0:1,
ymax = 120]
\addplot[red!30] {324*x^2 - 270*x + 36};
\addplot[red, only marks, samples = 11] {324*x^2 - 270*x + 36};
\legend{$f''(x)$};
\end{axis}
\end{tikzpicture}

\end{center}
\end{frame}

\begin{frame}[fragile]
\frametitle{Вторая производная}

Оператор дифференцирования вперед и назад
$$f_{x, i} = \frac{f_{i+1} - f_i}{h}$$
$$f_{\overline{x}, i} = \frac{f_{i} - f_{i-1}}{h}$$

Вторая производная
$$
f_{x\overline{x}, i} = ( f_{x, i} )_{\overline{x}}
 = 
\left( \frac{f_{i+1} - f_i}{h} \right)_{\overline{x}} = 
\frac{
\frac{f_{i+1} - f_i}{h} - \frac{f_i - f_{i-1}}{h}
}{h} = \frac{f_{i+1} - 2 f_i + f_{i-1}}{h^2}
$$

Из ряда Тейлора
$$f_{x\overline{x}, i}  = \frac{-f_{i + 2} + 4 f_{i + 1} - 3 f_i}{2 h} = f''(x_i) + O(h^2)$$
\end{frame}


\begin{frame}[fragile]
\frametitle{Вторая производная}
\begin{center}
\begin{tikzpicture}[scale=0.75]
\begin{axis}
[domain=0:1]
\addplot[blue!30] {27 * x^4 - 45 * x^3 + 18 * x^2 + 2 * x};
\addplot[blue, only marks, samples = 11] {27 * x^4 - 45 * x^3 + 18 * x^2 + 2*x};
\legend{$f(x)$};
\end{axis}
\end{tikzpicture}
\begin{tikzpicture}[scale=0.75]
\begin{axis}
[domain=0:1,
ymax = 120]
\addplot[red!30] {324*x^2 - 270*x + 36};
\addplot[red, only marks, samples = 11] {324*x^2 - 270*x + 36};
\legend{$f''(x)$};
\addplot[green, only marks] coordinates
{
( 0.1 , 12.78 )
( 0.2 , -4.5 )
( 0.3 , -15.3 )
( 0.4 , -19.62 )
( 0.5 , -17.46 )
( 0.6 , -8.82 )
( 0.7 , 6.3 )
( 0.8 , 27.9 )
( 0.9 , 55.98 )
};
\end{axis}
\end{tikzpicture}
\end{center}
\end{frame}


\begin{frame}[fragile]
\frametitle{Некорректность дифференцирования}

$$ \frac{f_{i+1} - f_i}{h} =  f_{x, i}$$
Допустим данные $f_i$ и $f_{i+1}$ уже содержат погрешность по сравнению с $f(x_{i+1})$ и $f(x_{i})$ (ошибка представления вещественных чисел).

$$ \frac{f_{i+1} - f_i}{h} = \frac{f(x_{i+1}) + \delta_{i+1} - f(x_i) -\delta_i}{h} = \frac{f(x_{i+1}) - f(x_i)}{h} + \frac{\delta_{i+1} - \delta_i}{h} $$

Считаем, что $\delta_i < \delta, \delta_{i+1} < \delta$.

$$ |f'(x_i) - f_{x, i}| < M_2 \frac{h}{2} + \frac{2 \delta}{h} $$

Минимум достигается при $h = 2 \sqrt{\frac{\delta}{M_2}}$

\end{frame}

\begin{frame}[fragile]
\frametitle{Задача Коши для дифференциального уравнения}
Дано $f(t, u)$ и $a$. Определить $u(t)$ при $t < T$.
$$
\begin{cases}
u'(t) = f(t, u(t))\\
u(0) = a
\end{cases}
$$

\end{frame}


\begin{frame}[fragile]
\frametitle{Задача Коши для дифференциального уравнения}

$$
\begin{cases}
u'(t) = t^2 + 1 \\
u(0) = 2
\end{cases}
$$

\vfill

\begin{center}
\begin{tikzpicture}[scale=0.75]
\begin{axis}
[domain=0:1,
xlabel=t,
ylabel=y]
\addplot[blue, thick] {x^3 / 3 + x + 2};
\node[draw] at (axis cs:0.0, 2) {};
\legend{$f(x)$};
\end{axis}
\end{tikzpicture}
\end{center}

Решение $u(t) = \frac{t^3}{3} + t + 2$
\end{frame}

\begin{frame}[fragile]
\frametitle{Задача Коши для дифференциального уравнения}


Введём равномерную сетку с шагом $\tau$: $t_i = i \cdot \tau$. Значения в узлах сетки $f_i$ и $y_i$. Проинтегрируем уравнение по $t$ на интервалах $t \in [t_i, t_{i+i}]$
$$ 
u(t_{i+1}) - u(t_i) = \int_{t_i}^{t_{i+1}} f(t, u(t)) dt
$$

В зависимости от аппроксимации интеграла получим различные методы.
\end{frame}


\begin{frame}[fragile]
\frametitle{Явный метод Эйлера}

Левые прямоугольники
$$
\begin{cases}
y_{i + 1} - y_i = \tau \cdot f(t_i, y_i) \\
y_0 = a
\end{cases}
$$

\begin{center}
\begin{tikzpicture}[scale=0.75]
\begin{axis}
[domain=0:1,
xlabel=t,
ylabel=y]
\addplot[blue, thick] {x^3 / 3 + x + 2};
\legend{$y(t)$};
\addplot[green, only marks] coordinates
{
( 0.0 , 2 )
( 0.1 , 2.1 )
( 0.2 , 2.201 )
( 0.3 , 2.305 )
( 0.4 , 2.414 )
( 0.5 , 2.53 )
( 0.6 , 2.655 )
( 0.7 , 2.791 )
( 0.8 , 2.94 )
( 0.9 , 3.104 )
( 1.0 , 3.285 )
};
\legend{$y_i$};
\end{axis}
\end{tikzpicture}
\end{center}
\end{frame}

\begin{frame}[fragile]
\frametitle{Неявныe методы}

Правые прямоугольники
$$
\begin{cases}
y_{i + 1} - y_i = \tau \cdot f(t_{i+1}, y_{i+1}) \\
y_0 = a
\end{cases}
$$

Метод трапеций
$$
\begin{cases}
y_{i + 1} - y_i = \frac{\tau}{2} \cdot \left(f(t_{i+1}, y_{i+1}) + f(t_{i}, y_{i}) \right) \\
y_0 = a
\end{cases}
$$

\end{frame}

\begin{frame}[fragile]
\frametitle{Методы Рунге Кутты}

$$y_{n+1} - y_{n} = \tau \sum_{i=1}^{m} b_i k_i$$

где $\sum_{i=1}^{m} {b_i} = 1$ (веса), $k_i$ --- некоторое значение функции $f$ на интервале $[t_n; t_{n+1}]$.

$$k_i = f(t_n + c_i \tau, y_n + \tau \sum_{j = 1}^{m}{a_{ij}} k_j)$$
где $\sum_{j=1}^{m} {a_{i, j}} = c_i$ (веса)
\end{frame}


\begin{frame}[fragile]
\frametitle{Методы Рунге Кутты}
Таблица Бутчера

\begin{center}
\begin{tabular}{c|cccc}
$c_1$ & $a_{1,1}$ & $a_{1,1}$ & $\ldots$ & $a_{1,1}$ \\
$c_2$ & $a_{2,1}$ & $a_{2,2}$ & $\ldots$ & $a_{2,n}$ \\
$\ldots$ & $\ldots$ & $\ldots$ & $\ldots$ & $\ldots$ \\
$c_n$ & $a_{n,1}$ & $a_{n,2}$ & $\ldots$ & $a_{n,n}$ \\
\hline
1 & $b_{1}$ & $b_{2}$ & $\ldots$ & $b_{n}$ \\
\end{tabular}
\end{center}

\end{frame}


\begin{frame}[fragile]
\frametitle{Методы Рунге Кутты}

\begin{center}
\begin{tabular}{c|cccc}
$0$ & 0 & 0 \\
0.5 & 0.5 & 0 \\
\hline
1 & 0 & 1 \\
\end{tabular}

Вспомогательные значения
$$
\begin{cases}
k_1 = f(t_n, y_n)\\
k_2 = f(t_n + \frac{\tau}{2}, y_n  + \frac{\tau}{2} k_1) \\
\end{cases}
$$

Метод
$$
y_{n+1} - y_n = \tau k_2
$$
\end{center}

\end{frame}


\begin{frame}[fragile]
\frametitle{Методы Рунге Кутты}

Порядок аппроксимации

Первый: $$\sum_{i=1}^{m} b_i = 1$$

Второй: $$2 \sum_{i=1}^{m} b_i c_i = 1$$

Третий: $$3 \sum_{i=1}^{m} b_i c_i^2 = 1$$  
$$6 \sum_{i=1}^{m} b_i \sum_{j=1}^{m} a_{i, j} c_j = 1$$

\end{frame}

\end{document}
