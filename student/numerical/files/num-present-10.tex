\documentclass[10pt]{beamer}
\usetheme{Madrid}
\usepackage[T2A]{fontenc}
\usepackage[utf8]{inputenc}
\usepackage[russian]{babel}
\usepackage{amsmath}
\usepackage{amsfonts}
\usepackage{amssymb}
\usepackage{comment}
\usepackage[all]{xy}
\usepackage{gnuplottex}

%\newtheorem{theorem}{Теорема}

\author{Баев А.Ж.}
\title{Введение в численные методы. \\ Дифференцирование и задача Коши}
\institute{Казахстанский филиал МГУ} 
\date{26 февраля 2019}
%\subject{} 

\usepackage{tikz, pgfplots}
\usepgfplotslibrary{fillbetween}
%\pgfplotsset{compat=1.10}


\usetikzlibrary{patterns} 
\usetikzlibrary{arrows, matrix, positioning, decorations.pathreplacing}

\usepackage{listings}
\lstset{language=python}
\lstset{basicstyle=\ttfamily}  
\lstset{keywordstyle=\color{blue}}
\lstset{frame=single}
\lstset{numbers=left}
\lstset{tabsize=4}



\usepackage{graphicx}

\begin{document}

\maketitle


\begin{frame}[fragile]
\frametitle{План на семестр}

\begin{enumerate}
\item СЛАУ (точные методы)
\item СЛАУ (итерационные методы)
\item решение нелинейных уравнений
\item интерполяция
\item аппроксимация
\item интегрирование
\item \textbf{дифференцирование}
\end{enumerate}
\end{frame}


\begin{frame}[fragile]
\frametitle{Краевая задача для дифференциального уравнения}
Дано $f(x)$, $u_L$ и $u_R$. Определить $u(x)$.
$$
\begin{cases}
u''(x) + f(x) = 0\\
u(0) = u_L\\
u(1) = u_R\\
\end{cases}
$$
\end{frame}

\begin{frame}[fragile]
\frametitle{Краевая задача для дифференциального уравнения}
Пример. Дан однородный стержень длины 1. $f(x) = 1 + x$ --- нагреватель. Температуры на концах равны $u_L = 1$, $u_R = 1$.
$$
\begin{cases}
u''(x) + (1 + 12 x^2) = 0\\
u(0) = 1\\
u(1) = 1\\
\end{cases}
$$

\begin{center}
\begin{tikzpicture}[scale=0.75]
\begin{axis}[ymin=0, ymax=16, xlabel=x]
\addplot[domain=0:1, red]{1 + 12*x*x};
\legend{Внешний нагреватель.};
\end{axis}
\end{tikzpicture}
\end{center}
\end{frame}

\begin{frame}[fragile]
\frametitle{Краевая задача для дифференциального уравнения}
Найдём решение
$$
\begin{cases}
u(x) = - x^4 - \frac{x^2}{2} + C_1 x + C_2\\
C_2 = 1\\
-1 - \frac12 + C_1 + C_2 = 1\\
\end{cases}
\Leftrightarrow
u(x) = - x^4 - \frac{x^2}{2} + \frac{3}{2} x + 1
$$

\begin{center}
\begin{tikzpicture}[scale=0.75]
\begin{axis}[ymin=0, ymax=2, xlabel=x]
\addplot[domain=0:1, blue]{- x^4 - x^2/2 + 3/2 * x + 1};
\legend{Аналитическое решение};
\addplot[only marks] coordinates {(0,1) (1,1)};
\end{axis}
\end{tikzpicture}
\end{center}
\end{frame}




\begin{frame}[fragile]
\frametitle{Краевая задача для дифференциального уравнения}
Введём равномерную сетку $x_i = i \cdot h$, где $h = \frac{1}{n}$ и $i = 0 \ldots n$. Апрроксимация $y_i = u(x_i)$.

$$
\begin{cases}
\frac{y_{i-1} - 2 y_i + y_{i+1}}{h^2} + f_i = 0,\\
y_0 = U_L,\\
y_{n} = U_R,\\
\end{cases}
\Leftrightarrow
\begin{cases}
-y_{i-1} + 2 y_i - y_{i+1} = h^2 f_i,\\
y_0 = U_L,\\
y_{n} = U_R,\\
\end{cases}
$$

Решение сводится к СЛАУ с трёхдиагональной матрицей:
$$
\begin{pmatrix}
1 & 0 & 0 & 0 & \ldots & 0 & 0 & 0 \\
-1 & 2 & -1 & 0 & \ldots & 0 & 0 & 0 \\
0 & -1 & 2 & -1 & \ldots & 0 & 0 & 0 \\
0 & 0 & -1 & 2 & \ldots & 0 & 0 & 0 \\
\ldots & \ldots & \ldots & \ldots & \ldots & \ldots & \ldots & \ldots \\
0 & 0 & 0 & 0 & \ldots & 2 & -1 & 0 \\
0 & 0 & 0 & 0 & \ldots & -1 & 2 & -1 \\
0 & 0 & 0 & 0 & \ldots & 0 & 0 & 1 \\
\end{pmatrix}
\begin{pmatrix}
y_0\\
y_1 \\
y_2 \\
y_3 \\\
\ldots\\
y_{n-2} \\
y_{n-1} \\
U_R
\end{pmatrix}
=
\begin{pmatrix}
U_L\\
h^2 f_1 \\
h^2 f_2 \\
h^2 f_3 \\\
\ldots\\
h^2 f_{n-2} \\
h^2 f_{n-1} \\
U_R
\end{pmatrix}
$$
\end{frame}

\begin{frame}[fragile]
\frametitle{Краевая задача для дифференциального уравнения}
\begin{lstlisting}
def f(x):
	return 1 + 12 * x * x

n = 10

u = np.array([0] + [0] + [-1] * (n-1))
d = np.array([1] + [2] * (n - 1) + [1])
l = np.array([-1] * (n-1) + [0] + [0])
A = np.array([u, d, l])

h = 1.0 / n
x = np.linspace(0, 1, n + 1)
b = h * h * np.vectorize(f)(x)
b[0] = 1
b[n] = 1

y = sl.solve_banded((1, 1), np.array([u, d, l]), b)
\end{lstlisting}
\end{frame}


\begin{frame}[fragile]
\frametitle{Краевая задача для дифференциального уравнения}

\begin{center}
\begin{tikzpicture}[scale=0.75]
\begin{axis}[ymin=0, ymax=2, xlabel=x]
\addplot[domain=0:1, blue]{- x^4 - x^2/2 + 3/2 * x + 1};
\addplot[only marks] coordinates {(0,1) (1,1)};
\addplot[domain=0:1, black, only marks] coordinates{
(0.0, 1.0)
(0.1, 1.144)
(0.2, 1.277)
(0.3, 1.395)
(0.4, 1.492)
(0.5, 1.56)
(0.6, 1.588)
(0.7, 1.563)
(0.8, 1.469)
(0.9, 1.288)
(1.0, 1.0)
};
\legend{Аналитическое решение, Численное решение};
\end{axis}
\end{tikzpicture}
\end{center}

\end{frame}

\begin{frame}[fragile]
\frametitle{Уравнение переноса}
Дана функция начального профиля $u_0(x)$ и скорость переноса $C$. Найти положение профиля в произвольный момент времени $T$.
$$
\begin{cases} 
\frac{\partial u(x,t)}{\partial t} + C \frac{\partial u(x,t)}{\partial x} = 0, x \in R, t > 0\\
u(x, 0) = u_0(x), x \in R\\
\end{cases}
$$

Например, $u_0(x) = \frac{1}{2 x^2 + 1}$.
\begin{center}
\begin{tikzpicture}[scale=0.75]
\begin{axis}
[xlabel=x]
\addplot[blue, samples=100] {1 / (2*x*x+1)};
\legend{t=0};
\end{axis}
\end{tikzpicture}
\end{center}
\end{frame}

\begin{frame}[fragile]
\frametitle{Уравнение переноса}

$$\frac{\partial u(x,t)}{\partial t} + C \frac{\partial u(x,t)}{\partial x} = 0$$

Аналитическое решение
$$u(x, t) = u_0(x - C t)$$

Например, $u_0(x) = \frac{1}{2 x^2 + 1}$, $C = 1$, то есть $u(x, t) = \frac{1}{2 (x-t)^2 + 1}$.
\begin{center}
\begin{tikzpicture}[scale=0.75]
\begin{axis}
[xlabel=x]
\addplot[blue, samples=100] {1 / (2*(x-0)*(x-0)+1)};
\addplot[green, samples=100] {1 / (2*(x-1)*(x-1)+1)};
\addplot[red, samples=100] {1 / (2*(x-2)*(x-2)+1)};
\legend{t=0, t=1, t=2};
\end{axis}
\end{tikzpicture}
\end{center}
\end{frame}


\begin{frame}[fragile]
\frametitle{Уравнение переноса}
Построим численное решение на двумерной равномерной сетке:
$x_i = i \cdot h$, $t^k = k \cdot \tau$, где $h = \frac{1}{n}$, $\tau = \frac{T}{m}$. Аппроксимация решения $y_i^k \approx u(x_i, t^k)$.

\begin{center}
\begin{tikzpicture}
\draw[rounded corners=1, line width=10pt, line cap=round] (5, 4) -- (5, 3) -- (4, 3);

\draw (0, 0) grid (10, 4);
\foreach \x in {0,...,10}{
	\fill (\x, 0) circle (0.1);
}
\node at (0, -0.3) {0};
\node at (1, -0.3) {1};
\node at (2, -0.3) {2};
\node at (3, -0.3) {...};
\node at (4, -0.3) {i-1};
\node at (5, -0.3) {i};
\node at (6, -0.3) {i+1};
\node at (7, -0.3) {i+2};
\node at (8, -0.3) {...};
\node at (9, -0.3) {n-1};
\node at (10, -0.3) {n};

\node at (-0.5, 0) {0};
\node at (-0.5, 1) {1};
\node at (-0.5, 2) {...};
\node at (-0.5, 3) {k};
\node at (-0.5, 4) {k+1};

\foreach \x in {0,...,10}{
	\foreach \y in {1,...,4}
		\draw[fill=white] (\x, \y) circle (0.1);
}
\end{tikzpicture}
\end{center}

$$
\begin{cases} 
\frac{\partial u(x,t)}{\partial t} + C \frac{\partial u(x,t)}{\partial x} = 0, x \in R, t > 0\\
u(x, 0) = u_0(x), x \in R\\
\end{cases}
\Rightarrow
\begin{cases} 
\frac{y^{k+1}_i - y^k_i}{\tau} + C \frac{y^{k}_{i} - y^k_{i-1}}{h} = 0, i = 0..n-1\\
u_i^k = u_0(x_i)\\
\end{cases}
$$
\end{frame}

\begin{frame}[fragile]
\frametitle{Уравнение переноса}
Параметры задачи:
\begin{lstlisting}
def u0(x):
    return 1.0 / (1 + 2 * x * x)
C = 1.0
T = 2.0
L, R = -5.0, 5.0
\end{lstlisting}

Параметры метода:
\begin{lstlisting}
n = 40
m = 40
h = (R - L) / n
tau = T / m
\end{lstlisting}

Сетки:
\begin{lstlisting}
x = np.linspace(L, R, n + 1)
t = np.linspace(0.0, T, m + 1)
y = np.zeros((m + 1, n + 1))
\end{lstlisting}
\end{frame}

\begin{frame}[fragile]
\frametitle{Уравнение переноса}
$$
\begin{cases}
y^{k+1}_i = y^k_i - \frac{C \tau}{h} ( y^{k}_{i+1} - y^k_i), i = 0..n-1\\
y_i^0 = u_0(x_i)\\
\end{cases}
$$

Метод:
\begin{lstlisting}
d = C * tau / h
y[0] = np.vectorize(u0)(x)
for k in range(m):
    for i in range(1, n + 1):
        y[k+1][i] = y[k][i] - d * (y[k][i] - y[k][i-1])
\end{lstlisting}

Точное решение
\begin{lstlisting}
def solution(x, t):
    return f(x - C * t)

vsolution = np.vectorize(solution, excluded=['t'])
u = np.zeros((m + 1, n + 1))
for k in range(m):
	u[k] = vsolution(x, tau * k)
\end{lstlisting}
\end{frame}



\begin{frame}[fragile]
\frametitle{Уравнение переноса}
Анимация

\begin{lstlisting}[showstringspaces=false]
import matplotlib.pyplot as plt
import matplotlib.animation as animation

def animate(k):
    plt.clf()
    plt.ylim(0, 1)
    plt.title(time = " + str(tau * k))
    plt.plot(x, y[k], marker='o')
    plt.legend("Numerical")
    plt.plot(x, u[k], marker='*')
    plt.legend("Analytical")

ani = animation.FuncAnimation(plt.figure(0), animate, 
                              frames=y.shape[0],
                              interval=100)
# ani.save('transfer.mp4')
plt.show()
\end{lstlisting}
\end{frame}


\begin{frame}[fragile]
\frametitle{Уравнение переноса}

Например, $u_0(x) = \frac{1}{2 x^2 + 1}$, $C = 1$, то есть $u(x, t) = \frac{1}{2 (x-t)^2 + 1}$.
\begin{center}
\begin{tikzpicture}
\begin{axis}
[xlabel=x,
legend style={at={(-0.7,0.8)},anchor=west}
]
\addplot[blue, samples=100] {1 / (2*(x-0)*(x-0)+1)};
\addplot[red, samples=100] {1 / (2*(x-2)*(x-2)+1)};
\addplot[blue, only marks] coordinates{
(-5.00,  0.02)
(-4.75,  0.02)
(-4.50,  0.02)
(-4.25,  0.03)
(-4.00,  0.03)
(-3.75,  0.03)
(-3.50,  0.04)
(-3.25,  0.05)
(-3.00,  0.05)
(-2.75,  0.06)
(-2.50,  0.07)
(-2.25,  0.09)
(-2.00,  0.11)
(-1.75,  0.14)
(-1.50,  0.18)
(-1.25,  0.24)
(-1.00,  0.33)
(-0.75,  0.47)
(-0.50,  0.67)
(-0.25,  0.89)
( 0.00,  1.00)
( 0.25,  0.89)
( 0.50,  0.67)
( 0.75,  0.47)
( 1.00,  0.33)
( 1.25,  0.24)
( 1.50,  0.18)
( 1.75,  0.14)
( 2.00,  0.11)
( 2.25,  0.09)
( 2.50,  0.07)
( 2.75,  0.06)
( 3.00,  0.05)
( 3.25,  0.05)
( 3.50,  0.04)
( 3.75,  0.03)
( 4.00,  0.03)
( 4.25,  0.03)
( 4.50,  0.02)
( 4.75,  0.02)
( 5.00,  0.02)
};
\addplot[red, only marks] coordinates{
(-5.00,  0.00)
(-4.75,  0.00)
(-4.50,  0.00)
(-4.25,  0.00)
(-4.00,  0.00)
(-3.75,  0.00)
(-3.50,  0.00)
(-3.25,  0.01)
(-3.00,  0.01)
(-2.75,  0.02)
(-2.50,  0.02)
(-2.25,  0.03)
(-2.00,  0.03)
(-1.75,  0.04)
(-1.50,  0.04)
(-1.25,  0.05)
(-1.00,  0.06)
(-0.75,  0.07)
(-0.50,  0.09)
(-0.25,  0.11)
( 0.00,  0.14)
( 0.25,  0.19)
( 0.50,  0.25)
( 0.75,  0.33)
( 1.00,  0.42)
( 1.25,  0.53)
( 1.50,  0.61)
( 1.75,  0.67)
( 2.00,  0.69)
( 2.25,  0.66)
( 2.50,  0.59)
( 2.75,  0.50)
( 3.00,  0.41)
( 3.25,  0.32)
( 3.50,  0.25)
( 3.75,  0.19)
( 4.00,  0.15)
( 4.25,  0.11)
( 4.50,  0.09)
( 4.75,  0.07)
( 5.00,  0.06)
};
\legend{Аналитика (t=0), Аналитика (t=2), Численное (t=0), Численное (t=2)};
\end{axis}
\end{tikzpicture}
\end{center}
\end{frame}



\begin{frame}[fragile]
\frametitle{Погрешность и устойчивость}

Погрешность аппроксимации (оператора):
$$ O(\tau + h)$$

Условие устойчивости:
$$ \tau < \frac{h}{2 C}$$

Погрешность решения:
$$ O(\tau + h)$$
\end{frame}


\begin{frame}[fragile]
\frametitle{Уравнение теплопроводности}
Дана функция начального профиля $u_0(x)$ и коэффициент теплопроводности $\mu$. Найти профиль в произвольный момент времени $T$.
$$
\begin{cases} 
\frac{\partial u(x,t)}{\partial t} - \mu \frac{\partial^2 u(x,t)}{\partial x^2} = 0, x \in R, t > 0\\
u(x, 0) = u_0(x), x \in R\\
\end{cases}
$$

Например, $u_0(x) = \frac{1}{2 x^2 + 1}$.
\begin{center}
\begin{tikzpicture}[scale=0.75]
\begin{axis}
[xlabel=x]
\addplot[blue, samples=100] {1 / (2*x*x+1)};
\legend{t=0};
\end{axis}
\end{tikzpicture}
\end{center}
\end{frame}

\begin{frame}[fragile]
\frametitle{Уравнение теплопроводности}

$$\frac{\partial u(x,t)}{\partial t} - \mu \frac{\partial^2 u(x,t)}{\partial x^2} = 0$$

Аналитическое решение
$$u(x, t) = \frac{1}{2 \sqrt{\mu \pi t}} \int_{R} e^{-\frac{(x - y)^2}{4 \mu t}} u_0(y) dy$$

Например, $u_0(x) = e^{-\frac{x^2}{4}}$, $\mu = 1$, то есть $u(x, t) = \frac{1}{\sqrt{t+1}} e^{-\frac{x^2}{4(t+1)}}$.
\begin{center}
\begin{tikzpicture}[scale=0.75]
\begin{axis}
[xlabel=x, domain=-10:10]
\addplot[blue, samples=100] {e^(-x*x/4)};
\addplot[green, samples=100] {e^(-x*x/4/(1+1)) / sqrt(1 + 1)};
\addplot[red, samples=100] {e^(-x*x/4/(2+1)) / sqrt(2 + 1)};
\legend{t=0, t=1, t=2};
\end{axis}
\end{tikzpicture}
\end{center}
\end{frame}


\begin{frame}[fragile]
\frametitle{Уравнение теплопроводности}
Построим численное решение на двумерной равномерной сетке:
$x_i = i \cdot h$, $t^k = k \cdot \tau$, где $h = \frac{1}{n}$, $\tau = \frac{T}{m}$. Аппроксимация решения $y_i^k \approx u(x_i, t^k)$.

\begin{center}
\begin{tikzpicture}
\draw[rounded corners=1, line width=10pt, line cap=round] (5, 4) -- (5, 3);
\draw[rounded corners=1, line width=10pt, line cap=round] (4, 3) -- (6, 3);

\draw (0, 0) grid (10, 4);
\foreach \x in {0,...,10}{
	\fill (\x, 0) circle (0.1);
}
\node at (0, -0.3) {0};
\node at (1, -0.3) {1};
\node at (2, -0.3) {2};
\node at (3, -0.3) {...};
\node at (4, -0.3) {i-1};
\node at (5, -0.3) {i};
\node at (6, -0.3) {i+1};
\node at (7, -0.3) {i+2};
\node at (8, -0.3) {...};
\node at (9, -0.3) {n-1};
\node at (10, -0.3) {n};

\node at (-0.5, 0) {0};
\node at (-0.5, 1) {1};
\node at (-0.5, 2) {...};
\node at (-0.5, 3) {k};
\node at (-0.5, 4) {k+1};

\foreach \x in {0,...,10}{
	\foreach \y in {1,...,4}
		\draw[fill=white] (\x, \y) circle (0.1);
}
\end{tikzpicture}
\end{center}

$$
\begin{cases} 
\frac{\partial u(x,t)}{\partial t} - \mu \frac{\partial^2 u(x,t)}{\partial x^2} = 0, x \in R, t > 0\\
u(x, 0) = u_0(x), x \in R\\
\end{cases}
\Rightarrow
\begin{cases} 
\frac{y^{k+1}_i - y^k_i}{\tau} - \mu \frac{y^{k}_{i+1} - 2 y^k_i + y^k_{i-1}}{h} = 0, i = 1..n-1\\
u_i^k = u_0(x_i)\\
\end{cases}
$$
\end{frame}

\begin{frame}[fragile]
\frametitle{Уравнение теплопроводности}
Параметры задачи:
\begin{lstlisting}
def u0(x):
    return np.exp(-x*x/4)

mu = 1.0
T = 2.0
L, R = -10.0, 10.0
\end{lstlisting}

Параметры метода:
\begin{lstlisting}
n = 40
m = 40
h = (R - L) / n
tau = T / m
\end{lstlisting}

Сетки:
\begin{lstlisting}
x = np.linspace(L, R, n + 1)
t = np.linspace(0.0, T, m + 1)
y = np.zeros((m + 1, n + 1))
\end{lstlisting}
\end{frame}

\begin{frame}[fragile]
\frametitle{Уравнение теплопроводности}
$$
\begin{cases}
y^{k+1}_i = y^k_i + \frac{\mu \tau}{h^2} ( y^{k}_{i+1} - 2 y^k_i + y^{k}_{i-1}), i = 1..n-1\\
y_i^0 = u_0(x_i)\\
\end{cases}
$$

Метод:
\begin{lstlisting}
d = mu * tau / (h * h) 
y[0] = np.vectorize(u0)(x)
for k in range(m):
    for i in range(1, n):
        y[k+1][i] = y[k][i] + d * (y[k][i-1] - 2 * y[k][i] + y[k][i+1])
\end{lstlisting}

Точное решение
\begin{lstlisting}
def solution(x, t):
    return 1 / np.sqrt(t + 1) * np.exp(- x * x / 4 / (t + 1))

vsolution = np.vectorize(solution, excluded=['t'])
u = np.zeros((m + 1, n + 1))
for k in range(m):
	u[k] = vsolution(x, tau * k)
\end{lstlisting}
\end{frame}



\begin{frame}[fragile]
\frametitle{Уравнение теплопроводности}

Например,$u_0(x) = e^{-\frac{x^2}{4}}$, $\mu = 1$, то есть $u(x, t) = \frac{1}{\sqrt{t+1}} e^{-\frac{x^2}{4(t+1)}}$.
\begin{center}
\begin{tikzpicture}
\begin{axis}
[xlabel=x,
legend style={at={(-0.7,0.8)},anchor=west}
]
\addplot[blue, samples=100] {e^(-x*x/4)};
\addplot[red, samples=100] {e^(-x*x/4/(2+1)) / sqrt(2 + 1)};
\addplot[blue, only marks] coordinates{
(-10.00,  0.00)
(-9.50,  0.00)
(-9.00,  0.00)
(-8.50,  0.00)
(-8.00,  0.00)
(-7.50,  0.00)
(-7.00,  0.00)
(-6.50,  0.00)
(-6.00,  0.00)
(-5.50,  0.00)
(-5.00,  0.00)
(-4.50,  0.01)
(-4.00,  0.02)
(-3.50,  0.05)
(-3.00,  0.11)
(-2.50,  0.21)
(-2.00,  0.37)
(-1.50,  0.57)
(-1.00,  0.78)
(-0.50,  0.94)
( 0.00,  1.00)
( 0.50,  0.94)
( 1.00,  0.78)
( 1.50,  0.57)
( 2.00,  0.37)
( 2.50,  0.21)
( 3.00,  0.11)
( 3.50,  0.05)
( 4.00,  0.02)
( 4.50,  0.01)
( 5.00,  0.00)
( 5.50,  0.00)
( 6.00,  0.00)
( 6.50,  0.00)
( 7.00,  0.00)
( 7.50,  0.00)
( 8.00,  0.00)
( 8.50,  0.00)
( 9.00,  0.00)
( 9.50,  0.00)
(10.00,  0.00)
};
\addplot[red, only marks] coordinates{
(-10.00,  0.00)
(-9.50,  0.00)
(-9.00,  0.00)
(-8.50,  0.00)
(-8.00,  0.00)
(-7.50,  0.01)
(-7.00,  0.01)
(-6.50,  0.02)
(-6.00,  0.03)
(-5.50,  0.05)
(-5.00,  0.07)
(-4.50,  0.11)
(-4.00,  0.15)
(-3.50,  0.21)
(-3.00,  0.27)
(-2.50,  0.34)
(-2.00,  0.41)
(-1.50,  0.48)
(-1.00,  0.53)
(-0.50,  0.57)
( 0.00,  0.58)
( 0.50,  0.57)
( 1.00,  0.53)
( 1.50,  0.48)
( 2.00,  0.41)
( 2.50,  0.34)
( 3.00,  0.27)
( 3.50,  0.21)
( 4.00,  0.15)
( 4.50,  0.11)
( 5.00,  0.07)
( 5.50,  0.05)
( 6.00,  0.03)
( 6.50,  0.02)
( 7.00,  0.01)
( 7.50,  0.01)
( 8.00,  0.00)
( 8.50,  0.00)
( 9.00,  0.00)
( 9.50,  0.00)
(10.00,  0.00)
};
\legend{Аналитика (t=0), Аналитика (t=2), Численное (t=0), Численное (t=2)};
\end{axis}
\end{tikzpicture}
\end{center}
\end{frame}



\begin{frame}[fragile]
\frametitle{Погрешность и устойчивость}

Погрешность аппроксимации (оператора):
$$ O(\tau + h^2)$$

Условие устойчивости:
$$ \tau < \frac{h^2}{2 \mu}$$

Погрешность решения:
$$ O(\tau + h^2)$$
\end{frame}
\end{document}
