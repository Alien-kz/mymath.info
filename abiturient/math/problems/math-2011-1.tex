\documentclass[11pt, a5paper]{article}

\usepackage[T2A]{fontenc}		%cyrillic output
\usepackage[utf8]{inputenc}		%cyrillic output
\usepackage[english, russian]{babel}	%word wrap
\usepackage{amssymb, amsfonts, amsmath}	%math symbols
\usepackage{mathtext}			%text in formulas
\usepackage{geometry}			%paper format attributes
\usepackage{fancyhdr}			%header
\usepackage{graphicx}			%input pictures
\usepackage[inline]{enumitem}	%itemize in row
\usepackage{tikz}				%draw pictures
\usetikzlibrary{patterns}		%draw pictures: fill
\usepackage{wrapfig}			%text around figure

\geometry{left=1cm, right=1cm, top=2cm, bottom=1cm, headheight=15pt}
\sloppy							%correct overfull

\newcommand{\head}[4]
{
	\pagestyle{fancy}
	\fancyhf{}
	\chead{#3, #4}

	\begin{center}
	\begin{large}
	#1 --- #2
	\end{large}
	\end{center}

}

\newcommand{\variant}[2]{
	\begin{center}
	\textit{Вариант #2}
	\end{center}
}

\newcommand{\unit}[1]{
	\text{\textit{ #1}}
}
\newcommand{\units}[2]{ 
	\frac{\text{\textit{#1}}}{\text{\textit{#2}}}
}

\begin{document}

\head{Вступительный экзамен по математике}{2011}{Казахстанский филиал МГУ имени М. В. Ломоносова}{г. Астана}

\variant{2011}{1}

\begin{enumerate}[wide]
\item Какие из чисел $2$, $\frac34$, $\sqrt{7}+2$, $\sqrt7-2$ являются корнями уравнения
$$4x^3 + 9 = 19 x^2 ?$$

\item Представьте число $\sqrt{33}$ в виде десятичной дроби с точностью до~$0{,}1$.

\item Решите уравнение
$$ \cos \left(\frac{3\pi}{2} + 2x \right) - \sin(\pi + 4x) = \sin{4x} + \sin{x} .$$

\item Решите систему уравнений
$$
\begin{cases}
4^x \cdot 32^y = 256,\\
\sqrt{2x-2} = y.
\end{cases}
$$

\item В арифметической прогрессии 34 члена, и разность этой прогрессии равна 12. Сумма всех членов прогрессии в 4 раза больше, чем сумма членов, стоящих на нечетных местах. Найдите первый член этой прогрессии.

\item В трапеции, описанной около окружности радиуса 4, разность длин боковых сторон равна 4, а длина средней линии равна 12. Найдите длины сторон трапеции.

\item Решите неравенство
$$\frac{\log_2(x+6) \cdot \log_5(x+5)}{x+4} \leqslant \frac{\log_5(x+6) \cdot \log_2(x+5)}{x+3} .$$

\item В пирамиде $ABCD$: $AB = 1$, $AC = 2$, $AD = 3$, $BC = \sqrt{5}$, $BD = \sqrt{10}$, $CD = \sqrt{13}$. Найдите радиус шара, вписанного в пирамиду $ABCD$.

\end{enumerate}

\newpage

\end{document} 
