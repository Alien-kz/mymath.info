\documentclass[11pt, a5paper]{article}

\usepackage[T2A]{fontenc}		%cyrillic output
\usepackage[utf8]{inputenc}		%cyrillic output
\usepackage[english, russian]{babel}	%word wrap
\usepackage{amssymb, amsfonts, amsmath}	%math symbols
\usepackage{mathtext}			%text in formulas
\usepackage{geometry}			%paper format attributes
\usepackage{fancyhdr}			%header
\usepackage{graphicx}			%input pictures
\usepackage[inline]{enumitem}	%itemize in row
\usepackage{tikz}				%draw pictures
\usetikzlibrary{patterns}		%draw pictures: fill
\usepackage{wrapfig}			%text around figure

\geometry{left=1cm, right=1cm, top=2cm, bottom=1cm, headheight=15pt}
\sloppy							%correct overfull

\newcommand{\head}[4]
{
	\pagestyle{fancy}
	\fancyhf{}
	\chead{#3, #4}

	\begin{center}
	\begin{large}
	#1 --- #2
	\end{large}
	\end{center}

}

\newcommand{\variant}[2]{
	\begin{center}
	\textit{Вариант #2}
	\end{center}
}

\newcommand{\unit}[1]{
	\text{\textit{ #1}}
}
\newcommand{\units}[2]{ 
	\frac{\text{\textit{#1}}}{\text{\textit{#2}}}
}

\begin{document}

\head{Вступительный экзамен по математике}{2018}{Казахстанский филиал МГУ имени М. В. Ломоносова}{г.~Астана}

\variant{2018}{1}

\begin{enumerate}[wide]
\item Какое целое число задано выражением $\frac{\sqrt{8}\cdot \left( \frac{5}{3} + \frac{1}{5} \right)}{\left( \frac{2}{3} - \frac{1}{5} \right) \cdot \sqrt{32} }$?

\item Решить уравнение:
$$ \sqrt{10 x + 6} = 5 x - 9.$$

\item Решить неравенство:
$$ \left( \frac{8}{27} \right)^{\frac{2}{x}} \leqslant \left( \frac{9}{4} \right)^{\frac{1}{3-x} }. $$

\item В геометрической прогрессии 50 членов (все положительные). Если просуммировать логарифмы по основанию 2 от каждого члена прогрессии, то получится 1325. Если вычислить сумму логарифмов по основанию 2 только первых 30 членов, то получится 495. Вычислите сумму первых 10 членов прогрессии.

\item Решите систему уравнений:
$$
\begin{cases}
5 \sin{y} - 3 \sqrt{5} \cos{x} = 7 - 2 \cos^2{y}, \\
\tg{x} = 2.
\end{cases}
$$

\item В треугольнике $ABC$ со сторонами: $AB = 4$, $BC = 5$, $AC = 6$ проведены высоты $AH_1$, $BH_2$, $CH_3$. Найдите отношение длин отрезков $H_1H_3 : H_2H_3$.

\item Найдите все значения параметра $a$, при которых уравнение
$$ |(2-a) x - a| = (2 - a) (x + 1)^2 + 2 a x - 2 x + 2 a $$
имеет ровно одно решение.

\item В треугольной пирамиде $SABC$ длины всех ребер одинаковы. Точка $M$ в пространстве такова, что $MA = MB = MC = \sqrt{3}$ см и прямая $AM$ пересекается с высотой треугольника $SBC$, опущенной из вершины $B$. Найдите объем пирамиды $SABC$.

\end{enumerate}

\newpage


\end{document} 
