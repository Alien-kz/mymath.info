\documentclass[11pt, a5paper]{article}

\usepackage[T2A]{fontenc}		%cyrillic output
\usepackage[utf8]{inputenc}		%cyrillic output
\usepackage[english, russian]{babel}	%word wrap
\usepackage{amssymb, amsfonts, amsmath}	%math symbols
\usepackage{mathtext}			%text in formulas
\usepackage{geometry}			%paper format attributes
\usepackage{fancyhdr}			%header
\usepackage{graphicx}			%input pictures
\usepackage[inline]{enumitem}	%itemize in row
\usepackage{tikz}				%draw pictures
\usetikzlibrary{patterns}		%draw pictures: fill
\usepackage{wrapfig}			%text around figure

\geometry{left=1cm, right=1cm, top=2cm, bottom=1cm, headheight=15pt}
\sloppy							%correct overfull

\newcommand{\head}[4]
{
	\pagestyle{fancy}
	\fancyhf{}
	\chead{#3, #4}

	\begin{center}
	\begin{large}
	#1 --- #2
	\end{large}
	\end{center}

}

\newcommand{\variant}[2]{
	\begin{center}
	\textit{Вариант #2}
	\end{center}
}

\newcommand{\unit}[1]{
	\text{\textit{ #1}}
}
\newcommand{\units}[2]{ 
	\frac{\text{\textit{#1}}}{\text{\textit{#2}}}
}

\begin{document}

\head{Вступительный экзамен по математике}{2012}{Казахстанский филиал МГУ имени М. В. Ломоносова}{г.~Астана}

\variant{2012}{1}

\begin{enumerate}[wide]
\item Определите, какие из чисел являются целыми, и вычислите эти целые числа:
\begin{enumerate}
\item[a)] $\frac{19}{7} + \frac{7}{5} - \frac{4}{35}$;
\item[б)] $\sqrt{22} \cdot \sqrt{33} \cdot \sqrt{6}$;
\item[в)] $\frac{2{,}9 \cdot 3{,}4}{4{,}93}$. 
\end{enumerate}

\item Решите уравнение
$$\sqrt{x^2 + 24} = 6 - x^2 .$$

\item Решите уравнение
$$ \left( x - \frac{\pi}{2} \right)^2 \left| \cos{x} + \sin{x} \right|= \frac{\pi^2}{4} \left( \cos{x} + \sin{x} \right) .$$

\item В арифметической прогрессии десятый член больше пятого члена на 15 и больше второго члена в 13 раз. Найдите сумму всех членов этой прогрессии, начиная с сотого члена и заканчивая двухсотым.

\item Решите неравенство
$$\log_7{x} \leqslant  5 + 2 \log_{\sqrt{x}} \left( \frac{1}{7} \right ) .$$

\item В выпуклом шестиугольнике все углы равны $120^{\circ}$ и четыре последовательные стороны имеют длины 2, 3, 3, 4. Найдите площадь шестиугольника.

\item Найдите все значения, которые принимает функция
$$f(x) = \frac{2 x^2 + x + 1}{3x^2 - x + 1} .$$

\item Кусок сыра в форме правильной четырехугольной пирамиды $SABCD$ ($S$ --- вершина пирамиды) разрезали одним плоским разрезом, который проходит через ребро $AB$ и делит ребро $SC$ в отношении $1:3$, считая от вершины $S$. Найдите отношение объемов полученных кусков сыра.

\end{enumerate}

\newpage


\end{document} 
