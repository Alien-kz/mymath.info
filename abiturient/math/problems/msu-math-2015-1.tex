\documentclass[11pt, a5paper]{article}

\usepackage[T2A]{fontenc}
\usepackage[utf8]{inputenc}
\usepackage[english, russian]{babel}
\usepackage{amssymb}
\usepackage{amsfonts}
\usepackage{amsmath}
\usepackage{mathtext}

\usepackage{comment}
\usepackage{geometry}
\geometry{left=1cm, right=1cm, top=1cm, bottom=1cm}
\usepackage[inline]{enumitem}

\usepackage{graphicx}
\usepackage{tikz}
\usetikzlibrary{patterns}

\usepackage{wrapfig}
\usepackage{fancybox,fancyhdr}
\sloppy

\setlength{\headheight}{28pt}
\newcommand{\variant}[2]{
	\begin{center}
	\textit{Вариант #2}
	\end{center}
}

\newcommand{\unit}[1]{\text{\textit{ #1}}}
\newcommand{\units}[2]{ \frac{\text{\textit{#1}}}{\text{\textit{#2}}}}

\newcommand{\head}[4]
{
	\fancyhf{}
	\pagestyle{fancy}
	\chead{#3, #4}

	\begin{center}
	\begin{large}
	#1 --- #2 \\
	\end{large}
	\end{center}

}

\begin{document}

\head{Вступительный экзамен по математике}{2015}{Казахстанский филиал МГУ имени М. В. Ломоносова}{г. Астана}

\variant{2015}{1}

\begin{enumerate}[wide]
\item Какое из чисел больше и почему: 4,5 или $\sqrt{\frac{21}{8}} + \frac{17}{6}$ ?

\item Решите уравнение
$$(x^2 - 8x + 16)(x^2 - 8x + 18) - 24 = 0 .$$

\item Решите уравнение
$$\sqrt{24} \cos{x} = \sqrt{11 \cos{x} - \cos{2x}} .$$

\item Решите систему уравнений
$$
\begin{cases}
y^2 + 2xy = 40x,\\
16x^2 + 8xy = 5y .
\end{cases}
$$

\item Решите неравенство
$$ \frac{\log_{25} \left(7 - \frac{x}{2} \right) }{\log_{125} (22 - x) } \leqslant \frac{3}{4} .$$

\item В треугольнике длины двух сторон равны 4 и 5, а длина биссектрисы угла между этими сторонами равна $\frac{20}{9}$. Найдите площадь этого треугольника.

\item Найдите все значения параметра $a$, при которых уравнение
$$(x+1)^4 - (a+3) (x^2 + 2x) + a^2 + 3a + 1 = 0$$
имеет 4 различных корня, образующих арифметическую прогрессию.

\item В правильной шестиугольной пирамиде с вершиной $S$ и основанием $ABCDEF$ площадь сечения $SAC$ относится к площади боковой грани $SAB$ как \mbox{$\sqrt{51} : \sqrt{19}$}. Сторона основания равна 3. Найти объем данной шестиугольной пирамиды.

\end{enumerate}

\newpage


\end{document} 
