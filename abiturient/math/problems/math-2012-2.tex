\documentclass[11pt, a5paper]{article}

\usepackage[T2A]{fontenc}		%cyrillic output
\usepackage[utf8]{inputenc}		%cyrillic output
\usepackage[english, russian]{babel}	%word wrap
\usepackage{amssymb, amsfonts, amsmath}	%math symbols
\usepackage{mathtext}			%text in formulas
\usepackage{geometry}			%paper format attributes
\usepackage{fancyhdr}			%header
\usepackage{graphicx}			%input pictures
\usepackage[inline]{enumitem}	%itemize in row
\usepackage{tikz}				%draw pictures
\usetikzlibrary{patterns}		%draw pictures: fill
\usepackage{wrapfig}			%text around figure

\geometry{left=1cm, right=1cm, top=2cm, bottom=1cm, headheight=15pt}
\sloppy							%correct overfull

\newcommand{\head}[4]
{
	\pagestyle{fancy}
	\fancyhf{}
	\chead{#3, #4}

	\begin{center}
	\begin{large}
	#1 --- #2
	\end{large}
	\end{center}

}

\newcommand{\variant}[2]{
	\begin{center}
	\textit{Вариант #2}
	\end{center}
}

\newcommand{\unit}[1]{
	\text{\textit{ #1}}
}
\newcommand{\units}[2]{ 
	\frac{\text{\textit{#1}}}{\text{\textit{#2}}}
}

\begin{document}

\head{Вступительный экзамен по математике}{2012}{Казахстанский филиал МГУ имени М. В. Ломоносова}{г. Астана}

\variant{2012}{2}

\begin{enumerate}[wide]
\item Определите, какие из чисел являются целыми, и вычислите эти целые числа:
\begin{enumerate}
\item[a)] $\frac{11}{3} + \frac{9}{11} + \frac{17}{33}$;
\item[б)] $\sqrt{14} \cdot \sqrt{35} \cdot \sqrt{10}$;
\item[в)] $\frac{3{,}1 \cdot 3{,}8}{5{,}89}$. 
\end{enumerate}

\item Решите уравнение
$$\sqrt{x^2 + 12} = 8 - x^2 .$$

\item Решите уравнение
$$ \left( x + \frac{\pi}{2} \right)^2 \left| \cos{x} - \sin{x} \right|= \frac{\pi^2}{4} \left( \cos{x} - \sin{x} \right) .$$

\item В арифметической прогрессии девятый член больше четвертого члена на 10 и больше третьего члена в 5 раз. Найдите сумму всех членов этой прогрессии, начиная с двухсотого члена и заканчивая трехсотым.

\item Решите неравенство
$$3 \log_{\sqrt{x}}{11} \leqslant 8 + 2 \log_{11} \left( \frac{1}{x} \right ) .$$

\item В выпуклом шестиугольнике все углы равны $120^{\circ}$ и четыре последовательные стороны имеют длины 4, 5, 5, 6. Найдите площадь шестиугольника.

\item Найдите все значения, которые принимает функция
$$f(x) = \frac{3 x^2 + x + 1}{2 x^2 - x + 1} .$$

\item Кусок сыра в форме правильной четырехугольной пирамиды $SABCD$ ($S$ --- вершина пирамиды) разрезали одним плоским разрезом, который проходит через ребро $AB$ и делит ребро $SC$ в отношении $2:3$, считая от вершины $S$. Найдите отношение объемов полученных кусков сыра.

\end{enumerate}

\newpage

\end{document} 
