\documentclass[11pt, a5paper]{article}

\usepackage[T2A]{fontenc}		%cyrillic output
\usepackage[utf8]{inputenc}		%cyrillic output
\usepackage[english, russian]{babel}	%word wrap
\usepackage{amssymb, amsfonts, amsmath}	%math symbols
\usepackage{mathtext}			%text in formulas
\usepackage{geometry}			%paper format attributes
\usepackage{fancyhdr}			%header
\usepackage{graphicx}			%input pictures
\usepackage[inline]{enumitem}	%itemize in row
\usepackage{tikz}				%draw pictures
\usetikzlibrary{patterns}		%draw pictures: fill
\usepackage{wrapfig}			%text around figure

\geometry{left=1cm, right=1cm, top=2cm, bottom=1cm, headheight=15pt}
\sloppy							%correct overfull

\newcommand{\head}[4]
{
	\pagestyle{fancy}
	\fancyhf{}
	\chead{#3, #4}

	\begin{center}
	\begin{large}
	#1 --- #2
	\end{large}
	\end{center}

}

\newcommand{\variant}[2]{
	\begin{center}
	\textit{Вариант #2}
	\end{center}
}

\newcommand{\unit}[1]{
	\text{\textit{ #1}}
}
\newcommand{\units}[2]{ 
	\frac{\text{\textit{#1}}}{\text{\textit{#2}}}
}

\begin{document}

\head{Вступительный экзамен по математике}{2013}{Казахстанский филиал МГУ имени М. В. Ломоносова}{г.~Астана}

\variant{2013}{2}

\begin{enumerate}[wide]
\item Докажите, что число
$$\left(\sqrt[3]{4} - \sqrt[6]{2} \right)^3 \cdot \left(20 + 14 \sqrt{2} \right)$$
является целым и найдите это целое число.

\item Решите неравенство
$$\frac{11 \cdot |x + 3| - 6}{6 \cdot |x + 3| + 5} < 1 .$$

\item Решите уравнение
$$\cos(11 x - \pi) = 4 + 7 \sin{\frac{\pi - 11 x}{2}}.$$

\item В возрастающей арифметической прогрессии произведение шестого и седьмого членов на 44 больше, чем произведение четвертого и восьмого членов, и на 104 больше, чем произведение второго и девятого членов. Чему равна сумма первых 23 членов этой прогрессии?

\item Решите неравенство
$$(6 - 2 x) \cdot \log_{3^x-6} \sqrt{3} \leqslant 1 .$$

\item В трапеции $ABCD$ длина боковой стороны $CD$ равна $6$. Через точки $A$, $B$, $C$ проходит окружность, пересекающая основание трапеции $AD$ в точке $F$. Длина отрезка $BF$ равна $6 \sqrt{2}$. Угол $AFB$ равен $45^{\circ}$. Найдите длину основания $AD$.

\item Произведение двух натуральных чисел уменьшили на 25. Результат разделили на сумму исходных натуральных чисел с остатком. В частном получили 4, а в остатке 50. Найдите исходные натуральные числа.

\item Квадрат $ABCD$ со стороной 6 см является основанием двух пирамид $MABCD$ и $NABCD$, причем $MA$ и $NC$ --- высоты этих пирамид и точки $M$, $N$ лежат по одну сторону от плоскости $ABCD$. Сумма длин высот $MA$ и $NC$ равна 8 см, а объем общей части пирамид равен 18 см$^3$. Найдите отношение высот $MA$ и $NC$. 

\end{enumerate}

\newpage

\end{document} 
