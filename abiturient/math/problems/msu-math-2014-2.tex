\documentclass[11pt, a5paper]{article}

\usepackage[T2A]{fontenc}
\usepackage[utf8]{inputenc}
\usepackage[english, russian]{babel}
\usepackage{amssymb}
\usepackage{amsfonts}
\usepackage{amsmath}
\usepackage{mathtext}

\usepackage{comment}
\usepackage{geometry}
\geometry{left=1cm, right=1cm, top=1cm, bottom=1cm}
\usepackage[inline]{enumitem}

\usepackage{graphicx}
\usepackage{tikz}
\usetikzlibrary{patterns}

\usepackage{wrapfig}
\usepackage{fancybox,fancyhdr}
\sloppy

\setlength{\headheight}{28pt}
\newcommand{\variant}[2]{
	\begin{center}
	\textit{Вариант #2}
	\end{center}
}

\newcommand{\unit}[1]{\text{\textit{ #1}}}
\newcommand{\units}[2]{ \frac{\text{\textit{#1}}}{\text{\textit{#2}}}}

\newcommand{\head}[4]
{
	\fancyhf{}
	\pagestyle{fancy}
	\chead{#3, #4}

	\begin{center}
	\begin{large}
	#1 --- #2 \\
	\end{large}
	\end{center}

}

\begin{document}

\head{Вступительный экзамен по математике}{2014}{Казахстанский филиал МГУ имени М. В. Ломоносова}{г. Астана}

\variant{2014}{2}

\begin{enumerate}[wide]
\item К какому целому числу находится ближе всего на числовой оси число
$$\frac{7,2 \cdot 3,1 + 10,14}{3,2 \cdot 2,1 - 4,52} ?$$

\item Решите уравнение
$$\frac{\sqrt{22-4x-x^2}}{2-x}=1 .$$

\item Решите уравнение
$$7 \sin^2{5x} + \cos^2{10x} = 2.$$

\item Даны арифметическая прогрессия, в которой разность отлична от 0, и геометрическая прогрессия. Известно, что 2-й, 3-й и 11-й члены арифметической прогрессии совпадают, соответственно, с 1-м, 4-м и 7-м членами геометрической прогрессии. Найдите отношение суммы 9 первых членов геометрической прогрессии к сумме 9 первых членов арифметической прогрессии.

\item Решите неравенство
$$\log_{x^2} \left( 2x^2 + \frac{13}{2}x - \frac{15}{2} \right)^2 \leqslant 2 .$$

\item Высота $CH$ и биссектриса $BL$ в треугольнике $ABC$ пересекаются в точке $K$. При этом $CK=8$, $KH=4$, $BL=18$. Найдите длину стороны $AB$.

\item Найдите все значения $a$, при которых уравнение 
$$a \left(x^2 + x^{-2} \right) - (a + 2) \left(x + x^{-1}\right) + 7 = 0$$
не имеет решений.

\item В треугольной пирамиде $ABCD$ суммы трех плоских углов при каждой из вершин $B$ и $D$ равны $180^{\circ}$ и $AC=BD$. Радиус шара, вписанного в эту пирамиду, равен 3 см. Найдите длину высоты пирамиды, опущенной из вершины $A$.

\end{enumerate}

\newpage

\end{document} 
