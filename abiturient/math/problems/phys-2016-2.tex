\documentclass[11pt, a5paper]{article}

\usepackage[T2A]{fontenc}		%cyrillic output
\usepackage[utf8]{inputenc}		%cyrillic output
\usepackage[english, russian]{babel}	%word wrap
\usepackage{amssymb, amsfonts, amsmath}	%math symbols
\usepackage{mathtext}			%text in formulas
\usepackage{geometry}			%paper format attributes
\usepackage{fancyhdr}			%header
\usepackage{graphicx}			%input pictures
\usepackage[inline]{enumitem}	%itemize in row
\usepackage{tikz}				%draw pictures
\usetikzlibrary{patterns}		%draw pictures: fill
\usepackage{wrapfig}			%text around figure

\geometry{left=1cm, right=1cm, top=2cm, bottom=1cm, headheight=15pt}
\sloppy							%correct overfull

\newcommand{\head}[4]
{
	\pagestyle{fancy}
	\fancyhf{}
	\chead{#3, #4}

	\begin{center}
	\begin{large}
	#1 --- #2
	\end{large}
	\end{center}

}

\newcommand{\variant}[2]{
	\begin{center}
	\textit{Вариант #2}
	\end{center}
}

\newcommand{\unit}[1]{
	\text{\textit{ #1}}
}
\newcommand{\units}[2]{ 
	\frac{\text{\textit{#1}}}{\text{\textit{#2}}}
}

\begin{document}

\head{Вступительный экзамен по физике}{2016}{Казахстанский филиал МГУ имени М. В. Ломоносова}{г. Астана}

\variant{2016}{2}

\begin{enumerate}[wide]
\item Дайте определение равномерного движения материальной точки по окружности. Каково по величине и направлению ускорение материальной точки при ее равномерном движении по окружности?

\item Сформулируйте основные положения мо\-ле\-ку\-ляр\-но--ки\-не\-ти\-чес\-кой теории. Какова масса и размер молекул по порядку величины.

\item Дайте определение потенциала электростатического поля. Запишите формулу для потенциала электростатического поля точечного заряда.

\item Какие линзы называются тонкими? Приведите примеры построения изображений в собирающей и рассеивающей линзах.

\item \textbf{Задача.} Под каким углом $\alpha$ к горизонту нужно бросить камень, чтобы отношение максимальный высоты подъема камня к дальности его полета составило $n = \frac{\sqrt{3}}{4}$?

\item \textbf{Задача.} Рабочим телом теплового двигателя является $v = 1 \unit{моль}$ идеального одноатомного газа. Вначале газ сжимают без теплообмена с окружающей средой так, что он нагревается на $\Delta T = 20 \unit{K}$. Затем газ изотермически расширяется, при этом ему сообщается количество теплоты $Q = 500 \unit{Дж}$. Наконец, при постоянном объеме газ переводят в исходное состояние. Найдите КПД этого теплового двигателя. Универсальную газовую постоянную примите равной $R = 8,3 \frac{\unit{Дж}}{\unit{моль} \cdot \unit{K}}$.

\item \textbf{Задача.}

\noindent \begin{minipage}{0.6\linewidth}
Точки $A$, $B$, $C$ и $D$ расположены на прямой и разделены равными промежутками $L$ (см. рисунок). В точке $A$ помещен заряд $q_1 = 16 \cdot 10^{-9} \unit{Кл}$, в точке $B$ -- заряд $q_2 = 2 \cdot 10^{-9} \unit{Кл}$. Какой заряд $q_3$ надо поместить в точку $D$, чтобы напряженность поля в точке $C$ была равна нулю?
\end{minipage}
\begin{minipage}{0.3\linewidth}
\begin{tikzpicture}[x=10,y=10]
\draw (0,0)--(10,0);
\fill (2,0) circle (0.2) node[below]{$A$} node[above]{$q_1$};
\fill (4,0) circle (0.2) node[below]{$B$} node[above]{$q_2$};
\fill (6,0) circle (0.2) node[below]{$C$};
\fill (8,0) circle (0.2) node[below]{$D$} node[above]{$q_3$};
\end{tikzpicture}
\end{minipage}

\item \textbf{Задача.} Объектив фотоаппарата имеет фокусное расстояние $F = 6 \unit{см}$, а размеры кадра на фотоплёнке $a \times b = 24 \times 36 \unit{мм}$. На каком расстоянии $D$ нужно расположить объектив фотоаппарата от чертежа размерами $A \times B =  280 \times 720 \unit{мм}$, чтобы изображение чертежа на фотопленке занимало весь кадр? Объектив фотоаппарата считайте тонкой линзой.

\item \textbf{Задача.} Согласно теории Бора энергию на $n$-м энергетическом уровне атом водорода можно представить в виде $E_n = -\frac{13,6}{n^2} \unit{эВ}$ ($1 \unit{эВ} = 1,6 \cdot 10^{-19} \unit{Дж}$). Атом водорода, поглощая фотон с частотой $v = 2,94 \cdot 10^{15} \unit{Гц}$, переходит из основного состояния в возбужденное. Найдите максимальную длину волны $\lambda_{max}$, которую может излучить атом при всех возможных вариантах его возвращения в основное (первое) состояние. Скорость света $e = 3 \cdot 10^8 \units{м}{c}$. Постоянную Планка примите равной $h = 6,6 \cdot 10^{-24} \units{Дж}{c}$.

\item \textbf{Задача.} В результате $\beta$--распада радиоактивный изотоп калия ${}_{19}^{40} K$ превращается в изотоп кальция ${}_{20}^{40}C$. Период полураспада изотопа калия равен $T = 1,24 \cdot 10^9$ лет. Какая часть $m$ ядер калия превратится в ядро кальция за $t = 3,72 \cdot 10^9$ лет?

\end{enumerate}

\newpage


\end{document} 
