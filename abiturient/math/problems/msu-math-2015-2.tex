\documentclass[11pt, a5paper]{article}

\usepackage[T2A]{fontenc}		%cyrillic output
\usepackage[utf8]{inputenc}		%cyrillic output
\usepackage[english, russian]{babel}	%word wrap
\usepackage{amssymb, amsfonts, amsmath}	%math symbols
\usepackage{mathtext}			%text in formulas
\usepackage{geometry}			%paper format attributes
\usepackage{fancyhdr}			%header
\usepackage{graphicx}			%input pictures
\usepackage[inline]{enumitem}	%itemize in row
\usepackage{tikz}				%draw pictures
\usetikzlibrary{patterns}		%draw pictures: fill
\usepackage{wrapfig}			%text around figure

\geometry{left=1cm, right=1cm, top=2cm, bottom=1cm, headheight=15pt}
\sloppy							%correct overfull

\newcommand{\head}[4]
{
	\pagestyle{fancy}
	\fancyhf{}
	\chead{#3, #4}

	\begin{center}
	\begin{large}
	#1 --- #2
	\end{large}
	\end{center}

}

\newcommand{\variant}[2]{
	\begin{center}
	\textit{Вариант #2}
	\end{center}
}

\newcommand{\unit}[1]{
	\text{\textit{ #1}}
}
\newcommand{\units}[2]{ 
	\frac{\text{\textit{#1}}}{\text{\textit{#2}}}
}

\begin{document}

\head{Вступительный экзамен по математике}{2015}{Казахстанский филиал МГУ имени М. В. Ломоносова}{г. Астана}

\variant{2015}{2}

\begin{enumerate}[wide]
\item Какое из чисел больше и почему: 5,5 или $\sqrt{\frac{20}{7}} + \frac{23}{6}$ ?

\item Решите уравнение
$$(x^2 - 7x + 16)(x^2 - 7x + 19) - 28 = 0 .$$

\item Решите уравнение
$$\sqrt{20} \sin{x} = \sqrt{9 \sin{x} + \cos{2x}} .$$

\item Решите систему уравнений
$$
\begin{cases}
2y^2 + xy = 15x,\\
5x^2 + 10xy = 12y .
\end{cases}
$$

\item Решите неравенство
$$ \frac{\log_{64} \left(5 + \frac{x}{2} \right) }{\log_{16} (18 + x) } \leqslant \frac{1}{3} .$$

\item В треугольнике длины двух сторон равны 8 и 3, а длина биссектрисы угла между этими сторонами равна $\frac{24}{11}$. Найдите площадь этого треугольника.

\item Найдите все значения параметра $a$, при которых уравнение
$$(x-1)^4 - (a+4) (x^2 - 2x) + a^2 + 5a + 5 = 0$$
имеет 4 различных корня, образующих арифметическую прогрессию.

\item В правильной шестиугольной пирамиде с вершиной $S$ и основанием $ABCDEF$ площадь сечения $SAC$ относится к площади боковой грани $SAB$ как \mbox{$\sqrt{37} : \sqrt{13}$}. Сторона основания равна 2. Найти объем данной шестиугольной пирамиды.

\end{enumerate}

\newpage


\end{document} 
