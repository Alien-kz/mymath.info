\documentclass[11pt,a5paper]{report}
\usepackage[utf8]{inputenc}
\usepackage[T2A]{fontenc}
\usepackage[russian]{babel}
\usepackage{amsmath}
\usepackage{amsfonts}
\usepackage{amssymb}
\usepackage[left=1cm,right=1cm,top=1cm,bottom=2cm]{geometry}
\pagestyle{empty}
\begin{document}

Данные 13 проверочных вариантов составлены по материалам вступительных экзаменов в МГУ в период с 1970 по 1995 год. В качестве исходного источника взят <<Математика абитуриенту>> (Ткачук) и <<Алгебра 7-9>> (Галицкий). Уровень сложности соответствует варианту экзамена по математике в филиал МГУ с одной особенностью: вместо стереометрии добавлена еще одна задача по алгебре средней сложности.

\newpage

\begin{center}
Проверочная работа 1
\end{center}

\begin{enumerate}
\item Упростить выражение
$$
\frac{\left(b^{\frac{5}{6}} a^{-\frac{1}{6}} + b^{\frac{1}{3}} a^{\frac{1}{3}}\right)^2 + \left(b^{\frac{5}{6}} a^{-\frac{1}{6}} - b^{\frac{1}{3}} a^{\frac{1}{3}}\right)^2}
{(\sqrt[3]{a^{-1}} - \sqrt[3]{b^{-1}}) (\sqrt[3]{a^2} + \sqrt[3]{b^2} + \sqrt[3]{ab})} - 2a + \frac{4a^2}{a-b}
$$

\item Решить уравнение

$$ 4^{\sqrt{3x^2 - 2x} + 1} + 2 = 9 \cdot 2^{\sqrt{3x^2 - 2x}}$$

\item Решить неравенство

$$ \left( \frac{1}{3} \right)^{\log_{1/4} (x^2 - 3x + 1)} < 9 $$

\item Решить уравнение

$$ \frac{1}{2} (\cos^2{x} + \cos^2{2x}) - 1 = 2 \sin{2x} - 2 \sin{x} - \sin{x} \sin{2x}$$


\item Восьмой член арифметической прогрессии с ненулевой разностью равен 60. Известно, что первый, седьмой и двадцать пятый член образуют геометрическую прогрессию. Найдите знаменатель геометрической прогрессии.

\item Хорда $AB$ стягивает дугу окружности, равную $120^{\circ}$. Точка $C$ лежит на этой дуге, а точка $D$ лежит на хорде $AB$. При этом $AD = 2$, $BD = 1$, $DC = \sqrt{2}$. Найти площадь треугольника $ABC$.

\item Найти все действительные решения системы:
$$
\begin{cases}
|xy - 2| = 6 - x^2, \\
2 + 3 y^2 = 2 xy
\end{cases}
$$

\item Определить все действительные значения $a$, при каждом из которых уравнение имеет решение и найти эти решения:
$$
\cos^4{x} - (a + 2) \cos^2{x} - (a + 3) = 0
$$
\end{enumerate}
\newpage

\begin{enumerate}

\item (гео-70-1) $2(b+a)$

\item (хим-70-2) $\{1; -\frac{1}{3}\}$

\item (физ-70-2) $(\frac{3 - \sqrt{69}}{2}; \frac{3 - \sqrt{5}}{2}) \cup (\frac{3 + \sqrt{5}}{2}; \frac{3 + \sqrt{69}}{2})$

\item (физ-70-1) $\pm \frac{\pi}{3} + 2 \pi n$, $\pi m$.

\item (галицкий-12.145) $3$

\item (хим-70-4) $\frac{3}{2\sqrt{2}}$

\item (мм-70-2) $\{ (\sqrt{6}; \frac{\sqrt{6}}{3}), (-\sqrt{6}; -\frac{\sqrt{6}}{3}) \}$

\item (вмк-70-1) при $-3 \leqslant a \leqslant -2$ ответ $\pm arccos(\sqrt{a+3}) + \pi n$.
\end{enumerate}
\newpage


\begin{center}
Проверочная работа 2


\end{center}

\begin{enumerate}
\item Упростить выражение

$$ 5^{\log_{1/5}{\frac{1}{2}}} + \log_{\sqrt{2}} {\frac{4}{\sqrt{3} + \sqrt{7}}} + \log_{\frac{1}{2}} \frac{1}{10 + 2 \sqrt{21}}$$

\item Решить уравнение

$$2 \cos{2x} - 1 = (2 \cos{2x} + 1) \tg{x}$$

\item Решите уравнение

$$ \log_{\sin{2x}}(\tg{x} + \ctg{x}) = 1 - \log_{\sin{2x}}^2 {2}$$

\item Решите уравнение

$$ 4^{3x^2 + x} - 8 = 2 \cdot 8^{x^2 + \frac{x}{3}}$$

\item В арифметической прогрессии, содержащей 9 чисел, первый член равен 1, а сумма всех членов равна 369. Геометрическая прогрессия также имеет 9 членов, причем первый и последний её члены совпадают с соответвующими членами данной арифметической прогрессии. Найдите пятый член геометрической прогрессии.

\item Через середину $M$ стороны $BC$ параллелограмма $ABCD$, площадь которого равна 1, и вершину $A$ проведена прямая, пересекающая диагональ $BD$ в точке $O$. Найти площадь четырехугольника $OMCD$.

\item Решите неравенство

$$4x + 8 \sqrt{2 - x^2} > 4 + (x^2 - x) \cdot 2^x + 2^{x+1} \cdot x \sqrt{2 - x^2}$$

\item Найти все значения действительного параметра $\alpha$, для которого неравенство

$$ 4^x - \alpha \cdot 2^x - \alpha + 3 \leqslant 0$$

имеет хотя бы одно решение.

\end{enumerate}
\newpage

\begin{enumerate}

\item (гео-73-2) 6

\item (био-73-2) $-\frac{\pi}{4} + \pi n$, $(-1)^n \frac{\pi}{12} + \frac{\pi n}{2}$

\item (био-73-3) $(-1)^n \frac{\pi}{8} + \frac{\pi n}{2}$

\item (почв-73-5) $\{ \frac23; -1\}$

\item (галицкий-12.153) 9

\item (био-73-4) $\frac{5}{12}$

\item (мм-73-3) $(-1; \sqrt{2} ]$.

\item (био-73-5) $\alpha \geqslant 2$.

\end{enumerate}
\newpage

\begin{center}
Проверочная работа 1


\end{center}

\begin{enumerate}
\item Решите неравенство

$$ (x^2 + 3 x + 1) (x^2 + 3 x - 3) \geqslant 5$$

\item Решите уравнение

$$ \sin^2{x} - \cos{x} \cos{3x} = \frac{1}{4}$$

\item Решите неравенство

$$ \sqrt{x^2 + 4 x - 5} - 2 x + 3 > 0$$

\item Решите уравнение

$$x + 27^{\frac52 | \log_{9 \sqrt{3}} {\sqrt[3]{x}} |} = \frac{10}{3} $$

\item Решите неравенство

$$\frac{2x^2 - 11 x + 15}{2^x - 6} < 0$$

\item Найти трехзначное число, если его цифры образуют геометрическую прогрессию со знаменателем, отличным от единицы, а цифры числа, меньшего на 200, образуют арифметическую прогрессию.

\item В треугольнике $ABC$ известны стороны $AB = 6$, $BC = 4$, $CA = 8$. Биссектриса угла $C$ пересекает сторону $AB$ в точке $D$. Через точки $A$, $D$, $C$ проведена окружность, пересекающая сторону $BC$ в точке $E$. Найти площадь $ADE$.

\item Найти все целые положительные решения уравнения

$$2x^2 + 2xy - x + y = 112.$$
\end{enumerate}

\newpage

\begin{enumerate}

\item (геофиз-74-1) $(-\infty; -4] \cup [-2; -1] \cup [1; +\infty)$

\item (гео-75-2) $\pm \frac{\pi}{6} + \pi n$

\item (вмк-75-1) $(-\infty; -5] \cup [1; \frac{8+\sqrt{22}}{3})$

\item (почв-75-3) $\{ \frac{1}{3}; \frac{5}{3} \}$

\item (физ-75-3) $(-\infty; \frac52) \cup (\log_2 6; 3)$

\item (галицкий-12.161) 842 или 248

\item (гео-общ-75-4) $\frac{3 \sqrt{15}}{2}$

\item (псих-75-5) $x=1, y=37$.

\end{enumerate}
\newpage

\begin{center}
Проверочная работа 2


\end{center}

\begin{enumerate}

\item Решите уравнение

$$ \cos 2x+4\sin^3 x = 1 $$

\item Решите уравнение

$$ 4\sqrt{x+1} = |2x-1|+3 $$

\item Решите неравенство

$$ \log_{1/\sqrt{5}}(6^{x+1}-36^x) \geqslant -2 $$

\item Решите уравнение

$$ \sqrt{5+4\sin x-4\cos^2 x} = 2 + \cos 2x $$

\item Сумма членов арифметической прогрессии и ее первый член положительны. Если увеличить разность этой прогрессии на 4, не меняя первого члена, то сумма ее членов увеличится в 3 раза. Если же первый член исходной прогрессии увеличить в 5 раз, не меняя ее разности, то сумма членов увеличится также в 3 раза. Найдите разность исходной прогрессии.

\item В треугольнике $ABC$ задана точка $M$ на стороне $AC$, соединенная с вершиной $B$ отрезком $MB$. Известно, что $AM = 6$, $MC = 2$, $\angle ABM = 60^{\circ}$, $\angle MBC = 30^{\circ}$. Найдите площадь треугольника $ABC$.

\item Решите неравенство

$$ \log_{x\sqrt[6]{3}}(3x^6+2x^2-6) > 6 $$

\item Найти все значения параметра $k$, при каждом из которых существует хотя бы одно общее решение у неравенств $x^2 + 4kx + 3k^2 > 1 + 2k$ и $x^2 + 2kx \leqslant 3k^2 - 8k + 4$

\end{enumerate}
\newpage

\begin{enumerate}

\item (физ-76-1) $\pi n, (-1)^n \frac{\pi}{6} + \pi n$

\item (гео-76-2) $\{0, 3\}$

\item (хим-77-3) $(-\infty, 0] \cup [\log_6 5, 1)$

\item (псих-76-4) $(-1)^n\arcsin\frac{\sqrt{5}-1}{2}+\pi n, -\frac{\pi}{2} + 2\pi n$

\item (галицкий-12.107) 1

\item (план.экон-76-3) $8\sqrt{3}$

\item (почв-76-5) $(\sqrt{3}, +\infty)$

\item (геофиз.геол-77-5) $(-\infty, \frac{1}{2})\cup(\frac{3}{2},+\infty)$
\end{enumerate}

Неплохие задачи:

\begin{itemize}
\item Тригонометрия: мехмат-76-1, общ.геол-76-3, био-76-1, почв-76-3

\item Показательные: общ.геол-77-1

\item Корни: био-77-1

\item Логарифмы: план.экон-76-1

\end{itemize}

\newpage

\begin{center}
Проверочная работа 3


\end{center}

\begin{enumerate}

\item Решите неравенство

$$(x-1)\sqrt{x^2-x-2}\geqslant 0$$

\item Решите уравнение

$$\sin 2x + \sin 6x = 3\cos^2 2x$$

\item Решите неравенство

$$\sqrt{\log_9(3x^2-4x+2)}+1>\log_3(3x^2-4x+2)$$

\item Решите неравенство

$$9^{\sqrt{x^2-3}}+3<28\times 3^{\sqrt{x^2-3}-1}$$

\item Решите уравнение

$$\log_{1-2x^2}x=\frac{1}{4}-\frac{3}{\log_{2}(1-2x^2)^4}$$

\item Диагонали трапеции $ABCD$ пересекаются в точке $E$. Найти площадь треугольника $BCE$, если длины оснований трапеции $AB=30$ см, $DC=24$ см, боковой стороны $AD=3$ см и $\angle DAB=60^{\circ}$. 

\item Сумма второго, четвертого и шестого членов арифметической прогрессии равна 18, а их произведение равно -168. Найдите первый член и разность прогрессии.

\item Найдите все значения $\alpha$ , при которых уравнение

$$x^2+\frac{6x}{\sqrt{\sin\alpha}}+\frac{9\sqrt{3}}{\cos\alpha}+36=0$$

имеет единственное решение.

\end{enumerate}

\newpage

\begin{enumerate}
\item (вмк-78-1) $\{-1\}\cup[2,+\infty)$

\item (хим-78-1) $\frac{\pi}{4}+\frac{\pi n}{2}, \frac{(-1)^n}{2}\arcsin\frac{3}{4}+\frac{\pi n}{2}$

\item (гео-78-4) $(-1, \frac{1}{3}] \cup [1, \frac{7}{3})$

\item (био-78-4) $(-\sqrt{7},-\sqrt{3}] \cup [\sqrt{3},\sqrt{7})$

\item (полит.экон-78-1) $\frac{1}{2}$

\item (почв-79-4) $10\sqrt{3}$ см$^2$

\item (галицкий-12.83) $a_1=-6, d=4$ или $a_1=18, d=-4$

\item (общ.геол-79-6) $\frac{5}{6}\pi +\pi n, \frac{\pi}{18}+2\pi n, \frac{13}{18}\pi + 2\pi n$
\end{enumerate}

\newpage

\begin{center}
Проверочная работа 4


\end{center}

\begin{enumerate}

\item Решите уравнение

$$x^2+4|x-3|-7x+11=0$$

\item Решите неравенство

$$\sqrt{-x^2+6x-5} > 8-2x$$

\item Решите уравнение

$$\sin \Bigl(2x-\frac{7\pi}{2}\Bigr)+\sin \Bigl(\frac{3\pi}{2}-8x\Bigr)+\cos 6x=1$$

\item Решите неравенство

$$\log_2^2(2-x)-8\log_{1/4}(2-x)\geqslant 5$$

\item Решите неравенство

$$(2^x+3\times 2^{-x})^{2\log_2 x - \log_2 (x+6)} > 1$$

\item В прямоугольном треугольнике $ABC$ из вершины $B$ прямого угла опущена высота $BD$ на гипотенузу $AC$. Известно, что $AB=13$, $BD=12$. Найти площадь треугольника $ABC$.

\item Найдите сумму членов геометрической прогрессии с пятнадцатого по двадцать первый включительно, если сумма первых семи членов прогрессии равна 14, а сумма первых четырнадцати ее членов равна 18.

\item Найти все целые значения параметра $k$, при каждом из которых уравнение $5-4\sin^2 x-8\cos^2\frac{x}{2}=3k$ имеет решения. Найти эти решения.
\end{enumerate}

\newpage

\begin{enumerate}
\item (геофиз-80-3) $\frac{3+\sqrt{13}}{2}, \frac{11-\sqrt{29}}{2}$

\item (био-80-3) $(3, 5]$

\item (фил-81-1) $\frac{\pi n}{3}, \frac{\pi}{8}+\frac{\pi n}{4}$

\item (фил-81-2) $(-\infty, 0] \cup [1\frac{31}{32}, 2)$

\item (био-81-4) $(3, +\infty)$

\item (вмк-80-3) 202.8

\item (галицкий-12.128)

\item (полит.экон-80-5) $\{-1,0,1\}$
\end{enumerate}

\newpage

\begin{center}
Проверочная работа 7


\end{center}

\begin{enumerate}

\item Решите уравнение

$$\sqrt{x+2}-\sqrt{x-1}=\sqrt{2x-3}$$

\item Решите неравенство

$$\log_3((x+2)(x+4))+\log_{1/3}(x+2)<\frac{1}{2}\log_{\sqrt{3}}7$$

\item Решите уравнение

$$\sqrt{5\sin x+\cos 2x}+2\cos x=0$$

\item Решите неравенство

$$8+6\times|3-\sqrt{x+5}|>x$$


\item Решите неравенство

$$5^x-3^{x+1}>2(5^{x-1}-3^{x-2})$$

\item Сумма трех чисел, составляющих арифметическую прогрессию, равна 15. Если к этим числам прибавить соответственно 1, 1 и 9, то получатся три числа, составляющих геометрическую прогрессию. Найдите исходные три числа.

\item В треугольнике $ABC$ величина угла $BAC$ равна $\frac{\pi}{3}$, длина высоты, опущенной из вершины $C$ на сторону $AB$ равна $\sqrt{3}$ см, а радиус окружности, описанной около треугольника $ABC$, равен 5 см. Найти длины сторон треугольника $ABC$.

\item При каких значениях параметра $a$ уравнение

$$(3a-1)x^2+2ax+3a-2=0$$

имеет два действительных корня.

\end{enumerate}

\newpage

\begin{enumerate}

\item (общ.геол-82-2) $2$

\item (полит.экон-82-3) $(-2, 3)$

\item (мехмат-82-3) $\frac{5}{6}\pi+2\pi n$

\item (био-83-3) $[-5, 20)$

\item (физ-82-4) $(3, +\infty)$

\item (галицкий-12.151) 1, 5, 9 или 17, 5, -7

\item (вмк-81-3) $AB=1+6\sqrt{2}, AC=2, BC=5\sqrt{3}$

\item (вмк-80-4) $(\frac{9-\sqrt{17}}{16}, \frac{1}{3}) \cup (\frac{1}{3}, \frac{9+\sqrt{17}}{16})$

\end{enumerate}

\newpage

\begin{center}
Проверочная работа 8


\end{center}

\begin{enumerate}

\item Решите неравенство

$$3|x-2|+|5x-4|\leqslant 10$$

\item Решите неравенство

$$\log_{1/2}^2(3x+1)>\log_{1/2}(3x+1)+6$$

\item Решите уравнение

$$9\cos 3x \cos 5x + 7 = 9\cos 3x \cos x + 12\cos 4x$$

\item Решите неравенство

$$7^{x-\frac{1}{8}x^2} < 7^{1-x} (\sqrt[8]{7})^{x^2} + 6$$

\item Решите уравнение

$$\sqrt{4+2\log_2\Bigl(1-\frac{8x}{(2x+1)^2}\Bigr)} = \log_2\Bigl(\frac{2x+1}{2x-1}\Bigr)+2\log_2\Bigl(\frac{1}{\sqrt{2}}\Bigr)$$

\item В треугольнике $ABC$ длина стороны $BC$ равна 4, сумма длин двух других сторон равна 6. Найти площадь треугольника $ABC$, если косинус угла $ACB$ равен $\frac{5}{12}$.

\item Сумма первых десяти членов арифметической прогрессии равна 155, а сумма первых двух членов геометрической прогрессии равна 9. Найдите эти прогрессии, если первый член арифметической прогрессии равен знаменателю геометрической прогрессии, а первый член геометрической прогрессии равен разности арифметической прогрессии.

\item При каждом $a$ решите уравнение

$$4^x - 2a(a+1)2^{x-1} + a^3 = 0$$

\end{enumerate}

\newpage

\begin{enumerate}

\item (полит.экон-84-3) $[-1, 0]$

\item (био-85-3) $(-\frac{1}{3}, \frac{7}{24}) \cup (1, +\infty)$

\item (вмк-84-2) $\pm\frac{1}{4}\arccos\frac{1}{6}+\frac{\pi n}{2}$

\item (хим-85-2) $(-\infty, 4-2\sqrt{2}) \cup (4+2\sqrt{2}, +\infty)$

\item (био-84-3) $\frac{3}{2}$

\item (геофиз-84-4) $\frac{8}{13}\sqrt{119}$

\item (галицкий-12.156) $2, 5, 8, \dots$ и $3, 6, 12, \dots$ или $\frac{25}{2}, \frac{79}{6}, \frac{83}{16}, \dots$ и $\frac{2}{3}, \frac{25}{3}, \frac{625}{5}, \dots$  

\item (физ-85-4) При $a > 0$ имеем $x=\log_2 a, x = 2\log_2 a$, при $a = 0$ решений нет, при $a < 0$ имеем $x=2\log_2 |a|$

\end{enumerate}

\newpage

\begin{center}
Проверочная работа 9


\end{center}

\begin{enumerate}

\item Решите уравнение

$$\sqrt{\frac{3}{4}-\cos x} = \sqrt{\frac{3}{4}-\cos 3x}$$

\item Решите неравенство

$$\sqrt{2x^2+15x-17} > x+3$$

\item Решите уравнение

$$\log_{5-x^2}(2x^2-8x-2) = 1 + \log_{5-x^2}2$$

\item Решите уравнение

$$4^{\sin x}+2^{5-2\sin x}=18$$

\item Сумма трех чисел, составляющих геометрическую прогрессию равна 14. Если от первого числа отнять 15, а второе и третье увеличить соответственно на 11 и 5, то полученные три числа составят арифметическую прогрессию. Найдите исходные три числа.

\item Внутри треугольника $ABC$ взята точка $K$. Известно, что $AK = 1$, $KC = \sqrt{3}$, а величины углов $AKC$, $ABK$ и $KBC$ равны $120^{\circ}, 15^{\circ}, 15^{\circ}$ соответственно. Найти длину отрезка $BK$.

\item Решите неравенство

$$\frac{1}{2}\log_{x-1}(x^2-8x+16)+\log_{4-x}(-x^2+5x-4) > 3$$

\item Решите уравнение

$$\sqrt{3x^2-1}+\sqrt{x^2-x+1} = \sqrt{3x^2+2x+1}+\sqrt{x^2+2x+4}$$

\end{enumerate}

\newpage

\begin{enumerate}

\item (псих-87-4) $\frac{\pi}{2}+\pi n, \pi + 2\pi n$

\item (фил-87-3) $(-\infty, -8.5) \cup (\frac{\sqrt{185}-9}{2}, +\infty)$

\item (фил-87-2) $-1$

\item (физ-87-2) $(-1)^n\frac{\pi}{6}+\pi n$

\item (галицкий-12.152) $18, -6, 2$ или $2, -6, 18$

\item (гео-86-4) $\sqrt{6-3\sqrt{3}}$

\item (полит.экон-87-4) $(2, \frac{5}{2}) \cup (\frac{5}{2}, 3)$

\item (геофиз-85-5) $-1$

\end{enumerate}

\newpage

\begin{center}
Проверочная работа 10


\end{center}

\begin{enumerate}

\item Решите уравнение

$$||3-x|-x+1|+x=6$$

\item Решите неравенство

$$\log_{x+2}(2x^2+x)\leqslant 2$$

\item Решите неравенство

$$3^x-3^{\frac{1}{2}-x} > \sqrt{3}-1$$

\item Решите уравнение

$$\cos 7x+\cos x = 2\cos 3x (\sin 2x -1)$$

\item Решите неравенство

$$2\log_{\sqrt{2}}2+\log_{\sqrt{2}}\left((2^{x^2-1}-\frac{1}{4}\right) < \log_{\sqrt{2}}31$$

\item В треугольнике $ABC$ проведена биссектриса $CD$, при этом величины углов $ADC$ и $CDB$ относятся как $7:5$. Найти длину $AD$, если известно, что $BC=1$, а угол $BAC$ равен $\frac{\pi}{6}$.

\item Сумма трех чисел, составляющих геометрическую прогрессию, равна 3, а сумма их квадратов равна 21. Найдите эти числа.

\item Найдите все значения параметра $a$, при которых уравнение
$$(a^2-6a+9)(2+2\sin x - \cos^2 x)+(12a-18-2a^2)(1+\sin x)+a+3=0$$
не имеет решений.
\end{enumerate}

\newpage

\begin{enumerate}

\item (экон-89-3) \{-2, 4\}

\item (почв-89-3) $(-2, -1)\cup(-1,-\frac{1}{2})\cup(0, 4]$

\item (гео-88-2) $(\frac{1}{2}, +\infty)$

\item (вмк-88-2) $\frac{\pi}{6}+\frac{\pi n}{3}, \frac{(-1)^n}{2}\arcsin\left(\frac{-1+\sqrt{17}}{4}\right)+\frac{\pi n}{2}$

\item (био-88-2) $(-2, 2)$

\item (геол-89-4) $3-\sqrt{3}$

\item (галицкий-12.122) $1, -2, 4$ или $4, -2, 1$

\item (геол-89-6) $(-\infty, -3)\cup(1, 6)$

\end{enumerate}

\newpage

\begin{center}
Проверочная работа 11


\end{center}

\begin{enumerate}

\item Решите уравнение

$$2\sin^2 x+\sin^2 2x = \frac{5}{4}-2\cos 2x$$

\item Решите уравнение

$$5\sqrt{1+|x^2-1|} = 3 + |5x+3|$$

\item Решите неравенство

$$\frac{2^{2+\sqrt{x-1}}-24}{2^{1+\sqrt{x-1}}-8} > 1$$

\item Решите неравенство

$$\log_{1/7}\log_{3}\frac{|-x+1|+|x+1|}{2x+1} \geqslant 0$$

\item Решите уравнение

$$\sqrt{4\cos 2x - 2\sin 2x} = 2\cos x$$

\item В арифметической прогрессии сумма четвертого, восьмого, девятнадцатого и двадцать третьего членов равна 30. Найдите сумму 26 первых членов прогрессии.

\item В прямоугольном треугольнике $ABC$ из вершины прямого угла $C$ проведены медиана $CM$ и высота $CH$. Найти отношение $AH:AM$, если $CM:CH=5:4$ и точка $H$ находится между точками $A$ и $M$.

\item Найти все пары действительных чисел $a$ и $b$, при которых уравнение
$$(3x-a^2+ab-b^2)^2+(2x^2-a^2-ab)^2+x^2+9=6x$$
имеет хотя бы одно решение $x$.

\end{enumerate}

\newpage

\begin{enumerate}

\item (хим-90-2) $\pm \frac{\pi}{3}+\pi n$

\item (экон-90-2) $(-\infty, -1]\cup\{\frac{1}{5}\}$

\item (гео-90-3) $[1, 5)\cup(10, +\infty)$

\item (экон-91-2) $[-\frac{1}{6}, \frac{1}{2})$

\item (мехмат-91-1) $2\pi n, -\frac{\pi}{4}+2\pi n$

\item (галицкий-12.90) $195$

\item (фил-90-3) $2:5$

\item (геол-90-5) $(3, 3), (-3, -3), (2\sqrt{3}, \sqrt{3}), (-2\sqrt{3}, -\sqrt{3})$

\end{enumerate}

\newpage

\begin{center}
Проверочная работа 12


\end{center}

\begin{enumerate}

\item Решите неравенство

$$\frac{\sqrt{x^2-5x+8}}{3-x} \geqslant 1$$

\item Решите неравенство

$$\sqrt{\sin x} > \sqrt{-\cos x}$$

\item Решите уравнение

$$\log_{x}(3x-2)-2=\sqrt{\log_{x}^2(3x-2)+4\log_{x}\left(\frac{x}{3x-2}\right)}$$

\item Решите неравенство

$$(x^2-8x+15)(2^{x-3}+2^{3-x}-2)^{-1}\sqrt{x-1} \leqslant 0$$

\item Найти первый член и разность арифметической прогрессии, если известно, что пятый и девятый члены дают в сумме $40$, а сумма седьмого и тринадцатого членов равна $58$.

\item В ромбе $ABCD$ угол при вершине $A$ равен $\frac{\pi}{3}$. Точка $N$ делит сторону $AB$ в отношении $AN:NB=2:1$. Определить тангенс угла $DNC$.

\item Найдите все значения $a$, при которых уравнение $$4^x+(a^2+5)2^x+9-a^2=0$$ не имеет решений.

\item Найдите все значения параметра $c$, при которых уравнение
$$|x^2-1|+|x^2-x-2|=x^2+3x+c$$
имеет ровно 3 различных решения.
\end{enumerate}

\newpage

\begin{enumerate}

\item (инст.стр.Азия.Африка-93-1) $[1, 3)$

\item (хим-92-2) $[\frac{\pi}{2}+2\pi n, \frac{3}{4}\pi +2\pi n)$

\item (инст.стр.Азия-Африка-93-4) $(\frac{2}{3}, 1)\cup(1, 2]$

\item (экон-92-2) $\{1\}\cup(3, 5]$

\item (физ-92-5) $a1=2, d=3$

\item (фил-92-3) $\frac{9}{11}\sqrt{3}$

\item (мехмат-93-2) $[-3, 3]$

\item (гео-92-5) $2, \frac{10}{3}$

\end{enumerate}

\newpage

\begin{center}
Проверочная работа 13


\end{center}

\begin{enumerate}

\item Решите уравнение

$$5\cos x + 2\sin x = 3$$

\item Решите неравенство

$$\sqrt{8\cdot 16^x-\frac{1}{2}\cdot 9^x} \leqslant 3\cdot 4^x-3^x$$

\item Решите уравнение

$$y^2+2\sqrt{y^2+3y-4}-4+3y=0$$

\item Решите неравенство

$$\log_{\cos x}\cos^2 x \geqslant \log_{\cos x-\frac{1}{2}}\left(\cos^2 x-\cos x-x^2-14x-\frac{51}{4}\right)$$

\item Прямоугольные треугольники $ABC$ и $ABD$ имеют общую гипотенузу $AB=5$. Точки $C$ и $D$ расположены по разные стороны от прямой, проходящей через точки $A$ и $B$, $BC=BD=3$. Точка $E$ лежит на $AC$, $EC=1$. Точка $F$ лежит на $AD$, $FD=2$. Найдите площадь пятиугольника $ECBDF$.

\item Найдите четыре целых числа, составляющих возрастающую арифметическую прогрессию, в которой наибольший член равен сумме квадратов остальных членов.

\item Найдите все значения параметра $p$, при которых уравнение $$x-2=\sqrt{-2(p+2)x+2}$$ имеет единственное решение.

\item Найдите все значения параметра $a$, при которых неравенство $$x^2+4x+6a|x+2|+9a^2\leqslant 0$$ имеет не более одного решения.

\end{enumerate}

\newpage

\begin{enumerate}

\item (физ-94-2) $\arccos\Bigl(\frac{5}{\sqrt{29}}\Bigr)\pm \arccos\Bigl(\frac{3}{\sqrt{29}}\Bigr)+2\pi n$

\item (мехмат-96-2) $[\log_{4/3}(3+\frac{1}{2}\sqrt{30}),+\infty)$

\item (геол-94-3) $\{-4, 1\}$

\item (вмк-95-4) $[-13, -4\pi)\cup(-4\pi, -\frac{11\pi}{3})\cup(-\frac{7\pi}{3}, -2\pi)\cup(-2\pi, -\frac{5\pi}{3})\cup(-\frac{\pi}{3}, -1]$

\item (хим-94-4) $\frac{228}{25}$

\item (галицкий-12.108) $-1, 0, 1, 2$

\item (менеджмент.экон-95-6) $(-\infty, -\frac{3}{2}]$

\item (инст.стр.Азия.Африка-95-6) $[\frac{2}{3}, +\infty)$

\end{enumerate}
\end{document}