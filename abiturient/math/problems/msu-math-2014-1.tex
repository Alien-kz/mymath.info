\documentclass[11pt, a5paper]{article}

\usepackage[T2A]{fontenc}
\usepackage[utf8]{inputenc}
\usepackage[english, russian]{babel}
\usepackage{amssymb}
\usepackage{amsfonts}
\usepackage{amsmath}
\usepackage{mathtext}

\usepackage{comment}
\usepackage{geometry}
\geometry{left=1cm, right=1cm, top=1cm, bottom=1cm}
\usepackage[inline]{enumitem}

\usepackage{graphicx}
\usepackage{tikz}
\usetikzlibrary{patterns}

\usepackage{wrapfig}
\usepackage{fancybox,fancyhdr}
\sloppy

\setlength{\headheight}{28pt}
\newcommand{\variant}[2]{
	\begin{center}
	\textit{Вариант #2}
	\end{center}
}

\newcommand{\unit}[1]{\text{\textit{ #1}}}
\newcommand{\units}[2]{ \frac{\text{\textit{#1}}}{\text{\textit{#2}}}}

\newcommand{\head}[4]
{
	\fancyhf{}
	\pagestyle{fancy}
	\chead{#3, #4}

	\begin{center}
	\begin{large}
	#1 --- #2 \\
	\end{large}
	\end{center}

}

\begin{document}

\head{Вступительный экзамен по математике}{2014}{Казахстанский филиал МГУ имени М. В. Ломоносова}{г. Астана}

\variant{2014}{1}

\begin{enumerate}[wide]
\item К какому целому числу находится ближе всего на числовой оси число
$$\frac{5,1 \cdot 4,2 + 11,76}{2,3 \cdot 2,2 - 2,46} ?$$

\item Решите уравнение
$$\frac{\sqrt{41-6x-x^2}}{3-x}=1 .$$

\item Решите уравнение
$$6 \sin^2{3x} + 2 \cos^2{6x} = 5 .$$

\item Даны арифметическая прогрессия, в которой разность отлична от 0, и геометрическая прогрессия. Известно, что 1-й, 2-й и 10-й члены арифметической прогрессии совпадают, соответственно, со 2-м, 5-м и 8-м членами геометрической прогрессии. Найдите отношение суммы 8 первых членов геометрической прогрессии к сумме 8 первых членов арифметической прогрессии.

\item Решите неравенство
$$\log_{x^2} \left( 5x^2 - \frac{20}{3}x - \frac{32}{3} \right)^2 \leqslant 2 .$$

\item Высота $AH$ и биссектриса $BL$ в треугольнике $ABC$ пересекаются в точке $K$. При этом $AK=4$, $KH=2$, $BL=11$. Найдите длину стороны $BC$.

\item Найдите все значения $a$, при которых уравнение 
$$a \left(x^2 + x^{-2} \right) - (a + 1) \left(x + x^{-1}\right) + 5 = 0$$
не имеет решений.

\item В треугольной пирамиде $ABCD$ суммы трех плоских углов при каждой из вершин $B$ и $C$ равны $180^{\circ}$ и $AD=BC$. Длина высоты пирамиды, опущенной из вершины $A$, равна 40 см. Найдите радиус шара, вписанного в эту пирамиду. 

\end{enumerate}

\newpage

\end{document} 
