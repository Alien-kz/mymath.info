\documentclass[11pt, a5paper]{article}

\usepackage[T2A]{fontenc}		%cyrillic output
\usepackage[utf8]{inputenc}		%cyrillic output
\usepackage[english, russian]{babel}	%word wrap
\usepackage{amssymb, amsfonts, amsmath}	%math symbols
\usepackage{mathtext}			%text in formulas
\usepackage{geometry}			%paper format attributes
\usepackage{fancyhdr}			%header
\usepackage{graphicx}			%input pictures
\usepackage[inline]{enumitem}	%itemize in row
\usepackage{tikz}				%draw pictures
\usetikzlibrary{patterns}		%draw pictures: fill
\usepackage{wrapfig}			%text around figure

\geometry{left=1cm, right=1cm, top=2cm, bottom=1cm, headheight=15pt}
\sloppy							%correct overfull

\newcommand{\head}[4]
{
	\pagestyle{fancy}
	\fancyhf{}
	\chead{#3, #4}

	\begin{center}
	\begin{large}
	#1 --- #2
	\end{large}
	\end{center}

}

\newcommand{\variant}[2]{
	\begin{center}
	\textit{Вариант #2}
	\end{center}
}

\newcommand{\unit}[1]{
	\text{\textit{ #1}}
}
\newcommand{\units}[2]{ 
	\frac{\text{\textit{#1}}}{\text{\textit{#2}}}
}

\begin{document}

\head{Вступительный экзамен по математике}{2017}{Казахстанский филиал МГУ имени М. В. Ломоносова}{г.~Астана}

\variant{2017}{2}

\begin{enumerate}
\item Найдите все целые числа, которые лежат между числами $\sqrt{5} \cdot \sqrt{39}$ и $\frac{15-2{,}6}{2-1{,}2}$.

\item Решите уравнение $|x^2-15x+56| = 15x - 52 - x^2$.

\item В 10 коробках с номерами от 1 до 10 лежат только красные и синие шары. Число красных шаров во второй коробке в $\frac54$ раз больше, чем в первой. Количества красных шаров в коробках образуют арифметическую прогрессию, а количества синих шаров в коробках образуют геометрическую прогрессию (в порядке номеров коробок). Количество синих шаров в первой коробке составляет 20\%, а в третьей --- 40\% от числа всех шаров в данной коробке. Найти отношение общего числа синих шаров к общему числу красных шаров.

\item Решите уравнение 
$$\cos\left(2x + \frac{\pi}{4}\right) + \cos{x} = -\frac{1}{\sqrt{2}}.$$

\item Решите систему уравнений
$$
\begin{cases}
\frac{1}{y^2} \cdot x^{\log_y{x}} = x,\\
\left( \log_{2} x^3 \right) \cdot \log_{x} \left( 5x - 6y \right) = 9
\end{cases}
$$

\item В выпуклом четырехугольнике $ABCD$ $\angle{A} = \angle{D} = 60^{\circ}$, $AD = 24$, $BC = 13$. Окружность с центром на стороне $AD$ касается сторон $AB$, $BC$ и $CD$. Найдите длины сторон $AB$ и $CD$.

\item Найдите все значения параметра $a$, при которых неравенство $$11 + \cos^2{x} > 3 a^2 + 5 a - (4a - 1) \sin{x}$$ выполняется для всех $x$.

\item В правильную четырехугольную пирамиду $SABCD$ ($S$ --- вершина пирамиды) вписан шар. Через центр шара и ребро $AB$ проведена плоскость, которая в пересечении с пирамидой дает четырехугольник $ABMN$. Объемы пирамид $SABMN$ и $SABCD$ относятся как $7 : 25$. Найдите косинус двугранного угла между боковой гранью и основанием исходной пирамиды.
\end{enumerate}

\newpage


\end{document} 
