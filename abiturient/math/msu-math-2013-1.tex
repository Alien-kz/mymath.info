\documentclass[11pt, a5paper]{article}

\usepackage[T2A]{fontenc}
\usepackage[utf8]{inputenc}
\usepackage[english, russian]{babel}
\usepackage{amssymb}
\usepackage{amsfonts}
\usepackage{amsmath}
\usepackage{mathtext}

\usepackage{comment}
\usepackage{geometry}
\geometry{left=1cm, right=1cm, top=1cm, bottom=1cm}
\usepackage[inline]{enumitem}

\usepackage{graphicx}
\usepackage{tikz}
\usetikzlibrary{patterns}

\usepackage{wrapfig}
\usepackage{fancybox,fancyhdr}
\sloppy

\setlength{\headheight}{28pt}
\newcommand{\variant}[2]{
	\begin{center}
	\textit{Вариант #2}
	\end{center}
}

\newcommand{\unit}[1]{\text{\textit{ #1}}}
\newcommand{\units}[2]{ \frac{\text{\textit{#1}}}{\text{\textit{#2}}}}

\newcommand{\head}[4]
{
	\fancyhf{}
	\pagestyle{fancy}
	\chead{#3, #4}

	\begin{center}
	\begin{large}
	#1 --- #2 \\
	\end{large}
	\end{center}

}

\begin{document}

\head{Вступительный экзамен по математике}{2013}{Казахстанский филиал МГУ имени М. В. Ломоносова}{г. Астана}

\variant{2013}{1}

\begin{enumerate}[wide]
\item Докажите, что число
$$\left(\sqrt[3]{9} - \sqrt[6]{3} \right)^3 \cdot \left(9 + 5 \sqrt{3} \right)$$
является целым и найдите это целое число.

\item Решите неравенство
$$\frac{13 \cdot |x + 2| -5}{2 \cdot |x + 2| + 1} < 4 .$$

\item Решите уравнение
$$2 + \cos(\pi + 9x) = 5 \sin{\frac{\pi - 9x}{2}}.$$

\item В возрастающей арифметической прогрессии произведение седьмого и восьмого членов на 46 больше, чем произведение пятого и девятого членов, и на 108 больше, чем произведение третьего и десятого членов. Чему равна сумма первых 25 членов этой прогрессии?

\item Решите неравенство
$$(18 - 3 x) \cdot \log_{2^x-12} \sqrt[3]{2} \leqslant 1 .$$

\item В трапеции $ABCD$ длина основания $AD$ равна 20, а длина боковой стороны $CD$ равна $10 \sqrt{3}$. Через точки $A$, $B$, $C$ проходит окружность, пересекающая основание трапеции $AD$ в точке $F$. Угол $AFB$ равен $60^{\circ}$. Найдите длину отрезка $BF$.

\item Произведение двух натуральных чисел уменьшили на 26. Результат разделили на сумму исходных натуральных чисел с остатком. В частном получили 5, а в остатке 60. Найдите исходные натуральные числа.

\item Квадрат $ABCD$ со стороной 3 см является основанием двух пирамид $MABCD$ и $NABCD$, причем $MA$ и $NC$ --- высоты этих пирамид и точки $M$, $N$ лежат по одну сторону от плоскости $ABCD$. Сумма длин высот $MA$ и $NC$ равна 9 см, а объем общей части пирамид равен 6 см$^3$. Найдите отношение высот $MA$ и $NC$. 

\end{enumerate}

\newpage

\end{document} 
