\documentclass[11pt, a5paper]{article}

\usepackage[T2A]{fontenc}
\usepackage[utf8]{inputenc}
\usepackage[english, russian]{babel}
\usepackage{amssymb}
\usepackage{amsfonts}
\usepackage{amsmath}
\usepackage{mathtext}

\usepackage{comment}
\usepackage{geometry}
\geometry{left=1cm, right=1cm, top=1cm, bottom=1cm}
\usepackage[inline]{enumitem}

\usepackage{graphicx}
\usepackage{tikz}
\usetikzlibrary{patterns}

\usepackage{wrapfig}
\usepackage{fancybox,fancyhdr}
\sloppy

\setlength{\headheight}{28pt}
\newcommand{\variant}[2]{
	\begin{center}
	\textit{Вариант #2}
	\end{center}
}

\newcommand{\unit}[1]{\text{\textit{ #1}}}
\newcommand{\units}[2]{ \frac{\text{\textit{#1}}}{\text{\textit{#2}}}}

\newcommand{\head}[4]
{
	\fancyhf{}
	\pagestyle{fancy}
	\chead{#3, #4}

	\begin{center}
	\begin{large}
	#1 --- #2 \\
	\end{large}
	\end{center}

}

\begin{document}

\head{Вступительный экзамен по математике}{2011}{Казахстанский филиал МГУ имени М. В. Ломоносова}{г. Астана}

\variant{2011}{2}

\begin{enumerate}[wide]
\item Какие из чисел $2$, $-\frac23$, $\sqrt{5}+3$, $\sqrt5-3$ являются корнями уравнения
$$3x^3 = 16 x^2 - 8?$$

\item Представьте число $\sqrt{39}$ в виде десятичной дроби с точностью до~$0{,}1$.

\item Решите уравнение
$$ \cos \left(\frac{3\pi}{2} - 2x \right) + \sin(\pi - 3x) = \sin{3x} - \cos{x} .$$

\item Решите систему уравнений
$$
\begin{cases}
8^x \cdot 64^y = 128,\\
\sqrt{12y-7} = x.
\end{cases}
$$

\item В арифметической прогрессии 26 членов, и разность этой прогрессии равна 15. Сумма всех членов прогрессии в 5 раз больше, чем сумма членов, стоящих на нечетных местах. Найдите первый член этой прогрессии.

\item В трапеции, описанной около окружности радиуса 6, разность длин боковых сторон равна 4, а длина средней линии равна 15. Найдите длины сторон трапеции.

\item Решите неравенство
$$\frac{\log_3(x-3) \cdot \log_4(x-4)}{x-5} \leqslant \frac{\log_4(x-3) \cdot \log_3(x-4)}{x-6} .$$

\item В пирамиде $ABCD$: $AB = 1$, $AC = 3$, $AD = 4$, $BC = \sqrt{10}$, $BD = \sqrt{17}$, $CD = 5$. Найдите радиус шара, вписанного в пирамиду $ABCD$.

\end{enumerate}

\newpage


\end{document} 
