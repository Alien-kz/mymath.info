\documentclass[11pt, a5paper]{article}

\usepackage[T2A]{fontenc}
\usepackage[utf8]{inputenc}
\usepackage[english, russian]{babel}
\usepackage{amssymb}
\usepackage{amsfonts}
\usepackage{amsmath}
\usepackage{mathtext}

\usepackage{comment}
\usepackage{geometry}
\geometry{left=1cm, right=1cm, top=1cm, bottom=1cm}
\usepackage[inline]{enumitem}

\usepackage{graphicx}
\usepackage{tikz}
\usetikzlibrary{patterns}

\usepackage{wrapfig}
\usepackage{fancybox,fancyhdr}
\sloppy

\setlength{\headheight}{28pt}
\newcommand{\variant}[2]{
	\begin{center}
	\textit{Вариант #2}
	\end{center}
}

\newcommand{\unit}[1]{\text{\textit{ #1}}}
\newcommand{\units}[2]{ \frac{\text{\textit{#1}}}{\text{\textit{#2}}}}

\newcommand{\head}[4]
{
	\fancyhf{}
	\pagestyle{fancy}
	\chead{#3, #4}

	\begin{center}
	\begin{large}
	#1 --- #2 \\
	\end{large}
	\end{center}

}

\begin{document}

\head{Вступительный экзамен по математике}{2017}{Казахстанский филиал МГУ имени М. В. Ломоносова}{г. Астана}

\variant{2017}{1}

\begin{enumerate}[wide]
\item Найдите все целые числа, которые лежат между числами $\sqrt{3} \cdot \sqrt{85}$ и $\frac{14-1{,}7}{3-2{,}3}$.

\item Решите уравнение $|x^2-14x+48| = 14x - 42 - x^2$.

\item В 9 коробках с номерами от 1 до 9 лежат только красные и синие шары. Число красных шаров во второй коробке в $\frac76$ раз больше, чем в первой. Количества красных шаров в коробках образуют арифметическую прогрессию, а количества синих шаров в коробках образуют геометрическую прогрессию (в пордяке номеров коробок). Количество синих шаров в первой коробке составляет 25\%, а в третьей --- 50\% от числа всех шаров в данной коробке. Найти отношение общего числа синих шаров к общему числу красных шаров.

\item Решите уравнение $\sin\left(2x + \frac{\pi}{4}\right) - \sin{x} = \frac{1}{\sqrt{2}}$.

\item Решите систему уравнений
$$
\begin{cases}
x^{\log_y{x}} = \frac{y^2}{x},\\
\left( \log_{3} x^2 \right) \cdot \log_{x} \left( 2x - \frac{3}{y} \right) = 4
\end{cases}
$$

\item В выпуклом четырехугольнике $ABCD$ $\angle{B} = \angle{C} = 60^{\circ}$, $AD = 21$, $BC = 40$. Окружность с центром на стороне $BC$ касается сторон $AB$, $AD$ и $CD$. Найдите длины сторон $AB$ и $CD$.

\item Найдите все значения параметра $a$, при которых неравенство $13 + \sin^2{x} > 3 a^2 - a + (4a - 5) \cos{x}$ выполняется для всех $x$.

\item В правильную четырехугольную пирамиду $SABCD$ ($S$ --- вершина пирамиды) вписан шар. Через центр шара и ребро $AB$ проведена плоскость, которая в пересечении с пирамидой дает четырехугольник $ABMN$. Объемы пирамид $SABMN$ и $SABCD$ относятся как $5 : 9$. Найдите косинус двугранного угла между боковой гранью и основанием исходной пирамиды.
\end{enumerate}

\newpage


\end{document} 
