\documentclass[11pt, a5paper]{article}

\usepackage[T2A]{fontenc}
\usepackage[utf8]{inputenc}
\usepackage[english, russian]{babel}
\usepackage{amssymb}
\usepackage{amsfonts}
\usepackage{amsmath}
\usepackage{mathtext}

\usepackage{comment}
\usepackage{geometry}
\geometry{left=1cm, right=1cm, top=1cm, bottom=1cm}
\usepackage[inline]{enumitem}

\usepackage{graphicx}
\usepackage{tikz}
\usetikzlibrary{patterns}

\usepackage{wrapfig}
\usepackage{fancybox,fancyhdr}
\sloppy

\setlength{\headheight}{28pt}
\newcommand{\variant}[2]{
	\begin{center}
	\textit{Вариант #2}
	\end{center}
}

\newcommand{\unit}[1]{\text{\textit{ #1}}}
\newcommand{\units}[2]{ \frac{\text{\textit{#1}}}{\text{\textit{#2}}}}

\newcommand{\head}[4]
{
	\fancyhf{}
	\pagestyle{fancy}
	\chead{#3, #4}

	\begin{center}
	\begin{large}
	#1 --- #2 \\
	\end{large}
	\end{center}

}

\begin{document}

\head{Вступительный экзамен по физике}{2017}{Казахстанский филиал МГУ имени М. В. Ломоносова}{г. Астана}

\variant{2017}{2}

\begin{enumerate}[wide]
\item Сформулируйте закон всемирного тяготения. Как зависит сила тяжести от высоты тела над поверхностью Земли?

\item Дайте определение идеального газа. Запишите основное уравнение молекулярно-кинетической теории идеального газа.

\item Дайте определение напряженности электрического поля. Напишите формулу для напряженности электростатического поля точечного заряда.

\item Сформулируйте законы преломления света. Нарисуйте ход лучей в призме.

\item \textbf{Задача}. Деревянная однородная линейка выдвинута за край стола на $\alpha = \frac{1}{4}$ часть своей длины. При этом она не опрокидывается, если на ее свешивающийся конец положить груз массой не более $m_1 = 250 \unit{г}.$ На какую часть длины $\beta$ можно выдвинуть за край стола эту линейку, если на ее свешивающийся конец положен груз массой $m_2 = 125 \unit{г}$.

\item \textbf{Задача}. В сосуде под поршнем находился воздух с относительной влажностью $\varphi = 40 \%$. Объем воздуха изотермически уменьшили в 5 раз. Какая часть $\alpha$ водяных паров сконденсировалась после сжатия?

\item \textbf{Задача}. По наклонной плоскости, составляющей угол $\alpha = 30^{\circ}$ с горизонтальной поверхностью (см. рисунок), соскальзывает с высоты $h = 50 \unit{см}$ небольшое тело, заряженное отрицательным зарядом $-q$ ($q = 4 \unit{мкКл}$). В точке пересечения вертикали, проведенной через начальное положение тела, с основанием наклонной плоскости находится заряд $+q$. Определить скорость $v$, с которой тело достигнет основания наклонной плоскости, если масса тела $M = 100 \unit{г}$. Значение электрической постоянной $\varepsilon_0 = 8{,}85 \cdot 10^{-12} \units{Ф}{м}$. Ускорение свободного падения принять равным $g = 10 \frac{\unit{м}}{\unit{c}^2}$. Трением пренебречь.

\begin{center}
\begin{tikzpicture}[scale=0.5]
\draw (0, 0) -- (10 * cos{30}, 10 * sin{30}) -- (10 * cos{30}, 0) -- (0, 0);
\fill[pattern=north east lines] (0, 0) rectangle (10 * cos{30}, -0.5);
\draw[rotate=30, pattern=north east lines] (9, 0) rectangle (10, 1);
\draw (2, 0) arc(0:30:2) node [midway, right]{$\alpha$};
\node at (6.5, 5) {$M$};
\node at (9.5, 0) {$+q$};
\node at (9.5, 5) {$-q$};
\end{tikzpicture}
\end{center}

\item \textbf{Задача}. Тонкая линза с фокусным расстоянием $F = 0{,}4 \unit{м}$ создает на экране увеличенное изображение предмета, который помещен на расстояние $L = 2{,}5 \unit{м}$ от экрана. Каково расстояние $d$ от предмета до линзы?

\item \textbf{Задача}. В соответствии с основами теории Бора энергию электрона на $n$-м энергетическом уровне атома водорода можно представить в виде $E_n = -13{,}6 n^{-2} \unit{эВ}$ ($1 \unit{эВ} = 1{,}6 \cdot 10^{-19} \unit{Дж}$). 
При переходе электрона в атоме водорода с четвертой стационарной орбиты на вторую излучается фотон. Какова длина волны этой линии спектра? Постоянная Планка 
$h = 6{,}62 \cdot 10^{-34} \unit{Дж } \cdot \unit{с}$, 
скорость света $c = 3 \cdot 10^8 \units{м}{с}$.

\item \textbf{Задача}. Радиоактивный препарат с большим периодом полураспада помещен в медный контейнер массой
$M = 0{,}5 \unit{кг}$. За $\tau = 2$ часа температура контейнера повысилась на $\Delta T = 5{,}2 \unit{K}$. Известно, что данный препарат, помещенный в контейнер, испускает $\alpha$--частицы с энергией $E = 5{,}3 \unit{МэВ}$ ($1 \unit{эВ} = 1{,}6 \cdot 10^{-19} \unit{Дж}$), причем энергия всех испущенных $\alpha$-частиц полностью переходит во внутреннюю энергию контейнера. Определить активность препарата $A$, то есть количество $\alpha$--частиц, испускаемых им за 1 $\unit{с}$. Удельная теплоемкость меди $c = 0{,}385 \frac{\unit{кДж}}{\unit{кг } \cdot \unit{К}}$. Теплоемкостью препарата и теплообменом с окружающей средой пренебречь.

\end{enumerate}

\newpage


\end{document} 
