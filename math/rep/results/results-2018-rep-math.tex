\documentclass[12pt, a4paper]{article}

\usepackage[T2A]{fontenc}
\usepackage[utf8]{inputenc}
\usepackage[english, russian]{babel}
\usepackage{amssymb}
\usepackage{amsfonts}
\usepackage{amsmath}
\usepackage{mathtext}

\usepackage{comment}
\usepackage{geometry}
\geometry{left=1cm, right=1cm, top=2cm, bottom=2cm}

\usepackage{graphicx}
\usepackage{tikz}

\usepackage{wrapfig}
\usepackage{fancybox,fancyhdr}
\sloppy

\setlength{\headheight}{28pt}

\newcommand{\head}[4]
{
	\fancyhf{}
	\pagestyle{fancy}
	\chead{#3, #4}

	\begin{center}
	\begin{large}
	#1 \\
	\end{large}
	#2 \\
	\end{center}

}

\begin{document}

\head{X Республиканская студенческая предметная олимпиада по направлению \\ <<Математика>>}{03 апреля 2018}{Казахстанско-Британский технический университет}{г. Алматы}

\begin{center}
\begin{tabular}{|l|l|l|c|c|c|c|c|c|c|c|}
\hline
№ & Участник & ВУЗ & Курс & 1 & 2 & 3 & 4 & 5 & $\Sigma$ & Диплом \\
\hline
1 & Бекмаганбетов Бекарыс & КФ МГУ & 2 & 10 & 8 & 10 & 8 & 10 & 46 & 1 степени \\
\hline
2 & Жанбырбаев Есеналы & КБТУ & 2 & 0 & 10 & 10 & 8 & 10 & 38 & 2 степени \\
\hline
3 & Шакиев Александр & МУИТ & 2 & 0 & 10 & 0 & 8 & 10 & 28 & 2 степени \\
\hline
4 & Куанышабай Максат & НУ & 2 & 0 & 4 & 10 & 1 & 10 & 25 & 3 степени \\
\hline
5 & Жакатаев Еркебулан & КазНУ & 4 & 10 & 10 & 0 & 2 & 0 & 22 & 3 степени \\
\hline
6 & Аскергалиев Ануар & КФ МГУ & 2 & 0 & 3 & 0 & 5 & 10 & 18 & 3 степени \\
\hline
\end{tabular}
\end{center}

\end{document} 
