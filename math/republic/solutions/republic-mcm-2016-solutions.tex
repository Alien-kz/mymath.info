\documentclass[12pt, a4paper]{article}

\usepackage[T2A]{fontenc}
\usepackage[utf8]{inputenc}
\usepackage[english, russian]{babel}
\usepackage{amssymb}
\usepackage{amsfonts}
\usepackage{amsmath}
\usepackage{mathtext}

\usepackage{comment}
\usepackage{geometry}
\geometry{left=1cm, right=1cm, top=2cm, bottom=2cm}

\usepackage{graphicx}
\usepackage{tikz}

\usepackage{wrapfig}
\usepackage{fancybox,fancyhdr}
\sloppy

\setlength{\headheight}{28pt}

\newcommand{\head}[4]
{
	\fancyhf{}
	\pagestyle{fancy}
	\chead{#3, #4}

	\begin{center}
	\begin{large}
	#1 \\
	\textit{#2} \\
	\end{large}
	\end{center}

}

\begin{document}

\head{VIII Республиканская студенческая предметная олимпиада по направлению \\ <<Математическое и компьютерное моделирование>>}{01 апреля 2016}{Казахстанский филиал МГУ имени М. В. Ломоносова}{г. Астана}

\begin{enumerate}

\item (Абдикалыков А.К.)

Максимальный балл давался за алгоритм с асимптотической сложностью $O(1)$ (явную формулу), промежуточные баллы --- за сложность $O(n)$ и $O(n^2)$.
\begin{multline*}
\sum\limits_{j=1}^{n^2}{[\sqrt j\,]}=
\sum\limits_{k=1}^{n}{\left(k\cdot\sum\limits_{\substack{[\sqrt j\,]=k \\ 1 \leqslant j \leqslant n^2}}{1}\right)}=
\sum\limits_{k=1}^{n-1}{\left(k\cdot\sum\limits_{\substack{[\sqrt j\,]=k \\ 1 \leqslant j \leqslant n^2}}{1}\right)}+n=\\
=\sum\limits_{k=1}^{n-1}{\left(k\cdot\sum\limits_{j=k^2}^{(k+1)^2-1}{1}\right)}+n=
\sum\limits_{k=1}^{n-1}{k(2k+1)}+n=\\
=\sum\limits_{k=1}^{n-1}{(2k^2+k)}+n=
2\cdot\frac{(n-1)n(2n-1)}{6}+\frac{(n-1)n}{2}+n=\\
=\frac{(n-1)n(4n+1)}{6}+n=
\frac{n(4n^2-3n+5)}{6}
\end{multline*}
    
\item (Абдикалыков А.К.)

а) Две полупараболы из условия задачи --- это графики функций $y=x^2$ и $y=-\sqrt{x}$ при $x\geqslant 0$. Поэтому искомая функция
$$
S(L)=\int\limits_{0}^{f^{-1}(L)}{f(x)~dx},
$$
где $f(x)=x^2+\sqrt{x}$. (В силу монотонности функции $f(x)$ корректно вводить $f^{-1}(L)$.) Так как, кроме прочего, подынтегральная функция в определении $S(L)$ положительная, то и сама функция $S(L)$ --- возрастающая. Поскольку $S(2)=1$ (это можно показать разными способами: как графически, составив квадрат, например, так и аналитически, посчитав явно интеграл), то $S(L)>1$ при $L>2$.\\
б) Найдём сначала $x_0=f^{-1}(L)$ с помощью бинарного поиска, затем вычислим
$$
S(L)=\left.\left(\frac{1}{3}x^3+\frac{2}{3}x^{\frac{3}{2}}\right)\right|_{x=0}^{x_0}=
\frac{x_0^3+2x_0^{\frac{3}{2}}}{3}=$$
$$=\frac{x_0(2x_0^2+2\sqrt{x_0}-x_0^2)}{3}=
\frac{x_0(2L-x_0^2)}{3}.
$$

\item (Баев А.Ж.)

Продифференцируем по $x$ и $y$:
$$f'(x - y) + f'(x + y) = 2 x f''(x^2 + y^2),$$
$$ - f'(x - y) + f'(x + y) = 2 y f''(x^2 + y^2).$$

Пусть $x \ne 0$, $y \ne 0$. Приравняем $f''(x^2 + y^2)$:
$$(y + x) f'(x - y) = (x - y) f'(x + y).$$

Пусть $|x| \ne |y|$.
$$\frac{f'(x + y)}{x + y} = \frac{f'(x - y)}{x - y}.$$

Зафиксируем величину $x - y = A$, отличную от нуля. Тогда выражение справа не зависит от $y$ и равно некоторой константе $2C$.
$$\frac{f'(2y + A)}{2y + A} = \frac{f'(A)}{A} = 2 C.$$

Заметим, что $t = 2y + A$ может принимать любые ненулевые значения. Значит, при $t \ne 0$:
$$f'(t) = 2Ct.$$
$$f(t) = Ct^2 + C_1.$$

При подстановке в исходное уравнение, получим: $C_1 = 0$. При $t = 0$ доопределяется из непрерывности $f'(t)$ (по соотношению в условии). Ответ: $f(t) = C t^2$.

\item (Баев А.Ж., Абдикалыков А.К.)

Пусть конечное значение $S=\sum\limits_{j=1}^{2n-1}c_j\cdot\frac{f_{j-1}+f_j+f_{j+1}}{3}$. Тогда пункт б) эквивалентен решению нижеуказанной системы линейных уравнений, причём в целых числах. Видно, что система состоит из $(2n+1)$ уравнения относительно $(2n-1)$ неизвестной.
\begin{equation*}
\begin{pmatrix}
1 & 0 & 0 & \cdots & 0 & 0 & 0 \\
1 & 1 & 0 & \cdots & 0 & 0 & 0 \\
1 & 1 & 1 & \cdots & 0 & 0 & 0 \\
0 & 1 & 1 & \cdots & 0 & 0 & 0 \\
\cdots & \cdots & \cdots & \cdots & \cdots & \cdots & \cdots\\
0 & 0 & 0 & \cdots & 1 & 1 & 0 \\
0 & 0 & 0 & \cdots & 1 & 1 & 1\\
0 & 0 & 0 & \cdots & 0 & 1 & 1\\
0 & 0 & 0 & \cdots & 0 & 0 & 1\\
\end{pmatrix}
\begin{pmatrix}
c_1 \\
c_2 \\
c_3 \\
\cdots \\
c_{2n-3} \\
c_{2n-2} \\
c_{2n-1}
\end{pmatrix}
=
\begin{pmatrix}
1 \\
4 \\
2 \\
4 \\
\cdots \\
4 \\
2 \\
4 \\
1
\end{pmatrix}
\end{equation*}

Рассматривая только первые $(2n-1)$ уравнения, мы получим систему с нижнетреугольной матрицей и определителем, равным единице, а значит, всеми уравнениями, кроме последних двух, все неизвестные определяются однозначно, принимая при этом целые значения. Таким образом, задача сводится к нахождению таких $n$, чтобы система из этих $(2n-1)$ уравнения имела решение, совместимое с дополнительными условиями $c_{2n-2}+c_{2n-1}=4$, $c_{2n-1}=1$. Решая эту систему методом Гаусса, получаем
$$
\begin{matrix}
c_1=1, & c_2=3, & c_3=-2,\\
c_4=3, & c_5=1, & c_6=0,\\
c_7=1, & c_8=3, & c_9=-2,\\
c_{10}=3, & c_{11}=1, & c_{12}=0,\\
\end{matrix}
$$
$$
\hdots
$$
Таким образом, равенства $c_{2n-2}=3$, $c_{2n-1}=1$ выполняются только в том случае, если $$2n-1=5\pmod 6,$$ или, что то же самое, $n$ кратно 3.

Пункт а) эквивалентен решению той же системы в целых числах, но уже без первого и последнего уравнений.
\begin{equation*}
\begin{pmatrix}
1 & 1 & 0 & \cdots & 0 & 0 & 0 \\
1 & 1 & 1 & \cdots & 0 & 0 & 0 \\
0 & 1 & 1 & \cdots & 0 & 0 & 0 \\
\cdots & \cdots & \cdots & \cdots & \cdots & \cdots & \cdots\\
0 & 0 & 0 & \cdots & 1 & 1 & 0 \\
0 & 0 & 0 & \cdots & 1 & 1 & 1\\
0 & 0 & 0 & \cdots & 0 & 1 & 1
\end{pmatrix}
\begin{pmatrix}
c_1 \\
c_2 \\
c_3 \\
\cdots \\
c_{2n-3} \\
c_{2n-2} \\
c_{2n-1}
\end{pmatrix}
=
\begin{pmatrix}
4 \\
2 \\
4 \\
\cdots \\
4 \\
2 \\
4
\end{pmatrix}
\end{equation*}

Фиксируя $c_1=c$ и используя все уравнения, кроме последнего ($c_{2n-2}+c_{2n-1}=4$), находим
$$
\begin{matrix}
c_1=c, & c_2=4-c, & c_3=-2, \\
c_4=2+c, & c_5=2-c, & c_6=0,\\
c_7=c, & c_8=4-c, & c_9=-2, \\
c_{10}=2+c, & c_{11}=2-c, & c_{12}=0,
\end{matrix}
$$
$$
\hdots
$$
Значит,
$$
c_{2n-2}+c_{2n-1}=
\begin{cases}
c, & 2n-2=0\pmod 6,\\
-2-c, & 2n-2=2\pmod 6,\\
4, & 2n-2=4\pmod 6.
\end{cases}
$$
Видно, что в любом случае можно подобрать такое $c$, чтобы выполнялось равенство $$c_{2n-2}+c_{2n-1}=4,$$ из чего следует, что система совместна при любом $n$.

\item  (Абдикалыков А.К.)

а) Уменьшить двоичное число на единицу: {\bf CSLAF}.

б) Поменять все биты: {\bf CLA}.

в) Поменять только старший бит: {\bf CLRCLA}.

\item (Баев А.Ж.)

а) Пусть исходный квадрат --- это квадрат $[0, 1] \times [0, 1]$ на плоскости, а центр вырезанного квадрата расположен в точке $(x_0, y_0)$. Квадрат целиком поместится, если $(x_0; y_0) \in [a, 1 - a] \times [a, 1 - a]$. 

1 шаг. Найдем центр тяжести. Запишем функцию плотности массы пластины по оси $OX$:
$$
f(x) =
\begin{cases}
1, x < x_0 - a\\
1 - 2a, x \in [x_0 - a, x_0 + a]\\
1, x < x_0 + a\\
\end{cases}
$$

Найдем проекцию центра тяжести $m$ на ось $OX$:
$$m = \frac{\int_0^1 x f(x) dx}{\int_0^1 f(x) dx} = \frac{1 - 8 a^2 x_0}{2 (1 - 4 a^2)} .$$

2 шаг. Найдем вероятность попадания центра тяжести в вырезанную часть.
$$P = P(m \in [x_0 - a; x_0 + a]) =$$
$$= P\left( \frac{1 - 8 a^2 x_0}{2 (1 - 4 a^2)} < x_0 + a \right) - P\left( \frac{1 - 8 a^2 x_0}{2 (1 - 4 a^2)} < x_0 - a \right) =$$
$$= P\left( x_0 + a > \frac{1}{2} + 4 a^3  \right) - P\left( x_0 - a > \frac{1}{2} - 4 a^3 \right). $$
Обознаим полученную разность $P_1 - P_2.$

Вычислим $P_1$. Заметим, что $x_0 + a$ равномерно распределено на $[2a, 1]$. Поэтому важно понять, попадает ли $\frac{1}{2} + 4 a^3$ в интервал $[2a, 1]$. $\frac{1}{2} + 4 a^3 < 1$ ввиду того, что $a \in [0, \frac12]$. Проверим левую границу:

$$\frac{1}{2} + 4 a^3 > 2a$$
$$ \left(a - \frac{1}{2} \right)\left(a - \varphi \right)\left(a - \overline{\varphi}\right) > 0$$
где $\varphi= \frac{-1 + \sqrt{5}}{4}$. Значит, $\frac{1}{2} + 4 a^3$ попадает в интервал $[2a, 1]$ при $a < \varphi$.

$$P_1 = 
\begin{cases}
\frac{1 - 8 a^3}{2(1 - 2a)}, &a < \varphi\\
1, &a > \varphi.
\end{cases}
$$

Аналогично вычислим $P_2$. $x_0 - a$ равномерно распределено на $[0, 1-2a]$. Поэтому важно понять, попадает ли $\frac{1}{2} - 4 a^3$ в интервал $[0, 1-2a]$. $\frac{1}{2} - 4 a^3 > 0$ ввиду того, что $a \in [0, \frac12]$. Значит:
$$P_2 = 
\begin{cases}
 \frac{1 - 4a + 8 a^3}{2(1 - 2a)}, &a < \varphi\\
0, &a > \varphi.
\end{cases}
$$

Найдем вероятность попадания центра тяжести в вырезанную часть:
$$(P_1 - P_2)^2 =
\begin{cases}
4a^2(1+2a)^2,& a \in [0, \frac{-1 + \sqrt{5}}{4}]\\
1,& a \in \left[\frac{-1 + \sqrt{5}}{4}, \frac12 \right]\\
\end{cases}
$$

б) Промоделируем методом Монте--Карло и подсчет центра тяжести, и подсчет ответа. Генерируем $N$ подходящих квадратов. У каждого из них генерируем $M$ случайных точек. Если центр тяжести данных точек находится внутри квадрата, то засчитываем этот квадрат. Иначе --- нет. Отметим, что порядок аппроксимации данного метода $O\left( \frac{1}{ \sqrt{NM} } \right)$.

\end{enumerate} 


\end{document} 
