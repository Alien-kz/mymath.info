\documentclass[11pt, a4paper]{article}

\usepackage[T2A]{fontenc}		%cyrillic output
\usepackage[utf8]{inputenc}		%cyrillic output
\usepackage[english, russian]{babel}	%word wrap
\usepackage{amssymb, amsfonts, amsmath}	%math symbols
\usepackage{mathtext}			%text in formulas
\usepackage{geometry}			%paper format attributes
\usepackage{fancyhdr}			%header
\usepackage{graphicx}			%input pictures
\usepackage{tabularx}			%smart table
\usepackage{tikz}				%draw pictures
\usetikzlibrary{patterns}		%draw pictures: fill
\usetikzlibrary{positioning}	%draw pictures: below of
\usetikzlibrary{calc}			%draw pictures: $\i$
\usepackage{enumitem}			%enumarate parameters

\geometry{left=2cm, right=2cm, top=2cm, bottom=2cm, headheight=15pt}
\setlist[enumerate]{leftmargin=*}	%remove enumarate indenttion
\sloppy							%correct overfull
\pagestyle{empty}				%no page numbers

\newcommand{\head}[4]
{
	\thispagestyle{fancy}
	\fancyhf{}
	\chead{#3, #4}

	\begin{center}
	\begin{large}
	#1 \\
	\textit{#2}\\
	\end{large}
	\end{center}

}

\DeclareMathOperator*{\argmin}{Arg\,min}

\begin{document}

\head{XI Республиканская студенческая предметная олимпиада по~направлению~<<Математика>>}{18 апреля 2019}{Казахстанский филиал МГУ имени М. В. Ломоносова}{г.~Алматы}

\begin{enumerate}

\item (Баев А.Ж.)

Умножим числитель и знаменатель отдельного элемента ряда на $2^{n-1}$:
$$
\frac{1}{2^n} \cdot \frac{n + 1}{n (n - 1)}
=
\frac{2^{n-1}(n + 1)}{2^{n} n \cdot 2^{n-1} (n - 1)}
=
\frac{2^{n} n - 2^{n-1} (n - 1)}{2^{n} n \cdot 2^{n-1} (n - 1)}
=
$$
$$
=
\frac{1}{2^{n-1} (n-1)} - \frac{1}{2^{n} n}.
$$
Обозначив $b_n = \frac{1}{2^n n}$, перепишем наш ряд в виде
$$
\sum_{n = 2}^{\infty} (b_{n - 1} - b_n) =  b_1 = \frac12.
$$

Ответ: $\frac12$.

\newpage

\item (Васильев А.Н.)

Допустим, $x = \frac{p}{q}$ и $y = \frac{m}{n}$ --- несократимые дроби. Тогда
$$
\left( \frac{p}{q} \right) ^ \frac{m}{n} = \frac{p}{q} + \frac{m}{n}
$$
Чтобы значение левой части равенства было рациональным, необходимо $p=a^n$, $q = b^n$ для некоторых натуральных $a$ и $b$.
$$
\frac{a^m}{b^m} = \frac{a^n}{b^n} + \frac{m}{n} \Longleftrightarrow \frac{a^m b^n - a^n b^m}{b^{n + m}} = \frac{m}{n}
$$
Приведём дробь в левой части к несократимому виду. При $n > m$
$$\frac{a^m (b^{n-m} - a^{n-m})}{b^{n}} = \frac{m}{n}$$
и при $n < m$
$$\frac{a^n (a^{m-n} - b^{m-n})}{b^{m}} = \frac{m}{n},$$
а случай $m=n$, очевидно, не возможен. Получаем, что $n=b^{\max(m, n)}$ и в любом случае $n\geqslant b^n$, или $b \leqslant n^{\frac{1}{n}}$. Но $n^\frac{1}{n} < 2$ при любом $n$, и поэтому решение возможно только при $b = 1$ и $n=1$.

Так, задача свелась к решению в целых числах. Обязательно должно выполняться $x \geqslant 2$ и $y\geqslant 2$, поскольку варианты $x=1$ и $y=1$ не возможны. Пусть $y>2$, тогда
$$x ( x^{y - 1} - 1) \geqslant 2 (2^{y-1} - 1) > y.$$
Следовательно, $y=2$, а $x$ можно найти, решив уравнение $x^2=x+2$.

Ответ: $(2, 2)$.

\newpage

\item (Васильев А.Н.) 

Обозначим $g(\lambda)=\int\limits_{0}^{1} |f(x) - \lambda| \; dx$. Функция $f(x)$ интегрируема на данном отрезке интегрирования, а значит, $g(\lambda)$ непрерывна на $\mathbb R$.

а) Функция $f(x)$ интегрируема на $[0, 1]$, а значит, ограничена на нём. Пусть $m=\inf\limits_{x\in [0, 1]} f(x)$, $M=\sup\limits_{x\in [0, 1]} f(x)$. Очевидно, что функция $g(\lambda)$ убывает на $(-\infty, m)$ и возрастает на $(M, +\infty)$. А так как $g(\lambda)$ непрерывна на сегменте $[m, M]$, то она достигает на нём своего минимума, который при этом является глобальным. Следовательно, множество $\mathcal{M}(f)$ не пустое.

б) Для $f(x)={\mathrm{sgn}}~\left(2x-1\right)$ функция $g(\lambda)$ будет постоянной на отрезке $[-1, 1]$, что даёт $\mathcal{M}(f)=[-1, 1]$.

в) Без ограничения общности можно считать, что $f(x)$ нестрого возрастающая. Пусть $f(c-0)\leqslant \lambda \leqslant f(c+0)$ для некоторого $c\in\left[0, \frac{1}{2}\right]$ (случай $c\in\left[\frac{1}{2}, 1\right]$ рассматривается аналогично). Тогда
$$
g(\lambda) = \int\limits_{0}^{c} (\lambda - f(x)) \; dx + \int\limits_{c}^{1} (f(x) - \lambda) \; dx = 
$$
$$
=\lambda(2c-1)+ \int\limits_{0}^{1}{f(x) \; dx}-2\int\limits_{0}^{c}{f(x) \; dx},
$$
$$
g\left(f\left(\frac{1}{2}\right)\right)=\int\limits_{0}^{1}{f(x) \; dx}-2\int\limits_{0}^{\frac{1}{2}}{f(x) \; dx},
$$
$$
g(\lambda)-g\left(f\left(\frac{1}{2}\right)\right) = \lambda(2c-1) + 2\int\limits_{c}^{\frac{1}{2}}{f(x) \; dx} \geqslant \lambda(2c-1) + 2\int\limits_{c}^{\frac{1}{2}}{\lambda \; dx}= 0.
$$
Следовательно, $f\left(\frac{1}{2}\right)\in \mathcal{M}(f)$.

г) Примером такой функции может служить $f(x)=|2x-1|$. Здесь $g(\lambda)=\lambda^2-\lambda+\frac{1}{2}$ на $[0, 1]$, и $f\left(\frac{1}{2}\right)=0\not\in\mathcal{M}(f)=\left\{\frac{1}{2}\right\}$.

\newpage

\item (Абдикалыков А.К.) 

Пусть $\lambda_1$, $\lambda_2$ --- собственные значения матрицы $A$, тогда $\mu_1=\lambda_1^2+\lambda_1+2019$ и $\mu_2=\lambda_2^2+\lambda_2+2019$ будут собственными значениями матрицы $A^2+A+2019E$. Но тогда по условию $(\lambda_1^2+\lambda_1+2019)(\lambda_2^2+\lambda_2+2019)=0$, и один из множителей должен быть равным нулю. Без ограничения общности будем считать, что $\lambda_1^2+\lambda_1+2019=0$. В таком случае $\lambda_1=\frac{1}{2}\left(-1\pm i\sqrt{4075}\right)$ --- комплексное число, а так как $A$ --- вещественная матрица, то $\lambda_2=\overline{\lambda_1}\not=\lambda_1$ и $A$ имеет два линейно независимых собственных вектора $x_1$ и $x_2$. Те же векторы будут собственными и для $A^2+A+2019E$, но уже для собственных чисел $\mu_1=\lambda_1^2+\lambda_1+2019=0$ и $\mu_2=\lambda_2^2+\lambda_2+2019=\overline{\lambda_1^2+\lambda_1+2019}=0$. Следовательно, $A^2 + A + 2019 E = O$.

\newpage

\item (Баев А.Ж.) 

\begin{center}
    \begin{tikzpicture}[scale=0.5,line cap=round,line join=round,x=1.0cm,y=1.0cm]
\clip(-2, -4) rectangle (12, 8);
\draw [shift={(7.338349140322256,4.887754991842263)},line width=1.pt,fill=black,fill opacity=0.10000000149011612] (0,0) -- (-165.20215481658687:0.6432853930610821) arc (-165.20215481658687:-105.46516419868773:0.6432853930610821) -- cycle;
\fill[line width=1.pt,fill=black,fill opacity=0.10000000149011612] (7.338349140322256,4.887754991842263) -- (-1.4565280044613138,5.513576830476024) -- (11.120849832113286,-3.0768043482686354) -- cycle;
\draw [rotate around={0.:(2.,0.)},line width=1.pt] (2.,0.) ellipse (3.878570204631876cm and 3.3231471276875713cm);
\draw [line width=1.pt,dash pattern=on 5pt off 5pt] (7.338349140322256,4.887754991842263)-- (0.8572283728212576,3.1756297801317372);
\draw [line width=1.pt,dash pattern=on 5pt off 5pt] (7.338349140322256,4.887754991842263)-- (5.773984248380404,-0.7665099781435818);
\draw [line width=1.pt] (-1.4565280044613138,5.513576830476024)-- (11.120849832113286,-3.0768043482686354);
\draw [line width=1.pt] (2.021600315592237,7.489082687020486)-- (7.689678021978569,-1.0208228285713292);
\draw [line width=1.pt] (7.338349140322256,4.887754991842263)-- (-1.4565280044613138,5.513576830476024);
\draw [line width=1.pt] (2.9852506486129937,5.068528391818816) -- (3.003514311833545,5.325193570824241);
\draw [line width=1.pt] (2.8783068240273995,5.076138251494046) -- (2.8965704872479505,5.332803430499471);
\draw [line width=1.pt] (7.338349140322256,4.887754991842263)-- (0.,0.);
\draw [line width=1.pt] (3.785111814056336,2.366515216673871) -- (3.6424700819134452,2.580673830227929);
\draw [line width=1.pt] (3.695879058408813,2.3070811616143345) -- (3.5532373262659225,2.5212397751683926);
\draw [line width=1.pt] (7.338349140322256,4.887754991842263)-- (11.120849832113286,-3.0768043482686354);
\draw [line width=1.pt] (9.322819091861932,1.009092269705072) -- (9.090385481354778,0.8987057115798681);
\draw [line width=1.pt] (9.368813491080768,0.9122449319937582) -- (9.136379880573614,0.8018583738685544);
\draw [line width=1.pt] (7.338349140322256,4.887754991842263)-- (4.,0.);
\draw [line width=1.pt] (5.775415965121896,2.3713143518752724) -- (5.562933175200359,2.516440639966991);
\draw [line width=1.pt] (7.338349140322256,4.887754991842263)-- (7.689678021978569,-1.0208228285713292);
\draw [line width=1.pt] (7.642443821807464,1.9411026490451406) -- (7.385583340493365,1.9258295142257935);
\draw [line width=1.pt] (7.338349140322256,4.887754991842263)-- (2.021600315592237,7.489082687020486);
\draw [line width=1.pt] (4.623431655821455,6.072852733368755) -- (4.736517800093041,6.303984945493994);
\draw [line width=1.pt] (7.338349140322256,4.887754991842263)-- (-1.4565280044613138,5.513576830476024);
\draw [line width=1.pt] (-1.4565280044613138,5.513576830476024)-- (11.120849832113286,-3.0768043482686354);
\draw [line width=1.pt] (11.120849832113286,-3.0768043482686354)-- (7.338349140322256,4.887754991842263);

\draw [fill=black] (0.,0.) circle (2.0pt);
\draw[color=black] (0.026338505861291064,-0.24780672942871096) node {$A$};
\draw [fill=black] (4.,0.) circle (2.5pt);
\draw[color=black] (3.993265096404631,-0.3335781151701886) node {$B$};
\draw [fill=black] (7.338349140322256,4.887754991842263) circle (2.5pt);
\draw[color=black] (7.488449065369844,5.284447650896596) node {$C$};
\draw [fill=black] (0.8572283728212576,3.1756297801317372) circle (2.0pt);
\draw[color=black] (0.45519543456867917,3.268820085971872) node {$D$};
\draw [fill=black] (5.773984248380404,-0.7665099781435818) circle (2.0pt);
\draw[color=black] (6.137549739941571,-0.8910921224897932) node {$E$};
\draw [fill=black] (-1.4565280044613138,5.513576830476024) circle (2.0pt);
\draw[color=black] (-1.7641391714920542,5.402383306291128) node {$A_1$};
\draw [fill=black] (2.021600315592237,7.489082687020486) circle (2.5pt);
\draw[color=black] (1.6238305652963116,7.482339410521961) node {$B_1$};
\draw [fill=black] (11.120849832113286,-3.0768043482686354) circle (2.0pt);
\draw[color=black] (11.315997154083284,-2.6601269534077696) node {$A_2$};
\draw [fill=black] (7.689678021978569,-1.0208228285713292) circle (2.5pt);
\draw[color=black] (7.928027417294917,-1.223456242238019) node {$B_2$};
\draw[color=black] (6.673820750920979,4.1980767148696035) node {$\alpha$};
\draw[color=black] (3.9503794035338924,2.1251349927204463) node {$a$};
\draw[color=black] (9.611290862471416,0.8243355923397595) node {$a$};
\draw[color=black] (2.921122774636161,5.491661883073443) node {$a$};

\end{tikzpicture}
\end{center}

В решении задачи никак не используется эллипс. Достаточно заметить, что $CA = CA_1 = CA_2 = a$, $CB = CB_1 = CB_2 = b$ в силу симметрии. Также верно, что $\angle ACD = \angle A_1CD$ и $\angle ACE = \angle A_2CE$. Значит, 
$$\angle DCE = \angle ACD  + \angle ACE = \frac12 \angle ACA_1 + \frac12 \angle ACA_2 = \frac12 \angle A_1CA_2$$
Таким образом угол $\angle A_1CA_2 = 2 \alpha$; аналогично $\angle B_1CB_2 = 2 \alpha$.  Осталось выразить все неизвестные через $a$, $b$, $\alpha$ по теореме косинусов:
$$
\frac{ A_1A_2^2 + B_1B_2^2 }{CA^2 + CB^2}
=
\frac{ 2 a^2 - 2 a^2 \cos{2 \alpha} + 2 b^2 - 2 b^2 \cos{2 \alpha} }{a^2 + b^2}
=
2 (1 - \cos{2 \alpha})
=
4 \sin^2 \alpha 
$$
Ответ: а) 2; б) 4.

\newpage

\item (Клячко А.А.) 

Очевидно, что если целочисленная система линейных уравнений разрешима, то она разрешима и по любому модулю. Предположим теперь, что у нас есть система с $n$ неизвестными $x_1, x_2, \hdots, x_n$, разрешимая по любому модулю. Элементарными преобразованиями эту систему можно привести к трапециевидной форме $Ax=b$ (также с целыми коэффициентами). Так как по предположению система разрешима по любому модулю, то в этой форме не будет строк, где ненулевым будет только свободный коэффициент. Пусть $M=\alpha_1\alpha_2\hdots\alpha_k>0$, где $\alpha_1, \alpha_2, \hdots, \alpha_k$ --- ведущие элементы строк матрицы $A$, которые без ограничения общности можно считать положительными. По условию $\exists~ x', c\in\mathbb{Z}^k$~: $Ax'=b+Mc$. Найдём такой вектор $\Delta x\in\mathbb{Z}^n$, что $A\Delta x=-Mc$. Это можно сделать обратным ходом метода Гаусса, предполагая нулевыми все компоненты $\Delta x$, не соответствующие ведущим коэффициентам рассматриваемой на данном шаге строки. Тогда $x=x'+\Delta x$ будет решением исходной системы.

\end{enumerate}


\end{document} 
