\documentclass[11pt, a4paper]{article}

\usepackage[T2A]{fontenc}		%cyrillic output
\usepackage[utf8]{inputenc}		%cyrillic output
\usepackage[english, russian]{babel}	%word wrap
\usepackage{amssymb, amsfonts, amsmath}	%math symbols
\usepackage{mathtext}			%text in formulas
\usepackage{geometry}			%paper format attributes
\usepackage{fancyhdr}			%header
\usepackage{graphicx}			%input pictures
\usepackage{tabularx}			%smart table
\usepackage{tikz}				%draw pictures
\usetikzlibrary{patterns}		%draw pictures: fill
\usetikzlibrary{positioning}	%draw pictures: below of
\usetikzlibrary{calc}			%draw pictures: $\i$
\usepackage{enumitem}			%enumarate parameters

\geometry{left=2cm, right=2cm, top=2cm, bottom=2cm, headheight=15pt}
\setlist[enumerate]{leftmargin=*}	%remove enumarate indenttion
\sloppy							%correct overfull
\pagestyle{empty}				%no page numbers

\newcommand{\head}[4]
{
	\thispagestyle{fancy}
	\fancyhf{}
	\chead{#3, #4}

	\begin{center}
	\begin{large}
	#1 \\
	\textit{#2}\\
	\end{large}
	\end{center}

}

\DeclareMathOperator*{\argmin}{Arg\,min}

\begin{document}

\head{XI Республиканская студенческая предметная олимпиада по~направлению~<<Математическое и компьютерное моделирование>>}{18 апреля 2019}{Казахстанский филиал МГУ имени М. В. Ломоносова}{г.~Астана}

\begin{enumerate}

\item (Баев А.Ж.) 
$$
A
= D = 
\begin{pmatrix}
E_{n-1} & 0\\
0 & 0
\end{pmatrix},
\quad
B
= C = 
\begin{pmatrix}
O_{n-1} & 0\\
0 & 1
\end{pmatrix}.
$$

\newpage

\item (Абдикалыков А.К.) 

Обозначим выражение, предел которого надо найти, как $S_n$. Эту сумму можно переписать как
$$
S_n = \frac{1}{n}\sum\limits_{k=1}^{n}\left(\frac{k-\sin\frac{k}{n}}{n}\right)^{2018}=\frac{1}{n}\sum\limits_{k=1}^{n}{f(\xi_k)},
$$
где $f(x)=x^{2018}$, а $\xi_k\in[\frac{k-1}{n}, \frac{k}{n}]$. Видно, что это некоторая интегральная сумма функции $f(x)$ при равномерном разбиении отрезка $[0, 1]$; значит, её предел при $n\to\infty$ равен интегралу
$$
\int\limits_{0}^{1}{x^{2018} \; dx} = \frac{1}{2019}.
$$

\newpage

\item (Баев А.Ж.) 

Пусть $x_1$, $x_2$, $\hdots$, $x_n$ --- искомые координаты. После $i$-го запроса мы получаем равенство:
$$a_{i 1} x_1 + a_{i 2} x_2 + \hdots + a_{i n} x_n = f_i,$$
где $a_{i j}$ равно либо $\frac{1}{K}$, либо 0, причём ненулевых коэффициентов ровно $K$. Таким образом, задача сводится к решению системы линейных уравнений
$$
\frac{1}{K} A x = f
$$
с матрицей специального вида. Покажем, что при любом $K<N$ существует невырожденная квадратная матрица порядка $N$ такая, чтобы в каждой строке было ровно $K$ единиц и $N-K$ нулей.

Чтобы узнать координаты первых $(K + 1)$-й точки, решим систему $\frac{1}{K} B \overline{x} = f_{1 .. k+1}$ с  матрицей порядка $K + 1$ вида
$$
B =
\begin{pmatrix}
0 & 1  &  \hdots & 1 & 1 & 1 \\
1 & 0  &  \hdots & 1 & 1 & 1\\
\vdots  & \vdots & \ddots & \vdots & \vdots & \vdots \\
1 & 1  &  \hdots & 0 & 1 & 1\\
1 & 1  &  \hdots & 1 & 0 & 1\\
1 & 1  &  \hdots & 1 & 1 & 0\\
\end{pmatrix}
$$
Система уравнения с данной матрицей всегда совместна: сложим все строки --- получим строку из всех единиц, вычтем из полученной строки остальные строки --- получим строку из одной единицы. Запросами, соответствующими данной матрице, можно получить координаты первых $(K+1)$-й точки. Далее будем делать запросы, в каждом из которых будут первые $(K-1)$ точка и какая-то новая точка.
$$
A =
\begin{pmatrix}
0 & 1  & \hdots & 1 & 1 & 1 & 0 & 0 & \hdots & 0 \\
1 & 0  &  \hdots & 1 & 1 & 1 & 0 & 0 & \hdots & 0 \\
\vdots  & \vdots & \ddots & \vdots & \vdots & \vdots & \vdots & \vdots & \vdots & \vdots\\
1 & 1  &  \hdots & 0 & 1 & 1 & 0 & 0 & \hdots & 0 \\
1 & 1  &  \hdots & 1 & 0 & 1 & 0 & 0 & \hdots & 0 \\
1 & 1  &  \hdots & 1 & 1 & 0 & 0 & 0 & \hdots & 0 \\
1 & 1 &  \hdots & 1 & 0 & 0 & 1 & 0 & \hdots & 0 \\
1 & 1  &  \hdots & 1 & 0 & 0 & 0 & 1 & \hdots & 0 \\
\vdots & \vdots  &  \vdots & \vdots & \vdots & \vdots & \vdots & \vdots & \ddots & \vdots \\
1 & 1  &  \hdots & 1 & 0 & 0 & 0 & 0 & \hdots & 1 \\
\end{pmatrix}
$$
Таким образом, при $K < N$, ответ --- $N$. А при $K = N$, очевидно, найти $x_i$ невозможно ни за какое количество запросов. 

\newpage

\item (Абдикалыков А.К., Баев А.Ж.) 

Представим искомый многочлен в виде суммы $P(x)=Q(x)+R(x)$, где $Q(x)$ --- многочлен с нечётными показателями при $x$, $R(x)$ --- с чётными. При таких обозначениях условие задачи равносильно
$$
3 \int\limits_{0}^{1} R(x) \; dx = 2 R(0) + R(1),
$$
а нечётная составляющая $Q(x)$ может быть таким образом произвольной. Пусть $R(x)=a_{2k}x^{2k} + \hdots + a_2x^2 + a_0$, тогда наше равенство эквивалентно
$$
\frac{3}{5}a_4+\frac{3}{7}a_6+\hdots+\frac{3}{2k+1}a_{2k}=a_4+a_6+\hdots+a_{2k},
$$
что невозможно, если хотя бы один из коэффициентов $a_4$, $a_6$, $\hdots$, $a_{2k}$ больше нуля. Поэтому ответом на задачу являются все многочлены вида $xT(x^2) + ax^2 + b$.

\newpage

\item (Абдикалыков А.К.) 


а) Допустим, что нашлась такая функция $f(x)$, что $f(f(x))=g(x)$. Тогда $f(x)$ обратима, поскольку $f(f(x - 2019)) = x$, причём обратная функция однозначно определена. Таким образом, $f$ осуществляет биективное отображение множества целых чисел $\{-2^{31}, -2^{31}+1, \hdots, 2^{31}-1\}$ на само себя; другими словами, $f$ --- некоторая перестановка чисел типа {\tt int}. То же самое можно сказать и про $g$: указанное множество данная функция преобразует в $\{-2^{31} + 2019, -2^{31}+2020, \hdots, 2^{31}-1, -2^{31}, -2^{31}+1, \hdots, -2^{31}+2018\}$. Нетрудно видеть, что $g$ --- нечётная перестановка (количество инверсий $2019 \cdot (2^{32} - 2019)$), но это приводит к противоречию, так как $g=f\circ f$ --- композиция двух чётных или двух нечётных перестановок и должна быть чётной. Следовательно, функции $f$, удовлетворяющей условиям, не существует.

б) Подойдёт функция
\begin{verbatim}
int f(int x)
{
    int t = (x % 4 + 4) % 4;
    if (t == 0 || t == 1)
        return x + 2;
    if (t == 2)
        return x - 1;
    if (t == 3)
        return x - 3;
}
\end{verbatim}

\newpage

\item (Абдикалыков А.К.) 

Допустим, что начальная клетка белая. Тогда робот будет двигаться вправо, пока не встретит чёрную клетку, после чего он зациклится. Выпишем возможные варианты: {\tt WB}, вероятность 1/4, посещены 2 клетки; {\tt WWB}, вероятность 1/8, посещены 3 клетки; {\tt WWWB}, вероятность 1/16, посещены 4 клетки; и так далее. Проведя аналогичные рассуждения для начальной чёрной клетки, получим те же числа. Искомое математическое ожидание будет равно
$$
2\left(\frac{2}{4}+\frac{3}{8}+\frac{4}{16}+\hdots\right)=\frac{2}{2}+\frac{3}{4}+\frac{4}{8}+\hdots+\frac{n}{2^{n-1}}+\hdots=M.
$$
Сумму $M$ можно найти несколькими способами, в том числе и так:
$$
M + 1 = M + \left(\frac{1}{2}+\frac{1}{4}+\frac{1}{8}+\hdots+\frac{1}{2^{n-1}}+\hdots\right)=\frac{3}{2}+\frac{4}{4}+\frac{5}{8}+\hdots+\frac{n+1}{2^{n-1}}+\hdots=2M-2 \Rightarrow
$$
$\Rightarrow M = 3.$
\end{enumerate}


\end{document} 
