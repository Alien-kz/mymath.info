\documentclass[11pt, a5paper, landscape]{article}

\usepackage[T2A]{fontenc}
\usepackage[utf8]{inputenc}
\usepackage[english, russian]{babel}
\usepackage{amssymb}
\usepackage{amsfonts}
\usepackage{amsmath}
\usepackage{mathtext}

\usepackage{comment}
\usepackage{geometry}
\geometry{left=0.5cm, right=1cm, top=1cm, bottom=1cm}
\usepackage[inline]{enumitem}

\usepackage{graphicx}
\usepackage{tikz}
\usetikzlibrary{patterns}

\usepackage{wrapfig}
\usepackage{fancybox,fancyhdr}
\sloppy

\setlength{\headheight}{28pt}
\newcommand{\variant}[2]{
	\begin{center}
	\textit{Вариант #2}
	\end{center}
}

\newcommand{\unit}[1]{\text{\textit{ #1}}}
\newcommand{\units}[2]{ \frac{\text{\textit{#1}}}{\text{\textit{#2}}}}

\newcommand{\head}[4]
{
	\fancyhf{}
	\pagestyle{fancy}
	\chead{#3, #4}

	\begin{center}
	\begin{large}
	#1 \\
	#2 \\
	\end{large}
	\end{center}

}

\begin{document}

\head{X Республиканская студенческая предметная олимпиада по~направлению~<<Математическое и компьютерное моделирование>>}{20 апреля 2018}{Назарбаев Университет}{г. Астана}

\begin{center}
\begin{tabular}{|l|l|l|c|c|c|c|c|c|c|c|}
\hline
№ & Участник & ВУЗ & Курс & 1 & 2 & 3 & 4 & 5  & $\Sigma$ & Диплом \\
\hline
1 & Бекмаганбетов Бекарыс & КФ МГУ & 2 & 7 & 7 & 7 & 7 & 7 & 35 & 1 \\
\hline
2 & Аскергалиев Ануар & КФ МГУ & 2 & 4 & 0 & 7 & 6 & 2 & 19 & 2 \\
\hline
3 & Шарипов Азат & КФ МГУ & 2 & 2 & 0 & 0 & 5 & 1.5 & 8.5 & 2 \\
\hline
4 & Сабирова Роза & КазНУ & ? & 3 & 0 & 0 & 2 & 3 & 8 & 3 \\
\hline
5 & Абай Азат & КазНУ & ? & 4 & 1 & 0 & 0 & 1 & 6 & 3 \\
\hline
6 & Ногаева Аида & КазНУ & 3 & 3 & 1 & 0 & 0 & 1 & 5 & 3 \\
\hline
\end{tabular}
\end{center}


\end{document} 
