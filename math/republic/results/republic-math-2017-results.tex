\documentclass[12pt, a5paper, landscape]{article}

\usepackage[T2A]{fontenc}		%cyrillic output
\usepackage[utf8]{inputenc}		%cyrillic output
\usepackage[english, russian]{babel}	%word wrap
\usepackage{amssymb, amsfonts, amsmath}	%math symbols
\usepackage{mathtext}			%text in formulas
\usepackage{geometry}			%paper format attributes
\usepackage{fancyhdr}			%header
\usepackage{graphicx}			%input pictures
\usepackage{tikz}				%draw pictures
\usetikzlibrary{patterns}		%draw pictures: fill

\geometry{left=1cm, right=1cm, top=2cm, bottom=1cm, headheight=15pt}
\sloppy							%correct overfull

\newcommand{\head}[4]
{
	\pagestyle{fancy}
	\fancyhf{}
	\chead{#3, #4}

	\begin{center}
	\begin{large}
	#1 \\
	\textit{#2}\\
	\end{large}
	\end{center}

}

\begin{document}

\head{IX Республиканская студенческая предметная олимпиада по~направлению~<<Математика>>}{13 апреля 2017}{Казахстанский филиал МГУ имени М. В. Ломоносова}{г.~Астана}

\begin{center}
\begin{tabular}{|l|l|l|c|c|c|c|c|c|c|c|c|}
\hline
№ & Участник & ВУЗ & Курс & 1 & 2 & 3 & 4 & 5 & 6 & $\Sigma$ & Диплом \\
\hline
1 & Жанбырбаев Есеналы & КБТУ & 2 & 10 & 10 & 10 & 3 & 0 & 10 & 43 & 1 \\
\hline
2 & Бекмаганбетов Бекарыс & КФ МГУ & 1 & 10 & 10 & 10 & 2 & 10 & 0 & 42 & 2 \\
\hline
3 & Сайланбаев Алибек & НУ & 4 & 10 & 10 & 10 & 2 & 0 & 8 & 40 & 2 \\
\hline
4 & Аманкелды Акежан & НУ & 4 & 9 & 9 & 10 & 5 & 0 & 0 & 33 & 3 \\
\hline
5 & Жанахметов Султан & НУ & 3 & 10 & 10 & 10 & 2 & 0 & 0 & 32 & 3 \\
\hline
6 & Шакиев Александр & МУИТ & 2 & 10 & 9 & 10 & 0 & 0 & 0 & 39 & 3 \\
\hline
\end{tabular}
\end{center}

\end{document} 
