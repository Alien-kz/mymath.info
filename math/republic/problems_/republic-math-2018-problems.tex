\documentclass[11pt, a4paper]{article}

\usepackage[T2A]{fontenc}		%cyrillic output
\usepackage[utf8]{inputenc}		%cyrillic output
\usepackage[english, russian]{babel}	%word wrap
\usepackage{amssymb, amsfonts, amsmath}	%math symbols
\usepackage{mathtext}			%text in formulas
\usepackage{geometry}			%paper format attributes
\usepackage{fancyhdr}			%header
\usepackage{graphicx}			%input pictures
\usepackage{tikz}				%draw pictures
\usetikzlibrary{patterns}		%draw pictures: fill
\usepackage{enumitem}			%enumarate parameters

\geometry{left=1cm, right=1cm, top=2cm, bottom=1cm, headheight=15pt}
\setlist[enumerate]{leftmargin=*}	%remove enumarate indenttion
\sloppy							%correct overfull

\newcommand{\head}[4]
{
	\pagestyle{fancy}
	\fancyhf{}
	\chead{#3, #4}

	\begin{center}
	\begin{large}
	#1 \\
	\textit{#2}\\
	\end{large}
	\end{center}

}

\begin{document}

\head{X Республиканская студенческая предметная олимпиада по~направлению~<<Математика>>}{03 апреля 2018}{Казахстанско-Британский технический университет}{г.~Алматы}

\begin{enumerate}

\item Последовательность многочленов $P_n$ равномерно сходится на всей оси $f: \mathbb{R} \to \mathbb{R}$. Докажите, что $f$ является многочленом.

\item Докажите, что если непрерывно дифференцируемая функция $f: \mathbb{R} \to \mathbb{R}$ удовлевторяет тождеству $f(x) = \alpha f(x / 2)$, а $|\alpha| < 2$, то при $|\alpha| = 1$ функция $f$ --- произвольная постоянная, а при остальных $\alpha$ --- нулевая.

\item Две непрерывно дифференцируемые на $[0; a]$ функции $f_0$, $f_1$ принимают неположительные значения и $f_0(0) = f_1(0) = 0$. Докажите, что если при всех $x$ выполняется неравенство $f'_0(x) + x f'_1(x) \geqslant 0$ на отрезке $[0; a]$, то обе функции являются тождественно нулевыми.

\item Известно, что на графике многочлена $P$ можно отметить $n$ точек, являющихся вершинами правильного $n$-угольника. Доказать, что его степень не меньше $n - 1$.

\item Если три вектора $(u_1, u_2, u_3)$, $(v_1, v_2, v_3)$, $(w_1, w_2, w_3)$ с ненулевыми координатами попарно ортогональны, то векторы $\left(\frac{1}{u_1}, \frac{1}{u_2}, \frac{1}{u_3} \right)$, $\left(\frac{1}{v_1}, \frac{1}{v_2}, \frac{1}{v_3} \right)$, $\left(\frac{1}{w_1}, \frac{1}{w_2}, \frac{1}{w_3} \right)$ с обратными координатами компланарны, т.е. лежат в одной плоскости.
\end{enumerate}


\end{document} 
