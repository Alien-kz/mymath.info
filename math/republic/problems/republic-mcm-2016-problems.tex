\documentclass[12pt, a4paper]{article}

\usepackage[T2A]{fontenc}
\usepackage[utf8]{inputenc}
\usepackage[english, russian]{babel}
\usepackage{amssymb}
\usepackage{amsfonts}
\usepackage{amsmath}
\usepackage{mathtext}

\usepackage{comment}
\usepackage{geometry}
\geometry{left=1cm, right=1cm, top=2cm, bottom=2cm}

\usepackage{graphicx}
\usepackage{tikz}

\usepackage{wrapfig}
\usepackage{fancybox,fancyhdr}
\sloppy

\setlength{\headheight}{28pt}

\newcommand{\head}[4]
{
	\fancyhf{}
	\pagestyle{fancy}
	\chead{#3, #4}

	\begin{center}
	\begin{large}
	#1 \\
	\textit{#2} \\
	\end{large}
	\end{center}

}

\begin{document}

\head{VIII Республиканская студенческая предметная олимпиада по направлению \\ <<Математическое и компьютерное моделирование>>}{01 апреля 2016}{Казахстанский филиал МГУ имени М. В. Ломоносова}{г. Астана}

\begin{flushright}
Стоимость задач: \\
10 баллов каждая задача.\\
\end{flushright}

\begin{enumerate}
\item Введём функцию
$$
f(n) = \bigl[\sqrt 1\,\bigr] + \bigl[\sqrt 2\,\bigr] + \bigl[\sqrt 3\,\bigr] + \hdots + \bigl[\sqrt{n^2-1}\,\bigr] + \bigl[\sqrt{n^2}\,\bigr],
$$
где $[x]$ --- наибольшее целое число, не превышающее $x$. Опишите функцию, которая вычисляет $f(n)$ для данного натурального $n$, не используя при этом операцию извлечения корня и вещественную арифметику.

\item На декартовой координатной плоскости нарисованы две полупараболы: график функции $y = x^2$ $(x \geqslant 0)$ и его копия, повёрнутая на прямой угол по часовой стрелке. Эти две кривые отсекают от прямой, параллельной оси ординат, отрезок длины $L$. Обозначим через $S(L)$ --- площадь отсечённой фигуры.

a) Докажите, что $S(L) > 1$ при $L > 2$;

б) Напишите функцию, которая вычисляет $S(L)$ для данного положительного вещественного числа $L$.



\item Найдите все дифференцируемые функции $f\colon \mathbb R\rightarrow\mathbb R$, удовлетворяющие соотношению
$$
f(x-y) + f(x+y) = f'(x^2 + y^2)
$$
для любых $x,y\in\mathbb R$.



\item  Дана функция $f\colon [0, 2n]\rightarrow\mathbb R$. Пусть $f_i = f(i)$ --- значения функции во всех целых $i$ от 0 до $2n$. Дана переменная $S$ вещественного типа с начальным значением 0. За один ход робот может выбрать целое $i$ от 1 до $2n-1$, затем добавить к переменной $S$ или вычесть из нее среднее арифметическое значений функции $f(x)$ в узлах~$i-1$,~$i$,~$i+1$:
$$S := S \pm \frac{f_{i-1}+f_i+f_{i+1}}{3}.$$ 
Может ли робот за конечное число ходов получить в переменной $S$ значение
$$I = \frac{1}{3} \left(f_0 + 4\sum\limits_{k=1}^{n}{f_{2k-1}} + 2\sum\limits_{k=1}^{n-1}{f_{2k}} + f_{2n}\right),$$
которое является приближением интеграла $\displaystyle \int\limits_0^{2n} f(x)\,dx$, если\\
а) $f(0) = f(2n) = 0$;\\
б) $f(0) \ne 0$, $f(2n) \ne 0$?

\item  Дана некоторая условная машина, состоящая из памяти в $n$ бит и указателя, который в каждый отдельный момент находится над какой-то из этих n ячеек. Перед запуском программы в память записывается некоторое натуральное число $m$ в двоичной системе счисления, а указатель устанавливается над крайним правым (младшим) битом числа. Язык программирования для этой машины состоит из следующих команд:

\begin{center}
\begin{tabular}{|c|c|p{14cm}|}
\hline
{\bf L} & left & сместить указатель налево на одну ячейку, если это возможно, иначе завершить программу\\
\hline
{\bf R} & right & сместить указатель направо на одну ячейку, если это возможно, иначе завершить программу\\
\hline
{\bf C} & change & изменить значение бита в текущей ячейке на противоположное\\
\hline
{\bf A} & again & перейти к выполнению первой команды\\
\hline
{\bf S} & skip & пропустить две следующие команды, если в текущей ячейке 0\\
\hline
{\bf F} & finish & завершить выполнение программы\\
\hline
\end{tabular}
\end{center}

Команды записываются в одну строку и выполняются в последовательном порядке, слева направо. При этом запись программы обязана оканчиваться командой \textbf{A} или \textbf{F}. Напишите для этой абстрактной машины следующие программы:

а) заменить данное число на $(m - 1)$;

б) заменить данное число на $(2^n - m - 1)$;

в) изменить на противоположный его старший (крайний слева) бит.

\textit{Примеры:}

а) программа, обнуляющая все ячейки: \textbf{SSCLA};

б) программа, которая изменяет второй справа бит, если крайний справа бит нулевой: \textbf{SFFLCF}.


\item Из квадратной однородной пластины со стороной 1 случайным образом вырезается квадрат со сторонами, равными $2a$ и параллельными сторонам исходного квадрата. При этом центр квадрата --- это случайная величина, равномерно распределённая по всем допустимым положениям (квадрат со стороной $(1 - 2a)$).

a) Найдите вероятность $p(a)$ того, что центр тяжести полученной фигуры лежит в вырезанной области.

б) Опишите функцию $p(a)$, которая вычисляет указанную вероятность приблизительно, считая при этом, что нам не известен метод нахождения центра тяжести произвольной фигуры, однако мы можем найти центр тяжести конечного множества точек одинаковой массы.

\end{enumerate}


\end{document} 
