\documentclass[11pt, a4paper]{article}

\usepackage[T2A]{fontenc}		%cyrillic output
\usepackage[utf8]{inputenc}		%cyrillic output
\usepackage[english, russian]{babel}	%word wrap
\usepackage{amssymb, amsfonts, amsmath}	%math symbols
\usepackage{mathtext}			%text in formulas
\usepackage{geometry}			%paper format attributes
\usepackage{fancyhdr}			%header
\usepackage{graphicx}			%input pictures
\usepackage{tikz}				%draw pictures
\usetikzlibrary{patterns}		%draw pictures: fill
\usepackage{enumitem}			%enumarate parameters

\geometry{left=1cm, right=1cm, top=2cm, bottom=1cm, headheight=15pt}
\setlist[enumerate]{leftmargin=*}	%remove enumarate indenttion
\sloppy							%correct overfull

\newcommand{\head}[4]
{
	\pagestyle{fancy}
	\fancyhf{}
	\chead{#3, #4}

	\begin{center}
	\begin{large}
	#1 \\
	\textit{#2}\\
	\end{large}
	\end{center}

}

\begin{document}

\head{VI Республиканская студенческая предметная олимпиада по~направлению~<<Математика>>}{27 марта 2014}{Казахский национальный университет имени аль-Фараби}{г.~Алматы}

\begin{enumerate}
\item Вычислить предел:
$$\lim_{x \to +\infty} \frac{(x - \sqrt{x^2 - 1})^n + (x + \sqrt{x^2 - 1})^n}{x^n}.$$

\item Решить систему уравнений:
$$
\begin{cases}
\frac{x^2}{y^2} + \frac{y^2}{z^2} + \frac{z^2}{x^2} = 3 \\
x + y + z = 3 \\
\frac{y^2}{x^2} + \frac{z^2}{y^2} + \frac{x^2}{z^2} = 3 \\
\end{cases}
$$

\item Дан многочлен с целыми коэффициентами. В трех целых точках он принимает значение 2. Докажите, что ни в какой целой точке он не принимает значение 3.

\item Найти все решения дифференциального уравнения второго порядка $$f''(x) + f(\pi - x) = 1.$$

\end{enumerate}

\end{document} 
