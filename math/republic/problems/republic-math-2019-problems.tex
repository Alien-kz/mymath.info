\documentclass[11pt, a4paper]{article}

\usepackage[T2A]{fontenc}		%cyrillic output
\usepackage[utf8]{inputenc}		%cyrillic output
\usepackage[english, russian]{babel}	%word wrap
\usepackage{amssymb, amsfonts, amsmath}	%math symbols
\usepackage{mathtext}			%text in formulas
\usepackage{geometry}			%paper format attributes
\usepackage{fancyhdr}			%header
\usepackage{graphicx}			%input pictures
\usepackage{tabularx}			%smart table
\usepackage{tikz}				%draw pictures
\usetikzlibrary{patterns}		%draw pictures: fill
\usetikzlibrary{positioning}	%draw pictures: below of
\usetikzlibrary{calc}			%draw pictures: $\i$
\usepackage{enumitem}			%enumarate parameters

\geometry{left=2cm, right=2cm, top=2cm, bottom=2cm, headheight=15pt}
\setlist[enumerate]{leftmargin=*}	%remove enumarate indenttion
\sloppy							%correct overfull
\pagestyle{empty}				%no page numbers

\newcommand{\head}[4]
{
	\thispagestyle{fancy}
	\fancyhf{}
	\chead{#3, #4}

	\begin{center}
	\begin{large}
	#1 \\
	\textit{#2}\\
	\end{large}
	\end{center}

}

\DeclareMathOperator*{\argmin}{Arg\,min}

\begin{document}

\head{XI Республиканская студенческая предметная олимпиада по~направлению~<<Математика>>}{18 апреля 2019}{Казахстанский филиал МГУ имени М. В. Ломоносова}{г.~Алматы}

\begin{enumerate}

\item Найдите сумму ряда
$$
\sum_{n = 2}^{\infty} \frac{1}{2^n} \cdot \frac{n + 1}{n (n - 1)}.
$$

\item Решите уравнение
$$
x^y = x + y
$$
в положительных рациональных числах.

\item Пусть $f$ --- интегрируемая по Риману на отрезке $[0, 1]$ функция. Введём обозначение
$$
\mathcal{M}(f) = \argmin_{\lambda \in \mathbb{R}} \int\limits_{0}^{1} |f(x) - \lambda| \; dx.
$$
\begin{enumerate}
\item [а)] Докажите, что множество $\mathcal{M}(f)$ не пустое.
\item [б)] Приведите пример такой функции $f$, что множество $\mathcal{M}(f)$ не одноэлементно.
\item [в)] Докажите, что если функция $f$ монотонная, то $f\left(\frac{1}{2}\right)\in \mathcal{M}(f)$.
\item [г)] Приведите пример такой функции $f$, что $f\left(\frac{1}{2}\right)\not\in \mathcal{M}(f)$.

\textit{Примечание:} $\argmin\limits_{\lambda \in S} g(\lambda) = \left\{\mu\in S\colon g(\mu)=\inf\limits_{\lambda \in S} g(\lambda)\right\}$.
\end{enumerate}

\item Дана матрица $A \in \mathbb{R}^{2 \times 2}$ такая, что $|A^2 + A + 2019E|= 0$. Докажите, что $A^2 + A + 2019 E = O$, где $E$ --- единичная матрица $2 \times 2$, а $O$ --- нулевая матрица $2 \times 2$.

\item Дан эллипс с фокусами $A$ и $B$ и точка $C$ вне эллипса. Из точки $C$ провели касательные к эллипсу $l_1$ и $l_2$, которые образуют угол $\alpha$. Точки $A_1$ и $A_2$ --- точки, симметричные $A$ относительно касательных, $B_1$ и $B_2$ --- точки, симметричные $B$ относительно касательных. Вычислите
$$
\frac{ A_1A_2^2 + B_1B_2^2 }{CA^2 + CB^2},
$$
если
\begin{enumerate}
\item [а)] $\alpha = \frac{\pi}{2}$;
\item [б)] $\alpha = \frac{\pi}{4}$.
\end{enumerate}

\item Докажите, что система линейных уравнений с целыми коэффицентами разрешима в целых числах тогда и только тогда, когда она разрешима по любому модулю.

\end{enumerate}


\end{document} 
