\documentclass[11pt, a4paper]{article}

\usepackage[T2A]{fontenc}		%cyrillic output
\usepackage[utf8]{inputenc}		%cyrillic output
\usepackage[english, russian]{babel}	%word wrap
\usepackage{amssymb, amsfonts, amsmath}	%math symbols
\usepackage{mathtext}			%text in formulas
\usepackage{geometry}			%paper format attributes
\usepackage{fancyhdr}			%header
\usepackage{graphicx}			%input pictures
\usepackage{tabularx}			%smart table
\usepackage{tikz}				%draw pictures
\usetikzlibrary{patterns}		%draw pictures: fill
\usetikzlibrary{positioning}	%draw pictures: below of
\usetikzlibrary{calc}			%draw pictures: $\i$
\usepackage{enumitem}			%enumarate parameters

\geometry{left=2cm, right=2cm, top=2cm, bottom=2cm, headheight=15pt}
\setlist[enumerate]{leftmargin=*}	%remove enumarate indenttion
\sloppy							%correct overfull
\pagestyle{empty}				%no page numbers

\newcommand{\head}[4]
{
	\thispagestyle{fancy}
	\fancyhf{}
	\chead{#3, #4}

	\begin{center}
	\begin{large}
	#1 \\
	\textit{#2}\\
	\end{large}
	\end{center}

}

\DeclareMathOperator*{\argmin}{Arg\,min}

\begin{document}

\head{XI Республиканская студенческая предметная олимпиада по~направлению~<<Математическое и компьютерное моделирование>>}{18 апреля 2019}{Казахстанский филиал МГУ имени М. В. Ломоносова}{г.~Астана}

\begin{enumerate}

\item Пусть $n>1$. Приведите пример четырёх квадратных вырожденных матриц $A, B, C, D$ порядка $n$ таких, что блочная матрица порядка $2n$
$$
\begin{pmatrix}
A & B\\
C & D
\end{pmatrix}
$$
невырождена.

\item Найдите
$$
\lim_{n \to \infty} \sum_{k=1}^{n} \frac{\left(k - \sin{\frac{k}{n}}\right)^{2018}}{n^{2019}}.
$$

\item Даны $N$ различных точек на координатной прямой и число $K \leqslant N$. Чёрный ящик позволяет узнать координаты центра тяжести любых $K$ точек из данных. За какое минимальное количество запросов можно определить координаты всех точек?

\item Найдите все многочлены $P(x)$ с неотрицательными коэффициентами, для которых верно
$$
3 \int\limits_{-1}^{1} P(x) \; dx = P(-1) + 4 P(0) + P(1).
$$

\item Существует ли такая функция {\tt int f(int x)}, что присваивание {\tt y = f(f(x))} при всех {\tt x} даёт тот же результат, что и {\tt y = g(x)}, где {\tt g(x)} описана следующим образом?
\begin{verbatim}
а)  int g(int x) {
        return x + 2019;
    }
\end{verbatim}
\begin{verbatim}
б)  int g(int x) {
        if (x % 2 == 0)
            return x + 1;
        else
            return x - 1;
    }
\end{verbatim}
\textit{Примечание:} Все арифметические действия производятся по правилам знаковой арифметики с модулем $2^{32}$. Например, если$x=2147483647$, то $x + 1$ принимает значение $-2147483648$.

\item Дана бесконечная лента из белых и чёрных клеток. Робот начинает своё движение с некоторой клетки и движется вправо на одну клетку, если текущая клетка белая, и влево, если чёрная. Найдите математическое ожидание количества различных клеток, которые посетит этот робот.

\end{enumerate}


\end{document} 
