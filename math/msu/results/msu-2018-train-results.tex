\documentclass[12pt, a4paper]{article}

\usepackage[T2A]{fontenc}
\usepackage[utf8]{inputenc}
\usepackage[english, russian]{babel}
\usepackage{amssymb}
\usepackage{amsfonts}
\usepackage{amsmath}
\usepackage{mathtext}

\usepackage{comment}
\usepackage{geometry}
\geometry{left=1cm, right=1cm, top=2cm, bottom=2cm}

\usepackage{graphicx}
\usepackage{tikz}

\usepackage{wrapfig}
\usepackage{fancybox,fancyhdr}
\sloppy

\setlength{\headheight}{28pt}

\newcommand{\head}[4]
{
	\fancyhf{}
	\pagestyle{fancy}
	\chead{#3, #4}

	\begin{center}
	\begin{large}
	#1 \\
	\textit{#2} \\
	\end{large}
	\end{center}

}

\begin{document}

\head{Тренировочная студенческая олимпиада по математике \\ Казахстанского филиала МГУ}{13 марта 2018}{Казахстанский филиал МГУ имени М. В. Ломоносова}{г. Астана}

\begin{center}
\begin{tabular}{|l|l|l|c|*{5}{p{0.3cm}|}c|}
\hline
№ & Участник & Факультет & Курс & 1 & 2 & 3 & 4 & 5 & $\Sigma$ \\
\hline
1 & Бекмаганбетов Бекарыс & ММ & 2 & 10 & 10 & 10 & 0 & 10 & 40 \\
\hline
2 & Аскергали Ануар & ВМК & 2 & 0 & 0 & 10 & 9 & 0 & 19 \\
\hline
3 & Дукенбай Аслан & ММ & 1 & 10 & 0 & 0 & 0 & 0 & 10 \\
\hline
4 & Ергалиев Иса  & ВМК & 2 & 10 & 0 & 0 & 0 & 0 & 10 \\
\hline
5 & Сурукпаев Аслан & ММ & 2 & 0 & 10 & 0 & 0 & 0 & 10 \\
\hline
6 & Макатова Батима & ММ & 2 & 9 & 0 & 0 & 0 & 0 & 9 \\
\hline
7 & Вагнер Алан & ВМК & 1 & 0 & 0 & 0 & 9 & 0 & 9 \\
\hline
8 & Джексембаев Руслан & ММ & 1 & 9 & 0 & 0 & 0 & 0 & 9 \\
\hline
9 & Ержанов Жалгас & ВМК & 2 & 0 & 0 & 0 & 0 & 0 & 0 \\
\hline
10 & Кунакбаев Рамазан & ММ & 2 & 0 & 0 & 0 & 0 & 0 & 0 \\
\hline
\end{tabular}
\end{center}

\end{document} 
