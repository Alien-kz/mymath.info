\documentclass[14pt, a5paper]{extarticle}

\usepackage[T2A]{fontenc}		%cyrillic output
\usepackage[utf8]{inputenc}		%cyrillic output
\usepackage[english, russian]{babel}	%word wrap
\usepackage{amssymb, amsfonts, amsmath}	%math symbols
\usepackage{mathtext}			%text in formulas
\usepackage{geometry}			%paper format attributes
\usepackage{fancyhdr}			%header
\usepackage{graphicx}			%input pictures
\usepackage{tikz}				%draw pictures
\usetikzlibrary{patterns}		%draw pictures: fill
\usetikzlibrary{positioning}	%draw pictures: below of
\usetikzlibrary{calc}			%draw pictures: $\i$
\usepackage{enumitem}			%enumarate parameters

\geometry{left=1cm, right=1cm, top=2cm, bottom=1cm, headheight=15pt}
\setlist[enumerate]{leftmargin=*}	%remove enumarate indenttion
\sloppy							%correct overfull
\pagestyle{empty}				%no page numbers

\newcommand{\head}[4]
{
	\thispagestyle{fancy}
	\fancyhf{}
	\chead{#3, #4}

	\begin{center}
	\begin{large}
	#1 \\
	\textit{#2}\\
	\end{large}
	\end{center}

}

\begin{document}

\head{Открытая студенческая олимпиада по математике \\ Казахстанского филиала МГУ \\ для непрофильных специальностей}{8 декабря 2018}{Казахстанский филиал МГУ имени М. В. Ломоносова}{г. Астана}

\begin{enumerate}
\item Вычислите определитель
$$
\begin{vmatrix}
2019 & 2018 & 2018 \\
2018 & 2019 & 2018 \\
2018 & 2018 & 2019 \\
\end{vmatrix}
$$

\item Найдите все натуральные числа $n$, представимые в виде 
$$n = x^y + y^z + z^x,$$
где $x$, $y$, $z$ --- натуральные числа.


\item Вычислите
$$
\sqrt{5 \cdot \sqrt{5 \cdot \sqrt{5 \cdot \sqrt{5 \cdot \ldots}}}}
$$

\item Вычислите интеграл
$$
\int_{-2018}^{2018} \left(e^{\sin{x}} - e^{-\sin{x}} \right) \left(e^{\cos{x}} - e^{-\cos{x}} \right)dx
$$

\item Разрежьте правильный шестиугольник:
\begin{enumerate}
\item на 8 равных трапеций;
\item на 6 равных трапеций.
\end{enumerate}

\item Два города находятся на одной параллели Земли. Расстояние между ними по этой параллели равно четверти длины этой параллели, а кратчайшее расстояние по поверхности равно $\frac16$ длины экватора. Вычислите, какую часть от экватора составляет расстояние от первого города до экватора. Землю считать идеальным шаром.

\item Совокупный ожидаемый приведенный доход индивида рассчитывается по формуле
$$I = \int_{0}^{+\infty} k \cdot t \cdot e^{-rt} dt,$$
где $k\cdot t$ --- ставка дохода индивида в момент времени $t$ ($k > 0$ --- коэффициент), $r$ --- ставка дисконтирования. Найдите зависимость коэффициента $k(r)$ от ставки дисконтирования $r$, который обеспечивает совокупный ожидаемый приведенный доход индивида $I$ равным 1. Нарисуйте график зависимости $k(r)$.

\end{enumerate}

\end{document} 
