\documentclass[14pt, a5paper]{extarticle}

\usepackage[T2A]{fontenc}		%cyrillic output
\usepackage[utf8]{inputenc}		%cyrillic output
\usepackage[english, russian]{babel}	%word wrap
\usepackage{amssymb, amsfonts, amsmath}	%math symbols
\usepackage{mathtext}			%text in formulas
\usepackage{geometry}			%paper format attributes
\usepackage{fancyhdr}			%header
\usepackage{graphicx}			%input pictures
\usepackage{tikz}				%draw pictures
\usetikzlibrary{patterns}		%draw pictures: fill
\usetikzlibrary{positioning}	%draw pictures: below of
\usetikzlibrary{calc}			%draw pictures: $\i$
\usepackage{enumitem}			%enumarate parameters

\geometry{left=1cm, right=1cm, top=2cm, bottom=1cm, headheight=15pt}
\setlist[enumerate]{leftmargin=*}	%remove enumarate indenttion
\sloppy							%correct overfull
\pagestyle{empty}				%no page numbers

\newcommand{\head}[4]
{
	\thispagestyle{fancy}
	\fancyhf{}
	\chead{#3, #4}

	\begin{center}
	\begin{large}
	#1 \\
	\textit{#2}\\
	\end{large}
	\end{center}

}

\begin{document}

\head{Тренировочная студенческая олимпиада по математике \\ Казахстанского филиала МГУ}{15 марта 2014}{Казахстанский филиал МГУ имени М. В. Ломоносова}{г. Астана}

\begin{enumerate}

\item Вычислить произведение двух чисел с помощью операций сложения, вычитания, возведения в квадрат и взятия обратного числа. Возможностью обращения знаменателя в 0 пренебречь.

\item Пусть $\alpha$, $\beta$, $\gamma$ --- три различных корня уравнения $$x^3 - x - 1 = 0.$$ 
Вычислите:
$$\frac{1-\alpha}{1+\alpha} + \frac{1-\beta}{1+\beta} + \frac{1-\gamma}{1+\gamma}.$$

\item Найти максимальное значение определителя третьего порядка, у которого 2 элемента равны 4, а остальные 1 или -1.

\item Решите в комплексных числах систему:
\begin{equation*}
\begin{cases}
\sqrt{3} z^{11} - z^{10} - 1 = 0 \\
|z| = 1
\end{cases}
\end{equation*}

\item Вычислите интеграл
$$\int\limits_{-1}^{1} \frac{dx}{(e^x + 1)(x^2 + 1)} .$$

\item Из точки $P$ к параболе с фокусом $F$ провели две касательные $PX$ и $PY$. Докажите, что:
$$FP^2 = FX \cdot FY .$$

\item Обозначим $f(x) = a x^2 + b x + c$, $a > 0$. Известно, что уравнение $f(x) = x$ не имеет решений. Докажите, что последовательность 
$$a_n = \min \underbrace{f(f( \cdots f}_{n} (x) { \cdots ))} $$
--- возрастающая.

\item Даны натуральные числа $m$, $n$ такие, что $m^2 - 2 n^2 = 1$, и даны функции: 
\begin{equation*}
\begin{aligned}
f_1(x) &= \cos(x),\\
f_2(x) &= \ctg(x),\\
f_3(x) &= \arctg(x). 
\end{aligned}
\end{equation*}
Докажите, что существует композиция $H(x)$ из указанных функций (некоторые функции могут быть использованы несколько раз, а некоторые ни разу) такая, что $H(1) = \frac{m}{n}$.

\item Пусть $p_1 = 2$, $p_2 = 3$, $p_3 = 5$, ..., $p_n$ --- возрастающая последовательность всех простых чисел, не превосходящих~$2^{100}$. Докажите, что:
$$\frac{1}{p_1} + \frac{1}{p_2} + ... + \frac{1}{p_n} < 10 .$$

\end{enumerate}



\end{document} 
