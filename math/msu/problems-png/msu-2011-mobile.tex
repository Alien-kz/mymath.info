\documentclass[14pt, a5paper]{extarticle}

\usepackage[T2A]{fontenc}		%cyrillic output
\usepackage[utf8]{inputenc}		%cyrillic output
\usepackage[english, russian]{babel}	%word wrap
\usepackage{amssymb, amsfonts, amsmath}	%math symbols
\usepackage{mathtext}			%text in formulas
\usepackage{geometry}			%paper format attributes
\usepackage{fancyhdr}			%header
\usepackage{graphicx}			%input pictures
\usepackage{tikz}				%draw pictures
\usetikzlibrary{patterns}		%draw pictures: fill
\usetikzlibrary{positioning}	%draw pictures: below of
\usetikzlibrary{calc}			%draw pictures: $\i$
\usepackage{enumitem}			%enumarate parameters

\geometry{left=1cm, right=1cm, top=2cm, bottom=1cm, headheight=15pt}
\setlist[enumerate]{leftmargin=*}	%remove enumarate indenttion
\sloppy							%correct overfull
\pagestyle{empty}				%no page numbers

\newcommand{\head}[4]
{
	\thispagestyle{fancy}
	\fancyhf{}
	\chead{#3, #4}

	\begin{center}
	\begin{large}
	#1 \\
	\textit{#2}\\
	\end{large}
	\end{center}

}

\begin{document}

\head{Открытая студенческая олимпиада по математике \\ Казахстанского филиала МГУ}{10 декабря 2011}{Казахстанский филиал МГУ имени М. В. Ломоносова}{г. Астана}

\begin{enumerate}
\item Назовем конечное числовое множество своеобразным, если оно содержит число, равное количеству его элементов, но никакое его собственное подмножество этим свойством не обладает. Определите количество своеобразных подмножеств множества $\{1,\;2,\;\dots,\;12\}$.

\item Определите множество значений функции, сопоставляющей каждому прямоугольному треугольнику отношение~$\displaystyle\frac{h}{r}$, где $h$ --- высота, проведенная к гипотенузе, а $r$~---~радиус вписанной в треугольник окружности.

\item Две последовательности $\{x_n\}_{n=0}^{\infty}$ и $\{y_n\}_{n=0}^\infty$ удовлетворяют условиям:
$$
\begin{cases}
x_{n+1}=2x_n-\alpha y_n,\\
y_{n+1}=2y_n-\displaystyle\frac{1}{\alpha} x_n
\end{cases}
$$
при всех $n\geqslant 0$, где $\alpha\not=0$ --- постоянная величина, а $x_0=1$ и $y_0=0$. Найдите $x_{2012}$ и $y_{2012}$.

\item Существует ли такая последовательность вещественных чисел $\{x_n\}_{n=1}^{\infty}$, что для неё справедливы соотношения: \begin{equation*}
\left\{
\begin{aligned}
\lim\limits_{k\to\infty}x_{12k} = 20,\\
\lim\limits_{k\to\infty}x_{20k} = 12?\\
\end{aligned}
\right.
\end{equation*}

\item Найдите $\displaystyle\int\limits_{0}^{\pi/2}\left(\sin^2(\sin^2x)+\cos^2(\cos^2x)\right)\,dx$.

\item Найдите $\displaystyle\sum\limits_{n=0}^{\infty}\frac{n+2}{n!+(n+1)!+(n+2)!}$.

\item Найдите все функции $f\colon[0,\;1]\to \mathbb R$, удовлетворяющие неравенству 
$$(x-y)^2\leqslant |f(x)-f(y)|\leqslant|x-y|$$
для любых $x, y\in[0,\;1]$.

\item Пусть $x_1,\;x_2,\;\dots,\;x_n,\;\dots$ --- все положительные корни уравнения $\mathrm{tg}\,x=x$, выписанные в порядке возрастания, $n_1,\;n_2,\;\dots,\;n_k,\;\dots$ --- некоторая возрастающая последовательность натуральных чисел. Докажите, что ряды $\displaystyle\sum\limits_{k=1}^{\infty} |\cos x_{n_k}|$ и $\displaystyle\sum\limits_{k=1}^{\infty}\frac{1}{n_k}$ сходятся или расходятся одновременно.

\item Квадратная матрица $A$ порядка $n$ состоит из чисел $+1$ и $-1$. При этом, в каждой строке и в каждом столбце этой матрицы находится ровно одно число $-1$. Найдите $|\det A|$.

\item Пусть $S$ --- сумма всех обратимых элементов конечного ассоциативного кольца с единицей. Докажите, что $S^2=0$ или $S^2=S$.
\end{enumerate}


\end{document} 
