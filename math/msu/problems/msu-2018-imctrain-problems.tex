\documentclass[11pt,a4paper]{article}
\usepackage[utf8]{inputenc}
\usepackage[T2A]{fontenc}
\usepackage[russian]{babel}
\usepackage{amsmath}
\usepackage{amsfonts}
\usepackage{amssymb}
\usepackage{comment}
\usepackage[left=2cm,right=2cm,top=2cm,bottom=2cm]{geometry}

\newcommand{\head}[2]{
	\section{#2}
	\subsection{Problems}
	\begin{center}
	#1\\
	\end{center}
}

\begin{document}

Данные материалы были использованы для подготовки команды Казахстанского филиала МГУ на IMC-2018, где ребята выступили очень успешно: Бекмаганбетов Бекарыс (ММ-2) взял золото (6 задач из 10), Аскергалиев Ануар (ВМК-2) взял серебро (4 задачи из 10), а Журавская Александра (ВМК-4) получила бронзовую награду (2 задачи из 10).

В брошюре собраны 219 задач из различных источников: задачи IMC, задачи Putnam, Putnam And Beyond (Andreescu), Задачи студенческих олимпиад (Садовничий, Григорьян, Конягин) и другие.
\tableofcontents
\newpage

\head{25.06.2018}{Telescope}
\input{25-06-tele}

\head{26.06.2018}{Recurrent}
\input{26-06-recur}

\head{27.06.2018}{Linear algebra}
\input{27-06-linalg}

\head{28.06.2018}{Polynomial}
\input{28-06-polynom}

\head{29.06.2018}{Calculus}
\input{29-06-calc}

\head{02.07.2018}{Series}
\input{02-07-series}

\head{03.07.2018}{Linear algebra}
\input{03-07-linear}

\head{04.07.2018}{Calculus}
\input{04-07-calc}

\head{05.07.2018}{Sequences and series}
\input{05-07-seq}

\head{09.07.2018}{Linear algebra}
\input{09-07-linear}

\head{10.07.2018}{Number Theory}
\input{10-07-number}

\input{15-07-train}

\end{document}