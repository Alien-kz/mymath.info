\documentclass[11pt, a4paper]{article}

\usepackage[T2A]{fontenc}
\usepackage[utf8]{inputenc}
\usepackage[english, russian]{babel}
\usepackage{amssymb}
\usepackage{amsfonts}
\usepackage{amsmath}
\usepackage{mathtext}

\usepackage{comment}
\usepackage{geometry}
\geometry{left=0.5cm, right=1cm, top=1cm, bottom=1cm}
\usepackage[inline]{enumitem}

\usepackage{graphicx}
\usepackage{tikz}
\usetikzlibrary{patterns}

\usepackage{wrapfig}
\usepackage{fancybox,fancyhdr}
\sloppy

\setlength{\headheight}{28pt}
\newcommand{\variant}[2]{
	\begin{center}
	\textit{Вариант #2}
	\end{center}
}

\newcommand{\unit}[1]{\text{\textit{ #1}}}
\newcommand{\units}[2]{ \frac{\text{\textit{#1}}}{\text{\textit{#2}}}}

\newcommand{\head}[4]
{
	\fancyhf{}
	\pagestyle{fancy}
	\chead{#3, #4}

	\begin{center}
	\begin{large}
	#1 \\
	\textit{#2}\\
	\end{large}
	\end{center}

}

\begin{document}

\head{Тренировочная студенческая олимпиада по математике \\ Казахстанского филиала МГУ}{15 марта 2014}{Казахстанский филиал МГУ имени М. В. Ломоносова}{г. Астана}

\begin{enumerate}

\item Вычислить произведение двух чисел с помощью операций сложения, вычитания, возведения в квадрат и взятия обратного числа. Возможностью обращения знаменателя в 0 пренебречь.

\item Пусть $\alpha$, $\beta$, $\gamma$ --- три различных корня уравнения $$x^3 - x - 1 = 0.$$ 
Вычислите:
$$\frac{1-\alpha}{1+\alpha} + \frac{1-\beta}{1+\beta} + \frac{1-\gamma}{1+\gamma}.$$

\item Найти максимальное значение определителя третьего порядка, у которого 2 элемента равны 4, а остальные 1 или -1.

\item Решите в комплексных числах систему:
\begin{equation*}
\begin{cases}
\sqrt{3} z^{11} - z^{10} - 1 = 0 \\
|z| = 1
\end{cases}
\end{equation*}

\item Вычислите интеграл
$$\int\limits_{-1}^{1} \frac{dx}{(e^x + 1)(x^2 + 1)} .$$

\item Из точки $P$ к параболе с фокусом $F$ провели две касательные $PX$ и $PY$. Докажите, что:
$$FP^2 = FX \cdot FY .$$

\item Обозначим $f(x) = a x^2 + b x + c$, $a > 0$. Известно, что уравнение $f(x) = x$ не имеет решений. Докажите, что последовательность $a_n = \min \underbrace{f(f( \cdots f}_{n} (x) { \cdots ))} $ --- возрастающая.

\item Даны натуральные числа $m$, $n$ такие, что $m^2 - 2 n^2 = 1$, и даны функции: 
\begin{equation*}
\begin{aligned}
f_1(x) &= \cos(x),\\
f_2(x) &= \ctg(x),\\
f_3(x) &= \arctg(x). 
\end{aligned}
\end{equation*}
Докажите, что существует композиция $H(x)$ из указанных функций (некоторые функции могут быть использованы несколько раз, а некоторые ни разу) такая, что $H(1) = \frac{m}{n}$.

\item Пусть $p_1 = 2$, $p_2 = 3$, $p_3 = 5$, ..., $p_n$ --- возрастающая последовательность всех простых чисел, не превосходящих~$2^{100}$. Докажите, что:
$$\frac{1}{p_1} + \frac{1}{p_2} + ... + \frac{1}{p_n} < 10 .$$

\end{enumerate}



\end{document} 
