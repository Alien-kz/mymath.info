\documentclass[11pt, a4paper]{article}

\usepackage[T2A]{fontenc}
\usepackage[utf8]{inputenc}
\usepackage[english, russian]{babel}
\usepackage{amssymb}
\usepackage{amsfonts}
\usepackage{amsmath}
\usepackage{mathtext}

\usepackage{comment}
\usepackage{geometry}
\geometry{left=1cm, right=1cm, top=2cm, bottom=2cm}

\usepackage{graphicx}
\usepackage{tikz}

\usepackage{wrapfig}
\usepackage{fancybox,fancyhdr}
\sloppy


\newcommand{\head}[2]
{
	\fancyhf{}
	\pagestyle{fancy}
	\chead{Казахстанский филиал МГУ имени М.В.Ломоносова}

	\begin{center}
	\begin{LARGE}
	#1
	\end{LARGE}
	\end{center}

	\begin{center}
	\begin{large}
	#2
	\end{large}
	\end{center}

	\textit{
	\begin{flushright}
	Время работы:  180 минут\\
	Каждая задача оценивается в 10 баллов.\\
	\end{flushright}
	}

}


\renewcommand\thesubsubsection{}
\renewcommand\thesubsection{}
\renewcommand\thesection{}
\usepackage[indentfirst, explicit]{titlesec}
\titleformat{\section}{\Large\bfseries\center}{}{1em}{#1}
\titleformat{\subsection}{\Large\bfseries\center}{}{1em}{#1}
\titleformat{\subsubsection}{\normalsize\bfseries\center}{}{1em}{#1}

\begin{document}

\head{Открытая олимпиада по математике}{19 декабря 2015}

\begin{enumerate}

\item $a_n$, $b_n$, $x_n$, $y_n$ --- четыре арифметические прогрессии. Известно, что $a_n b_n = x_n y_n$ для трёх различных натуральных $n$. Доказать, что $a_n b_n = x_n y_n$ для всех натуральных $n$.

\item Пусть $f(n)$ --- вещественнозначная функция, определённая на множестве натуральных чисел и удовлетворяющая следующему условию: для любого $n > 1$ существует такой его простой делитель $p$, что $f(n) = f\left( \frac{n}{p} \right) - f(p)$. Известно, что $f(2015) = 2015$. Найдите $f(2016)$.

\item Найдите все дифференцируемые функции $f: \mathbb{R} \rightarrow \mathbb{R}$, удовлетворяющие следующим двум условиям:\\
а) $f'(x) = 0$ для всех $x \in \mathbb{Z}$;\\
б) если для некоторого $x_0 \in \mathbb{R}$ справедливо равенство $f'(x_0) = 0$, то для него справедливо также равенство $f(x_0) = 0$.

\item На параболе выбраны 4 точки: $A_1$, $A_2$, $A_3$ и $A_4$. Через эти точки к параболе проведены 4 касательные $l_1$, $l_2$, $l_3$ и $l_4$ соответственно. $l_1$ пересекает $l_2$ в точке $M$. $l_3$ пересекает $l_4$ в точке $N$. Докажите, что $MN$, $A_1A_3$ и $A_2A_4$ пересекаются в одной точке.

\item Решить в вещественных числах уравнение
$$4x + 2 \sin{x} + \sin( 2x + \sin{x} ) + 12 \pi = 0.$$

\item Дано $n$ натуральных чисел $a_1$, $a_2$, $\hdots$, $a_n$, при этом все они не превосходят $n$. Доказать, что существует непустое подмножество $P = \{p_1, \hdots, p_k \}$ множества $\{1, 2, \hdots, n\}$ такое, что множества $P$ и $\{ a_{p_1}, a_{p_2}, \hdots, a_{p_n} \}$ совпадают.

\item Найдите все матрицы $A$, которые обладают следующими свойствами:\\
а) имеет всего одно собственное значение (без учёта кратности);\\
б) ранг равен 1;\\
в) след равен 1.

\end{enumerate}

\end{document} 
