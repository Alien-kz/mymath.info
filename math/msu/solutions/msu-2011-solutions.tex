\documentclass[12pt, a4paper]{article}

\usepackage[T2A]{fontenc}
\usepackage[utf8]{inputenc}
\usepackage[english, russian]{babel}
\usepackage{amssymb}
\usepackage{amsfonts}
\usepackage{amsmath}
\usepackage{mathtext}

\usepackage{comment}
\usepackage{geometry}
\geometry{left=1cm, right=1cm, top=2cm, bottom=2cm}

\usepackage{graphicx}
\usepackage{tikz}

\usepackage{wrapfig}
\usepackage{fancybox,fancyhdr}
\sloppy

\setlength{\headheight}{28pt}

\newcommand{\head}[4]
{
	\fancyhf{}
	\pagestyle{fancy}
	\chead{#3, #4}

	\begin{center}
	\begin{large}
	#1 \\
	\textit{#2} \\
	\end{large}
	\end{center}

}

\begin{document}

\head{Открытая студенческая олимпиада по математике \\ Казахстанского филиала МГУ}{10 декабря 2011}{Казахстанский филиал МГУ имени М. В. Ломоносова}{г. Астана}

\begin{enumerate}
\item Ответ: 144. Так как минимальный элемент множества равен мощности множества, то указанное количество равно:
$$\sum_{k=0}^{5} C_{11-k}^k = 144.$$

\item Ответ: $(2; 1 + \sqrt{2}]$. Пусть $\alpha$ --- один из острых углов треугольника. Тогда:
$$\frac{h}{r} = \sin{\alpha} + \cos{\alpha} + 1 = \sqrt{2} \sin\left(\alpha + \frac{\pi}{4}\right) + 1.$$

\item (Абдикалыков А.К.) Ответ: $x_{2012}=\frac{1+3^{2012}}{2}$, $y_{2012}=\frac{1-3^{2012}}{2\alpha}.$

Заметим, что из условия следует $x_{n+1}+\alpha y_{n+1}=x_n+\alpha y_n$. Таким образом, $x_n+\alpha y_n=x_0+\alpha y_0=1$ для всех $n$. Исключив $y_n$ из первого рекуррентного соотношения, получим $x_{n+1}=3x_n-1$. Решив полученное с помощью замены $t_n=x_n-\displaystyle\frac{1}{2}$, найдём
$$
\begin{cases}
x_{2012}=\frac{1+3^{2012}}{2}, \\ y_{2012}=\displaystyle\frac{1-x_{2012}}{\alpha}=\frac{1-3^{2012}}{2\alpha}.
\end{cases}
$$

\item Ответ: нет. Достаточно рассмотреть подпоследовательность $\left\lbrace x_{60k} \right\rbrace$.

\item Ответ: $\frac{\pi}{2}$. Обозначим искомый интеграл как $I$. Сделаем подстановку $x=\displaystyle\frac{\pi}{2}-t$:
\begin{multline*}
I=\int\limits_{0}^{\pi/2}\left(\sin^2\left(\cos^2t\right)+\cos^2\left(\sin^2t\right)\right)\,dx=\\
=\int\limits_{0}^{\pi/2}\left(1-\cos^2\left(\cos^2t\right)+1-\sin^2\left(\sin^2t\right)\right)\,dx=\pi-I.
\end{multline*}

\item Ответ: 1. 
$$\frac{n+2}{n!+(n+1)!+(n+2)!}=$$
$$=\frac{n+2}{n!(1+n+1+(n+1)(n+2))}=$$
$$=\frac{1}{n!(n+2)}=\frac{n+1}{(n+2)!}=\frac{1}{(n+1)!}-\frac{1}{(n+2)!}.$$

\item Ответ: $x+C$ и $-x+C$, где $C$ --- постоянная. Заметим, что $|f(1)-f(0)| = 1$ (из условия). Для $t \in (0; 1)$ имеем: 
$$1 = |f(1) - f(0)| \leqslant |f(1) - f(t)| + |f(t) - f(0)| \leqslant$$
$$\leqslant (1 - t) + t = 1.$$
Следовательно, либо $f(t) = t + f(1) - 1$ для всех $t$, либо $f(t) = -t + f(1) + 1$ для всех $t$ (в зависимости от знака $f(1) - f(0)$).

\item (Абдикалыков А.К.) Нетрудно доказать, что уравнение $\mathrm{tg}\,x=x$ имеет ровно один корень на любом из отрезков вида $\left[\pi l - \displaystyle\frac{\pi}{2}, \pi l + \frac{\pi}{2}\right]$, $l\in\mathbb Z$. Таким образом, $x_n\in\left[\pi n - \displaystyle\frac{\pi}{2}, \pi n + \frac{\pi}{2}\right]$ и поэтому при $k\to\infty$
$$
|\cos x_{n_k}|=\frac{1}{\sqrt{1+\mathrm{tg}^2\,x_{n_k}}}=\frac{1}{\sqrt{1+x^2_{n_k}}}=$$
$$=O^*\left(\frac{1}{x_{n_k}}\right)=O^*\left(\frac{1}{n_k}\right),
$$
откуда и следует, что два данных ряда сходятся или расходятся одновременно.

\item (Абдикалыков А.К.) Ответ: $(n-2)\cdot 2^{n-1}$. Переставим строки матрицы $A$ так, чтобы все минус единицы стали на главную диагональ; при этом модуль определителя не изменится. Определитель изменённой матрицы находится с помощью элементарных преобразований и равен $(n-2)\cdot(-2)^{n-1}$.

\item Пусть $a_1$, ..., $a_n$ --- все обратимые элементы $S$. Тогда $\left\lbrace a_1, a_2, ..., a_n \right\rbrace = \left\lbrace a_i a_1, a_i a_2, ..., a_i a_n \right\rbrace$ для всех $i \in \lbrace 1, ..., n \rbrace$. Следовательно, $a_i S = S$ для всех $i$ для всех $i \in \lbrace 1, ..., n \rbrace$. Отсюда $S^2 = n S$.

Если $1 \neq -1$, то все обратимые элементы разбиваются на пары противоположных, т.е. $S = 0$. Если $1 = -1$, то $n = 0$ или $1$.
\end{enumerate}


\end{document} 
