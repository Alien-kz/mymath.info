\documentclass[11pt, a4paper]{article}

\usepackage[T2A]{fontenc}		%cyrillic output
\usepackage[utf8]{inputenc}		%cyrillic output
\usepackage[english, russian]{babel}	%word wrap
\usepackage{amssymb, amsfonts, amsmath}	%math symbols
\usepackage{mathtext}			%text in formulas
\usepackage{geometry}			%paper format attributes
\usepackage{fancyhdr}			%header
\usepackage{graphicx}			%input pictures
\usepackage{tikz}				%draw pictures
\usetikzlibrary{patterns}		%draw pictures: fill
\usepackage{enumitem}			%enumarate parameters

\geometry{left=1cm, right=1cm, top=2cm, bottom=1cm, headheight=15pt}
\setlist[enumerate]{leftmargin=*}	%remove enumarate indenttion
\sloppy							%correct overfull

\newcommand{\head}[4]
{
	\pagestyle{fancy}
	\fancyhf{}
	\chead{#3, #4}

	\begin{center}
	\begin{large}
	#1 \\
	\textit{#2}\\
	\end{large}
	\end{center}

}

\begin{document}

\head{Открытая студенческая олимпиада по математике \\ Казахстанского филиала МГУ}{19 декабря 2017}{Казахстанский филиал МГУ имени М. В. Ломоносова}{г. Астана}

\begin{enumerate}
\item Пример
$$
\begin{pmatrix}
1 & 0  & 0 \\
0 &  0 & 1 \\
0 & -1 & 0 \\
\end{pmatrix}
$$

\item Пример
$$
x_n = 
\begin{cases}
1, & n \text{ --- простое},\\
0, & n \text{ --- непростое}.\\ 
\end{cases}
$$

\item Обозначим интеграл
$$
I = \int\limits_{-1}^{1} \frac{x^{2k} + 2017}{2018^x + 1} \;dx ,
$$
и сделаем замену $t = -x$.
$$
I = \int\limits_{-1}^{1} \frac{t^{2k} + 2017}{2018^t + 1} \cdot 2018^t \;dt ,
$$
После сложения данных интегралов 
$$2I = \int\limits_{-1}^{1} (x^{2k} + 2017) \;dx = \frac{2}{2k+1} + 2 \cdot 2017.$$

\item Заметим, что для функции $g(x) = 3 - \frac{9}{x}$ верно, что $g(g(g(x))) = x$. Откуда получаем систему из трех неизвестных:
$$
\begin{cases}
f(x) + f\left(3 - \frac{9}{x}\right) = x - \frac{9}{x} \\
f\left(3 - \frac{9}{x}\right) + f\left(-\frac{9}{x - 3}\right) = 3 - \frac{9}{x} - \frac{9}{3 - \frac{9}{x}} \\
f\left(-\frac{9}{x - 3}\right) + f\left(x\right) = -\frac{9}{x - 3} - \frac{9}{-\frac{9}{x - 3}}
\end{cases}
$$
для всех $x \neq 0$, $x \neq 3$. Откуда находим решение $f(x) = x - \frac32$ для всех $x \neq 0$, $x \neq 3$. С учетом непрерывности, получаем, что функция доопределяется на всех числовой прямой в таком же виде.

\item Достаточно использовать 2 факта: 

1) все лучи исходящие из фокуса параболы после отражения идут параллельно оси симметрии параболы; 

2) биссектрисы внутренних односторонних углов при параллельных прямых перпендикулярны.

\item Все нечетные $n$ подходят. Для этого достаточно вспомнить, что биномиальные коффициенты обладают свойством симметрии $C_{n}^k = C_{n}^{n-k}$. В случае четного $n$, согласно постулату Бертрана среди чисел от $n/2$ до $n$ существует простое. Причем степень данного простого числа будет нечетна, чего не может быть.

\item Замена $b_n = \pi - a_n$ дает последовательность: 
$$
\begin{cases}
b_0 = \pi - 1,\\
b_{n+1} = b_n - \sin{b_n}.\\
\end{cases}. 
$$
Правая часть $b_n - \sin{b_n}$ положительна и ограничена на $(0; \pi - 1)$. Несложно показать, что последовательность будет убывающей. Следовательно, она сходится. Предел легко найти из предельного перехода: $b = \pi n$. Так как предел должен лежать в $[0; \pi - 1]$, то это 0. Ответ: $\pi$.

\item Заметим, что $f(n, k) + f(k, n) = 2^{n + k + 1}$. Это можно доказать по индукции. Ответ: $f(n, n) = 4^n$.
\end{enumerate}


\end{document} 
