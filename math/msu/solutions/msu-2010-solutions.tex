\documentclass[11pt, a4paper]{article}

\usepackage[T2A]{fontenc}		%cyrillic output
\usepackage[utf8]{inputenc}		%cyrillic output
\usepackage[english, russian]{babel}	%word wrap
\usepackage{amssymb, amsfonts, amsmath}	%math symbols
\usepackage{mathtext}			%text in formulas
\usepackage{geometry}			%paper format attributes
\usepackage{fancyhdr}			%header
\usepackage{graphicx}			%input pictures
\usepackage{tikz}				%draw pictures
\usetikzlibrary{patterns}		%draw pictures: fill
\usepackage{enumitem}			%enumarate parameters

\geometry{left=1cm, right=1cm, top=2cm, bottom=1cm, headheight=15pt}
\setlist[enumerate]{leftmargin=*}	%remove enumarate indenttion
\sloppy							%correct overfull

\newcommand{\head}[4]
{
	\pagestyle{fancy}
	\fancyhf{}
	\chead{#3, #4}

	\begin{center}
	\begin{large}
	#1 \\
	\textit{#2}\\
	\end{large}
	\end{center}

}

\begin{document}

\head{Открытая студенческая олимпиада по математике \\ Казахстанского филиала МГУ}{12 декабря 2010}{Казахстанский филиал МГУ имени М. В. Ломоносова}{г. Астана}

\begin{enumerate}

\item Композиция непрерывных функций является непрерывной функцией. Композиция сюръективных функций --- сюръективной. Значит, $h_1(x) = f(g(x))$ и $h_2(x) = g(f(x))$ --- непрерывные сюръективные функции. 

Из сюръективности $h_1(x)$ и $h_2(x)$ следует, что существуют такая точка $x_1$, что  $h_1(x_1) = 1$, и существует такая точка $x_2$, что $h_2(x_2) = 1$. Значит для функции $h(x) = h_1(x) - h_2(x)$ верно, что $h(x_1) \geqslant 0$ и $h(x_2) \leqslant 0$. По теореме Вейерштрасса, существует такая точка, что $h(x_0) = 0$, что и требовалось.

\item Ответ: 
$$\sin(1) + \sin(1) \sum_{k=1}^{\infty} \left( (-1)^k \prod_{s=0}^{2k-1} (r-s) \right) + $$
$$ + \cos(1) \sum_{k=0}^{\infty} \left( (-1)^k \prod_{s=0}^{2k} (r-s) \right) - r \bmod 2.$$

Сумма под пределом является суммой Дарбу для функции $x^r \cos{x}$ на отрезке $[0; 1]$ при равномерном разбиении:
$$\lim_{n \rightarrow \infty} \frac{1}{n} \sum_{k=1}^n \left( \frac{k}{n} \right)^r \cos \frac{k}{n} = \int_{0}^{1} x^r \cos(x) dx = I_r,$$
где $I_r$ находятся стандартным интегрированием по частям: 
$$I_r = 
\int_{0}^{1} x^r \cos{x} dx = 
\int_{0}^{1} x^r d(\sin{x}) = $$
$$ = \sin(1) + r \int_{0}^{1} x^{r-1} d(\cos{x}) = $$
$$ =\sin(1) + r \cos(1) - r(r-1) I_{r-2}.$$

После замены $J_r = \frac{I_r}{r!}$ получается простое рекуррентное соотношение:
$$J_r = \frac{1}{r!} \sin(1) + \frac{1}{(r-1)!} \cos(1) - J_{r-2},$$
которое позволяет выписать в явном виде $I_r$ при четном и нечетном $r$.


\item Рассмотрим степенной ряд $\sum_{n=1}^{\infty} a_n x^{n-1}$. Поскольку он сходится в точке $x=1$, то, по второй теореме Абеля, он сходится равномерно на $[0, 1]$. Следовательно, предельная функция непрерывна на $[0; 1]$. Отсюда следует утверждение задачи.

\item Сначала докажем, что если неравенство верно для $n$, то оно верно и для $2n$. Для этого достаточно применить неравенство для $n$ точек:
$$\frac{x_1 + x_2}{2}, \frac{x_3 + x_4}{2}, \ldots, \frac{x_{2n-1} + x_{2n}}{2}.$$
Методом математической индукции получается неравенство для всех $n = 2^k$.

Далее докажем, что если неравенство верно для $n$, то оно верно и для $n-1$. Для этого достаточно применить неравенство для $n$ точек:
$$x_1, x_2, \ldots, x_{n-1}, \frac{x_1 + x_2 + ... + x_{n-1}}{n-1}.$$
С учетом первой части, получаем, что неравенство верно для всех $n$.

\item Натуральное число $a = p_1^{\alpha_1} p_2^{\alpha_2} ... p_k^{\alpha_k}$ имеет в точности $(\alpha_1 + 1) (\alpha_2 + 1) ... (\alpha_k + 1)$ делителей. По условию задачи это число простое. Значит,  без ограничения общности, $\alpha_1 = p - 1$, а все остальные $\alpha_i = 0$. То есть $a = q ^ {p-1}$. Если $q = p$, то $a (a^k - 1)$ делится на $p$ явно. Если $q \neq p$, то $(p, q) = 1$ и можно применить малую теорему Ферма: $q^{p-1} \equiv 1 \pmod p$. Значит, $a^k \equiv 1 \pmod p$.

\item Легко заметить, что следы произведений матриц $AB$ и $BA$ всегда совпадают:
$$\mathbf{tr} AB = \sum_i (AB)_{ii} = \sum_i \sum_j a_{ij} b_{ji} =  \sum_j (BA)_{jj} = \mathbf{tr} BA.$$
Матриц, удовлетворяющих условию, не существует, так как след матрицы слева равен 0, а след матрицы справа равен размерности матрицы.

\item Ответ: $-1$. Данный ряд сходится условно. Значит, последовательность $S_N^{+}+S_N^{-}$ сходится к некоторой константе, а $S_N^{+}-S_N^{-}$ к $+\infty$. $$\lim\limits_{N \rightarrow \infty}(S_N^{+}+S_N^{-}) = C$$

Так как, $S_N^{-}$ стремится к $-\infty$, то 
$$\lim\limits_{N \rightarrow \infty} \frac{S_N^{+}}{S_N^{-}} = \lim\limits_{N \rightarrow \infty} \frac{C - S_N^{-}}{S_N^{-}} = -1.$$

\item Предлагаем решить эту задачу самостоятельно. И не забудьте прислать решение авторам пособия.

\end{enumerate}

\end{document} 
