\documentclass[11pt, a4paper]{article}

\usepackage[T2A]{fontenc}		%cyrillic output
\usepackage[utf8]{inputenc}		%cyrillic output
\usepackage[english, russian]{babel}	%word wrap
\usepackage{amssymb, amsfonts, amsmath}	%math symbols
\usepackage{mathtext}			%text in formulas
\usepackage{geometry}			%paper format attributes
\usepackage{fancyhdr}			%header
\usepackage{graphicx}			%input pictures
\usepackage{tikz}				%draw pictures
\usetikzlibrary{patterns}		%draw pictures: fill
\usetikzlibrary{positioning}	%draw pictures: below of
\usetikzlibrary{calc}			%draw pictures: $\i$
\usepackage{enumitem}			%enumarate parameters

\geometry{left=2cm, right=2cm, top=2cm, bottom=2cm, headheight=15pt}
\setlist[enumerate]{leftmargin=*}	%remove enumarate indenttion
\sloppy							%correct overfull
\pagestyle{empty}				%no page numbers

\newcommand{\head}[4]
{
	\thispagestyle{fancy}
	\fancyhf{}
	\chead{#3, #4}

	\begin{center}
	\begin{large}
	#1 \\
	\textit{#2}\\
	\end{large}
	\end{center}

}

\begin{document}

\head{Тренировочная студенческая олимпиада по математике \\ Казахстанского филиала МГУ}{13 марта 2018}{Казахстанский филиал МГУ имени М. В. Ломоносова}{г. Астана}

\begin{enumerate}
\item Последовательно для $i$ от $n$ до $2$ из $i$-й строки вычитаем $(i-1)$-ю:
$$
\det
\begin{pmatrix}
C_{0}^0 & C_{1}^1 & C_{2}^2 & \dots & C_{n-1}^{n-1} \\
0 & C_{1}^0 & C_{2}^1 & \dots & C_{n-1}^{n-2} \\
\dots & \dots & \dots & \dots & \dots \\
0 & C_{n-1}^0 & C_{n}^2 & \dots & C_{2n-3}^{n-2} \\
\end{pmatrix}
= ... = 1
$$

\item Идея.
$$
\frac{3n^2-1}{(n^3 - n)^2} = 
\frac{6n^2-2}{2 n^2 (n-1)^2 (n+1)^2} = 
$$
$$
=\frac{(n^2 + n)^2 + (n^2 - n)^2 - 2(n^2 - 1)^2 }{n^2 (n-1)^2 (n+1)^2}
=
$$
$$
=\frac{1}{2(n-1)^2} - \frac{1}{n^2}+\frac{1}{2(n+1)^2}
$$
Все слагаемые $\frac{1}{n^2}$, начиная с $n = 4$ встречаются в трех подряд идущих начальных слагаемых и в сумме сокращаются. Остаются лишь два слагаемых при $n = 2$ и одно при $n = 3$ общей суммой: $\frac{1}{2} - \frac{1}{4} + \frac{1}{8} = \frac{3}{8}$

\item $f(x) = 2 - f(x^2) = f(x^4)$. Так как функция четная, рассмотрим только положительные $x$.

Во--первых, $f(x^{4n}) = f(x)$. Пусть $0 \leqslant x < 1$. В силу непрерывности:
$$f(x) = \lim_{n \to \infty} f(x^{4n}) = f(0).$$

Во--вторых, $f(x^{\frac{1}{4n}}) = f(x)$. Пусть $x > 1$. В силу непрерывности:
$$f(x) = \lim_{n \to \infty} f(x^{\frac{1}{4n}}) = f(1).$$

В силу непрерывности и $f(1 - \varepsilon) = f(0)$, получаем, что $f(1) = f(0)$. Значит, $f(x) = const$. Из условия получаем, что $f(x) = 1$.

\item Пусть некоторый угол $n$-угольника содержит несколько углов $m$-угольников. Угол правильного $m$-угольника равен $180 \frac{m-2}{m}$. Если $m \geqslant 4$, то угол будет не менее $90^{\circ}$. Значит, $m = 3$, $n = 6$.

Пусть углы $n$-угольника и $m$-угольников равны: $n = m$. Так как сами многоугольники не совпадают, значит, есть точка на границе $n$-угольника в которой сходятся все несколько правильных $m$-угольников. То есть $180^{\circ}$ делится на $180^{\circ} \frac{n-2}{n}$, что равносильно условию $n$ делится на $n-2$, или $2$ делится на $n-2$. Получаем: $n = 4$ или $n = 3$.

Ответ: $(n, m) \in \{ (6; 3), (3; 3), (4, 4) \}$.


\item Рассмотрим функцию $g(x) = f(x) - x$:
$g(x)$ выпукла вниз, $g'(c) = 0$, $g(c) < 0$. Из выпуклости следует, что существует такая точка $a < c$, что $g(a) = g(b) = 0$. Существуют точки $a < c < b$ такие, что $f(a) = a$, $f(b) = b$.

Графики функций $f(x)$ и $f^{-1}(x)$ в квадрате $[a; b] \times [a; b]$ симметричны относительно прямой $y = x$. То есть область, ограниченная кривыми $y = x$ и $y = f(x)$, симметрична области, ограниченной кривыми $y = x$ и $y = f^{-1}(x)$:
$$
\int_{a}^{b} (f(x) + f^{-1}(x)) dx =
$$
$$= - \int_{a}^{b} (x - f(x)) dx + \int_{a}^{b} (f^{-1}(x) - x) dx + 2 \int_{a}^{b} x dx = b^2 - a^2
$$

\end{enumerate}

\end{document} 
